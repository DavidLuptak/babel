\documentclass[a4paper]{book}

\usepackage{titlesec}
\usepackage[shortlabels]{enumitem}

\usepackage[
  nil,
  bidi=basic,
  layout=tabular % counters % not for hebrew
  ]{babel}

\babelprovide[import,main]{hebrew}

\babelfont{rm}{Libertinus Serif}
\babelfont{sf}[Scale=MatchLowercase]{FreeSans}

\babeltags{lr = nil, he = hebrew}

% A quick dirty trick for Hebrew alph counters:
\makeatletter
\def\@alph#1{\char\numexpr1487+#1\relax}
\makeatother

% See below:
\newcommand\refrange[2]{\babelsublr{\texthe{\ref{#1}}-\texthe{\ref{#2}}}} 

\begin{document}

\tableofcontents

\setcounter{chapter}{5}
\chapter{אנרגיה קינטית}

\textlr{From Wikipedia. Examples with \textsf{titlesec} and
\textsf{enumitem}.}

במערכת מכנית, העבודה הכוללת הנעשית 
לאורך דרך מסוימת, על ידי כל הכוחות 
הפועלים על גוף, שווה להפרש בין האנרגיה 
הקינטית של הגוף במצבו הסופי לאנרגיה 
הקינטית במצבו ההתחלתי. אם העבודה מסומנת 
ב-$W$ והאנרגיה הקינטית ב-$E_k$, אז:
\begin{equation}
W=\Delta E_k 
\end{equation}
משוואה זו יכולה לשמש כהגדרה של אנרגיה
קינטית, ולהצטרף בכך להגדרה לפיה אנרגיה
קינטית היא צורת אנרגיה הנובעת מתנועה
בלבד.

\section{הקשר בין עבודה לאנרגיה קינטית}

במערכת מכנית, העבודה הכוללת הנעשית
לאורך דרך מסוימת, על ידי כל הכוחות
הפועלים על גוף, שווה להפרש בין האנרגיה
הקינטית של הגוף במצבו הסופי לאנרגיה
הקינטית במצבו ההתחלתי. 

\begin{enumerate}[(a),noitemsep]
  \item מסת התמד: מדד להתמדו של הגוף, כלומר
  להתנגדות שלו לתאוצה (בניסוח פיזיקלי:
  הכוח חלקי התאוצה).

  \item מסת כבידה אקטיבית: מדד לכוח הכבידה
  שהגוף מפעיל על גופים אחרים.
\end{enumerate}

\begin{enumerate}[a.,noitemsep]
  \item מסת התמד: מדד להתמדו של הגוף, כלומר
  להתנגדות שלו לתאוצה (בניסוח פיזיקלי:
  הכוח חלקי התאוצה).

  \item מסת כבידה אקטיבית: מדד לכוח הכבידה
  שהגוף מפעיל על גופים אחרים.
\begin{enumerate}[\theenumi a,noitemsep]
  \item מסת כבידה פסיבית: מדד להשפעה של כוח הכבידה על הגוף.
\end{enumerate}
\end{enumerate}

\begin{enumerate}[(1), start=4, noitemsep]
  \item מסת התמד: מדד להתמדו של הגוף, כלומר
  להתנגדות שלו לתאוצה (בניסוח פיזיקלי:
  הכוח חלקי התאוצה).

  \item מסת כבידה אקטיבית: מדד לכוח הכבידה
  שהגוף מפעיל על גופים אחרים.
\end{enumerate}

\begin{enumerate}[1., start=4, noitemsep]
  \item מסת התמד: מדד להתמדו של הגוף, כלומר
  להתנגדות שלו לתאוצה (בניסוח פיזיקלי:
  הכוח חלקי התאוצה).

  \item מסת כבידה אקטיבית: מדד לכוח הכבידה
  שהגוף מפעיל על גופים אחרים.
\begin{enumerate}[\theenumi 1, noitemsep]
  \item מסת כבידה פסיבית: מדד להשפעה של כוח הכבידה על הגוף.
\end{enumerate}
\end{enumerate}

\begin{enumerate}[(i), start=4, noitemsep]
  \item מסת התמד: מדד להתמדו של הגוף, כלומר
  להתנגדות שלו לתאוצה (בניסוח פיזיקלי:
  הכוח חלקי התאוצה).

  \item מסת כבידה אקטיבית: מדד לכוח הכבידה
  שהגוף מפעיל על גופים אחרים.
\end{enumerate}

\section{Cross references}

\begin{selectlanguage}{nil}
\textlr{This document (\textit{not} \textsf{babel}) defines (see 
the source file):}
\begin{verbatim}
\newcommand\refrange[2]{%
  \babelsublr{\texthe{\ref{#1}}-\texthe{\ref{#2}}}} 
\end{verbatim}
\end{selectlanguage}

\begin{center}
  \begin{tabular}{ll}
  \hline 
  אבג \ref{D} דהו &  {\lr\verb|\ref{D}|} \\
  אבג \ref{C}-\ref{E} דהו &  {\lr\verb|\ref{C}-\ref{E}|} \\
  אבג \refrange{C}{E} דהו &  {\lr\verb|\refrange{C}{E}|} \\
  \hline
  \end{tabular}
\end{center}

\titleformat{\section}[block]
{\normalfont\sffamily}
{\thesection}{.5em}{\titlerule\\[.8ex]\bfseries}

\section{הקשר בין עבודה לאנרגיה קינטית}\label{B}

במערכת מכנית, העבודה הכוללת הנעשית
לאורך דרך מסוימת, על ידי כל הכוחות
הפועלים על גוף, שווה להפרש בין האנרגיה
הקינטית של הגוף במצבו הסופי לאנרגיה
הקינטית במצבו ההתחלתי. 

\titleformat{\section}[frame]
  {\normalfont}
  {\filright
  \footnotesize
  \enspace סעיף \thesection\enspace}
  {8pt}
  {\Large\bfseries\filcenter}

\section{הקשר בין עבודה לאנרגיה קינטית}\label{C}

במערכת מכנית, העבודה הכוללת הנעשית
לאורך דרך מסוימת, על ידי כל הכוחות
הפועלים על גוף, שווה להפרש בין האנרגיה
הקינטית של הגוף במצבו הסופי לאנרגיה
הקינטית במצבו ההתחלתי. 

\titleformat{\section}[wrap]
  {\normalfont\sffamily\bfseries\filright}
  {\thesection.}{.5em}{}
  \titlespacing{\section}
  {8pc}{1.5ex plus .1ex minus .2ex}{1pc}

\section{הקשר בין עבודה לאנרגיה קינטית}\label{D}

במערכת מכנית, העבודה הכוללת הנעשית
לאורך דרך מסוימת, על ידי כל הכוחות
הפועלים על גוף, שווה להפרש בין האנרגיה
הקינטית של הגוף במצבו הסופי לאנרגיה
הקינטית במצבו ההתחלתי. משוואה זו,
המכונה עקרון עבודה-אנרגיה (Work-Energy
principle), מאפשרת להבין את חשיבותה של
האנרגיה הקינטית בחישובים מכניים.

\titleformat{\section}[leftmargin]
  {\normalfont
  \titlerule*[.6em]{\bfseries.}%
  \vspace{6pt}%
  \sffamily\bfseries\filright}
  {\thesection}{.8em}{}
  \titlespacing{\section}
  {6pc}{1.5ex plus .1ex minus .2ex}{1pc}

\section{הקשר בין עבודה לאנרגיה קינטית}\label{E}

במערכת מכנית, העבודה הכוללת הנעשית
לאורך דרך מסוימת, על ידי כל הכוחות
הפועלים על גוף, שווה להפרש בין האנרגיה
הקינטית של הגוף במצבו הסופי לאנרגיה
הקינטית במצבו ההתחלתי. משוואה זו,
המכונה עקרון עבודה-אנרגיה (Work-Energy
principle), מאפשרת להבין את חשיבותה של
האנרגיה הקינטית בחישובים מכניים.

\end{document}