\documentclass{article}

\usepackage[bidi=basic,english, layout=sectioning]{babel}

\babelprovide[onchar=ids fonts]{english}
\babelprovide[import, onchar=ids fonts]{syriac}

\babelfont[syriac]{rm}[Renderer=Harfbuzz]{Segoe UI Historic}

% \babelcharproperty{`0}[`9]{locale}{} 

\setlength{\parindent}{0pt}
\setlength{\parskip}{1.5mm}

\begin{document}

The font for Syriac is Segoe UI Historic, a Windows font. Estrangelo
Edessa is an alternative, but it does not render abbreviations
correctly, nor does FreeSans or Noto. I have found no Mac font for
Syriac.

\section{Numerals}

\selectlanguage{syriac}

\localenumeral{letters}{765}

\localenumeral{letters}{9164}

\localenumeral{letters}{834}

\localenumeral{letters}{21}

\localenumeral{letters}{9}

\section{From the Gospels}

% \today

1 ܟܬܒܐ ܕܝܠܝܕܘܬܗ ܕܝܫܘܥ ܡܫܝܚܐ ܒܪܗ ܕܕܘܝܕ ܒܪܗ ܕܐܒܪܗܡ ܀

2 ܐܒܪܗܡ ܐܘܠܕ ܠܐܝܤܚܩ ܐܝܤܚܩ ܐܘܠܕ ܠܝܥܩܘܒ ܝܥܩܘܒ ܐܘܠܕ ܠܝܗܘܕܐ ܘܠܐܚܘܗܝ ܀

3 ܝܗܘܕܐ ܐܘܠܕ ܠܦܪܨ ܘܠܙܪܚ ܡܢ ܬܡܪ ܦܪܨ ܐܘܠܕ ܠܚܨܪܘܢ ܚܨܪܘܢ ܐܘܠܕ ܠܐܪܡ ܀

4 ܐܪܡ ܐܘܠܕ ܠܥܡܝܢܕܒ ܥܡܝܢܕܒ ܐܘܠܕ ܠܢܚܫܘܢ ܢܚܫܘܢ ܐܘܠܕ ܠܤܠܡܘܢ ܀

\selectlanguage{english}

\section{From Wikipedia}

The East Syriac dialect is usually written in the \textit{Maḏnḥāyā}
(ܡܲܕ݂ܢܚܵܝܵܐ‎, ‘Eastern’) form of the alphabet. Other names for the
script include \textit{Swāḏāyā} (ܣܘܵܕ݂ܵܝܵܐ‎, ‘conversational’ or
‘vernacular’, often translated as ‘contemporary’, reflecting its use in
writing modern Neo-Aramaic), \textit{ʾĀṯūrāyā} (ܐܵܬ݂ܘܼܪܵܝܵܐ‎,
‘Assyrian’, not to be confused with the traditional name for the Hebrew
alphabet), \textit{Kaldāyā} (ܟܲܠܕܵܝܵܐ‎, ‘Chaldean’), and, inaccurately,
“Nestorian” (a term that was originally used to refer to the Church of
the East in the Sasanian Empire). The Eastern script resembles
\textit{ʾEsṭrangēlā} somewhat more closely than the Western script.

abcd ܐ܏ܒܓܕ efgh

\end{document}