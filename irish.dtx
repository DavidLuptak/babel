% \iffalse meta-comment
%
% Copyright 1989-2005 Johannes L. Braams and any individual authors
% listed elsewhere in this file.  All rights reserved.
% 
% This file is part of the Babel system.
% --------------------------------------
% 
% It may be distributed and/or modified under the
% conditions of the LaTeX Project Public License, either version 1.3
% of this license or (at your option) any later version.
% The latest version of this license is in
%   http://www.latex-project.org/lppl.txt
% and version 1.3 or later is part of all distributions of LaTeX
% version 2003/12/01 or later.
% 
% This work has the LPPL maintenance status "maintained".
% 
% The Current Maintainer of this work is Johannes Braams.
% 
% The list of all files belonging to the Babel system is
% given in the file `manifest.bbl. See also `legal.bbl' for additional
% information.
% 
% The list of derived (unpacked) files belonging to the distribution
% and covered by LPPL is defined by the unpacking scripts (with
% extension .ins) which are part of the distribution.
% \fi
% \CheckSum{122}
% \iffalse
%    Tell the \LaTeX\ system who we are and write an entry on the
%    transcript.
%<*dtx>
\ProvidesFile{irish.dtx}
%</dtx>
%<code>\ProvidesLanguage{irish}
%\fi
%\ProvidesFile{irish.dtx}
        [2005/03/30 v1.0h Irish support from the babel system]
%\iffalse
%% File `irish.dtx'
%% Babel package for LaTeX version 2e
%% Copyright (C) 1989 -- 2005
%%           by Johannes Braams, TeXniek
%
%% Please report errors to: J.L. Braams
%%                          babel at braams.cistron.nl
%
%    This file is part of the babel system, it provides the source
%    code for the Irish language definition file.
%
%    The Gaeilge or Irish Gaelic terms were tranlated from those
%    provided by Fraser Grant \texttt{FRASER@CERNVM} by Marion Gunn.
%<*filedriver>
\documentclass{ltxdoc}
\newcommand*{\TeXhax}{\TeX hax}
\newcommand*{\babel}{\textsf{babel}}
\newcommand*{\langvar}{$\langle \mathit lang \rangle$}
\newcommand*{\note}[1]{}
\newcommand*{\Lopt}[1]{\textsf{#1}}
\newcommand*{\file}[1]{\texttt{#1}}
\begin{document}
 \DocInput{irish.dtx}
\end{document}
%</filedriver>
%\fi
% \GetFileInfo{irish.dtx}
%
% \changes{irish-1.0b}{1995/06/14}{Corrected typo (PR1652)}
% \changes{irish-1.0e}{1996/10/10}{Replaced \cs{undefined} with
%    \cs{@undefined} and \cs{empty} with \cs{@empty} for consistency
%    with \LaTeX, moved the definition of \cs{atcatcode} right to the
%    beginning.}
%
%  \section{The Irish language}
%
%    The file \file{\filename}\footnote{The file described in this
%    section has version number \fileversion\ and was last revised on
%    \filedate. A contribution was made by Marion Gunn.}  defines all
%    the language definition macros for the Irish language.
%
%    For this language currently no special definitions are needed or
%    available.
%
% \StopEventually{}
%
%    The macro |\LdfInit| takes care of preventing that this file is
%    loaded more than once, checking the category code of the
%    \texttt{@} sign, etc.
% \changes{irish-1.0e}{1996/11/03}{Now use \cs{LdfInit} to perform
%    initial checks} 
%    \begin{macrocode}
%<*code>
\LdfInit{irish}\captionsirish
%    \end{macrocode}
%
%    When this file is read as an option, i.e. by the |\usepackage|
%    command, \texttt{irish} could be an `unknown' language in which
%    case we have to make it known.  So we check for the existence of
%    |\l@irish| to see whether we have to do something here.
%
%    \begin{macrocode}
\ifx\l@irish\@undefined
  \@nopatterns{irish}
  \adddialect\l@irish0\fi
%    \end{macrocode}
%
%    The next step consists of defining commands to switch to (and
%    from) the Irish language.
%
%  \begin{macro}{\irishhyphenmins}
%    This macro is used to store the correct values of the hyphenation
%    parameters |\lefthyphenmin| and |\righthyphenmin|.
% \changes{irish-1.0f}{1998/06/08}{Added definition of
%    \cs{hyphenmins}}
% \changes{irish-1.0h}{2000/09/22}{Now use \cs{providehyphenmins} to
%    provide a default value}
%    \begin{macrocode}
\providehyphenmins{\CurrentOption}{\tw@\thr@@}
%    \end{macrocode}
%  \end{macro}
%
% \begin{macro}{\captionsirish}
%    The macro |\captionsirish| defines all strings used in the
%    four standard documentclasses provided with \LaTeX.
% \changes{irish-1.0c}{1995/07/04}{Added \cs{proofname} for
%    AMS-\LaTeX}
% \changes{irish-1.0f}{1998/06/08}{Added missing translations provided
%    in PR 2719} 
% \changes{irish-1.0h}{2000/09/20}{Added \cs{glossaryname}}
%    \begin{macrocode}
\addto\captionsirish{%
  \def\prefacename{R\'eamhr\'a}%    <-- also "Brollach"
  \def\refname{Tagairt\'{\i}}%
  \def\abstractname{Achoimre}%
  \def\bibname{Leabharliosta}%
  \def\chaptername{Caibidil}%
  \def\appendixname{Aguis\'{\i}n}%
  \def\contentsname{Cl\'ar \'Abhair}%
  \def\listfigurename{L\'ear\'aid\'{\i}}%
  \def\listtablename{T\'abla\'{\i}}%
  \def\indexname{Inn\'eacs}%
  \def\figurename{L\'ear\'aid}%
  \def\tablename{T\'abla}%
  \def\partname{Cuid}%
  \def\enclname{faoi iamh}%
  \def\ccname{cc}%                 abrv. `c\'oip chuig'
  \def\headtoname{Go}%
  \def\pagename{Leathanach}%
  \def\seename{f\'each}%    
  \def\alsoname{f\'each freisin}% 
  \def\proofname{Cruth\'unas}% 
  \def\glossaryname{Glossary}% <-- Needs translation
  }
%    \end{macrocode}
% \end{macro}
%
% \begin{macro}{\dateirish}
%    The macro |\dateirish| redefines the command |\today| to produce
%    Irish dates.
% \changes{irish-1.0f}{1997/10/01}{Use \cs{edef} to define
%    \cs{today} to save memory}
% \changes{irish-1.0f}{1998/03/28}{use \cs{def} instead of \cs{edef}}
%    \begin{macrocode}
\def\dateirish{%
  \def\today{%
    \number\day\space \ifcase\month\or
    Ean\'air\or Feabhra\or M\'arta\or Aibre\'an\or
    Bealtaine\or Meitheamh\or I\'uil\or L\'unasa\or
    Me\'an F\'omhair\or Deireadh F\'omhair\or
    M\'{\i} na Samhna\or M\'{\i} na Nollag\fi
    \space \number\year}}
%    \end{macrocode}
% \end{macro}
%
% \begin{macro}{\extrasirish}
% \begin{macro}{\noextrasirish}
%    The macro |\extrasirish| will perform all the extra definitions
%    needed for the Irish language. The macro |\noextrasirish| is used
%    to cancel the actions of |\extrasirish|.  For the moment these
%    macros are empty but they are defined for compatibility with the
%    other language definition files.
%
%    \begin{macrocode}
\addto\extrasirish{}
\addto\noextrasirish{}
%    \end{macrocode}
% \end{macro}
% \end{macro}
%
%    The macro |\ldf@finish| takes care of looking for a
%    configuration file, setting the main language to be switched on
%    at |\begin{document}| and resetting the category code of
%    \texttt{@} to its original value.
% \changes{irish-1.0e}{1996/11/03}{Now use \cs{ldf@finish} to wrap up}
%    \begin{macrocode}
\ldf@finish{irish}
%</code>
%    \end{macrocode}
%
% \Finale
%\endinput
%% \CharacterTable
%%  {Upper-case    \A\B\C\D\E\F\G\H\I\J\K\L\M\N\O\P\Q\R\S\T\U\V\W\X\Y\Z
%%   Lower-case    \a\b\c\d\e\f\g\h\i\j\k\l\m\n\o\p\q\r\s\t\u\v\w\x\y\z
%%   Digits        \0\1\2\3\4\5\6\7\8\9
%%   Exclamation   \!     Double quote  \"     Hash (number) \#
%%   Dollar        \$     Percent       \%     Ampersand     \&
%%   Acute accent  \'     Left paren    \(     Right paren   \)
%%   Asterisk      \*     Plus          \+     Comma         \,
%%   Minus         \-     Point         \.     Solidus       \/
%%   Colon         \:     Semicolon     \;     Less than     \<
%%   Equals        \=     Greater than  \>     Question mark \?
%%   Commercial at \@     Left bracket  \[     Backslash     \\
%%   Right bracket \]     Circumflex    \^     Underscore    \_
%%   Grave accent  \`     Left brace    \{     Vertical bar  \|
%%   Right brace   \}     Tilde         \~}
%%
