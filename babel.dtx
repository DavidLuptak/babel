% \iffalse meta-comment
%
% Copyright 1989-2008 Johannes L. Braams and any individual authors
% listed elsewhere in this file.  All rights reserved.
% 
% This file is part of the Babel system.
% --------------------------------------
% 
% It may be distributed and/or modified under the
% conditions of the LaTeX Project Public License, either version 1.3
% of this license or (at your option) any later version.
% The latest version of this license is in
%   http://www.latex-project.org/lppl.txt
% and version 1.3 or later is part of all distributions of LaTeX
% version 2003/12/01 or later.
% 
% This work has the LPPL maintenance status "maintained".
% 
% The Current Maintainer of this work is Johannes Braams.
% 
% The list of all files belonging to the Babel system is
% given in the file `manifest.bbl. See also `legal.bbl' for additional
% information.
% 
% The list of derived (unpacked) files belonging to the distribution
% and covered by LPPL is defined by the unpacking scripts (with
% extension .ins) which are part of the distribution.
% \fi
% \CheckSum{3865}
%%
% \def\filename{babel.dtx}
% \let\thisfilename\filename
%
%\iffalse
% \changes{babel~3.5g}{1996/10/10}{We need at least \LaTeX\ from
%    December 1994}
% \changes{babel~3.6k}{1999/03/18}{We need at least \LaTeX\ from
%    June 1998}
%    \begin{macrocode}
%<+package>\NeedsTeXFormat{LaTeX2e}[2005/12/01]
%    \end{macrocode}
%
%% File 'babel.dtx'
%\fi
%%\ProvidesFile{babel.dtx}[2008/03/16 v3.8l The Babel package]
%\iffalse
%
% Babel DOCUMENT-STYLE option for LaTeX version 2.09 or plain TeX;
%% Babel package for LaTeX2e.
%
%% Copyright (C) 1989 -- 2008 by Johannes Braams,
%%                            TeXniek
%%                            all rights reserved.
%
%% Please report errors to: J.L. Braams
%%                          babel at braams.xs4all.nl
%<*filedriver>
\documentclass{ltxdoc}
\usepackage{supertabular}
\font\manual=logo10 % font used for the METAFONT logo, etc.
\newcommand*\MF{{\manual META}\-{\manual FONT}}
\newcommand*\TeXhax{\TeX hax}
\newcommand*\babel{\textsf{babel}}
\newcommand*\Babel{\textsf{Babel}}
\newcommand*\m[1]{\mbox{$\langle$\it#1\/$\rangle$}}
\newcommand*\langvar{\m{lang}}
\newcommand*\note[1]{}
\newcommand*\bsl{\protect\bslash}
\newcommand*\Lopt[1]{\textsf{#1}}
\newcommand*\Lenv[1]{\textsf{#1}}
\newcommand*\file[1]{\texttt{#1}}
\newcommand*\cls[1]{\texttt{#1}}
\newcommand*\pkg[1]{\texttt{#1}}
\begin{document}
 \DocInput{babel.dtx}
\end{document}
%</filedriver>
% \changes{babel~3.7a}{1997/04/16}{Make multiple loading of
%    \file{babel.def} impossible} 
%    \begin{macrocode}
%<*core>
\ifx\bbl@afterfi\@undefined
\else
  \bbl@afterfi\endinput
\fi
%</core>
%    \end{macrocode}
%<*dtx>
\ProvidesFile{babel.dtx}
%</dtx>
%\fi
%
% \GetFileInfo{babel.dtx}
%
% \changes{babel~2.0a}{1990/04/02}{Added text about \file{german.sty}}
% \changes{babel~2.0b}{1990/04/18}{Changed order of code to prevent
%    plain \TeX from seeing all of it}
% \changes{babel~2.1}{1990/04/24}{Modified user interface,
%    \cs{langTeX} no longer necessary}
% \changes{babel~2.1a}{1990/05/01}{Incorporated Nico's comments}
% \changes{babel~2.1b}{1990/05/01}{rename \cs{language} to
%    \cs{current@language}}
% \changes{babel~2.1c}{1990/05/22}{abstract for report fixed, missing
%    \texttt{\}}, found by Nicolas Brouard}
% \changes{babel~2.1d}{1990/07/04}{Missing right brace in definition of
%    abstract environment, found by Werenfried Spit}
% \changes{babel~2.1e}{1990/07/16}{Incorporated more comments from
%    Nico}
% \changes{babel~2.2}{1990/07/17}{Renamed \cs{newlanguage} to
%    \cs{addlanguage}}
% \changes{babel~2.2a}{1990/08/27}{Modified the documentation
%    somewhat}
% \changes{babel~3.0}{1991/04/23}{Moved part of the code to hyphen.doc
%    in preparation for \TeX~3.0}
% \changes{babel~3.0a}{1991/05/21}{Updated comments in various places}
% \changes{babel~3.0b}{1991/05/25}{Removed some problems in change log}
% \changes{babel~3.0c}{1991/07/15}{Renamed \file{babel.sty} and
%    \file{latexhax.sty} to \file{.com}}
% \changes{babel~3.1}{1991/10/31}{Added the support for active
%    characters and for extending a macro}
% \changes{babel~3.1}{1991/11/05}{Removed the need for
%    \file{latexhax}}
% \changes{babel~3.2}{1991/11/10}{Some Changes by br}
% \changes{babel~3.2a}{1992/02/15}{Fixups of the code and
%    documentation}
% \changes{babel~3.3}{1993/07/06}{Included driver file, and prepared
%    for distribution}
% \changes{babel~3.4}{1994/01/30}{Updated for \LaTeXe}
% \changes{babel~3.4}{1994/02/28}{Added language definition file for
%    bahasa}
% \changes{babel~3.4b}{1994/05/18}{Added a small driver to be able to
%    process just this file}
% \changes{babel~3.5a}{1995/02/03}{Provided common code to handle the
%    active double quote}
% \changes{babel~3.5c}{1995/06/14}{corrected a few typos (PR1652)}
% \changes{babel~3.5d}{1995/07/02}{Merged glyphs.dtx into this file}
% \changes{babel~3.5f}{1995/07/13}{repaired a typo}
% \changes{babel~3.5f}{1996/01/09}{replaced \cs{tmp}, \cs{bbl@tmp} and
%    \cs{bbl@temp} with \cs{bbl@tempa}}
% \changes{babel~3.5g}{1996/07/09}{replaced \cs{undefined} with
%    \cs{@undefined} to be consistent with \LaTeX}
% \changes{babel~3.7d}{1999/05/05}{Fixed a few typos in \cs{changes}
%    entries which made typesetting the code impossible}
% \changes{babel~3.7h}{2001/03/01}{Added a number of missing comment
%    characters which caused spurious white space}
% \changes{babel~3.8e}{2005/03/24}{Many enhancements to the text by
%    Andrew Young} 
%
% \title {Babel, a multilingual package for use with \LaTeX's standard
%    document classes\thanks{During the development ideas from Nico
%    Poppelier, Piet van Oostrum and many others have been used.
%    Bernd Raichle has provided many helpful suggestions.}}
%
% \author{Johannes Braams\\
%         Kersengaarde 33\\
%         2723 BP Zoetermeer\\
%         The Netherlands\\
%         \texttt{babel\char64 braams.xs4all.nl}}
%
% \date{Printed \today}
%
% \maketitle
%
%  \begin{abstract}
%    The standard distribution of \LaTeX\ contains a number of
%    document classes that are meant to be used, but also serve as
%    examples for other users to create their own document classes.
%    These document classes have become very popular among \LaTeX\
%    users. But it should be kept in mind that they were designed for
%    American tastes and typography. At one time they contained a
%    number of hard-wired texts. This report describes \babel{}, a
%    package that makes use of the new capabilities of \TeX\ version 3
%    to provide an environment in which documents can be typeset in
%    a language other than US English, or in more than one language.
%  \end{abstract}
%
%    \tableofcontents
%
% \section{The user interface}\label{U-I}
%
%    The user interface of this package is quite simple. It consists
%    of a set of commands that switch from one language to another, and
%    a set of commands that deal with shorthands. It is also possible
%    to find out what the current language is.
%
%  \DescribeMacro{\selectlanguage}
%    When a user wants to switch from one language to another he can
%    do so using the macro |\selectlanguage|. This macro takes the
%    language, defined previously by a language definition file, as
%    its argument. It calls several macros that should be defined in
%    the language definition files to activate the special definitions
%    for the language chosen.
%
%  \DescribeEnv{otherlanguage}
%    The environment \Lenv{otherlanguage} does basically the same as
%    |\selectlanguage|, except the language change is local to the
%    environment. This environment is required for intermixing
%    left-to-right typesetting with right-to-left typesetting.
%    The language to switch to is specified as an
%    argument to |\begin{otherlanguage}|.
%
%  \DescribeMacro{\foreignlanguage}
%    The command |\foreignlanguage| takes two arguments; the second
%    argument is a phrase to be typeset according to the rules of the
%    language named in its first argument. This command only switches
%    the extra definitions and the hyphenation rules for the language,
%    \emph{not} the names and dates.
%
%  \DescribeEnv{otherlanguage*}
%    In the environment \Lenv{otherlanguage*} only the typesetting
%    is done according to the rules of the other language, but the
%    text-strings such as `figure', `table', etc. are left as they
%    were set outside this environment.
%
%  \DescribeEnv{hyphenrules}
%    The environment \Lenv{hyphenrules} can be used to select
%    \emph{only} the hyphenation rules to be used. This can for
%    instance be used to select `nohyphenation', provided that in
%    \file{language.dat} the `language' nohyphenation is defined by
%    loading \file{serohyph.tex}.
%
%  \DescribeMacro{\languagename}
%    The control sequence |\languagename| contains the name of the
%    current language.
%
%  \DescribeMacro{\iflanguage}
%    If more than one language is used, it might be necessary to know
%    which language is active at a specific time. This can be checked
%    by a call to |\iflanguage|. This macro takes three arguments.
%    The first argument is the name of a language; the second and
%    third arguments are the actions to take if the result of the test
%    is \texttt{true} or \texttt{false} respectively.
%
%  \DescribeMacro{\useshorthands}
%    The command |\useshorthands| initiates the definition of
%    user-defined shorthand sequences. It has one argument, the
%    character that starts these personal shorthands.
%
%  \DescribeMacro{\defineshorthand}
%     The command |\defineshorthand| takes two arguments: the first
%     is a one- or two-character shorthand sequence, and the second is
%     the code the shorthand should expand to.
%
%  \DescribeMacro{\aliasshorthand}
%    The command |\aliasshorthand| can be used to let another
%    character perform the same functions as the default shorthand
%    character. If one prefers for example to use the character |/|
%    over |"| in typing polish texts, this can be achieved by entering
%    |\aliasshorthand{"}{/}|. \emph{Please note} that the substitute
%    shorthand character must have been declared in the preamble of
%    your document, using a command such as |\useshorthands{/}| in this
%    example.
%
%  \DescribeMacro{\languageshorthands}
%     The command |\languageshorthands| can be used to switch the
%     shorthands on the language level. It takes one argument, the
%     name of a language. Note that for this to work the language
%     should have been specified as an option when loading the \babel\
%     package.
%
%  \DescribeMacro{\shorthandon}
%  \DescribeMacro{\shorthandoff}
%    It is sometimes necessary to switch a shorthand
%    character off temporarily, because it must be used in an
%    entirely different way. For this purpose, the user commands
%    |\shorthandoff| and |\shorthandon| are provided. They each take a
%    list of characters as their arguments. The command |\shorthandoff|
%    sets the |\catcode| for each of the characters in its argument to
%    other (12); the command |\shorthandon| sets the |\catcode| to
%    active (13). Both commands only work on `known'
%    shorthand characters. If a character is not known to be a
%    shorthand character its category code will be left unchanged.
%
%  \DescribeMacro{\languageattribute}
%    This is a user-level command, to be used in the preamble of a
%    document (after |\usepackage[...]{babel}|), that declares which
%    attributes are to be used for a given language. It takes two
%    arguments: the first is the name of the language; the second,
%    a (list of) attribute(s) to used.
%    The command checks whether the language is known in this document
%    and whether the attribute(s) are known for this language.
%
% \subsection{Languages supported by \Babel}
%
%    In the following table all the languages supported by \Babel\ are
%    listed, together with the names of the options with which you can
%    load \babel\ for each language.
%
%    \begin{center}
%      \tablehead{Language & Option(s)\\\hline}
%      \tabletail{\hline}
%      \begin{supertabular}{l p{8cm}}
%        Afrikaans  & afrikaans\\
%        Bahasa     & bahasa, indonesian, indon, bahasai,
%                     bahasam, malay, meyalu\\
%        Basque     & basque\\
%        Breton     & breton\\
%        Bulgarian  & bulgarian\\
%        Catalan    & catalan\\
%        Croatian   & croatian\\
%        Czech      & czech\\
%        Danish     & danish\\
%        Dutch      & dutch\\
%        English    & english, USenglish, american, UKenglish,
%                     british, canadian, australian, newzealand\\
%        Esperanto  & esperanto\\
%        Estonian   & estonian\\
%        Finnish    & finnish\\
%        French     & french, francais, canadien, acadian\\
%        Galician   & galician\\
%        German     & austrian, german, germanb, ngerman, naustrian\\
%        Greek      & greek, polutonikogreek \\
%        Hebrew     & hebrew \\
%        Hungarian  & magyar, hungarian\\
%        Icelandic  & icelandic \\
%        Interlingua & interlingua \\
%        Irish Gaelic & irish\\
%        Italian    & italian\\
%^^A        Kannada    & kannada \\
%        Latin      & latin \\
%        Lower Sorbian & lowersorbian\\
%^^A        Devnagari  & nagari \\
%        North Sami & samin \\
%        Norwegian  & norsk, nynorsk\\
%        Polish     & polish\\
%        Portuguese & portuges, portuguese, brazilian, brazil\\
%        Romanian   & romanian\\
%        Russian    & russian\\
%^^A        Sanskrit   & sanskrit\\
%        Scottish Gaelic & scottish\\
%        Spanish    & spanish\\
%        Slovakian  & slovak\\
%        Slovenian  & slovene\\
%        Swedish    & swedish\\
%        Serbian    & serbian\\
%^^A        Tamil      & tamil \\
%        Turkish    & turkish\\
%        Ukrainian  & ukrainian\\
%        Upper Sorbian & uppersorbian\\
%        Welsh      & welsh\\
%      \end{supertabular}
%    \end{center}
%
%    For some languages \babel\ supports the options
%    \Lopt{activeacute} and \Lopt{activegrave}; for typestting Russian
%    texts, \babel\ knows about the options \Lopt{LWN} and \Lopt{LCY}
%    to specify the fontencoding of the cyrillic font used. Currently
%    only \Lopt{LWN} is supported.
%
% \subsection{Workarounds}
%
%    If you use the document class \cls{book} \emph{and} you use
%    |\ref| inside the argument of |\chapter|,
%    \LaTeX\ will keep complaining about an undefined
%    label. The reason is that the argument of |\ref| is passed through
%    |\uppercase| at some time during processing. To prevent such
%    problems, you could revert to using uppercase labels, or you can
%    use |\lowercase{\ref{foo}}| inside the argument of |\chapter|.
%
% \section{Changes for \LaTeXe}
%
%    With the advent of \LaTeXe\ the interface to \babel\ in the
%    preamble of the document has changed. With \LaTeX2.09 one used to
%    call up the \babel\ system with a line such as:
%
%\begin{verbatim}
%\documentstyle[dutch,english]{article}
%\end{verbatim}
%
%    which would tell \LaTeX\ that the document would be written in
%    two languages, Dutch and English, and that English would be the
%    first language in use.
%
%    The \LaTeXe\ way of providing the same information is:
%
%\begin{verbatim}
%\documentclass{article}
%\usepackage[dutch,english]{babel}
%\end{verbatim}
%
%    or, making \Lopt{dutch} and \Lopt{english} global options in
%    order to let other packages detect and use them:
%
%\begin{verbatim}
%\documentclass[dutch,english]{article}
%\usepackage{babel}
%\usepackage{varioref}
%\end{verbatim}
%
%    In this last example, the package \texttt{varioref} will also see
%    the options and will be able to use them.
%
% \section{Changes in \Babel\ version 3.7}
%
%    In \Babel\ version 3.7 a number of bugs that were found in
%    version~3.6 are fixed. Also a number of changes and additions
%    have occurred:
%    \begin{itemize}
%    \item Shorthands are expandable again. The disadvantage is that
%      one has to type |'{}a| when the acute accent is used as a
%      shorthand character. The advantage is that a number of other
%      problems (such as the breaking of ligatures, etc.) have
%      vanished.
%    \item Two new commands, |\shorthandon| and |\shorthandoff| have
%      been introduced to enable to temporarily switch off one or more
%      shorthands.
%^^A    \item Support for typesetting Sanskrit in transliteration is now
%^^A      available, thanks to Jun Takashima.
%^^A    \item Support for typesetting Kannada, Devnagari and Tamil is now
%^^A      available thanks to Jun Takashima.
%    \item Support for typesetting Greek has been enhanced. Code from
%      the \pkg{kdgreek} package (suggested by the author) was added
%      and |\greeknumeral| has been added.
%    \item Support for typesetting Basque is now available thanks to
%      Juan Aguirregabiria.
%    \item Support for typesetting Serbian with Latin script is now
%      available thanks to Dejan Muhamedagi\'{c} and Jankovic
%      Slobodan.
%    \item Support for typesetting Hebrew (and potential support for
%      typesetting other right-to-left written languages) is now
%      available thanks to Rama Porrat and Boris Lavva.
%    \item Support for typesetting Bulgarian is now available thanks to
%      Georgi Boshnakov.
%    \item Support for typesetting Latin is now available, thanks to
%      Claudio Beccari and Krzysztof Konrad \.Zelechowski.
%    \item Support for typesetting North Sami is now available, thanks
%      to Regnor Jernsletten.
%    \item The options \Lopt{canadian}, \Lopt{canadien} and
%      \Lopt{acadien} have been added for Canadian English and French
%      use.
%    \item A language attribute has been added to the |\mark...|
%      commands in order to make sure that a Greek header line comes
%      out right on the last page before a language switch.
%    \item Hyphenation pattern files are now read \emph{inside a
%      group}; therefore any changes a pattern file needs to make to
%      lowercase codes, uppercase codes, and category codes are kept
%      local to that group. If they are needed for the language, these
%      changes will need to be repeated and stored in |\extras...|
%    \item The concept of language attributes is introduced. It is
%      intended to give the user some control over the
%      features a language-definition file provides. Its
%      first use is for the Greek language, where the user can choose
%      the  $\pi o\lambda\upsilon\tau o\nu\kappa\acute{o}$
%      (``Polutoniko'' or multi-accented) Greek way of typesetting
%      texts. These attributes will possibly find wider use in future
%      releases.
%    \item The environment \Lenv{hyphenrules} is introduced.
%    \item The syntax of the file \file{language.dat} has been
%      extended to allow (optionally) specifying the font
%      encoding to be used while processing the patterns file.
%    \item The command |\providehyphenmins| should now be used in
%      language definition files in order to be able to keep any
%      settings provided by the pattern file.
%    \end{itemize}
%
% \section{Changes in \Babel\ version 3.6}
%
%    In \Babel\ version 3.6 a number of bugs that were found in
%    version~3.5 are fixed. Also a number of changes and additions
%    have occurred:
%    \begin{itemize}
%    \item A new environment \Lenv{otherlanguage*} is introduced. it
%      only switches the `specials', but leaves the `captions'
%      untouched.
%    \item The shorthands are no longer fully expandable. Some
%      problems could only be solved by peeking at the token following
%      an active character. The advantage is that |'{}a| works as
%      expected for languages that have the |'| active.
%    \item Support for typesetting french texts is much enhanced; the
%      file \file{francais.ldf} is now replaced by \file{frenchb.ldf}
%      which is maintained by Daniel Flipo.
%    \item Support for typesetting the russian language is again
%      available. The language definition file was originally
%      developed by Olga Lapko from CyrTUG. The fonts needed to
%      typeset the russian language are now part of the \babel\
%      distribution. The support is not yet up to the level which is
%      needed according to Olga, but this is a start.
%    \item Support for typesetting greek texts is now also
%      available. What is offered in this release is a first attempt;
%      it will be enhanced later on by Yannis Haralambous.
%    \item in \babel\ 3.6j some hooks have been added for the
%      development of support for Hebrew typesetting.
%    \item Support for typesetting texts in Afrikaans (a variant of
%      Dutch, spoken in South Africa) has been added to
%      \file{dutch.ldf}.
%    \item Support for typesetting Welsh texts is now available.
%    \item A new command |\aliasshorthand| is introduced. It seems
%      that in Poland various conventions are used to type the
%      necessary Polish letters. It is now possible to use the
%      character~|/| as a shorthand character instead of the
%      character~|"|, by issuing the command |\aliasshorthand{"}{/}|.
%    \item The shorthand mechanism now deals correctly with characters
%      that are already active.
%    \item Shorthand characters are made active at |\begin{document}|,
%      not earlier. This is to prevent problems with other packages.
%    \item A \emph{preambleonly} command |\substitutefontfamily| has
%      been added to create \file{.fd} files on the fly when the font
%      families of the Latin text differ from the families used for
%      the Cyrillic or Greek parts of the text.
%    \item Three new commands |\LdfInit|, |\ldf@quit| and
%      |\ldf@finish| are introduced that perform a number of standard
%      tasks.
%    \item In babel 3.6k the language Ukrainian has been added and the
%      support for Russian typesetting has been adapted to the package
%      'cyrillic' to be released with the December 1998 release of
%      \LaTeXe.
%    \end{itemize}
%
% \section{Changes in \Babel\ version 3.5}
%
%    In \Babel\ version 3.5 a lot of changes have been made when
%    compared with the previous release. Here is a list of the most
%    important ones:
%    \begin{itemize}
%    \item the selection of the language is delayed until
%      |\begin{document}|, which means you must
%      add appropriate |\selectlanguage| commands if you include
%      |\hyphenation| lists in the preamble of your document.
%    \item \babel\ now has a \Lenv{language} environment and a new
%      command |\foreignlanguage|;
%    \item the way active characters are dealt with is completely
%      changed. They are called `shorthands'; one can have three
%      levels of shorthands: on the user level, the language level, and
%      on `system level'. A consequence of the new way of handling
%      active characters is that they are now written to auxiliary
%      files `verbatim';
%    \item A language change now also writes information in the
%      \file{.aux} file, as the change might also affect typesetting
%      the table of contents. The consequence is that an .aux file
%      generated by a LaTeX format with babel preloaded gives errors
%      when read with a LaTeX format without babel; but I think this
%      probably doesn't occur;
%    \item \babel\ is now compatible with the \pkg{inputenc} and
%      \pkg{fontenc} packages;
%    \item the language definition files now have a new extension,
%      \file{ldf};
%    \item the syntax of the file \file{language.dat} is extended to
%      be compatible with the \pkg{french} package by Bernard Gaulle;
%    \item each language definition file looks for a configuration
%      file which has the same name, but the extension \file{.cfg}. It
%    can contain any valid \LaTeX\ code.
%    \end{itemize}
%
% \section{The interface between the core of \babel{} and the language
%    definition files}
%
%    In the core of the \babel{} system, several macros are defined
%    for use in language definition files. Their purpose
%    is to make a new language known.
%
%  \DescribeMacro{\addlanguage}
%    The macro |\addlanguage| is a non-outer version of the macro
%    |\newlanguage|, defined in \file{plain.tex} version~3.x. For
%    older versions of \file{plain.tex} and \file{lplain.tex} a
%    substitute definition is used.
%
%  \DescribeMacro{\adddialect}
%    The macro |\adddialect| can be used when two
%    languages can (or must) use the same hyphenation
%    patterns. This can also be useful for
%    languages for which no patterns are preloaded in the format. In
%    such cases the default behaviour of the \babel{} system is to
%    define this language as a `dialect' of the language for which the
%    patterns were loaded as |\language0|.
%
%    The language definition files must conform to a number of
%    conventions, because these files have to fill
%    in the gaps left by the common code in \file{babel.def}, i.\,e.,
%    the definitions of the macros that produce texts.  Also the
%    language-switching possibility which has been built into the
%    \babel{} system has its implications.
%
%    The following assumptions are made:
%   \begin{itemize}
%    \item Some of the language-specific definitions might be used by
%    plain \TeX\ users, so the files have to be coded so that they
%    can be read by both \LaTeX\ and plain \TeX. The current
%    format can be checked by looking at the value of the macro
%    |\fmtname|.
%
%    \item The common part of the \babel{} system redefines a number
%    of macros and environments (defined previously in the document
%    style) to put in the names of macros that replace the previously
%    hard-wired texts.  These macros have to be defined in the
%    language definition files.
%
%    \item The language definition files define five macros, used to
%    activate and deactivate the language-specific definitions.  These
%    macros are |\|\langvar|hyphenmins|, |\captions|\langvar,
%    |\date|\langvar, |\extras|\langvar\ and |\noextras|\langvar; where
%    \langvar\ is either the name of the language definition file or
%    the name of the \LaTeX\ option that is to be used. These
%    macros and their functions are discussed below.
%
%    \item When a language definition file is loaded, it can define
%    |\l@|\langvar\ to be a dialect of |\language0| when
%    |\l@|\langvar\ is undefined.
%
%    \item The language definition files can be read in the preamble of
%    the document, but also in the middle of document processing. This
%    means that they have to function independently of the current
%    |\catcode| of the \texttt{@}~sign.
%   \end{itemize}
%
%  \DescribeMacro{\providehyphenmins}
%    The macro |\providehyphenmins| should be used in the language
%    definition files to set the |\lefthyphenmin| and
%    |\righthyphenmin|. This macro will check whether these parameters
%    were provided by the hyphenation file before it takes any action.
%
%  \DescribeMacro{\langhyphenmins}
%    The macro |\|\langvar|hyphenmins| is used to store the values of
%    the |\lefthyphenmin| and |\righthyphenmin|.
%
%  \DescribeMacro{\captionslang}
%    The macro |\captions|\langvar\ defines the macros that
%    hold the texts to replace the original hard-wired texts.
%
%  \DescribeMacro{\datelang}
%    The macro |\date|\langvar\ defines |\today| and
%
%  \DescribeMacro{\extraslang}
%    The macro |\extras|\langvar\ contains all the extra definitions
%    needed for a specific language.
%
%  \DescribeMacro{\noextraslang}
%    Because we want to let the user switch
%    between languages, but we do not know what state \TeX\ might be in
%    after the execution of |\extras|\langvar, a macro that brings
%    \TeX\ into a predefined state is needed. It will be no surprise
%    that the name of this macro is |\noextras|\langvar.
%
%  \DescribeMacro{\bbl@declare@ttribute}
%    This is a command to be used in the language definition files for
%    declaring a language attribute. It takes three arguments: the
%    name of the language, the attribute to be defined, and the code
%    to be executed when the attribute is to be used.
%
%  \DescribeMacro{\main@language}
%    To postpone the activation of the definitions needed for a
%    language until the beginning of a document, all language
%    definition files should use |\main@language| instead of
%    |\selectlanguage|. This will just store the name of the language,
%    and the proper language will be activated at the start of the
%    document.
%
%  \DescribeMacro{\ProvidesLanguage}
%    The macro |\ProvidesLanguage| should be used to identify the
%    language definition files. Its syntax is similar to the syntax
%    of the \LaTeX\ command |\ProvidesPackage|.
%
%  \DescribeMacro{\LdfInit}
%    The macro |\LdfInit| performs a couple of standard checks that
%    must be made at the beginning of a language definition file,
%    such as checking the category code of the @-sign, preventing
%    the \file{.ldf} file from being processed twice, etc.
%
%  \DescribeMacro{\ldf@quit}
%    The macro |\ldf@quit| does work needed
%    if a \file{.ldf} file was processed
%    earlier. This includes resetting the category code
%    of the @-sign, preparing the language to be activated at
%    |\begin{document}| time, and ending the input stream.
%
%  \DescribeMacro{\ldf@finish}
%    The macro |\ldf@finish| does work needed
%    at the end of each \file{.ldf} file. This
%    includes resetting the category code of the @-sign,
%    loading a local configuration file, and preparing the language
%    to be activated at |\begin{document}| time.
%
%  \DescribeMacro{\loadlocalcfg}
%    After processing a language definition file,
%    \LaTeX\ can be instructed to load a local configuration
%    file. This file can, for instance, be used to add strings to
%    |\captions|\langvar\ to support local document
%    classes. The user will be informed that this
%    configuration file has been loaded. This macro is called by
%    |\ldf@finish|.
%
%  \DescribeMacro{\substitutefontfamily}
%    This command takes three arguments, a font encoding and two font
%    family names. It creates a font description file for the first
%    font in the given encoding. This \file{.fd} file will instruct
%    \LaTeX\ to use a font from the second family when a font from the
%    first family in the given encoding seems to be needed.
%
% \subsection{Support for active characters}
%
%    In quite a number of language definition files, active characters
%    are introduced. To facilitate this, some support macros are
%    provided.
%
% \DescribeMacro{\initiate@active@char}
%    The internal macro |\initiate@active@char| is used in language
%    definition files to instruct \LaTeX\ to give a character the
%    category code `active'. When a character has been made active it
%    will remain that way until the end of the document. Its
%    definition may vary.
%
% \DescribeMacro{\bbl@activate}
% \DescribeMacro{\bbl@deactivate}
%    The command |\bbl@activate| is used to change the way an active
%    character expands. |\bbl@activate| `switches on' the active
%    behaviour of the character. |\bbl@deactivate| lets the active
%    character expand to its former (mostly) non-active self.
%
% \DescribeMacro{\declare@shorthand}
%    The macro |\declare@shorthand| is used to define the various
%    shorthands. It takes three arguments: the name for the collection
%    of shorthands this definition belongs to; the character
%    (sequence) that makes up the shorthand, i.e.\ |~| or |"a|; and the
%    code to be executed when the shorthand is encountered.
%
% \DescribeMacro{\bbl@add@special}
% \DescribeMacro{\bbl@remove@special}
%    The \TeX book states: ``Plain \TeX\ includes a macro called
%    |\dospecials| that is
%    essentially a set macro, representing the set of all characters
%    that have a special category code.'' \cite[p.~380]{DEK} It is
%    used to set text `verbatim'.  To make this work if more
%    characters get a special category code, you have to add this
%    character to the macro |\dospecial|.  \LaTeX\ adds another macro
%    called |\@sanitize| representing the same character set, but
%    without the curly braces.  The macros
%    |\bbl@add@special|\meta{char} and
%    |\bbl@remove@special|\meta{char} add and remove the character
%    \meta{char} to these two sets.
%
% \subsection{Support for saving macro definitions}
%
%    Language definition files may want to \emph{re}define macros that
%    already exist. Therefor a mechanism for saving (and restoring)
%    the original definition of those macros is provided. We provide
%    two macros for this\footnote{This mechanism was introduced by
%    Bernd Raichle.}.
%
% \DescribeMacro{\babel@save} To save the current meaning of any
%    control sequence, the macro |\babel@save| is provided. It takes
%    one argument, \meta{csname}, the control sequence for which the
%    meaning has to be saved.
%
% \DescribeMacro{\babel@savevariable} A second macro is provided to
%    save the current value of a variable.  In this context, anything
%    that is allowed after the |\the| primitive is considered to be a
%    variable. The macro takes one argument, the \meta{variable}.
%
%    The effect of the preceding macros is to append a piece of code
%    to the current definition of |\originalTeX|. When
%    |\originalTeX| is expanded, this code restores the previous
%    definition of the control sequence or the previous value of the
%    variable.
%
% \subsection{Support for extending macros}
%
% \DescribeMacro{\addto}
%    The macro |\addto{|\meta{control sequence}|}{|\meta{\TeX\
%    code}|}| can be used to extend the definition of a macro. The
%    macro need not be defined. This macro can, for instance, be used
%    in adding instructions to a macro like |\extrasenglish|.
%
% \subsection{Macros common to a number of languages}
%
% \DescribeMacro{\allowhyphens}
%    In a couple of European languages compound words are used. This
%    means that when \TeX\ has to hyphenate such a compound word, it
%    only does so at the `\texttt{-}' that is used in such words. To
%    allow hyphenation in the rest of such a compound word, the macro
%    |\allowhyphens| can be used.
%
% \DescribeMacro{\set@low@box}
%    For some languages, quotes need to be lowered to the baseline. For
%    this purpose the macro |\set@low@box| is available. It takes one
%    argument and puts that argument in an |\hbox|, at the
%    baseline. The result is available in |\box0| for further
%    processing.
%
% \DescribeMacro{\save@sf@q}
%    Sometimes it is necessary to preserve the |\spacefactor|.  For
%    this purpose the macro |\save@sf@q| is available. It takes one
%    argument, saves the current spacefactor, executes the argument,
%    and restores the spacefactor.
%
% \DescribeMacro{\bbl@frenchspacing}
% \DescribeMacro{\bbl@nonfrenchspacing}
%    The commands |\bbl@frenchspacing| and |\bbl@nonfrenchspacing| can
%    be used to properly switch French spacing on and off.
%
% \section{Compatibility with \file{german.sty}}\label{l-h}
%
%    The file \file{german.sty} has been
%    one of the sources of inspiration for the \babel{}
%    system. Because of this I wanted to include \file{german.sty} in
%    the \babel{} system.  To be able to do that I had to allow for
%    one incompatibility: in the definition of the macro
%    |\selectlanguage| in \file{german.sty} the argument is used as the
%    {$\langle \it number \rangle$} for an |\ifcase|. So in this case
%    a call to |\selectlanguage| might look like
%    |\selectlanguage{\german}|.
%
%    In the definition of the macro |\selectlanguage| in
%    \file{babel.def} the argument is used as a part of other
%    macronames, so a call to |\selectlanguage| now looks like
%    |\selectlanguage{german}|.  Notice the absence of the escape
%    character.  As of version~3.1a of \babel{} both syntaxes are
%    allowed.
%
%    All other features of the original \file{german.sty} have been
%    copied into a new file, called \file{germanb.sty}\footnote{The
%    `b' is added to the name to distinguish the file from Partls'
%    file.}.
%
%    Although the \babel{} system was developed to be used with
%    \LaTeX, some of the features implemented in the language
%    definition files might be needed by plain \TeX\ users. Care has
%    been taken that all files in the system can be processed by plain
%    \TeX.
%
% \section{Compatibility with \file{ngerman.sty}}
%
%    When used with the options \Lopt{ngerman} or \Lopt{naustrian},
%    \babel{} will provide all features of the package \pkg{ngerman}.
%    There is however one exception:  The commands for special
%    hyphenation of double consonants (|"ff| etc.) and ck (|"ck|),
%    which are no longer required with the new German orthography, are
%    undefined. With the \pkg{ngerman} package, however, these
%    commands will generate appropriate warning messages only.
%
% \section{Compatibility with the \pkg{french} package}
%
%    It has been reported to me that the package \pkg{french} by
%    Bernard Gaulle (\texttt{gaulle@idris.fr}) works
%    together with \babel. On the other hand, it seems \emph{not} to
%    work well together with a lot of other packages. Therefore I have
%    decided to no longer load \file{french.ldf} by default. Instead,
%    when you want to use the package by Bernard Gaulle, you will have
%    to request it specifically, by passing either \Lopt{frenchle} or
%    \Lopt{frenchpro} as an option to \babel.
%
%\StopEventually{%
% \clearpage
% \let\filename\thisfilename
% \section{Conclusion}
%
%    A system of document options has been presented that enable the
%    user of \LaTeX\ to adapt the standard document classes of \LaTeX\
%    to the language he or she prefers to use. These options offer the
%    possibility of switching between languages in one document. The
%    basic interface consists of using one option, which is the same
%    for \emph{all} standard document classes.
%
%    In some cases the language definition files provide macros that
%    can be useful to plain \TeX\ users as well as to \LaTeX\ users.
%    The \babel{} system has been implemented so that it
%    can be used by both groups of users.
%
% \section{Acknowledgements}
%
%    I would like to thank all who volunteered as $\beta$-testers for
%    their time. I would like to mention Julio Sanchez who supplied
%    the option file for the Spanish language and Maurizio Codogno who
%    supplied the option file for the Italian language. Michel Goossens
%    supplied contributions for most of the other languages.  Nico
%    Poppelier helped polish the text of the documentation and
%    supplied parts of the macros for the Dutch language.  Paul
%    Wackers and Werenfried Spit helped find and repair bugs.
%
%    During the further development of the babel system I received
%    much help from Bernd Raichle, for which I am grateful.
%
%  \begin{thebibliography}{9}
%    \bibitem{DEK} Donald E. Knuth,
%      \emph{The \TeX book}, Addison-Wesley, 1986.
%    \bibitem{LLbook} Leslie Lamport,
%       \emph{\LaTeX, A document preparation System}, Addison-Wesley,
%       1986.
%    \bibitem{treebus} K.F. Treebus.
%       \emph{Tekstwijzer, een gids voor het grafisch verwerken van
%       tekst.}
%       SDU Uitgeverij ('s-Gravenhage, 1988). A Dutch book on layout
%       design and typography.
%    \bibitem{HP} Hubert Partl,
%      \emph{German \TeX}, \emph{TUGboat} 9 (1988) \#1, p.~70--72.
%     \bibitem{LLth} Leslie Lamport,
%       in: \TeXhax\ Digest, Volume 89, \#13, 17 February 1989.
%    \bibitem{BEP} Johannes Braams, Victor Eijkhout and Nico Poppelier,
%      \emph{The development of national \LaTeX\ styles},
%      \emph{TUGboat} 10 (1989) \#3, p.~401--406.
%    \bibitem{ilatex} Joachim Schrod,
%      \emph{International \LaTeX\ is ready to use},
%      \emph{TUGboat} 11 (1990) \#1, p.~87--90.
%  \end{thebibliography}
% }
%
% \section{Identification}
%
%    The file \file{babel.sty}\footnote{The file described in this
%    section is called \texttt{\filename}, has version
%    number~\fileversion\ and was last revised on~\filedate.} is meant
%    for \LaTeXe, therefor we make sure that the format file used is
%    the right one.
%
%  \begin{macro}{\ProvidesLanguage}
% \changes{babel~3.7a}{1997/03/18}{Added macro to prevent problems
%    with unexpected \cs{ProvidesFile} in plain formats because of
%    \babel.}
%    The identification code for each file is something that was
%    introduced in \LaTeXe. When the command |\ProvidesFile| does not
%    exist, a dummy definition is provided temporarily. For use in the
%    language definition file the command |\ProvidesLanguage| is
%    defined by \babel.
% \changes{babel~3.4e}{1994/06/24}{Redid the identification code,
%    provided dummy definition of \cs{ProvidesFile} for plain \TeX}
% \changes{babel~3.5f}{1995/07/26}{Store version in \cs{fileversion}}
% \changes{babel~3.5f}{1995/12/18}{Need to temporarily change the
%    definition of \cs{ProvidesFile} for December 1995 release}
% \changes{babel~3.5g}{1996/07/09}{Save a few csnames; use
%    \cs{bbl@tempa} instead of \cs{\@ProvidesFile} and store message
%    in \cs{toks8}}
%    \begin{macrocode}
%<*!package>
\ifx\ProvidesFile\@undefined
  \def\ProvidesFile#1[#2 #3 #4]{%
    \wlog{File: #1 #4 #3 <#2>}%
%<*kernel&patterns>
    \toks8{Babel <#3> and hyphenation patterns for }%
%</kernel&patterns>
    \let\ProvidesFile\@undefined
    }
%    \end{macrocode}
%    As an alternative for |\ProvidesFile| we define
%    |\ProvidesLanguage| here to be used in the language definition
%    files.
%    \begin{macrocode}
%<*kernel>
  \def\ProvidesLanguage#1[#2 #3 #4]{%
    \wlog{Language: #1 #4 #3 <#2>}%
    }
\else
%    \end{macrocode}
%    In this case we save the original definition of |\ProvidesFile| in
%    |\bbl@tempa| and restore it after we have stored the version of
%    the file in |\toks8|.
% \changes{babel~3.7a}{1997/11/04}{Removed superfluous braces}
%    \begin{macrocode}
%<*kernel&patterns>
  \let\bbl@tempa\ProvidesFile
  \def\ProvidesFile#1[#2 #3 #4]{%
    \toks8{Babel <#3> and hyphenation patterns for }%
    \bbl@tempa#1[#2 #3 #4]%
    \let\ProvidesFile\bbl@tempa}
%</kernel&patterns>
%    \end{macrocode}
%    When |\ProvidesFile| is defined we give |\ProvidesLanguage| a
%    similar definition.
%    \begin{macrocode}
  \def\ProvidesLanguage#1{%
    \begingroup
      \catcode`\ 10 %
      \@makeother\/%
      \@ifnextchar[%]
        {\@provideslanguage{#1}}{\@provideslanguage{#1}[]}}
  \def\@provideslanguage#1[#2]{%
    \wlog{Language: #1 #2}%
    \expandafter\xdef\csname ver@#1.ldf\endcsname{#2}%
    \endgroup}
%</kernel>
\fi
%</!package>
%    \end{macrocode}
%  \end{macro}
%
%    Identify each file that is produced from this source file.
% \changes{babel~3.4c}{1995/04/28}{lhyphen.cfg has become
%    lthyphen.cfg}
% \changes{babel~3.5b}{1995/01/25}{lthyphen.cfg has become hyphen.cfg}
%    \begin{macrocode}
%<+package>\ProvidesPackage{babel}
%<+core>\ProvidesFile{babel.def}
%<+kernel&patterns>\ProvidesFile{hyphen.cfg}
%<+kernel&!patterns>\ProvidesFile{switch.def}
%<+driver&!user>\ProvidesFile{babel.drv}
%<+driver&user>\ProvidesFile{user.drv}
                [2008/07/06 v3.8l %
%<+package>     The Babel package]
%<+core>         Babel common definitions]
%<+kernel>      Babel language switching mechanism]
%<+driver>]
%    \end{macrocode}
%
% \section{The Package File}
%
%    In order to make use of the features of \LaTeXe, the \babel\
%    system contains a package file, \file{babel.sty}. This file
%    is loaded by the |\usepackage| command and defines all the
%    language options known in the \babel\ system. It also takes care
%    of a number of compatibility issues with other packages.
%
%  \subsection{Language options}
%
% \changes{babel~3.6c}{1997/01/05}{When \cs{LdfInit} is undefined we
%    need to load \file{babel.def} from \file{babel.sty}}
% \changes{babel~3.6l}{1999/04/03}{Don't load \file{babel.def} now,
%    but rather define \cs{LdfInit} temporarily in order to load
%    \file{babel.def} at the right time, preventing problems with the
%    temporary definition of \cs{bbl@redefine}}
% \changes{babel~3.6r}{1999/04/12}{We \textbf{do} need to load
%    \file{babel.def} right now as \cs{ProvidesLanguage} needs to be
%    defined before the \file{.ldf} files are read and the reason for
%    for 3.6l has been removed}
%    \begin{macrocode}
%<*package>
\ifx\LdfInit\@undefined\input babel.def\relax\fi
%    \end{macrocode}
%
%    For all the languages supported we need to declare an option.
% \changes{babel~3.5a}{1995/03/14}{Changed extension of language
%    definition files to \texttt{ldf}}
% \changes{babel~3.5d}{1995/07/02}{Load language definition files
%    \emph{after} the check for the hyphenation patterns}
% \changes{babel~3.5g}{1996/10/04}{Added option \Lopt{afrikaans}}
% \changes{babel~3.7g}{2001/02/09}{Added option \Lopt{acadian}}
% \changes{babel~3.8c}{2004/06/12}{Added option \Lopt{australian}}
% \changes{babel~3.8h}{2005/11/23}{Added option \Lopt{albanian}}
%    \begin{macrocode}
\DeclareOption{acadian}{%%
%% This file will generate fast loadable files and documentation
%% driver files from the doc files in this package when run through
%% LaTeX or TeX.
%%
%% Copyright 1989-2011 Johannes L. Braams and any individual authors
%% listed elsewhere in this file.  All rights reserved.
%% 
%% This file is part of the Babel system.
%% --------------------------------------
%% 
%% It may be distributed and/or modified under the
%% conditions of the LaTeX Project Public License, either version 1.3
%% of this license or (at your option) any later version.
%% The latest version of this license is in
%%   http://www.latex-project.org/lppl.txt
%% and version 1.3 or later is part of all distributions of LaTeX
%% version 2003/12/01 or later.
%% 
%% This work has the LPPL maintenance status "maintained".
%% 
%% The Current Maintainer of this work is Johannes Braams.
%% 
%% The list of all files belonging to the LaTeX base distribution is
%% given in the file `manifest.bbl. See also `legal.bbl' for additional
%% information.
%% 
%% The list of derived (unpacked) files belonging to the distribution
%% and covered by LPPL is defined by the unpacking scripts (with
%% extension .ins) which are part of the distribution.
%%
%% --------------- start of docstrip commands ------------------
%%
\def\filedate{1999/10/30}
\def\batchfile{frenchb.ins}
\input docstrip.tex

{\ifx\generate\undefined
\Msg{**********************************************}
\Msg{*}
\Msg{* This installation requires docstrip}
\Msg{* version 2.3c or later.}
\Msg{*}
\Msg{* An older version of docstrip has been input}
\Msg{*}
\Msg{**********************************************}
\errhelp{Move or rename old docstrip.tex.}
\errmessage{Old docstrip in input path}
\batchmode
\csname @@end\endcsname
\fi}

\declarepreamble\mainpreamble
This is a generated file.

Copyright 1989-2011 Johannes L. Braams and any individual authors
listed elsewhere in this file.  All rights reserved.

This file was generated from file(s) of the Babel system.
---------------------------------------------------------

It may be distributed and/or modified under the
conditions of the LaTeX Project Public License, either version 1.3
of this license or (at your option) any later version.
The latest version of this license is in
  http://www.latex-project.org/lppl.txt
and version 1.3 or later is part of all distributions of LaTeX
version 2003/12/01 or later.

This work has the LPPL maintenance status "maintained".

The Current Maintainer of this work is Johannes Braams.

This file may only be distributed together with a copy of the Babel
system. You may however distribute the Babel system without
such generated files.

The list of all files belonging to the Babel distribution is
given in the file `manifest.bbl'. See also `legal.bbl for additional
information.

The list of derived (unpacked) files belonging to the distribution
and covered by LPPL is defined by the unpacking scripts (with
extension .ins) which are part of the distribution.
\endpreamble

\declarepreamble\fdpreamble
This is a generated file.

Copyright 1989-2011 Johannes L. Braams and any individual authors
listed elsewhere in this file.  All rights reserved.

This file was generated from file(s) of the Babel system.
---------------------------------------------------------

It may be distributed and/or modified under the
conditions of the LaTeX Project Public License, either version 1.3
of this license or (at your option) any later version.
The latest version of this license is in
  http://www.latex-project.org/lppl.txt
and version 1.3 or later is part of all distributions of LaTeX
version 2003/12/01 or later.

This work has the LPPL maintenance status "maintained".

The Current Maintainer of this work is Johannes Braams.

This file may only be distributed together with a copy of the Babel
system. You may however distribute the Babel system without
such generated files.

The list of all files belonging to the Babel distribution is
given in the file `manifest.bbl'. See also `legal.bbl for additional
information.

In particular, permission is granted to customize the declarations in
this file to serve the needs of your installation.

However, NO PERMISSION is granted to distribute a modified version
of this file under its original name.

\endpreamble

\keepsilent

\usedir{tex/generic/babel} 

\usepreamble\mainpreamble
\generate{\file{frenchb.ldf}{\from{frenchb.dtx}{code}}
          }
\nopreamble
\nopostamble
\generate{\file{frenchb.cfg}{\from{frenchb.dtx}{cfg}}
          }

\ifToplevel{
\Msg{***********************************************************}
\Msg{*}
\Msg{* To finish the installation you have to move the following}
\Msg{* files into a directory searched by TeX:}
\Msg{*}
\Msg{* \space\space All *.def, *.fd, *.ldf, *.sty}
\Msg{*}
\Msg{* To produce the documentation run the files ending with}
\Msg{* '.dtx' and `.fdd' through LaTeX.}
\Msg{*}
\Msg{* Happy TeXing}
\Msg{***********************************************************}
}
 
\endinput




}
\DeclareOption{albanian}{% \iffalse meta-com

% Copyright 1989-2007 Johannes L. Braams and any individual aut
% listed elsewhere in this file.  All rights reser

% This file is part of the Babel sys
% ----------------------------------

% It may be distributed and/or modified under
% conditions of the LaTeX Project Public License, either version
% of this license or (at your option) any later vers
% The latest version of this license i
%   http://www.latex-project.org/lppl
% and version 1.3 or later is part of all distributions of L
% version 2003/12/01 or la

% This work has the LPPL maintenance status "maintain

% The Current Maintainer of this work is Johannes Bra

% The list of all files belonging to the Babel syste
% given in the file `manifest.bbl. See also `legal.bbl' for additi
% informat

% The list of derived (unpacked) files belonging to the distribu
% and covered by LPPL is defined by the unpacking scripts (
% extension .ins) which are part of the distribut
%
% \CheckSum{
% \iff
%    Tell the \LaTeX\ system who we are and write an entry on
%    transcr
%<*
\ProvidesFile{albanian.
%</
%<code>\ProvidesLanguage{alban

%\ProvidesFile{albanian.
       [2007/10/20 v1.0c albanian support from the babel sys
%\iff
% Babel package for LaTeX versio
% Copyright (C) 1989 -
%           by Johannes Braams, TeX

% Please report errors to: J.L. Br
%                          babel at braamsdot xs4all do

%    This file is part of the babel system, it provides the so
%    code for the albanian language definition file.  A contribu
%    was made by Adi Zaimi (adizaimi at yahoo.c

%<*filedri
\documentclass{ltx
\newcommand*\TeXhax{\TeX
\newcommand*\babel{\textsf{bab
\newcommand*\langvar{$\langle \it lang \rang
\newcommand*\note[
\newcommand*\Lopt[1]{\textsf{
\newcommand*\file[1]{\texttt{
\begin{docum
 \DocInput{albanian.
\end{docum
%</filedri

% \GetFileInfo{albanian.
% \changes{albanian-1.0a}{2005/11/21}{Started first version of the f
% \changes{albanian-1.0b}{2005/12/10}{A number of corrections in
%    translations from Adi Za
% \changes{albanian-1.0c}{2006/06/05}{Small documentation

%  \section{The Albanian langu

%    The file \file{\filename}\footnote{The file described in
%    section has version number \fileversion\ and was last revise
%    \filedate} defines all the language definition macros for
%    Albanian languag

%    Albanian is written in a latin script, but it has 36 lette
%    9 which are diletters (dh, gj, ll, nj, rr, sh, th, xh, z
%    and two extra special charact

%    For this language the character |"| is made active
%    table~\ref{tab:albanian-quote} an overview is given of
%    purpo

%    \begin{table}[
%     \begin{cen
%     \begin{tabular}{lp{8
%      |"c| & |\"c|, also implemented for the uppercase
%      |"-| & an explicit hyphen sign, allowing hyphena
%                  in the rest of the word.
%      \verb="|= & disable ligature at this position
%      |""| & like |"-|, but producing no hyphen
%                  (for compund words with hyphen, e.g.\ |x-""y|)
%      |"`| & for Albanian left double quotes (looks like ,,).
%      |"'| & for Albanian right double quotes.
%      |"<| & for French left double quotes (similar to $<<$).
%      |">| & for French right double quotes (similar to $>>$).
%     \end{tabu
%     \caption{The extra definitions
%              by \file{albanian.ldf}}\label{tab:albanian-qu
%     \end{cen
%    \end{ta

%    Apart from defining shorthands we need to make sure that
%    first paragraph of each section is intended. Furthermore
%    following new math operators are defined (|\tg|, |\c
%    |\arctg|, |\arcctg|, |\sh|, |\ch|, |\th|, |\cth|, |\ar
%    |\arch|, |\arth|, |\arcth|, |\Prob|, |\Expect|, |\Varianc

% \StopEventual

%    The macro |\LdfInit| takes care of preventing that this fil
%    loaded more than once, checking the category code of
%    \texttt{@} sign,
%    \begin{macroc
%<*c
\LdfInit{albanian}\captionsalba
%    \end{macroc

%    When this file is read as an option, i.e. by the |\usepack
%    command, \texttt{albanian} will be an `unknown' language in w
%    case we have to make it known. So we check for the existenc
%    |\l@albanian| to see whether we have to do something h

%    \begin{macroc
\ifx\l@albanian\@undef
    \@nopatterns{Alban
    \adddialect\l@albanian
%    \end{macroc

%    The next step consists of defining commands to switch to
%    from) the Albanian langu

%  \begin{macro}{\captionsalban
%    The macro |\captionsalbanian| defines all strings
%    in the four standard documentclasses provided with \La
%    \begin{macroc
\addto\captionsalbani
  \def\prefacename{Parathen
  \def\refname{Referenc
  \def\abstractname{P\"ermbledh
  \def\bibname{Bibliograf
  \def\chaptername{Kapitul
  \def\appendixname{Shte
  \def\contentsname{P\"ermbaj
  \def\listfigurename{Figur
  \def\listtablename{Tabel
  \def\indexname{Indek
  \def\figurename{Figu
  \def\tablename{Tabe
  \def\partname{Pje
  \def\enclname{Lidh
  \def\ccname{Kop
  \def\headtoname{P\"
  \def\pagename{Fa
  \def\seename{shi
  \def\alsoname{shiko d
  \def\proofname{V\"ertet
  \def\glossaryname{P\"erhasja e Fjal\"ev

%    \end{macroc
%  \end{ma

%  \begin{macro}{\datealban
%    The macro |\datealbanian| redefines the command |\today
%    produce Albanian da
%    \begin{macroc
\def\datealbani
  \def\today{\number\day~\ifcase\mont
    Janar\or Shkurt\or Mars\or Prill\or Ma
    Qershor\or Korrik\or Gusht\or Shtator\or Tetor\or N\"ento
    Dhjetor\fi \space \number\ye
%    \end{macroc
%  \end{ma

%  \begin{macro}{\extrasalban
%  \begin{macro}{\noextrasalban
%    The macro |\extrasalbanian| will perform all the e
%    definitions needed for the Albanian language. The m
%    |\noextrasalbanian| is used to cancel the action
%    |\extrasalbanian

%    For Albanian the \texttt{"} character is made active. This is
%    once, later on its definition may vary. Other languages in
%    same document may also use the \texttt{"} character
%    shorthands; we specify that the albanian group of shorth
%    should be u

%    \begin{macroc
\initiate@active@cha
\addto\extrasalbanian{\languageshorthands{albani
\addto\extrasalbanian{\bbl@activate
%    \end{macroc
%    Don't forget to turn the shorthands off ag
%    \begin{macroc
\addto\noextrasalbanian{\bbl@deactivate
%    \end{macroc
%    First we define shorthands to facilitate the occurence of let
%    such as \v
%    \begin{macroc
\declare@shorthand{albanian}{"c}{\textormath{\v c}{\check
\declare@shorthand{albanian}{"e}{\textormath{\v e}{\check
\declare@shorthand{albanian}{"C}{\textormath{\v C}{\check
\declare@shorthand{albanian}{"E}{\textormath{\v E}{\check
%    \end{macroc

%    Then we define access to two forms of quotation marks, sim
%    to the german and french quotation ma
%    \begin{macroc
\declare@shorthand{albanian}{"
  \textormath{\quotedblbase{}}{\mbox{\quotedblbas
\declare@shorthand{albanian}{"
  \textormath{\textquotedblleft{}}{\mbox{\textquotedbllef
\declare@shorthand{albanian}{"
  \textormath{\guillemotleft{}}{\mbox{\guillemotlef
\declare@shorthand{albanian}{"
  \textormath{\guillemotright{}}{\mbox{\guillemotrigh
%    \end{macroc
%    then we define two shorthands to be able to specify hyphena
%    breakpoints that behave a little different from |
%    \begin{macroc
\declare@shorthand{albanian}{"-}{\nobreak-\bbl@allowhyph
\declare@shorthand{albanian}{""}{\hskip\z@s
%    \end{macroc
%    And we want to have a shorthand for disabling a ligat
%    \begin{macroc
\declare@shorthand{albanian}{"
  \textormath{\discretionary{-}{}{\kern.03em}
%    \end{macroc
%  \end{ma
%  \end{ma

%  \begin{macro}{\bbl@frenchind
%  \begin{macro}{\bbl@nonfrenchind
%    In albanian the first paragraph of each section should be inden
%    Add this code only in \La
%    \begin{macroc
\ifx\fmtname plain \
  \let\@aifORI\@afterindentf
  \def\bbl@frenchindent{\let\@afterindentfalse\@afterindent
                        \@afterindentt
  \def\bbl@nonfrenchindent{\let\@afterindentfalse\@ai
                          \@afterindentfa
  \addto\extrasalbanian{\bbl@frenchind
  \addto\noextrasalbanian{\bbl@nonfrenchind

%    \end{macroc
%  \end{ma
%  \end{ma

%  \begin{macro}{\mathalban
%    Some math functions in Albanian math books have other na
%    e.g. |sinh| in Albanian is written as |sh| etc. So we defi
%    number of new math operat
%    \begin{macroc
\def\sh{\mathop{\operator@font sh}\nolimits} % same as \
\def\ch{\mathop{\operator@font ch}\nolimits} % same as \
\def\th{\mathop{\operator@font th}\nolimits} % same as \
\def\cth{\mathop{\operator@font cth}\nolimits} % same as \
\def\arsh{\mathop{\operator@font arsh}\nolim
\def\arch{\mathop{\operator@font arch}\nolim
\def\arth{\mathop{\operator@font arth}\nolim
\def\arcth{\mathop{\operator@font arcth}\nolim
\def\tg{\mathop{\operator@font tg}\nolimits} % same as
\def\ctg{\mathop{\operator@font ctg}\nolimits} % same as
\def\arctg{\mathop{\operator@font arctg}\nolimits} % same as \ar
\def\arcctg{\mathop{\operator@font arcctg}\nolim
\def\Prob{\mathop{\mathsf P\hskip0pt}\nolim
\def\Expect{\mathop{\mathsf E\hskip0pt}\nolim
\def\Variance{\mathop{\mathsf D\hskip0pt}\nolim
%    \end{macroc
%  \end{ma

%    The macro |\ldf@finish| takes care of looking f
%    configuration file, setting the main language to be switche
%    at |\begin{document}| and resetting the category cod
%    \texttt{@} to its original va
%    \begin{macroc
\ldf@finish{alban
%</c
%    \end{macroc

% \Fi
%% \CharacterT
%%  {Upper-case    \A\B\C\D\E\F\G\H\I\J\K\L\M\N\O\P\Q\R\S\T\U\V\W\X
%%   Lower-case    \a\b\c\d\e\f\g\h\i\j\k\l\m\n\o\p\q\r\s\t\u\v\w\x
%%   Digits        \0\1\2\3\4\5\6\7
%%   Exclamation   \!     Double quote  \"     Hash (number
%%   Dollar        \$     Percent       \%     Ampersand
%%   Acute accent  \'     Left paren    \(     Right paren
%%   Asterisk      \*     Plus          \+     Comma
%%   Minus         \-     Point         \.     Solidus
%%   Colon         \:     Semicolon     \;     Less than
%%   Equals        \=     Greater than  \>     Question mar
%%   Commercial at \@     Left bracket  \[     Backslash
%%   Right bracket \]     Circumflex    \^     Underscore
%%   Grave accent  \`     Left brace    \{     Vertical bar
%%   Right brace   \}     Tilde

\endi

}
\DeclareOption{afrikaans}{%%
%% This file will generate fast loadable files and documentation
%% driver files from the doc files in this package when run through
%% LaTeX or TeX.
%%
%% Copyright 1989-2005 Johannes L. Braams and any individual authors
%% listed elsewhere in this file.  All rights reserved.
%% 
%% This file is part of the Babel system.
%% --------------------------------------
%% 
%% It may be distributed and/or modified under the
%% conditions of the LaTeX Project Public License, either version 1.3
%% of this license or (at your option) any later version.
%% The latest version of this license is in
%%   http://www.latex-project.org/lppl.txt
%% and version 1.3 or later is part of all distributions of LaTeX
%% version 2003/12/01 or later.
%% 
%% This work has the LPPL maintenance status "maintained".
%% 
%% The Current Maintainer of this work is Johannes Braams.
%% 
%% The list of all files belonging to the LaTeX base distribution is
%% given in the file `manifest.bbl. See also `legal.bbl' for additional
%% information.
%% 
%% The list of derived (unpacked) files belonging to the distribution
%% and covered by LPPL is defined by the unpacking scripts (with
%% extension .ins) which are part of the distribution.
%%
%% --------------- start of docstrip commands ------------------
%%
\def\filedate{1999/04/11}
\def\batchfile{dutch.ins}
\input docstrip.tex

{\ifx\generate\undefined
\Msg{**********************************************}
\Msg{*}
\Msg{* This installation requires docstrip}
\Msg{* version 2.3c or later.}
\Msg{*}
\Msg{* An older version of docstrip has been input}
\Msg{*}
\Msg{**********************************************}
\errhelp{Move or rename old docstrip.tex.}
\errmessage{Old docstrip in input path}
\batchmode
\csname @@end\endcsname
\fi}

\declarepreamble\mainpreamble
This is a generated file.

Copyright 1989-2005 Johannes L. Braams and any individual authors
listed elsewhere in this file.  All rights reserved.

This file was generated from file(s) of the Babel system.
---------------------------------------------------------

It may be distributed and/or modified under the
conditions of the LaTeX Project Public License, either version 1.3
of this license or (at your option) any later version.
The latest version of this license is in
  http://www.latex-project.org/lppl.txt
and version 1.3 or later is part of all distributions of LaTeX
version 2003/12/01 or later.

This work has the LPPL maintenance status "maintained".

The Current Maintainer of this work is Johannes Braams.

This file may only be distributed together with a copy of the Babel
system. You may however distribute the Babel system without
such generated files.

The list of all files belonging to the Babel distribution is
given in the file `manifest.bbl'. See also `legal.bbl for additional
information.

The list of derived (unpacked) files belonging to the distribution
and covered by LPPL is defined by the unpacking scripts (with
extension .ins) which are part of the distribution.
\endpreamble

\declarepreamble\fdpreamble
This is a generated file.

Copyright 1989-2005 Johannes L. Braams and any individual authors
listed elsewhere in this file.  All rights reserved.

This file was generated from file(s) of the Babel system.
---------------------------------------------------------

It may be distributed and/or modified under the
conditions of the LaTeX Project Public License, either version 1.3
of this license or (at your option) any later version.
The latest version of this license is in
  http://www.latex-project.org/lppl.txt
and version 1.3 or later is part of all distributions of LaTeX
version 2003/12/01 or later.

This work has the LPPL maintenance status "maintained".

The Current Maintainer of this work is Johannes Braams.

This file may only be distributed together with a copy of the Babel
system. You may however distribute the Babel system without
such generated files.

The list of all files belonging to the Babel distribution is
given in the file `manifest.bbl'. See also `legal.bbl for additional
information.

In particular, permission is granted to customize the declarations in
this file to serve the needs of your installation.

However, NO PERMISSION is granted to distribute a modified version
of this file under its original name.

\endpreamble

\keepsilent

\usedir{tex/generic/babel} 

\usepreamble\mainpreamble
\generate{\file{dutch.ldf}{\from{dutch.dtx}{code}}
          }
\usepreamble\fdpreamble

\ifToplevel{
\Msg{***********************************************************}
\Msg{*}
\Msg{* To finish the installation you have to move the following}
\Msg{* files into a directory searched by TeX:}
\Msg{*}
\Msg{* \space\space All *.def, *.fd, *.ldf, *.sty}
\Msg{*}
\Msg{* To produce the documentation run the files ending with}
\Msg{* '.dtx' and `.fdd' through LaTeX.}
\Msg{*}
\Msg{* Happy TeXing}
\Msg{***********************************************************}
}
 
\endinput
}
\DeclareOption{american}{%%
%% This file will generate fast loadable files and documentation
%% driver files from the doc files in this package when run through
%% LaTeX or TeX.
%%
%% Copyright 1989-2005 Johannes L. Braams and any individual authors
%% listed elsewhere in this file.  All rights reserved.
%% 
%% This file is part of the Babel system.
%% --------------------------------------
%% 
%% It may be distributed and/or modified under the
%% conditions of the LaTeX Project Public License, either version 1.3
%% of this license or (at your option) any later version.
%% The latest version of this license is in
%%   http://www.latex-project.org/lppl.txt
%% and version 1.3 or later is part of all distributions of LaTeX
%% version 2003/12/01 or later.
%% 
%% This work has the LPPL maintenance status "maintained".
%% 
%% The Current Maintainer of this work is Johannes Braams.
%% 
%% The list of all files belonging to the LaTeX base distribution is
%% given in the file `manifest.bbl. See also `legal.bbl' for additional
%% information.
%% 
%% The list of derived (unpacked) files belonging to the distribution
%% and covered by LPPL is defined by the unpacking scripts (with
%% extension .ins) which are part of the distribution.
%%
%% --------------- start of docstrip commands ------------------
%%
\def\filedate{1999/04/11}
\def\batchfile{english.ins}
\input docstrip.tex

{\ifx\generate\undefined
\Msg{**********************************************}
\Msg{*}
\Msg{* This installation requires docstrip}
\Msg{* version 2.3c or later.}
\Msg{*}
\Msg{* An older version of docstrip has been input}
\Msg{*}
\Msg{**********************************************}
\errhelp{Move or rename old docstrip.tex.}
\errmessage{Old docstrip in input path}
\batchmode
\csname @@end\endcsname
\fi}

\declarepreamble\mainpreamble
This is a generated file.

Copyright 1989-2005 Johannes L. Braams and any individual authors
listed elsewhere in this file.  All rights reserved.

This file was generated from file(s) of the Babel system.
---------------------------------------------------------

It may be distributed and/or modified under the
conditions of the LaTeX Project Public License, either version 1.3
of this license or (at your option) any later version.
The latest version of this license is in
  http://www.latex-project.org/lppl.txt
and version 1.3 or later is part of all distributions of LaTeX
version 2003/12/01 or later.

This work has the LPPL maintenance status "maintained".

The Current Maintainer of this work is Johannes Braams.

This file may only be distributed together with a copy of the Babel
system. You may however distribute the Babel system without
such generated files.

The list of all files belonging to the Babel distribution is
given in the file `manifest.bbl'. See also `legal.bbl for additional
information.

The list of derived (unpacked) files belonging to the distribution
and covered by LPPL is defined by the unpacking scripts (with
extension .ins) which are part of the distribution.
\endpreamble

\declarepreamble\fdpreamble
This is a generated file.

Copyright 1989-2005 Johannes L. Braams and any individual authors
listed elsewhere in this file.  All rights reserved.

This file was generated from file(s) of the Babel system.
---------------------------------------------------------

It may be distributed and/or modified under the
conditions of the LaTeX Project Public License, either version 1.3
of this license or (at your option) any later version.
The latest version of this license is in
  http://www.latex-project.org/lppl.txt
and version 1.3 or later is part of all distributions of LaTeX
version 2003/12/01 or later.

This work has the LPPL maintenance status "maintained".

The Current Maintainer of this work is Johannes Braams.

This file may only be distributed together with a copy of the Babel
system. You may however distribute the Babel system without
such generated files.

The list of all files belonging to the Babel distribution is
given in the file `manifest.bbl'. See also `legal.bbl for additional
information.

In particular, permission is granted to customize the declarations in
this file to serve the needs of your installation.

However, NO PERMISSION is granted to distribute a modified version
of this file under its original name.

\endpreamble

\keepsilent

\usedir{tex/generic/babel} 

\usepreamble\mainpreamble
\generate{\file{english.ldf}{\from{english.dtx}{code}}
          }
\usepreamble\fdpreamble

\ifToplevel{
\Msg{***********************************************************}
\Msg{*}
\Msg{* To finish the installation you have to move the following}
\Msg{* files into a directory searched by TeX:}
\Msg{*}
\Msg{* \space\space All *.def, *.fd, *.ldf, *.sty}
\Msg{*}
\Msg{* To produce the documentation run the files ending with}
\Msg{* '.dtx' and `.fdd' through LaTeX.}
\Msg{*}
\Msg{* Happy TeXing}
\Msg{***********************************************************}
}
 
\endinput
}
\DeclareOption{australian}{%%
%% This file will generate fast loadable files and documentation
%% driver files from the doc files in this package when run through
%% LaTeX or TeX.
%%
%% Copyright 1989-2005 Johannes L. Braams and any individual authors
%% listed elsewhere in this file.  All rights reserved.
%% 
%% This file is part of the Babel system.
%% --------------------------------------
%% 
%% It may be distributed and/or modified under the
%% conditions of the LaTeX Project Public License, either version 1.3
%% of this license or (at your option) any later version.
%% The latest version of this license is in
%%   http://www.latex-project.org/lppl.txt
%% and version 1.3 or later is part of all distributions of LaTeX
%% version 2003/12/01 or later.
%% 
%% This work has the LPPL maintenance status "maintained".
%% 
%% The Current Maintainer of this work is Johannes Braams.
%% 
%% The list of all files belonging to the LaTeX base distribution is
%% given in the file `manifest.bbl. See also `legal.bbl' for additional
%% information.
%% 
%% The list of derived (unpacked) files belonging to the distribution
%% and covered by LPPL is defined by the unpacking scripts (with
%% extension .ins) which are part of the distribution.
%%
%% --------------- start of docstrip commands ------------------
%%
\def\filedate{1999/04/11}
\def\batchfile{english.ins}
\input docstrip.tex

{\ifx\generate\undefined
\Msg{**********************************************}
\Msg{*}
\Msg{* This installation requires docstrip}
\Msg{* version 2.3c or later.}
\Msg{*}
\Msg{* An older version of docstrip has been input}
\Msg{*}
\Msg{**********************************************}
\errhelp{Move or rename old docstrip.tex.}
\errmessage{Old docstrip in input path}
\batchmode
\csname @@end\endcsname
\fi}

\declarepreamble\mainpreamble
This is a generated file.

Copyright 1989-2005 Johannes L. Braams and any individual authors
listed elsewhere in this file.  All rights reserved.

This file was generated from file(s) of the Babel system.
---------------------------------------------------------

It may be distributed and/or modified under the
conditions of the LaTeX Project Public License, either version 1.3
of this license or (at your option) any later version.
The latest version of this license is in
  http://www.latex-project.org/lppl.txt
and version 1.3 or later is part of all distributions of LaTeX
version 2003/12/01 or later.

This work has the LPPL maintenance status "maintained".

The Current Maintainer of this work is Johannes Braams.

This file may only be distributed together with a copy of the Babel
system. You may however distribute the Babel system without
such generated files.

The list of all files belonging to the Babel distribution is
given in the file `manifest.bbl'. See also `legal.bbl for additional
information.

The list of derived (unpacked) files belonging to the distribution
and covered by LPPL is defined by the unpacking scripts (with
extension .ins) which are part of the distribution.
\endpreamble

\declarepreamble\fdpreamble
This is a generated file.

Copyright 1989-2005 Johannes L. Braams and any individual authors
listed elsewhere in this file.  All rights reserved.

This file was generated from file(s) of the Babel system.
---------------------------------------------------------

It may be distributed and/or modified under the
conditions of the LaTeX Project Public License, either version 1.3
of this license or (at your option) any later version.
The latest version of this license is in
  http://www.latex-project.org/lppl.txt
and version 1.3 or later is part of all distributions of LaTeX
version 2003/12/01 or later.

This work has the LPPL maintenance status "maintained".

The Current Maintainer of this work is Johannes Braams.

This file may only be distributed together with a copy of the Babel
system. You may however distribute the Babel system without
such generated files.

The list of all files belonging to the Babel distribution is
given in the file `manifest.bbl'. See also `legal.bbl for additional
information.

In particular, permission is granted to customize the declarations in
this file to serve the needs of your installation.

However, NO PERMISSION is granted to distribute a modified version
of this file under its original name.

\endpreamble

\keepsilent

\usedir{tex/generic/babel} 

\usepreamble\mainpreamble
\generate{\file{english.ldf}{\from{english.dtx}{code}}
          }
\usepreamble\fdpreamble

\ifToplevel{
\Msg{***********************************************************}
\Msg{*}
\Msg{* To finish the installation you have to move the following}
\Msg{* files into a directory searched by TeX:}
\Msg{*}
\Msg{* \space\space All *.def, *.fd, *.ldf, *.sty}
\Msg{*}
\Msg{* To produce the documentation run the files ending with}
\Msg{* '.dtx' and `.fdd' through LaTeX.}
\Msg{*}
\Msg{* Happy TeXing}
\Msg{***********************************************************}
}
 
\endinput
}
%    \end{macrocode}
%    Austrian is really a dialect of German.
% \changes{babel~3.6i}{1997/02/20}{Added the \Lopt{Basque} option}
%    \begin{macrocode}
\DeclareOption{austrian}{% \iffalse meta-comm

% Copyright 1989-2008 Johannes L. Braams and any individual auth
% listed elsewhere in this file.  All rights reserv

% This file is part of the Babel syst
% -----------------------------------

% It may be distributed and/or modified under
% conditions of the LaTeX Project Public License, either version
% of this license or (at your option) any later versi
% The latest version of this license is
%   http://www.latex-project.org/lppl.
% and version 1.3 or later is part of all distributions of La
% version 2003/12/01 or lat

% This work has the LPPL maintenance status "maintaine

% The Current Maintainer of this work is Johannes Braa

% The list of all files belonging to the Babel system
% given in the file `manifest.bbl. See also `legal.bbl' for additio
% informati

% The list of derived (unpacked) files belonging to the distribut
% and covered by LPPL is defined by the unpacking scripts (w
% extension .ins) which are part of the distributi
%
% \CheckSum{3

% \iffa
%    Tell the \LaTeX\ system who we are and write an entry on
%    transcri
%<*d
\ProvidesFile{germanb.d
%</d
%<code>\ProvidesLanguage{germa
%
%\ProvidesFile{germanb.d
        [2008/06/01 v2.6m German support from the babel syst
%\iffa
%% File `germanb.d
%% Babel package for LaTeX version
%% Copyright (C) 1989 - 2
%%           by Johannes Braams, TeXn

%% Germanb Language Definition F
%% Copyright (C) 1989 - 2
%%           by Bernd Raichle raichle at azu.Informatik.Uni-Stuttgart
%%              Johannes Braams, TeXn
% This file is based on german.tex version 2.
%                       by Bernd Raichle, Hubert Partl et.

%% Please report errors to: J.L. Bra
%%                          babel at braams.xs4all

%<*filedriv
\documentclass{ltxd
\font\manual=logo10 % font used for the METAFONT logo, e
\newcommand*\MF{{\manual META}\-{\manual FON
\newcommand*\TeXhax{\TeX h
\newcommand*\babel{\textsf{babe
\newcommand*\langvar{$\langle \it lang \rangl
\newcommand*\note[1
\newcommand*\Lopt[1]{\textsf{#
\newcommand*\file[1]{\texttt{#
\begin{docume
 \DocInput{germanb.d
\end{docume
%</filedriv
%
% \GetFileInfo{germanb.d

% \changes{germanb-1.0a}{1990/05/14}{Incorporated Nico's commen
% \changes{germanb-1.0b}{1990/05/22}{fixed typo in definition
%    austrian language found by Werenfried S
%    \texttt{nspit@fys.ruu.n
% \changes{germanb-1.0c}{1990/07/16}{Fixed some typ
% \changes{germanb-1.1}{1990/07/30}{When using PostScript fonts w
%    the Adobe fontencoding, the dieresis-accent is located elsewhe
%    modified co
% \changes{germanb-1.1a}{1990/08/27}{Modified the documentat
%    somewh
% \changes{germanb-2.0}{1991/04/23}{Modified for babel 3
% \changes{germanb-2.0a}{1991/05/25}{Removed some problems in cha
%    l
% \changes{germanb-2.1}{1991/05/29}{Removed bug found by van der Me
% \changes{germanb-2.2}{1991/06/11}{Removed global assignmen
%    brought uptodate with \file{german.tex} v2.
% \changes{germanb-2.2a}{1991/07/15}{Renamed \file{babel.sty}
%    \file{babel.co
% \changes{germanb-2.3}{1991/11/05}{Rewritten parts of the code to
%    the new features of babel version 3
% \changes{germanb-2.3e}{1991/11/10}{Brought up-to-date w
%    \file{german.tex} v2.3e (plus some bug fixes) [b
% \changes{germanb-2.5}{1994/02/08}{Update or \LaTe
% \changes{germanb-2.5c}{1994/06/26}{Removed the use of \cs{fileda
%    and moved the identification after the loading
%    \file{babel.de
% \changes{germanb-2.6a}{1995/02/15}{Moved the identification to
%    top of the fi
% \changes{germanb-2.6a}{1995/02/15}{Rewrote the code that handles
%    active double quote charact
% \changes{germanb-2.6d}{1996/07/10}{Replaced \cs{undefined} w
%    \cs{@undefined} and \cs{empty} with \cs{@empty} for consiste
%    with \LaTe
% \changes{germanb-2.6d}{1996/10/10}{Moved the definition
%    \cs{atcatcode} right to the beginning

%  \section{The German langua

%    The file \file{\filename}\footnote{The file described in t
%    section has version number \fileversion\ and was last revised
%    \filedate.}  defines all the language definition macros for
%    German language as well as for the Austrian dialect of t
%    language\footnote{This file is a re-implementation of Hub
%    Partl's \file{german.sty} version 2.5b, see~\cite{HP}

%    For this language the character |"| is made active.
%    table~\ref{tab:german-quote} an overview is given of
%    purpose. One of the reasons for this is that in the Ger
%    language some character combinations change when a word is bro
%    between the combination. Also the vertical placement of
%    umlaut can be controlled this w
%    \begin{table}[h
%     \begin{cent
%     \begin{tabular}{lp{8c
%      |"a| & |\"a|, also implemented for the ot
%                  lowercase and uppercase vowels.
%      |"s| & to produce the German \ss{} (like |\ss{}|).
%      |"z| & to produce the German \ss{} (like |\ss{}|).
%      |"ck|& for |ck| to be hyphenated as |k-k|.
%      |"ff|& for |ff| to be hyphenated as |ff-
%                  this is also implemented for l, m, n, p, r and
%      |"S| & for |SS| to be |\uppercase{"s}|.
%      |"Z| & for |SZ| to be |\uppercase{"z}|.
%      \verb="|= & disable ligature at this position.
%      |"-| & an explicit hyphen sign, allowing hyphenat
%             in the rest of the word.
%      |""| & like |"-|, but producing no hyphen s
%             (for compund words with hyphen, e.g.\ |x-""y|).
%      |"~| & for a compound word mark without a breakpoint.
%      |"=| & for a compound word mark with a breakpoint, allow
%             hyphenation in the composing words.
%      |"`| & for German left double quotes (looks like ,,).
%      |"'| & for German right double quotes.
%      |"<| & for French left double quotes (similar to $<<$).
%      |">| & for French right double quotes (similar to $>>$).
%     \end{tabul
%     \caption{The extra definitions m
%              by \file{german.ldf}}\label{tab:german-quo
%     \end{cent
%    \end{tab
%    The quotes in table~\ref{tab:german-quote} can also be typeset
%    using the commands in table~\ref{tab:more-quot
%    \begin{table}[h
%     \begin{cent
%     \begin{tabular}{lp{8c
%      |\glqq| & for German left double quotes (looks like ,,).
%      |\grqq| & for German right double quotes (looks like ``).
%      |\glq|  & for German left single quotes (looks like ,).
%      |\grq|  & for German right single quotes (looks like `).
%      |\flqq| & for French left double quotes (similar to $<<$).
%      |\frqq| & for French right double quotes (similar to $>>$)
%      |\flq|  & for (French) left single quotes (similar to $<$).
%      |\frq|  & for (French) right single quotes (similar to $>$).
%      |\dq|   & the original quotes character (|"|).
%     \end{tabul
%     \caption{More commands which produce quotes, defi
%              by \file{german.ldf}}\label{tab:more-quo
%     \end{cent
%    \end{tab

% \StopEventuall

%    When this file was read through the option \Lopt{germanb} we m
%    it behave as if \Lopt{german} was specifi
% \changes{german-2.6l}{2008/03/17}{Making germanb behave like ger
%    needs some more work besides defining \cs{CurrentOptio
% \changes{germanb-2.6m}{2008/06/01}{Correted a ty
%    \begin{macroco
\def\bbl@tempa{germa
\ifx\CurrentOption\bbl@te
  \def\CurrentOption{germ
  \ifx\l@german\@undefi
    \@nopatterns{Germ
    \adddialect\l@germ

  \let\l@germanb\l@ger
  \AtBeginDocumen
    \let\captionsgermanb\captionsger
    \let\dategermanb\dateger
    \let\extrasgermanb\extrasger
    \let\noextrasgermanb\noextrasger


%    \end{macroco

%    The macro |\LdfInit| takes care of preventing that this file
%    loaded more than once, checking the category code of
%    \texttt{@} sign, e
% \changes{germanb-2.6d}{1996/11/02}{Now use \cs{LdfInit} to perf
%    initial check
%    \begin{macroco
%<*co
\LdfInit\CurrentOption{captions\CurrentOpti
%    \end{macroco

%    When this file is read as an option, i.e., by the |\usepacka
%    command, \texttt{german} will be an `unknown' language, so
%    have to make it known.  So we check for the existence
%    |\l@german| to see whether we have to do something he

% \changes{germanb-2.0}{1991/04/23}{Now use \cs{adddialect}
%    language undefin
% \changes{germanb-2.2d}{1991/10/27}{Removed use of \cs{@ifundefine
% \changes{germanb-2.3e}{1991/11/10}{Added warning, if no ger
%    patterns load
% \changes{germanb-2.5c}{1994/06/26}{Now use \cs{@nopatterns}
%    produce the warni
%    \begin{macroco
\ifx\l@german\@undefi
  \@nopatterns{Germ
  \adddialect\l@germ

%    \end{macroco

%    For the Austrian version of these definitions we just add anot
%    languag
% \changes{germanb-2.0}{1991/04/23}{Now use \cs{adddialect}
%    austri
%    \begin{macroco
\adddialect\l@austrian\l@ger
%    \end{macroco

%    The next step consists of defining commands to switch to (
%    from) the German langua

%  \begin{macro}{\captionsgerm
%  \begin{macro}{\captionsaustri
%    Either the macro |\captionsgerman| or the ma
%    |\captionsaustrian| will define all strings used in the f
%    standard document classes provided with \LaT

% \changes{germanb-2.2}{1991/06/06}{Removed \cs{global} definitio
% \changes{germanb-2.2}{1991/06/06}{\cs{pagename} should
%    \cs{headpagenam
% \changes{germanb-2.3e}{1991/11/10}{Added \cs{prefacenam
%    \cs{seename} and \cs{alsonam
% \changes{germanb-2.4}{1993/07/15}{\cs{headpagename} should
%    \cs{pagenam
% \changes{germanb-2.6b}{1995/07/04}{Added \cs{proofname}
%    AMS-\LaT
% \changes{germanb-2.6d}{1996/07/10}{Construct control sequence on
%    f
% \changes{germanb-2.6j}{2000/09/20}{Added \cs{glossarynam
%    \begin{macroco
\@namedef{captions\CurrentOption
  \def\prefacename{Vorwor
  \def\refname{Literatu
  \def\abstractname{Zusammenfassun
  \def\bibname{Literaturverzeichni
  \def\chaptername{Kapite
  \def\appendixname{Anhan
  \def\contentsname{Inhaltsverzeichnis}%    % oder nur: Inh
  \def\listfigurename{Abbildungsverzeichni
  \def\listtablename{Tabellenverzeichni
  \def\indexname{Inde
  \def\figurename{Abbildun
  \def\tablename{Tabelle}%                  % oder: Ta
  \def\partname{Tei
  \def\enclname{Anlage(n)}%                 % oder: Beilage
  \def\ccname{Verteiler}%                   % oder: Kopien
  \def\headtoname{A
  \def\pagename{Seit
  \def\seename{sieh
  \def\alsoname{siehe auc
  \def\proofname{Bewei
  \def\glossaryname{Glossa

%    \end{macroco
%  \end{mac
%  \end{mac

%  \begin{macro}{\dategerm
%    The macro |\dategerman| redefines the comm
%    |\today| to produce German dat
% \changes{germanb-2.3e}{1991/11/10}{Added \cs{month@germa
% \changes{germanb-2.6f}{1997/10/01}{Use \cs{edef} to def
%    \cs{today} to save memo
% \changes{germanb-2.6f}{1998/03/28}{use \cs{def} instead
%    \cs{ede
%    \begin{macroco
\def\month@german{\ifcase\month
  Januar\or Februar\or M\"arz\or April\or Mai\or Juni
  Juli\or August\or September\or Oktober\or November\or Dezember\
\def\dategerman{\def\today{\number\day.~\month@ger
    \space\number\yea
%    \end{macroco
%  \end{mac

%  \begin{macro}{\dateaustri
%    The macro |\dateaustrian| redefines the comm
%    |\today| to produce Austrian version of the German dat
% \changes{germanb-2.6f}{1997/10/01}{Use \cs{edef} to def
%    \cs{today} to save memo
% \changes{germanb-2.6f}{1998/03/28}{use \cs{def} instead
%    \cs{ede
%    \begin{macroco
\def\dateaustrian{\def\today{\number\day.~\ifnum1=\mo
  J\"anner\else \month@german\fi \space\number\yea
%    \end{macroco
%  \end{mac

%  \begin{macro}{\extrasgerm
%  \begin{macro}{\extrasaustri
% \changes{germanb-2.0b}{1991/05/29}{added some comment chars
%    prevent white spa
% \changes{germanb-2.2}{1991/06/11}{Save all redefined macr
%  \begin{macro}{\noextrasgerm
%  \begin{macro}{\noextrasaustri
% \changes{germanb-1.1}{1990/07/30}{Added \cs{dieresi
% \changes{germanb-2.0b}{1991/05/29}{added some comment chars
%    prevent white spa
% \changes{germanb-2.2}{1991/06/11}{Try to restore everything to
%    former sta
% \changes{germanb-2.6d}{1996/07/10}{Construct control seque
%    \cs{extrasgerman} or \cs{extrasaustrian} on the f

%    Either the macro |\extrasgerman| or the macros |\extrasaustri
%    will perform all the extra definitions needed for the Ger
%    language. The macro |\noextrasgerman| is used to cancel
%    actions of |\extrasgerman

%    For German (as well as for Dutch) the \texttt{"} character
%    made active. This is done once, later on its definition may va
%    \begin{macroco
\initiate@active@char
\@namedef{extras\CurrentOption
  \languageshorthands{germa
\expandafter\addto\csname extras\CurrentOption\endcsnam
  \bbl@activate{
%    \end{macroco
%    Don't forget to turn the shorthands off aga
% \changes{germanb-2.6i}{1999/12/16}{Deactivate shorthands ouside
%    Germ
%    \begin{macroco
\addto\noextrasgerman{\bbl@deactivate{
%    \end{macroco

% \changes{germanb-2.6a}{1995/02/15}{All the code to handle the act
%    double quote has been moved to \file{babel.de

%    In order for \TeX\ to be able to hyphenate German words wh
%    contain `\ss' (in the \texttt{OT1} position |^^Y|) we have
%    give the character a nonzero |\lccode| (see Appendix H, the \
%    boo
% \changes{germanb-2.6c}{1996/04/08}{Use decimal number instead
%    hat-notation as the hat may be activat
%    \begin{macroco
\expandafter\addto\csname extras\CurrentOption\endcsnam
  \babel@savevariable{\lccode2
  \lccode25=
%    \end{macroco
% \changes{germanb-2.6a}{1995/02/15}{Removeed \cs{3} as it is
%    longer in \file{german.ld

%    The umlaut accent macro |\"| is changed to lower the umlaut do
%    The redefinition is done with the help of |\umlautlo
%    \begin{macroco
\expandafter\addto\csname extras\CurrentOption\endcsnam
  \babel@save\"\umlautl
\@namedef{noextras\CurrentOption}{\umlauthi
%    \end{macroco
%    The german hyphenation patterns can be used with |\lefthyphenm
%    and |\righthyphenmin| set to
% \changes{germanb-2.6a}{1995/05/13}{use \cs{germanhyphenmins} to st
%    the correct valu
% \changes{germanb-2.6j}{2000/09/22}{Now use \cs{providehyphenmins}
%    provide a default val
%    \begin{macroco
\providehyphenmins{\CurrentOption}{\tw@\t
%    \end{macroco
%    For German texts we need to make sure that |\frenchspacing|
%    turned
% \changes{germanb-2.6k}{2001/01/26}{Turn frenchspacing on, as
%    \texttt{german.st
%    \begin{macroco
\expandafter\addto\csname extras\CurrentOption\endcsnam
  \bbl@frenchspaci
\expandafter\addto\csname noextras\CurrentOption\endcsnam
  \bbl@nonfrenchspaci
%    \end{macroco
%  \end{mac
%  \end{mac
%  \end{mac
%  \end{mac

% \changes{germanb-2.6a}{1995/02/15}{\cs{umlautlow}
%    \cs{umlauthigh} moved to \file{glyphs.dtx}, as well
%    \cs{newumlaut} (now \cs{lower@umlau

%    The code above is necessary because we need an extra act
%    character. This character is then used as indicated
%    table~\ref{tab:german-quot

%    To be able to define the function of |"|, we first defin
%    couple of `support' macr

% \changes{germanb-2.3e}{1991/11/10}{Added \cs{save@sf@q} macro
%    rewrote all quote macros to use
% \changes{germanb-2.3h}{1991/02/16}{moved definition
%    \cs{allowhyphens}, \cs{set@low@box} and \cs{save@sf@q}
%    \file{babel.co
% \changes{germanb-2.6a}{1995/02/15}{Moved all quotation characters
%    \file{glyphs.dt

%  \begin{macro}{\
%    We save the original double quote character in |\dq| to k
%    it available, the math accent |\"| can now be typed as |
%    \begin{macroco
\begingroup \catcode`\
\def\x{\endgr
  \def\@SS{\mathchar"701
  \def\dq{

%    \end{macroco
%  \end{mac
% \changes{germanb-2.6c}{1996/01/24}{Moved \cs{german@dq@disc}
%    babel.def, calling it \cs{bbl@dis

% \changes{germanb-2.6a}{1995/02/15}{Use \cs{ddot} instead
%    \cs{@MATHUMLAU

%    Now we can define the doublequote macros: the umlau
% \changes{germanb-2.6c}{1996/05/30}{added the \cs{allowhyphen
%    \begin{macroco
\declare@shorthand{german}{"a}{\textormath{\"{a}\allowhyphens}{\ddot
\declare@shorthand{german}{"o}{\textormath{\"{o}\allowhyphens}{\ddot
\declare@shorthand{german}{"u}{\textormath{\"{u}\allowhyphens}{\ddot
\declare@shorthand{german}{"A}{\textormath{\"{A}\allowhyphens}{\ddot
\declare@shorthand{german}{"O}{\textormath{\"{O}\allowhyphens}{\ddot
\declare@shorthand{german}{"U}{\textormath{\"{U}\allowhyphens}{\ddot
%    \end{macroco
%    trem
%    \begin{macroco
\declare@shorthand{german}{"e}{\textormath{\"{e}}{\ddot
\declare@shorthand{german}{"E}{\textormath{\"{E}}{\ddot
\declare@shorthand{german}{"i}{\textormath{\"{\i
                              {\ddot\imat
\declare@shorthand{german}{"I}{\textormath{\"{I}}{\ddot
%    \end{macroco
%    german es-zet (sharp
% \changes{germanb-2.6f}{1997/05/08}{use \cs{SS} instead
%    \texttt{SS}, removed braces after \cs{ss
%    \begin{macroco
\declare@shorthand{german}{"s}{\textormath{\ss}{\@SS{
\declare@shorthand{german}{"S}{\
\declare@shorthand{german}{"z}{\textormath{\ss}{\@SS{
\declare@shorthand{german}{"Z}{
%    \end{macroco
%    german and french quot
%    \begin{macroco
\declare@shorthand{german}{"`}{\gl
\declare@shorthand{german}{"'}{\gr
\declare@shorthand{german}{"<}{\fl
\declare@shorthand{german}{">}{\fr
%    \end{macroco
%    discretionary comma
%    \begin{macroco
\declare@shorthand{german}{"c}{\textormath{\bbl@disc ck}{
\declare@shorthand{german}{"C}{\textormath{\bbl@disc CK}{
\declare@shorthand{german}{"F}{\textormath{\bbl@disc F{FF}}{
\declare@shorthand{german}{"l}{\textormath{\bbl@disc l{ll}}{
\declare@shorthand{german}{"L}{\textormath{\bbl@disc L{LL}}{
\declare@shorthand{german}{"m}{\textormath{\bbl@disc m{mm}}{
\declare@shorthand{german}{"M}{\textormath{\bbl@disc M{MM}}{
\declare@shorthand{german}{"n}{\textormath{\bbl@disc n{nn}}{
\declare@shorthand{german}{"N}{\textormath{\bbl@disc N{NN}}{
\declare@shorthand{german}{"p}{\textormath{\bbl@disc p{pp}}{
\declare@shorthand{german}{"P}{\textormath{\bbl@disc P{PP}}{
\declare@shorthand{german}{"r}{\textormath{\bbl@disc r{rr}}{
\declare@shorthand{german}{"R}{\textormath{\bbl@disc R{RR}}{
\declare@shorthand{german}{"t}{\textormath{\bbl@disc t{tt}}{
\declare@shorthand{german}{"T}{\textormath{\bbl@disc T{TT}}{
%    \end{macroco
%    We need to treat |"f| a bit differently in order to preserve
%    ff-ligatur
% \changes{germanb-2.6f}{1998/06/15}{Copied the coding for \texttt{
%    from german.dtx version 2.5
%    \begin{macroco
\declare@shorthand{german}{"f}{\textormath{\bbl@discff}{
\def\bbl@discff{\penalty
  \afterassignment\bbl@insertff \let\bbl@nextff
\def\bbl@insertf
  \if f\bbl@nex
    \expandafter\@firstoftwo\else\expandafter\@secondoftwo
  {\relax\discretionary{ff-}{f}{ff}\allowhyphens}{f\bbl@nextf
\let\bbl@nextf
%    \end{macroco
%    and some additional comman
%    \begin{macroco
\declare@shorthand{german}{"-}{\nobreak\-\bbl@allowhyphe
\declare@shorthand{german}{"|
  \textormath{\penalty\@M\discretionary{-}{}{\kern.03e
              \allowhyphens}
\declare@shorthand{german}{""}{\hskip\z@sk
\declare@shorthand{german}{"~}{\textormath{\leavevmode\hbox{-}}{
\declare@shorthand{german}{"=}{\penalty\@M-\hskip\z@sk
%    \end{macroco

%  \begin{macro}{\mdq
%  \begin{macro}{\mdqo
%  \begin{macro}{\
%    All that's left to do now is to  define a couple of comma
%    for reasons of compatibility with \file{german.st
% \changes{germanb-2.6f}{1998/06/07}{Now use \cs{shorthandon}
%    \cs{shorthandoff
%    \begin{macroco
\def\mdqon{\shorthandon{
\def\mdqoff{\shorthandoff{
\def\ck{\allowhyphens\discretionary{k-}{k}{ck}\allowhyphe
%    \end{macroco
%  \end{mac
%  \end{mac
%  \end{mac

%    The macro |\ldf@finish| takes care of looking fo
%    configuration file, setting the main language to be switched
%    at |\begin{document}| and resetting the category code
%    \texttt{@} to its original val
% \changes{germanb-2.6d}{1996/11/02}{Now use \cs{ldf@finish} to w
%    u
%    \begin{macroco
\ldf@finish\CurrentOpt
%</co
%    \end{macroco

% \Fin

%% \CharacterTa
%%  {Upper-case    \A\B\C\D\E\F\G\H\I\J\K\L\M\N\O\P\Q\R\S\T\U\V\W\X\
%%   Lower-case    \a\b\c\d\e\f\g\h\i\j\k\l\m\n\o\p\q\r\s\t\u\v\w\x\
%%   Digits        \0\1\2\3\4\5\6\7\
%%   Exclamation   \!     Double quote  \"     Hash (number)
%%   Dollar        \$     Percent       \%     Ampersand
%%   Acute accent  \'     Left paren    \(     Right paren
%%   Asterisk      \*     Plus          \+     Comma
%%   Minus         \-     Point         \.     Solidus
%%   Colon         \:     Semicolon     \;     Less than
%%   Equals        \=     Greater than  \>     Question mark
%%   Commercial at \@     Left bracket  \[     Backslash
%%   Right bracket \]     Circumflex    \^     Underscore
%%   Grave accent  \`     Left brace    \{     Vertical bar
%%   Right brace   \}     Tilde

\endin
}
%    \end{macrocode}
% \changes{babel~3.8h}{2005/11/23}{added synonyms \Lopt{indonesian},
%    \Lopt{indon} and \Lopt{bahasai} for the original bahasa
%    (indonesia) support}
% \changes{babel~3.8h}{2005/11/23}{added \Lopt{malay}, \Lopt{meyaluy}
%    and \Lopt{bahasam} for the Bahasa Malaysia support}
%    \begin{macrocode}
\DeclareOption{bahasa}{\input{bahasai.ldf}}
\DeclareOption{indonesian}{\input{bahasai.ldf}}
\DeclareOption{indon}{\input{bahasai.ldf}}
\DeclareOption{bahasai}{\input{bahasai.ldf}}
\DeclareOption{malay}{% \iffalse meta-com

% Copyright 1989-2008 Johannes L. Braams and any individual aut
% listed elsewhere in this file.  All rights reser

% This file is part of the Babel sys
% ----------------------------------

% It may be distributed and/or modified under
% conditions of the LaTeX Project Public License, either version
% of this license or (at your option) any later vers
% The latest version of this license i
%   http://www.latex-project.org/lppl
% and version 1.3 or later is part of all distributions of L
% version 2003/12/01 or la

% This work has the LPPL maintenance status "maintain

% The Current Maintainer of this work is Johannes Bra

% The list of all files belonging to the Babel syste
% given in the file `manifest.bbl. See also `legal.bbl' for additi
% informat

% The list of derived (unpacked) files belonging to the distribu
% and covered by LPPL is defined by the unpacking scripts (
% extension .ins) which are part of the distribut
%
% \CheckSum{
%\iff
%    Tell the \LaTeX\ system who we are and write an entry on
%    transcr
%<*
\ProvidesFile{bahasam.
%</
%<code>\ProvidesLanguage{baha

%\ProvidesFile{bahasam.
       [2008/01/27 v1.0k Bahasa Malaysia support from the babel sys
%\iff
%% File `bahasam.
%% Babel package for LaTeX versio
%% Copyright (C) 1989 -
%%           by Johannes Braams, TeX

%% Bahasa Malaysia Language Definition
%% Copyright (C) 1994 -
%%           by J"org Knappen, (joerg.knappen at alpha.ntp.springer
%              Terry Mart (mart at vkpmzd.kph.uni-mainz
%              Institut f\"ur Kernph
%              Johannes Gutenberg-Universit\"at M
%              D-55099 M
%              Ger

%% Copyright (C) 2005,
%%           by Bob Margolis, (bob.margolis at ntlworld.
%              derived from J"ork Knappen's work - see ab
%%           [With help from Awangku Merali Pengiran Mohamed (Saraw
%               gratefully acknowled
%               Yate
%

%% Please report errors to: Bob Marg
%%                          bob.margolis at ntlworld
%%                          J.L. Br
%%                          babel at braams.xs4al

%    This file is part of the babel system, it provides the so
%    code for the  Bahasa Malaysia language defini
%    file.  The original version of this file was written by T
%    Mart (mart@vkpmzd.kph.uni-mainz.de) and J"org Kna
%    (knappen@vkpmzd.kph.uni-mainz.
%<*filedri
\documentclass{ltx
\newcommand*\TeXhax{\TeX
\newcommand*\babel{\textsf{bab
\newcommand*\langvar{$\langle \it lang \rang
\newcommand*\note[
\newcommand*\Lopt[1]{\textsf{
\newcommand*\file[1]{\texttt{
\begin{docum
 \DocInput{bahasam.
\end{docum
%</filedri

% \GetFileInfo{bahasam.

% \changes{bahasa-0.9c}{1994/06/26}{Removed the use of \cs{filed
%    and moved identification after the loading of \file{babel.d
% \changes{bahasa-1.0d}{1996/07/10}{Replaced \cs{undefined}
%    \cs{@undefined} and \cs{empty} with \cs{@empty} for consist
%    with \LaT
% \changes{bahasa-1.0e}{1996/10/10}{Moved the definitio
%    \cs{atcatcode} right to the beginni
% \changes{bahasam-0.9f}{2005/11/22}{A number of changes to make
%    specific to Bahasa Maya

%  \section{The Bahasa Malaysia langu

%    The file \file{\filename}\footnote{The file described in
%    section has version number \fileversion\ and was last revise
%    \filedate.}  defines all the language definition macros for
%    Bahasa Malaysia language. Bahasa just m
%    `language' in Bahasa Malaysia. A number of terms differ from those
%    in bahasa indone

%    For this language currently no special definitions are neede
%    availa

% \StopEventual

%    The macro |\LdfInit| takes care of preventing that this fil
%    loaded more than once, checking the category code of
%    \texttt{@} sign,
% \changes{bahasa-1.0e}{1996/11/02}{Now use \cs{LdfInit} to per
%    initial che
% \changes{bahasam-v1.0j}{2005/11/23}{Make it possible that this
%    is loaded by variuos opti
%    \begin{macroc
%<*c
\LdfInit\CurrentOption{date\CurrentOpt
%    \end{macroc

%    When this file is read as an option, i.e. by the |\usepack
%    command, \texttt{bahasa} could be an `unknown' language in w
%    case we have to make it known. So we check for the existenc
%    |\l@bahasa| to see whether we have to do something h

%    For both Bahasa Malaysia and Bahasa Indonesia the same se
%    hyphenation patterns can be used which are available in the
%    \file{inhyph.tex}. However it could be loaded using any of
%    possible Babel options fot the Malaysian and Indone
%    languase. So first we try to find out whether this is the c

% \changes{bahasa-0.9c}{1994/06/26}{Now use \cs{@patterns} to pro
%    the warn
%    \begin{macroc
\ifx\l@malay\@undef
  \ifx\l@meyalu\@undef
    \ifx\l@bahasam\@undef
      \ifx\l@bahasa\@undef
        \ifx\l@bahasai\@undef
          \ifx\l@indon\@undef
            \ifx\l@indonesian\@undef
              \@nopatterns{Bahasa Malay
              \adddialect\l@malay0\r
            \
              \let\l@malay\l@indone

          \
            \let\l@malay\l@i

        \
          \let\l@malay\l@bah

      \
        \let\l@malay\l@ba

    \
      \let\l@malay\l@bah

  \
    \let\l@malay\l@me


%    \end{macroc

%    Now that we are sure the |\l@malay| has some valid definitio
%    need to make sure that a name to access the hyphenation patte
%    corresponding to the option used, is availa
%    \begin{macroc
\expandafter\expandafter\expandafter
  \expandafter\cs
  \expandafter l\expandafter @\CurrentOption\endcs
  \l@m
%    \end{macroc

%    The next step consists of defining commands to switch to
%    from) the Bahasa langu

% \begin{macro}{\captionsbaha
%    The macro |\captionsbahasam| defines all strings used in the
%    standard documentclasses provided with \La
% \changes{bahasa-1.0b}{1995/07/04}{Added \cs{proofname}
%    AMS-\La
% \changes{bahasa-1.0d}{1996/07/09}{Replaced `Proof' by `Bu
%    (PR2214
% \changes{bahasa-1.0h}{2000/09/19}{Added \cs{glossaryna
% \changes{bahasa-1.0i}{2003/11/17}{Inserted translation for Gloss
% \changes{bahasam-1.0k}{2008/01/27}{Inserted changes from Awangku Mera
%    \begin{macroc
\@namedef{captions\CurrentOptio
  \def\prefacename{Praka
  \def\refname{Rujuk
  \def\abstractname{Abstrak}% (sometime it's called 'intis
                              %  or 'ikhtis
  \def\bibname{Bibliogra
  \def\chaptername{B
  \def\appendixname{Lampir
  \def\contentsname{Kandung
  \def\listfigurename{Senarai Gamb
  \def\listtablename{Senarai Jadu
  \def\indexname{Inde
  \def\figurename{Gamb
  \def\tablename{Jadu
  \def\partname{Bahagi
%  Subject:  Per
%  From:
  \def\enclname{Lampir
  \def\ccname{sk}% (short form for 'Salinan Kepa
  \def\headtoname{Kepa
  \def\pagename{Halam
%  Notes (Endnotes): Cat
  \def\seename{sila  ruj
  \def\alsoname{rujuk ju
  \def\proofname{Buk
  \def\glossaryname{Istil

%    \end{macroc
% \end{ma

% \begin{macro}{\datebaha
%    The macro |\datebahasam| redefines the command |\today| to pro
%    Bahasa Malaysian da
% \changes{bahasa-1.0f}{1997/10/01}{Use \cs{edef} to define \cs{tod
% \changes{bahasa~1.0f}{1998/03/28}{use \cs{def} instead of \cs{e
%    to save mem
% \changes{bahasa-1.0g}{1999/03/12}{Februari should be spelle
%    Pebru
% \changes{bahasam-1.0k}{2008/01/27}{Februari restored to BM spelli
%    see Collins Kamus Dwibahasa 2
%    \begin{macroc
\@namedef{date\CurrentOptio
  \def\today{\number\day~\ifcase\mont
    Januari\or Februari\or Mac\or April\or Mei\or Ju
    Julai\or Ogos\or September\or Oktober\or November\or Disembe
    \space \number\ye
%    \end{macroc
% \end{ma


% \begin{macro}{\extrasbaha
% \begin{macro}{\noextrasbaha
%    The macro |\extrasbahasa| will perform all the extra definit
%    needed for the Bahasa language. The macro |\extrasbahasa| is
%    to cancel the actions of |\extrasbahasa|.  For the moment t
%    macros are empty but they are defined for compatibility with
%    other language definition fi

%    \begin{macroc
\@namedef{extras\CurrentOptio
\@namedef{noextras\CurrentOptio
%    \end{macroc
% \end{ma
% \end{ma

%  \begin{macro}{\bahasamhyphenm
%    The bahasam hyphenation patterns should be used
%    |\lefthyphenmin| set to~2 and |\righthyphenmin| set t
% \changes{bahasa-1.0e}{1996/08/07}{use \cs{bahasamhyphenmins} to s
%    the correct val
% \changes{bahasa-1.0h}{2000/09/22}{Now use \cs{providehyphenmins
%    provide a default va
%    \begin{macroc
\providehyphenmins{\CurrentOption}{\tw@\
%    \end{macroc
%  \end{ma

%    The macro |\ldf@finish| takes care of looking f
%    configuration file, setting the main language to be switche
%    at |\begin{document}| and resetting the category cod
%    \texttt{@} to its original va
% \changes{bahasa-1.0e}{1996/11/02}{Now use \cs{ldf@finish} to wrap
%    \begin{macroc
\ldf@finish{\CurrentOpt
%</c
%    \end{macroc

% \Fi

%% \CharacterT
%%  {Upper-case    \A\B\C\D\E\F\G\H\I\J\K\L\M\N\O\P\Q\R\S\T\U\V\W\X
%%   Lower-case    \a\b\c\d\e\f\g\h\i\j\k\l\m\n\o\p\q\r\s\t\u\v\w\x
%%   Digits        \0\1\2\3\4\5\6\7
%%   Exclamation   \!     Double quote  \"     Hash (number
%%   Dollar        \$     Percent       \%     Ampersand
%%   Acute accent  \'     Left paren    \(     Right paren
%%   Asterisk      \*     Plus          \+     Comma
%%   Minus         \-     Point         \.     Solidus
%%   Colon         \:     Semicolon     \;     Less than
%%   Equals        \=     Greater than  \>     Question mar
%%   Commercial at \@     Left bracket  \[     Backslash
%%   Right bracket \]     Circumflex    \^     Underscore
%%   Grave accent  \`     Left brace    \{     Vertical bar
%%   Right brace   \}     Tilde

\endi
}
\DeclareOption{meyalu}{% \iffalse meta-com

% Copyright 1989-2008 Johannes L. Braams and any individual aut
% listed elsewhere in this file.  All rights reser

% This file is part of the Babel sys
% ----------------------------------

% It may be distributed and/or modified under
% conditions of the LaTeX Project Public License, either version
% of this license or (at your option) any later vers
% The latest version of this license i
%   http://www.latex-project.org/lppl
% and version 1.3 or later is part of all distributions of L
% version 2003/12/01 or la

% This work has the LPPL maintenance status "maintain

% The Current Maintainer of this work is Johannes Bra

% The list of all files belonging to the Babel syste
% given in the file `manifest.bbl. See also `legal.bbl' for additi
% informat

% The list of derived (unpacked) files belonging to the distribu
% and covered by LPPL is defined by the unpacking scripts (
% extension .ins) which are part of the distribut
%
% \CheckSum{
%\iff
%    Tell the \LaTeX\ system who we are and write an entry on
%    transcr
%<*
\ProvidesFile{bahasam.
%</
%<code>\ProvidesLanguage{baha

%\ProvidesFile{bahasam.
       [2008/01/27 v1.0k Bahasa Malaysia support from the babel sys
%\iff
%% File `bahasam.
%% Babel package for LaTeX versio
%% Copyright (C) 1989 -
%%           by Johannes Braams, TeX

%% Bahasa Malaysia Language Definition
%% Copyright (C) 1994 -
%%           by J"org Knappen, (joerg.knappen at alpha.ntp.springer
%              Terry Mart (mart at vkpmzd.kph.uni-mainz
%              Institut f\"ur Kernph
%              Johannes Gutenberg-Universit\"at M
%              D-55099 M
%              Ger

%% Copyright (C) 2005,
%%           by Bob Margolis, (bob.margolis at ntlworld.
%              derived from J"ork Knappen's work - see ab
%%           [With help from Awangku Merali Pengiran Mohamed (Saraw
%               gratefully acknowled
%               Yate
%

%% Please report errors to: Bob Marg
%%                          bob.margolis at ntlworld
%%                          J.L. Br
%%                          babel at braams.xs4al

%    This file is part of the babel system, it provides the so
%    code for the  Bahasa Malaysia language defini
%    file.  The original version of this file was written by T
%    Mart (mart@vkpmzd.kph.uni-mainz.de) and J"org Kna
%    (knappen@vkpmzd.kph.uni-mainz.
%<*filedri
\documentclass{ltx
\newcommand*\TeXhax{\TeX
\newcommand*\babel{\textsf{bab
\newcommand*\langvar{$\langle \it lang \rang
\newcommand*\note[
\newcommand*\Lopt[1]{\textsf{
\newcommand*\file[1]{\texttt{
\begin{docum
 \DocInput{bahasam.
\end{docum
%</filedri

% \GetFileInfo{bahasam.

% \changes{bahasa-0.9c}{1994/06/26}{Removed the use of \cs{filed
%    and moved identification after the loading of \file{babel.d
% \changes{bahasa-1.0d}{1996/07/10}{Replaced \cs{undefined}
%    \cs{@undefined} and \cs{empty} with \cs{@empty} for consist
%    with \LaT
% \changes{bahasa-1.0e}{1996/10/10}{Moved the definitio
%    \cs{atcatcode} right to the beginni
% \changes{bahasam-0.9f}{2005/11/22}{A number of changes to make
%    specific to Bahasa Maya

%  \section{The Bahasa Malaysia langu

%    The file \file{\filename}\footnote{The file described in
%    section has version number \fileversion\ and was last revise
%    \filedate.}  defines all the language definition macros for
%    Bahasa Malaysia language. Bahasa just m
%    `language' in Bahasa Malaysia. A number of terms differ from those
%    in bahasa indone

%    For this language currently no special definitions are neede
%    availa

% \StopEventual

%    The macro |\LdfInit| takes care of preventing that this fil
%    loaded more than once, checking the category code of
%    \texttt{@} sign,
% \changes{bahasa-1.0e}{1996/11/02}{Now use \cs{LdfInit} to per
%    initial che
% \changes{bahasam-v1.0j}{2005/11/23}{Make it possible that this
%    is loaded by variuos opti
%    \begin{macroc
%<*c
\LdfInit\CurrentOption{date\CurrentOpt
%    \end{macroc

%    When this file is read as an option, i.e. by the |\usepack
%    command, \texttt{bahasa} could be an `unknown' language in w
%    case we have to make it known. So we check for the existenc
%    |\l@bahasa| to see whether we have to do something h

%    For both Bahasa Malaysia and Bahasa Indonesia the same se
%    hyphenation patterns can be used which are available in the
%    \file{inhyph.tex}. However it could be loaded using any of
%    possible Babel options fot the Malaysian and Indone
%    languase. So first we try to find out whether this is the c

% \changes{bahasa-0.9c}{1994/06/26}{Now use \cs{@patterns} to pro
%    the warn
%    \begin{macroc
\ifx\l@malay\@undef
  \ifx\l@meyalu\@undef
    \ifx\l@bahasam\@undef
      \ifx\l@bahasa\@undef
        \ifx\l@bahasai\@undef
          \ifx\l@indon\@undef
            \ifx\l@indonesian\@undef
              \@nopatterns{Bahasa Malay
              \adddialect\l@malay0\r
            \
              \let\l@malay\l@indone

          \
            \let\l@malay\l@i

        \
          \let\l@malay\l@bah

      \
        \let\l@malay\l@ba

    \
      \let\l@malay\l@bah

  \
    \let\l@malay\l@me


%    \end{macroc

%    Now that we are sure the |\l@malay| has some valid definitio
%    need to make sure that a name to access the hyphenation patte
%    corresponding to the option used, is availa
%    \begin{macroc
\expandafter\expandafter\expandafter
  \expandafter\cs
  \expandafter l\expandafter @\CurrentOption\endcs
  \l@m
%    \end{macroc

%    The next step consists of defining commands to switch to
%    from) the Bahasa langu

% \begin{macro}{\captionsbaha
%    The macro |\captionsbahasam| defines all strings used in the
%    standard documentclasses provided with \La
% \changes{bahasa-1.0b}{1995/07/04}{Added \cs{proofname}
%    AMS-\La
% \changes{bahasa-1.0d}{1996/07/09}{Replaced `Proof' by `Bu
%    (PR2214
% \changes{bahasa-1.0h}{2000/09/19}{Added \cs{glossaryna
% \changes{bahasa-1.0i}{2003/11/17}{Inserted translation for Gloss
% \changes{bahasam-1.0k}{2008/01/27}{Inserted changes from Awangku Mera
%    \begin{macroc
\@namedef{captions\CurrentOptio
  \def\prefacename{Praka
  \def\refname{Rujuk
  \def\abstractname{Abstrak}% (sometime it's called 'intis
                              %  or 'ikhtis
  \def\bibname{Bibliogra
  \def\chaptername{B
  \def\appendixname{Lampir
  \def\contentsname{Kandung
  \def\listfigurename{Senarai Gamb
  \def\listtablename{Senarai Jadu
  \def\indexname{Inde
  \def\figurename{Gamb
  \def\tablename{Jadu
  \def\partname{Bahagi
%  Subject:  Per
%  From:
  \def\enclname{Lampir
  \def\ccname{sk}% (short form for 'Salinan Kepa
  \def\headtoname{Kepa
  \def\pagename{Halam
%  Notes (Endnotes): Cat
  \def\seename{sila  ruj
  \def\alsoname{rujuk ju
  \def\proofname{Buk
  \def\glossaryname{Istil

%    \end{macroc
% \end{ma

% \begin{macro}{\datebaha
%    The macro |\datebahasam| redefines the command |\today| to pro
%    Bahasa Malaysian da
% \changes{bahasa-1.0f}{1997/10/01}{Use \cs{edef} to define \cs{tod
% \changes{bahasa~1.0f}{1998/03/28}{use \cs{def} instead of \cs{e
%    to save mem
% \changes{bahasa-1.0g}{1999/03/12}{Februari should be spelle
%    Pebru
% \changes{bahasam-1.0k}{2008/01/27}{Februari restored to BM spelli
%    see Collins Kamus Dwibahasa 2
%    \begin{macroc
\@namedef{date\CurrentOptio
  \def\today{\number\day~\ifcase\mont
    Januari\or Februari\or Mac\or April\or Mei\or Ju
    Julai\or Ogos\or September\or Oktober\or November\or Disembe
    \space \number\ye
%    \end{macroc
% \end{ma


% \begin{macro}{\extrasbaha
% \begin{macro}{\noextrasbaha
%    The macro |\extrasbahasa| will perform all the extra definit
%    needed for the Bahasa language. The macro |\extrasbahasa| is
%    to cancel the actions of |\extrasbahasa|.  For the moment t
%    macros are empty but they are defined for compatibility with
%    other language definition fi

%    \begin{macroc
\@namedef{extras\CurrentOptio
\@namedef{noextras\CurrentOptio
%    \end{macroc
% \end{ma
% \end{ma

%  \begin{macro}{\bahasamhyphenm
%    The bahasam hyphenation patterns should be used
%    |\lefthyphenmin| set to~2 and |\righthyphenmin| set t
% \changes{bahasa-1.0e}{1996/08/07}{use \cs{bahasamhyphenmins} to s
%    the correct val
% \changes{bahasa-1.0h}{2000/09/22}{Now use \cs{providehyphenmins
%    provide a default va
%    \begin{macroc
\providehyphenmins{\CurrentOption}{\tw@\
%    \end{macroc
%  \end{ma

%    The macro |\ldf@finish| takes care of looking f
%    configuration file, setting the main language to be switche
%    at |\begin{document}| and resetting the category cod
%    \texttt{@} to its original va
% \changes{bahasa-1.0e}{1996/11/02}{Now use \cs{ldf@finish} to wrap
%    \begin{macroc
\ldf@finish{\CurrentOpt
%</c
%    \end{macroc

% \Fi

%% \CharacterT
%%  {Upper-case    \A\B\C\D\E\F\G\H\I\J\K\L\M\N\O\P\Q\R\S\T\U\V\W\X
%%   Lower-case    \a\b\c\d\e\f\g\h\i\j\k\l\m\n\o\p\q\r\s\t\u\v\w\x
%%   Digits        \0\1\2\3\4\5\6\7
%%   Exclamation   \!     Double quote  \"     Hash (number
%%   Dollar        \$     Percent       \%     Ampersand
%%   Acute accent  \'     Left paren    \(     Right paren
%%   Asterisk      \*     Plus          \+     Comma
%%   Minus         \-     Point         \.     Solidus
%%   Colon         \:     Semicolon     \;     Less than
%%   Equals        \=     Greater than  \>     Question mar
%%   Commercial at \@     Left bracket  \[     Backslash
%%   Right bracket \]     Circumflex    \^     Underscore
%%   Grave accent  \`     Left brace    \{     Vertical bar
%%   Right brace   \}     Tilde

\endi
}
\DeclareOption{bahasam}{% \iffalse meta-com

% Copyright 1989-2008 Johannes L. Braams and any individual aut
% listed elsewhere in this file.  All rights reser

% This file is part of the Babel sys
% ----------------------------------

% It may be distributed and/or modified under
% conditions of the LaTeX Project Public License, either version
% of this license or (at your option) any later vers
% The latest version of this license i
%   http://www.latex-project.org/lppl
% and version 1.3 or later is part of all distributions of L
% version 2003/12/01 or la

% This work has the LPPL maintenance status "maintain

% The Current Maintainer of this work is Johannes Bra

% The list of all files belonging to the Babel syste
% given in the file `manifest.bbl. See also `legal.bbl' for additi
% informat

% The list of derived (unpacked) files belonging to the distribu
% and covered by LPPL is defined by the unpacking scripts (
% extension .ins) which are part of the distribut
%
% \CheckSum{
%\iff
%    Tell the \LaTeX\ system who we are and write an entry on
%    transcr
%<*
\ProvidesFile{bahasam.
%</
%<code>\ProvidesLanguage{baha

%\ProvidesFile{bahasam.
       [2008/01/27 v1.0k Bahasa Malaysia support from the babel sys
%\iff
%% File `bahasam.
%% Babel package for LaTeX versio
%% Copyright (C) 1989 -
%%           by Johannes Braams, TeX

%% Bahasa Malaysia Language Definition
%% Copyright (C) 1994 -
%%           by J"org Knappen, (joerg.knappen at alpha.ntp.springer
%              Terry Mart (mart at vkpmzd.kph.uni-mainz
%              Institut f\"ur Kernph
%              Johannes Gutenberg-Universit\"at M
%              D-55099 M
%              Ger

%% Copyright (C) 2005,
%%           by Bob Margolis, (bob.margolis at ntlworld.
%              derived from J"ork Knappen's work - see ab
%%           [With help from Awangku Merali Pengiran Mohamed (Saraw
%               gratefully acknowled
%               Yate
%

%% Please report errors to: Bob Marg
%%                          bob.margolis at ntlworld
%%                          J.L. Br
%%                          babel at braams.xs4al

%    This file is part of the babel system, it provides the so
%    code for the  Bahasa Malaysia language defini
%    file.  The original version of this file was written by T
%    Mart (mart@vkpmzd.kph.uni-mainz.de) and J"org Kna
%    (knappen@vkpmzd.kph.uni-mainz.
%<*filedri
\documentclass{ltx
\newcommand*\TeXhax{\TeX
\newcommand*\babel{\textsf{bab
\newcommand*\langvar{$\langle \it lang \rang
\newcommand*\note[
\newcommand*\Lopt[1]{\textsf{
\newcommand*\file[1]{\texttt{
\begin{docum
 \DocInput{bahasam.
\end{docum
%</filedri

% \GetFileInfo{bahasam.

% \changes{bahasa-0.9c}{1994/06/26}{Removed the use of \cs{filed
%    and moved identification after the loading of \file{babel.d
% \changes{bahasa-1.0d}{1996/07/10}{Replaced \cs{undefined}
%    \cs{@undefined} and \cs{empty} with \cs{@empty} for consist
%    with \LaT
% \changes{bahasa-1.0e}{1996/10/10}{Moved the definitio
%    \cs{atcatcode} right to the beginni
% \changes{bahasam-0.9f}{2005/11/22}{A number of changes to make
%    specific to Bahasa Maya

%  \section{The Bahasa Malaysia langu

%    The file \file{\filename}\footnote{The file described in
%    section has version number \fileversion\ and was last revise
%    \filedate.}  defines all the language definition macros for
%    Bahasa Malaysia language. Bahasa just m
%    `language' in Bahasa Malaysia. A number of terms differ from those
%    in bahasa indone

%    For this language currently no special definitions are neede
%    availa

% \StopEventual

%    The macro |\LdfInit| takes care of preventing that this fil
%    loaded more than once, checking the category code of
%    \texttt{@} sign,
% \changes{bahasa-1.0e}{1996/11/02}{Now use \cs{LdfInit} to per
%    initial che
% \changes{bahasam-v1.0j}{2005/11/23}{Make it possible that this
%    is loaded by variuos opti
%    \begin{macroc
%<*c
\LdfInit\CurrentOption{date\CurrentOpt
%    \end{macroc

%    When this file is read as an option, i.e. by the |\usepack
%    command, \texttt{bahasa} could be an `unknown' language in w
%    case we have to make it known. So we check for the existenc
%    |\l@bahasa| to see whether we have to do something h

%    For both Bahasa Malaysia and Bahasa Indonesia the same se
%    hyphenation patterns can be used which are available in the
%    \file{inhyph.tex}. However it could be loaded using any of
%    possible Babel options fot the Malaysian and Indone
%    languase. So first we try to find out whether this is the c

% \changes{bahasa-0.9c}{1994/06/26}{Now use \cs{@patterns} to pro
%    the warn
%    \begin{macroc
\ifx\l@malay\@undef
  \ifx\l@meyalu\@undef
    \ifx\l@bahasam\@undef
      \ifx\l@bahasa\@undef
        \ifx\l@bahasai\@undef
          \ifx\l@indon\@undef
            \ifx\l@indonesian\@undef
              \@nopatterns{Bahasa Malay
              \adddialect\l@malay0\r
            \
              \let\l@malay\l@indone

          \
            \let\l@malay\l@i

        \
          \let\l@malay\l@bah

      \
        \let\l@malay\l@ba

    \
      \let\l@malay\l@bah

  \
    \let\l@malay\l@me


%    \end{macroc

%    Now that we are sure the |\l@malay| has some valid definitio
%    need to make sure that a name to access the hyphenation patte
%    corresponding to the option used, is availa
%    \begin{macroc
\expandafter\expandafter\expandafter
  \expandafter\cs
  \expandafter l\expandafter @\CurrentOption\endcs
  \l@m
%    \end{macroc

%    The next step consists of defining commands to switch to
%    from) the Bahasa langu

% \begin{macro}{\captionsbaha
%    The macro |\captionsbahasam| defines all strings used in the
%    standard documentclasses provided with \La
% \changes{bahasa-1.0b}{1995/07/04}{Added \cs{proofname}
%    AMS-\La
% \changes{bahasa-1.0d}{1996/07/09}{Replaced `Proof' by `Bu
%    (PR2214
% \changes{bahasa-1.0h}{2000/09/19}{Added \cs{glossaryna
% \changes{bahasa-1.0i}{2003/11/17}{Inserted translation for Gloss
% \changes{bahasam-1.0k}{2008/01/27}{Inserted changes from Awangku Mera
%    \begin{macroc
\@namedef{captions\CurrentOptio
  \def\prefacename{Praka
  \def\refname{Rujuk
  \def\abstractname{Abstrak}% (sometime it's called 'intis
                              %  or 'ikhtis
  \def\bibname{Bibliogra
  \def\chaptername{B
  \def\appendixname{Lampir
  \def\contentsname{Kandung
  \def\listfigurename{Senarai Gamb
  \def\listtablename{Senarai Jadu
  \def\indexname{Inde
  \def\figurename{Gamb
  \def\tablename{Jadu
  \def\partname{Bahagi
%  Subject:  Per
%  From:
  \def\enclname{Lampir
  \def\ccname{sk}% (short form for 'Salinan Kepa
  \def\headtoname{Kepa
  \def\pagename{Halam
%  Notes (Endnotes): Cat
  \def\seename{sila  ruj
  \def\alsoname{rujuk ju
  \def\proofname{Buk
  \def\glossaryname{Istil

%    \end{macroc
% \end{ma

% \begin{macro}{\datebaha
%    The macro |\datebahasam| redefines the command |\today| to pro
%    Bahasa Malaysian da
% \changes{bahasa-1.0f}{1997/10/01}{Use \cs{edef} to define \cs{tod
% \changes{bahasa~1.0f}{1998/03/28}{use \cs{def} instead of \cs{e
%    to save mem
% \changes{bahasa-1.0g}{1999/03/12}{Februari should be spelle
%    Pebru
% \changes{bahasam-1.0k}{2008/01/27}{Februari restored to BM spelli
%    see Collins Kamus Dwibahasa 2
%    \begin{macroc
\@namedef{date\CurrentOptio
  \def\today{\number\day~\ifcase\mont
    Januari\or Februari\or Mac\or April\or Mei\or Ju
    Julai\or Ogos\or September\or Oktober\or November\or Disembe
    \space \number\ye
%    \end{macroc
% \end{ma


% \begin{macro}{\extrasbaha
% \begin{macro}{\noextrasbaha
%    The macro |\extrasbahasa| will perform all the extra definit
%    needed for the Bahasa language. The macro |\extrasbahasa| is
%    to cancel the actions of |\extrasbahasa|.  For the moment t
%    macros are empty but they are defined for compatibility with
%    other language definition fi

%    \begin{macroc
\@namedef{extras\CurrentOptio
\@namedef{noextras\CurrentOptio
%    \end{macroc
% \end{ma
% \end{ma

%  \begin{macro}{\bahasamhyphenm
%    The bahasam hyphenation patterns should be used
%    |\lefthyphenmin| set to~2 and |\righthyphenmin| set t
% \changes{bahasa-1.0e}{1996/08/07}{use \cs{bahasamhyphenmins} to s
%    the correct val
% \changes{bahasa-1.0h}{2000/09/22}{Now use \cs{providehyphenmins
%    provide a default va
%    \begin{macroc
\providehyphenmins{\CurrentOption}{\tw@\
%    \end{macroc
%  \end{ma

%    The macro |\ldf@finish| takes care of looking f
%    configuration file, setting the main language to be switche
%    at |\begin{document}| and resetting the category cod
%    \texttt{@} to its original va
% \changes{bahasa-1.0e}{1996/11/02}{Now use \cs{ldf@finish} to wrap
%    \begin{macroc
\ldf@finish{\CurrentOpt
%</c
%    \end{macroc

% \Fi

%% \CharacterT
%%  {Upper-case    \A\B\C\D\E\F\G\H\I\J\K\L\M\N\O\P\Q\R\S\T\U\V\W\X
%%   Lower-case    \a\b\c\d\e\f\g\h\i\j\k\l\m\n\o\p\q\r\s\t\u\v\w\x
%%   Digits        \0\1\2\3\4\5\6\7
%%   Exclamation   \!     Double quote  \"     Hash (number
%%   Dollar        \$     Percent       \%     Ampersand
%%   Acute accent  \'     Left paren    \(     Right paren
%%   Asterisk      \*     Plus          \+     Comma
%%   Minus         \-     Point         \.     Solidus
%%   Colon         \:     Semicolon     \;     Less than
%%   Equals        \=     Greater than  \>     Question mar
%%   Commercial at \@     Left bracket  \[     Backslash
%%   Right bracket \]     Circumflex    \^     Underscore
%%   Grave accent  \`     Left brace    \{     Vertical bar
%%   Right brace   \}     Tilde

\endi
}
\DeclareOption{basque}{% \iffalse meta-com

% Copyright 1989-2005 Johannes L. Braams and any individual aut
% listed elsewhere in this file.  All rights reser

% This file is part of the Babel sys
% ----------------------------------

% It may be distributed and/or modified under
% conditions of the LaTeX Project Public License, either version
% of this license or (at your option) any later vers
% The latest version of this license i
%   http://www.latex-project.org/lppl
% and version 1.3 or later is part of all distributions of L
% version 2003/12/01 or la

% This work has the LPPL maintenance status "maintain

% The Current Maintainer of this work is Johannes Bra

% The list of all files belonging to the Babel syste
% given in the file `manifest.bbl. See also `legal.bbl' for additi
% informat

% The list of derived (unpacked) files belonging to the distribu
% and covered by LPPL is defined by the unpacking scripts (
% extension .ins) which are part of the distribut
%
% \CheckSum{
% \iff
%    Tell the \LaTeX\ system who we are and write an entry on
%    transcr
%<*
\ProvidesFile{basque.
%</
%<code>\ProvidesLanguage{bas

%\ProvidesFile{basque.
        [2005/03/29 v1.0f Basque support from the babel sys
%\iff
%% File `Basque.
%% Babel package for LaTeX versio
%% Copyright (C) 1989 -
%%           by Johannes Braams, TeX

%% Basque Language Definition
%% Copyright (C) 1997 -
%%           by Juan M. Aguirregab
%              The University of the Basque Cou
%              Dept. of Theoretical Phy
%              Apdo.
%              E-48080 Bi
%              S
%              tel: +34 946 012
%              fax: +34 944 648
%              e-mail: wtpagagj at lg.eh
%              WWW:    http://tp.lc.ehu.es/jma.
% based on
% Spanish Language Definition
% Copyright (C) 1991 -
%           by Julio San
%              GMV
%              c/ Isaac Newto
%              PTM - Tres Ca
%              E-28760 Ma
%              S
%              tel: +34 1 807 2
%              fax +34 1 807 2
%              jsanchez at gm

% Acknowledgements: I am indebted to Zunbeltz Iza
%                   who suggested the use of \discretionary i

%% Please report errors to: Juan M. Aguirregabiria <wtpagagj at lg.ehu
%%                          (or J.L. Braams <babel at braams.cistron.

%    This file is part of the babel system, it provides the so
%    code for the basque language definition f
%<*filedri
\documentclass{ltx
\newcommand*\TeXhax{\TeX
\newcommand*\babel{\textsf{bab
\newcommand*\langvar{$\langle \it lang \rang
\newcommand*\note[
\newcommand*\Lopt[1]{\textsf{
\newcommand*\file[1]{\texttt{
\begin{docum
 \DocInput{basque.
\end{docum
%</filedri

% \GetFileInfo{basque.

%  \section{The Basque langu

%    The file \file{\filename}\footnote{The file described in
%    section has version number \fileversion\ and was last revise
%    \filedate. The original author is Juan M. Aguirregabi
%    (\texttt{wtpagagj@lg.ehu.es}) and is based on the Spanish
%    by Julio S\'anc
%    (\texttt{jsanchez@gmv.es}).} defines all the language defini
%    macro's for the Basque langu

%    For this language the characters |~| and |"| are
%    active. In table~\ref{tab:basque-quote} an overview is give
%    their purp
%    \begin{table}[
%     \cente
%     \begin{tabular}{lp{8
%      \verb="|= & disable ligature at this positio
%      |"-| & an explicit hyphen sign, allowing hyphena
%             in the rest of the wor
%      |\-| & like the old |\-|, but allowing hyphena
%             in the rest of the word
%      |"<| & for French left double quotes (similar to $<<$
%      |">| & for French right double quotes (similar to $>>$
%      |~n| & a n with tilde. Works for uppercase
%     \end{tabu
%     \caption{The extra definitions made by \file{basque.l
%     \label{tab:basque-qu
%    \end{ta
%    These active accent characters behave according to their orig
%    definitions if not followed by one of the characters indicate
%    that ta

% \changes{basque-1.0f}{2002/01/07}{Changed url's for the patt
%    f
%    This option includes support for working with extended, 8
%    fonts, if available. Support is base
%    providing an appropriate definition for the accent macro
%    entry to the Basque language. This is automatically don
%    \LaTeXe\ or NFSS2. If T1 encoding is chosen, and provided
%    adequate hyphenation patterns\footnote{One source for
%    patterns is the archive at \texttt{tp.lc.ehu.es} that ca
%    accessed by anonymous FTP o
%    \texttt{http://tp.lc.ehu.es/jma/basque.html}} are availab
%    The easiest way to use the new encoding with \LaTeXe{
%    to load the package \texttt{t1enc} with |\usepackage|. This
%    be done before loading \ba

% \StopEventual

%    The macro |\LdfInit| takes care of preventing that this fil
%    loaded more than once, checking the category code of
%    \texttt{@} sign,
%    \begin{macroc
%<*c
\LdfInit{basque}\captionsba
%    \end{macroc

%    When this file is read as an option, i.e. by the |\usepack
%    command, \texttt{basque} could be an `unknown' language in w
%    case we have to make it known.  So we check for the existenc
%    |\l@basque| to see whether we have to do something h

%    \begin{macroc
\ifx\l@basque\@undef
  \@nopatterns{Bas
  \adddialect\l@bas

%    \end{macroc

%    The next step consists of defining commands to switch to
%    from) the Basque langu

% \begin{macro}{\captionsbas
%    The macro |\captionsbasque| defines all strings used
%    the four standard documentclasses provided with \La
% \changes{basque-1.0e}{2000/09/19}{Added \cs{glossaryna
% \changes{basque-1.0f}{2002/01/07}{Added translation for Gloss
%    \begin{macroc
\addto\captionsbasq
  \def\prefacename{Hitzaurr
  \def\refname{Erreferentzi
  \def\abstractname{Laburpe
  \def\bibname{Bibliograf
  \def\chaptername{Kapitul
  \def\appendixname{Eranski
  \def\contentsname{Gaien Aurkibid
  \def\listfigurename{Irudien Zerren
  \def\listtablename{Taulen Zerren
  \def\indexname{Kontzeptuen Aurkibid
  \def\figurename{Irud
  \def\tablename{Tau
  \def\partname{Ata
  \def\enclname{Erants
  \def\ccname{Kop
  \def\headtoname{No
  \def\pagename{Orr
  \def\seename{Iku
  \def\alsoname{Ikusi, halab
  \def\proofname{Frogape
  \def\glossaryname{Glosari

%    \end{macroc
% \end{ma

% \begin{macro}{\datebas
%    The macro |\datebasque| redefines the command |\today
%    produce Ba
% \changes{basque-1.0b}{1997/10/01}{Use \cs{edef} to define \cs{to
%    to save mem
% \changes{basque-1.0b}{1998/03/28}{use \cs{def} instead of \cs{ed
% \changes{basque-1.0c}{1999/11/22}{fixed typo in April's n
%    \begin{macroc
\def\datebasq
  \def\today{\number\year.eko\space\ifcase\mont
    urtarrilaren\or otsailaren\or martxoaren\or apirilare
    maiatzaren\or ekainaren\or uztailaren\or abuztuare
    irailaren\or urriaren\or azaroare
    abenduaren\fi~\number\d
%    \end{macroc
% \end{ma

% \begin{macro}{\extrasbas
% \begin{macro}{\noextrasbas
%    The macro |\extrasbasque| will perform all the extra definit
%    needed for the Basque language. The macro |\noextrasbasque
%    used to cancel the actions of |\extrasbasque|. For Basque,
%    characters are made active or are redefined. In particular,
%    \texttt{"} character and the |~| character receive
%    meanings. Therefore these characters have to be treate
%    `special' characte

%    \begin{macroc
\addto\extrasbasque{\languageshorthands{basq
\initiate@active@cha
\initiate@active@cha
\addto\extrasbasq
  \bbl@activate
  \bbl@activate
%    \end{macroc
%    Don't forget to turn the shorthands off ag
% \changes{basque-1.0d}{1999/12/16}{Deactivate shorthands ousid
%    Bas
%    \begin{macroc
\addto\noextrasbas
  \bbl@deactivate{"}\bbl@deactivate
%    \end{macroc

%    Apart from the active characters some other macros get a
%    definition. Therefore we store the current one to be abl
%    restore them la
%    \begin{macroc
\addto\extrasbasq
  \babel@sav
  \babel@sav
  \def\"{\protect\@umla
  \def\~{\protect\@til
%    \end{macroc
% \end{ma
% \end{ma

%  \begin{macro}{\basquehyphenm
%    Basque hyphenation uses |\lefthyphenmin| and |\righthyphen
%    both set t
% \changes{basque-1.0e}{2000/09/22}{Now use \cs{providehyphenmins
%    provide a default va
%    \begin{macroc
\providehyphenmins{\CurrentOption}{\tw@\
%    \end{macroc
% \end{ma

%  \begin{macro}{\diere
%  \begin{macro}{\textti
%    The original definition of |\"| is stored as |\dieresia|, bec
%    the we do not know what is its definition, since it depend
%    the encoding we are using or on special macros that the
%    might have loaded. The expansion of the macro might use the \
%    |\accent| primitive using some particular accent that the
%    provides or might check if a combined accent exists in the f
%    These two cases happen with respectively OT1 and T1 encodi
%    For this reason we save the definition of |\"| and use tha
%    the definition of other macros. We do likewise for |\'|
%    |\~|. The present coding of this option file is incorrect in
%    it can break when the encoding changes. We do not
%    |\tilde| as the macro name because it is already define
%    |\mathacce
%    \begin{macroc
\let\dieres
\let\texttil
%    \end{macroc
%  \end{ma
%  \end{ma

%  \begin{macro}{\@uml
%  \begin{macro}{\@ti
%    We check the encoding and if not using T1, we make the acc
%    expand but enabling hyphenation beyond the accent. If this is
%    case, not all break positions will be found in words that con
%    accents, but this is a limitation in \TeX. An unsolved pro
%    here is that the encoding can change at any time. The definit
%    below are made in such a way that a change between two 256-
%    encodings are supported, but changes between a 128-char a
%    256-char encoding are not properly supported. We check if T
%    in use. If not, we will give a warning and proceed redefining
%    accent macros so that \TeX{} at least finds the breaks that
%    not too close to the accent. The warning will only be printe
%    the log f
%    \begin{macroc
\ifx\DeclareFontShape\@undef
  \wlog{Warning: You are using an old La
  \wlog{Some word breaks will not be fou
  \def\@umlaut#1{\allowhyphens\dieresia{#1}\allowhyph
  \def\@tilde#1{\allowhyphens\texttilde{#1}\allowhyph
\
  \edef\bbl@next
  \ifx\f@encoding\bbl@
    \let\@umlaut\dier
    \let\@tilde\textt
  \
    \wlog{Warning: You are using encoding \f@encoding\s
      instead of
    \wlog{Some word breaks will not be fou
    \def\@umlaut#1{\allowhyphens\dieresia{#1}\allowhyph
    \def\@tilde#1{\allowhyphens\texttilde{#1}\allowhyph


%    \end{macroc
%  \end{ma
%  \end{ma

%     Now we can define our shorthands: the french quo
% \changes{basque-1.0b}{1997/04/03}{Removed empty groups a
%    guillemot character
%    \begin{macroc
\declare@shorthand{basque}{"
  \textormath{\guillemotleft}{\mbox{\guillemotlef
\declare@shorthand{basque}{"
  \textormath{\guillemotright}{\mbox{\guillemotrigh
%    \end{macroc
%    ordinals\footnote{The code for the ordinals was taken from
%    answer provided by Raymond
%    (\texttt{raymond@math.berkeley.edu}) to a question by Joseph
%    (\texttt{yogi@cs.ubc.ca}) in \texttt{comp.text.tex
%    \begin{macroc
  \declare@shorthand{basque}{'
    \textormath{\textquotedblright}{\sp\bgroup\prim@
%    \end{macroc
%    til
%    \begin{macroc
\declare@shorthand{basque}{~n}{\textormath{\~n}{\@tilde
\declare@shorthand{basque}{~N}{\textormath{\~N}{\@tilde
%    \end{macroc
%    and some additional comma

%    The shorthand |"-| should be used in places where a word cont
%    an explictit hyphenation character. According to the Academ
%    the Basque language, when a word break occurs at an expl
%    hyphen it must appear \emph{both} at the end of the first
%    \emph{and} at the beginning of the second l
% \changes{changes-1.0f}{2002/01/07}{The hyphen char needs to ap
%    at the beginning of the line as we
%    \begin{macroc
\declare@shorthand{basque}{"
  \nobreak\discretionary{-}{-}{-}\bbl@allowhyph
\declare@shorthand{basque}{"
  \textormath{\nobreak\discretionary{-}{}{\kern.03
              \allowhyphens
%    \end{macroc

%    The macro |\ldf@finish| takes care of looking f
%    configuration file, setting the main language to be switche
%    at |\begin{document}| and resetting the category cod
%    \texttt{@} to its original va
%    \begin{macroc
\ldf@finish{bas
%</c
%    \end{macroc

% \Fi


%% \CharacterT
%%  {Upper-case    \A\B\C\D\E\F\G\H\I\J\K\L\M\N\O\P\Q\R\S\T\U\V\W\X
%%   Lower-case    \a\b\c\d\e\f\g\h\i\j\k\l\m\n\o\p\q\r\s\t\u\v\w\x
%%   Digits        \0\1\2\3\4\5\6\7
%%   Exclamation   \!     Double quote  \"     Hash (number
%%   Dollar        \$     Percent       \%     Ampersand
%%   Acute accent  \'     Left paren    \(     Right paren
%%   Asterisk      \*     Plus          \+     Comma
%%   Minus         \-     Point         \.     Solidus
%%   Colon         \:     Semicolon     \;     Less than
%%   Equals        \=     Greater than  \>     Question mar
%%   Commercial at \@     Left bracket  \[     Backslash
%%   Right bracket \]     Circumflex    \^     Underscore
%%   Grave accent  \`     Left brace    \{     Vertical bar
%%   Right brace   \}     Tilde

\endi
}
\DeclareOption{brazil}{%%
%% This file will generate fast loadable files and documentation
%% driver files from the doc files in this package when run through
%% LaTeX or TeX.
%%
%% Copyright 1989-2008 Johannes L. Braams and any individual authors
%% listed elsewhere in this file.  All rights reserved.
%% 
%% This file is part of the Babel system.
%% --------------------------------------
%% 
%% It may be distributed and/or modified under the
%% conditions of the LaTeX Project Public License, either version 1.3
%% of this license or (at your option) any later version.
%% The latest version of this license is in
%%   http://www.latex-project.org/lppl.txt
%% and version 1.3 or later is part of all distributions of LaTeX
%% version 2003/12/01 or later.
%% 
%% This work has the LPPL maintenance status "maintained".
%% 
%% The Current Maintainer of this work is Johannes Braams.
%% 
%% The list of all files belonging to the LaTeX base distribution is
%% given in the file `manifest.bbl. See also `legal.bbl' for additional
%% information.
%% 
%% The list of derived (unpacked) files belonging to the distribution
%% and covered by LPPL is defined by the unpacking scripts (with
%% extension .ins) which are part of the distribution.
%%
%% --------------- start of docstrip commands ------------------
%%
\def\filedate{1999/04/11}
\def\batchfile{portuges.ins}
\input docstrip.tex

{\ifx\generate\undefined
\Msg{**********************************************}
\Msg{*}
\Msg{* This installation requires docstrip}
\Msg{* version 2.3c or later.}
\Msg{*}
\Msg{* An older version of docstrip has been input}
\Msg{*}
\Msg{**********************************************}
\errhelp{Move or rename old docstrip.tex.}
\errmessage{Old docstrip in input path}
\batchmode
\csname @@end\endcsname
\fi}

\declarepreamble\mainpreamble
This is a generated file.

Copyright 1989-2008 Johannes L. Braams and any individual authors
listed elsewhere in this file.  All rights reserved.

This file was generated from file(s) of the Babel system.
---------------------------------------------------------

It may be distributed and/or modified under the
conditions of the LaTeX Project Public License, either version 1.3
of this license or (at your option) any later version.
The latest version of this license is in
  http://www.latex-project.org/lppl.txt
and version 1.3 or later is part of all distributions of LaTeX
version 2003/12/01 or later.

This work has the LPPL maintenance status "maintained".

The Current Maintainer of this work is Johannes Braams.

This file may only be distributed together with a copy of the Babel
system. You may however distribute the Babel system without
such generated files.

The list of all files belonging to the Babel distribution is
given in the file `manifest.bbl'. See also `legal.bbl for additional
information.

The list of derived (unpacked) files belonging to the distribution
and covered by LPPL is defined by the unpacking scripts (with
extension .ins) which are part of the distribution.
\endpreamble

\declarepreamble\fdpreamble
This is a generated file.

Copyright 1989-2008 Johannes L. Braams and any individual authors
listed elsewhere in this file.  All rights reserved.

This file was generated from file(s) of the Babel system.
---------------------------------------------------------

It may be distributed and/or modified under the
conditions of the LaTeX Project Public License, either version 1.3
of this license or (at your option) any later version.
The latest version of this license is in
  http://www.latex-project.org/lppl.txt
and version 1.3 or later is part of all distributions of LaTeX
version 2003/12/01 or later.

This work has the LPPL maintenance status "maintained".

The Current Maintainer of this work is Johannes Braams.

This file may only be distributed together with a copy of the Babel
system. You may however distribute the Babel system without
such generated files.

The list of all files belonging to the Babel distribution is
given in the file `manifest.bbl'. See also `legal.bbl for additional
information.

In particular, permission is granted to customize the declarations in
this file to serve the needs of your installation.

However, NO PERMISSION is granted to distribute a modified version
of this file under its original name.

\endpreamble

\keepsilent

\usedir{tex/generic/babel} 

\usepreamble\mainpreamble
\generate{\file{portuges.ldf}{\from{portuges.dtx}{code}}
          }
\usepreamble\fdpreamble

\ifToplevel{
\Msg{***********************************************************}
\Msg{*}
\Msg{* To finish the installation you have to move the following}
\Msg{* files into a directory searched by TeX:}
\Msg{*}
\Msg{* \space\space All *.def, *.fd, *.ldf, *.sty}
\Msg{*}
\Msg{* To produce the documentation run the files ending with}
\Msg{* '.dtx' and `.fdd' through LaTeX.}
\Msg{*}
\Msg{* Happy TeXing}
\Msg{***********************************************************}
}
 
\endinput
}
%    \end{macrocode}
% \changes{babel~3.5b}{1995/05/25}{Added \Lopt{brazilian} as
%    alternative for \Lopt{brazil}}
%    \begin{macrocode}
\DeclareOption{brazilian}{%%
%% This file will generate fast loadable files and documentation
%% driver files from the doc files in this package when run through
%% LaTeX or TeX.
%%
%% Copyright 1989-2008 Johannes L. Braams and any individual authors
%% listed elsewhere in this file.  All rights reserved.
%% 
%% This file is part of the Babel system.
%% --------------------------------------
%% 
%% It may be distributed and/or modified under the
%% conditions of the LaTeX Project Public License, either version 1.3
%% of this license or (at your option) any later version.
%% The latest version of this license is in
%%   http://www.latex-project.org/lppl.txt
%% and version 1.3 or later is part of all distributions of LaTeX
%% version 2003/12/01 or later.
%% 
%% This work has the LPPL maintenance status "maintained".
%% 
%% The Current Maintainer of this work is Johannes Braams.
%% 
%% The list of all files belonging to the LaTeX base distribution is
%% given in the file `manifest.bbl. See also `legal.bbl' for additional
%% information.
%% 
%% The list of derived (unpacked) files belonging to the distribution
%% and covered by LPPL is defined by the unpacking scripts (with
%% extension .ins) which are part of the distribution.
%%
%% --------------- start of docstrip commands ------------------
%%
\def\filedate{1999/04/11}
\def\batchfile{portuges.ins}
\input docstrip.tex

{\ifx\generate\undefined
\Msg{**********************************************}
\Msg{*}
\Msg{* This installation requires docstrip}
\Msg{* version 2.3c or later.}
\Msg{*}
\Msg{* An older version of docstrip has been input}
\Msg{*}
\Msg{**********************************************}
\errhelp{Move or rename old docstrip.tex.}
\errmessage{Old docstrip in input path}
\batchmode
\csname @@end\endcsname
\fi}

\declarepreamble\mainpreamble
This is a generated file.

Copyright 1989-2008 Johannes L. Braams and any individual authors
listed elsewhere in this file.  All rights reserved.

This file was generated from file(s) of the Babel system.
---------------------------------------------------------

It may be distributed and/or modified under the
conditions of the LaTeX Project Public License, either version 1.3
of this license or (at your option) any later version.
The latest version of this license is in
  http://www.latex-project.org/lppl.txt
and version 1.3 or later is part of all distributions of LaTeX
version 2003/12/01 or later.

This work has the LPPL maintenance status "maintained".

The Current Maintainer of this work is Johannes Braams.

This file may only be distributed together with a copy of the Babel
system. You may however distribute the Babel system without
such generated files.

The list of all files belonging to the Babel distribution is
given in the file `manifest.bbl'. See also `legal.bbl for additional
information.

The list of derived (unpacked) files belonging to the distribution
and covered by LPPL is defined by the unpacking scripts (with
extension .ins) which are part of the distribution.
\endpreamble

\declarepreamble\fdpreamble
This is a generated file.

Copyright 1989-2008 Johannes L. Braams and any individual authors
listed elsewhere in this file.  All rights reserved.

This file was generated from file(s) of the Babel system.
---------------------------------------------------------

It may be distributed and/or modified under the
conditions of the LaTeX Project Public License, either version 1.3
of this license or (at your option) any later version.
The latest version of this license is in
  http://www.latex-project.org/lppl.txt
and version 1.3 or later is part of all distributions of LaTeX
version 2003/12/01 or later.

This work has the LPPL maintenance status "maintained".

The Current Maintainer of this work is Johannes Braams.

This file may only be distributed together with a copy of the Babel
system. You may however distribute the Babel system without
such generated files.

The list of all files belonging to the Babel distribution is
given in the file `manifest.bbl'. See also `legal.bbl for additional
information.

In particular, permission is granted to customize the declarations in
this file to serve the needs of your installation.

However, NO PERMISSION is granted to distribute a modified version
of this file under its original name.

\endpreamble

\keepsilent

\usedir{tex/generic/babel} 

\usepreamble\mainpreamble
\generate{\file{portuges.ldf}{\from{portuges.dtx}{code}}
          }
\usepreamble\fdpreamble

\ifToplevel{
\Msg{***********************************************************}
\Msg{*}
\Msg{* To finish the installation you have to move the following}
\Msg{* files into a directory searched by TeX:}
\Msg{*}
\Msg{* \space\space All *.def, *.fd, *.ldf, *.sty}
\Msg{*}
\Msg{* To produce the documentation run the files ending with}
\Msg{* '.dtx' and `.fdd' through LaTeX.}
\Msg{*}
\Msg{* Happy TeXing}
\Msg{***********************************************************}
}
 
\endinput
}
\DeclareOption{breton}{% \iffalse meta-com

% Copyright 1989-2005 Johannes L. Braams and any individual aut
% listed elsewhere in this file.  All rights reser

% This file is part of the Babel sys
% ----------------------------------

% It may be distributed and/or modified under
% conditions of the LaTeX Project Public License, either version
% of this license or (at your option) any later vers
% The latest version of this license i
%   http://www.latex-project.org/lppl
% and version 1.3 or later is part of all distributions of L
% version 2003/12/01 or la

% This work has the LPPL maintenance status "maintain

% The Current Maintainer of this work is Johannes Bra

% The list of all files belonging to the Babel syste
% given in the file `manifest.bbl. See also `legal.bbl' for additi
% informat

% The list of derived (unpacked) files belonging to the distribu
% and covered by LPPL is defined by the unpacking scripts (
% extension .ins) which are part of the distribut
%
% \CheckSum{

% \iff
%    Tell the \LaTeX\ system who we are and write an entry on
%    transcr
%<*
\ProvidesFile{breton.
%</
%<code>\ProvidesLanguage{bre

%\ProvidesFile{breton.
        [2005/03/29 v1.0h Breton support from the babel sys
%\iff
%% File `breton.
%% Babel package for LaTeX versio
%% Copyright (C) 1989 -
%%           by Johannes Braams, TeX

%% Breton Language Definition
%% Copyright (C) 1994 -
%%           by Christian Rol
%              Universite de Bretagne occiden
%              Departement d'informat
%              6, avenue Le Go
%              BP
%              29275 Brest Cedex -- FR
%              Christian.Rolland at univ-brest.fr (Inter

%% Please report errors to: J.L. Br
%%                          babel at braams.cistro

%    This file is part of the babel system, it provides the so
%    code for the Breton language definition file. It is based on
%    language definition file for french, version
%<*filedri
\documentclass{ltx
\newcommand*\TeXhax{\TeX
\newcommand*\babel{\textsf{bab
\newcommand*\langvar{$\langle \it lang \rang
\newcommand*\note[
\newcommand*\Lopt[1]{\textsf{
\newcommand*\file[1]{\texttt{
\begin{docum
 \DocInput{breton.
\end{docum
%</filedri

% \GetFileInfo{breton.
% \changes{breton-1.0e}{1996/10/10}{Replaced \cs{undefined}
%    \cs{@undefined} and \cs{empty} with \cs{@empty} for consist
%    with \LaTeX, moved the definition of \cs{atcatcode} right to
%    beginnin

% \changes{breton-1.0}{1994/09/21}{First rele

%  \section{The Breton langu

%    The file \file{\filename}\footnote{The file described in
%    section has version number \fileversion\ and was last revise
%    \filedate.} defines all the language-specific macros for the Br
%    langua

%    There are not really typographic rules for the Br
%    language. It is a local language (it's one of the ce
%    languages) which is spoken in Brittany (West of France). S
%    have a synthesis between french typographic rules and eng
%    typographic rules. The characters \texttt{:}, \texttt
%    \texttt{!} and \texttt{?} are made active in order to g
%    whitespace automatically before these charact

% \StopEventual

%    The macro |\LdfInit| takes care of preventing that this fil
%    loaded more than once, checking the category code of
%    \texttt{@} sign,
% \changes{breton-1.0e}{1996/11/02}{Now use \cs{LdfInit} to per
%    initial chec
%    \begin{macroc
%<*c
\LdfInit{breton}\captionsbr
%    \end{macroc

%    When this file is read as an option, i.e. by the |\usepack
%    command, \texttt{breton} will be an `unknown' language in w
%    case we have to make it known.  So we check for the existenc
%    |\l@breton| to see whether we have to do something h

%    \begin{macroc
\ifx\l@breton\@undef
    \@nopatterns{Bre
    \adddialect\l@breton
%    \end{macroc
%    The next step consists of defining commands to switch to
%    English language. The reason for this is that a user might
%    to switch back and forth between langua

% \begin{macro}{\captionsbre
%    The macro |\captionsbreton| defines all strings used in
%    four standard document classes provided with \La
% \changes{breton-1.0b}{1995/07/04}{Added \cs{proofname}
%    AMS-\La
% \changes{breton-1.0h}{2000/09/19}{Added \cs{glossaryna
%    \begin{macroc
\addto\captionsbret
  \def\prefacename{Rakskr
  \def\refname{Daveenno\
  \def\abstractname{Dvierra\
  \def\bibname{Lennadure
  \def\chaptername{Penn
  \def\appendixname{Stagade
  \def\contentsname{Taole
  \def\listfigurename{Listenn ar Figurenno\
  \def\listtablename{Listenn an taolenno\
  \def\indexname{Meneg
  \def\figurename{Figure
  \def\tablename{Taole
  \def\partname{Lode
  \def\enclname{Diello\`u kevr
  \def\ccname{Eilskrid
  \def\headtoname{e
  \def\pagename{Paje
  \def\seename{Gwelo
  \def\alsoname{Gwelout iv
  \def\proofname{Proof}%  <-- needs transla
  \def\glossaryname{Glossary}% <-- Needs transla

%    \end{macroc
% \end{ma

% \begin{macro}{\datebre
%    The macro |\datebreton| redefines the com
%    |\today| to produce Breton da
% \changes{breton-1.0f}{1997/10/01}{Use \cs{edef} to define \cs{to
%    to save mem
% \changes{breton-1.0f}{1998/03/28}{use \cs{def} instead of \cs{ed
%    \begin{macroc
\def\datebret
  \def\today{\ifnum\day=1\relax 1\/$^{\rm a\tilde{n}}$\
    \number\day\fi \space a\space viz\space\ifcase\mont
    Genver\or C'hwevrer\or Meurzh\or Ebrel\or Mae\or Mezheve
    Gouere\or Eost\or Gwengolo\or Here\or Du\or Kerz
    \space\number\ye
%    \end{macroc
% \end{ma

% \begin{macro}{\extrasbre
% \begin{macro}{\noextrasbre
%    The macro |\extrasbreton| will perform all the e
%    definitions needed for the Breton language. The m
%    |\noextrasbreton| is used to cancel the action
%    |\extrasbret

%    The category code of the characters \texttt{:}, \texttt
%    \texttt{!} and \texttt{?} is made |\active| to insert a li
%    white sp
% \changes{breton-1.0b}{1995/03/07}{Use the new mechanism for dea
%    with active ch
%    \begin{macroc
\initiate@active@cha
\initiate@active@cha
\initiate@active@cha
\initiate@active@cha
%    \end{macroc
%    We specify that the breton group of shorthands should be u
%    \begin{macroc
\addto\extrasbreton{\languageshorthands{bret
%    \end{macroc
%    These characters are `turned on' once, later their definition
%    va
%    \begin{macroc
\addto\extrasbret
  \bbl@activate{:}\bbl@activate
  \bbl@activate{!}\bbl@activate
%    \end{macroc
%    Don't forget to turn the shorthands off ag
% \changes{breton-1.0g}{1999/12/16}{Deactivate shorthands ouside of Bre
%    \begin{macroc
\addto\noextrasbret
 \bbl@deactivate{:}\bbl@deactivate
 \bbl@deactivate{!}\bbl@deactivate
%    \end{macroc

%    The last thing |\extrasbreton| needs to do is to make sure
%    |\frenchspacing| is in effect.  If this is not the case
%    execution of |\noextrasbreton| will switch it of ag
%    \begin{macroc
\addto\extrasbreton{\bbl@frenchspac
\addto\noextrasbreton{\bbl@nonfrenchspac
%    \end{macroc
% \end{ma
% \end{ma

% \begin{macro}{\breton@sh
%    We have to reduce the amount of white space before \texttt
%    \texttt{:} and \texttt{!} when the user types a space in fron
%    these characters. This should only happen outside mathmode, h
%    the test with |\ifmmo

%    \begin{macroc
\declare@shorthand{breton}{
    \ifm
      \string;\s
    \else\r
%    \end{macroc
%    In horizontal mode we check for the presence of a `space'
%    replace it by a |\thinspa
%    \begin{macroc
      \ifh
        \ifdim\lastskip
          \unskip\penalty\@M\thins


      \string;\s
    \
%    \end{macroc
% \end{ma

% \begin{macro}{\breton@sh
% \begin{macro}{\breton@sh
%    Because these definitions are very similar only one is displ
%    in a way that the definition can be easily chec
%    \begin{macroc
\declare@shorthand{breton}{
  \ifmmode\string:\s
  \else\r
    \ifh
      \ifdim\lastskip>\z@\unskip\penalty\@M\thinspac

    \string:\s

\declare@shorthand{breton}{
  \ifmmode\string!\s
  \else\r
    \ifh
      \ifdim\lastskip>\z@\unskip\penalty\@M\thinspac

    \string!\s

%    \end{macroc
% \end{ma
% \end{ma

% \begin{macro}{\breton@sh
%    For the question mark something different has to be done. In
%    case the amount of white space that replaces the space chara
%    depends on the dimensions of the f
%    \begin{macroc
\declare@shorthand{breton}{
  \ifm
    \string?\s
  \else\r
    \ifh
      \ifdim\lastskip
        \un
        \kern\fontdimen2\
        \kern-1.4\fontdimen3\


    \string?\s

%    \end{macroc
% \end{ma

%    All that is left to do now is provide the breton user with
%    extra utilit

%    Some definitions for special charact
%    \begin{macroc
\DeclareTextSymbol{\at}{OT1}
\DeclareTextSymbol{\at}{T1}
\DeclareTextSymbolDefault{\at}{
\DeclareTextSymbol{\boi}{OT1}
\DeclareTextSymbol{\boi}{T1}
\DeclareTextSymbolDefault{\boi}{
\DeclareTextSymbol{\circonflexe}{OT1}
\DeclareTextSymbol{\circonflexe}{T1
\DeclareTextSymbolDefault{\circonflexe}{
\DeclareTextSymbol{\tild}{OT1}{
\DeclareTextSymbol{\tild}{T1
\DeclareTextSymbolDefault{\tild}{
\DeclareTextSymbol{\degre}{OT1}
\DeclareTextSymbol{\degre}{T1
\DeclareTextSymbolDefault{\degre}{
%    \end{macroc

%    The following macros are used in the redefinition of |\^|
%    |\"| to handle the lette
% \changes{breton-1.0c}{1995/07/07}{Postpone
%    \cs{DeclareTextCompositeCommand}s untill \cs{AtBeginDocume

%    \begin{macroc
\AtBeginDocume
  \DeclareTextCompositeCommand{\^}{OT1}{i}{\
  \DeclareTextCompositeCommand{\"}{OT1}{i}{\"\
%    \end{macroc

%    And some more macros for number
%    \begin{macroc
\def\kentan{1\/${}^{\rm a\tilde{n
\def\eil{2\/${}^{\rm
\def\re{\/${}^{\rm r
\def\trede{3
\def\pevare{4
\def\vet{\/${}^{\rm ve
\def\pempvet{5\
%    \end{macroc

%    The macro |\ldf@finish| takes care of looking f
%    configuration file, setting the main language to be switche
%    at |\begin{document}| and resetting the category cod
%    \texttt{@} to its original va
% \changes{breton-1.0e}{1996/11/02}{Now use \cs{ldf@finish} to
%
%    \begin{macroc
\ldf@finish{bre
%</c
%    \end{macroc

% \Fi

%% \CharacterT
%%  {Upper-case    \A\B\C\D\E\F\G\H\I\J\K\L\M\N\O\P\Q\R\S\T\U\V\W\X
%%   Lower-case    \a\b\c\d\e\f\g\h\i\j\k\l\m\n\o\p\q\r\s\t\u\v\w\x
%%   Digits        \0\1\2\3\4\5\6\7
%%   Exclamation   \!     Double quote  \"     Hash (number
%%   Dollar        \$     Percent       \%     Ampersand
%%   Acute accent  \'     Left paren    \(     Right paren
%%   Asterisk      \*     Plus          \+     Comma
%%   Minus         \-     Point         \.     Solidus
%%   Colon         \:     Semicolon     \;     Less than
%%   Equals        \=     Greater than  \>     Question mar
%%   Commercial at \@     Left bracket  \[     Backslash
%%   Right bracket \]     Circumflex    \^     Underscore
%%   Grave accent  \`     Left brace    \{     Vertical bar
%%   Right brace   \}     Tilde

\endi
}
%    \end{macrocode}
% \changes{babel~3.5d}{1995/07/02}{Added \Lopt{british} as an
%    alternative for \Lopt{english} with a preference for british
%    hyphenation}
% \changes{babel~3.7f}{2000/09/21}{Added the \Lopt{bulgarian} option}
% \changes{babel~3.7g}{2001/02/07}{Added option \Lopt{canadian}}
% \changes{babel~3.7g}{2001/02/09}{Added option \Lopt{canadien}}
%    \begin{macrocode}
\DeclareOption{british}{%%
%% This file will generate fast loadable files and documentation
%% driver files from the doc files in this package when run through
%% LaTeX or TeX.
%%
%% Copyright 1989-2005 Johannes L. Braams and any individual authors
%% listed elsewhere in this file.  All rights reserved.
%% 
%% This file is part of the Babel system.
%% --------------------------------------
%% 
%% It may be distributed and/or modified under the
%% conditions of the LaTeX Project Public License, either version 1.3
%% of this license or (at your option) any later version.
%% The latest version of this license is in
%%   http://www.latex-project.org/lppl.txt
%% and version 1.3 or later is part of all distributions of LaTeX
%% version 2003/12/01 or later.
%% 
%% This work has the LPPL maintenance status "maintained".
%% 
%% The Current Maintainer of this work is Johannes Braams.
%% 
%% The list of all files belonging to the LaTeX base distribution is
%% given in the file `manifest.bbl. See also `legal.bbl' for additional
%% information.
%% 
%% The list of derived (unpacked) files belonging to the distribution
%% and covered by LPPL is defined by the unpacking scripts (with
%% extension .ins) which are part of the distribution.
%%
%% --------------- start of docstrip commands ------------------
%%
\def\filedate{1999/04/11}
\def\batchfile{english.ins}
\input docstrip.tex

{\ifx\generate\undefined
\Msg{**********************************************}
\Msg{*}
\Msg{* This installation requires docstrip}
\Msg{* version 2.3c or later.}
\Msg{*}
\Msg{* An older version of docstrip has been input}
\Msg{*}
\Msg{**********************************************}
\errhelp{Move or rename old docstrip.tex.}
\errmessage{Old docstrip in input path}
\batchmode
\csname @@end\endcsname
\fi}

\declarepreamble\mainpreamble
This is a generated file.

Copyright 1989-2005 Johannes L. Braams and any individual authors
listed elsewhere in this file.  All rights reserved.

This file was generated from file(s) of the Babel system.
---------------------------------------------------------

It may be distributed and/or modified under the
conditions of the LaTeX Project Public License, either version 1.3
of this license or (at your option) any later version.
The latest version of this license is in
  http://www.latex-project.org/lppl.txt
and version 1.3 or later is part of all distributions of LaTeX
version 2003/12/01 or later.

This work has the LPPL maintenance status "maintained".

The Current Maintainer of this work is Johannes Braams.

This file may only be distributed together with a copy of the Babel
system. You may however distribute the Babel system without
such generated files.

The list of all files belonging to the Babel distribution is
given in the file `manifest.bbl'. See also `legal.bbl for additional
information.

The list of derived (unpacked) files belonging to the distribution
and covered by LPPL is defined by the unpacking scripts (with
extension .ins) which are part of the distribution.
\endpreamble

\declarepreamble\fdpreamble
This is a generated file.

Copyright 1989-2005 Johannes L. Braams and any individual authors
listed elsewhere in this file.  All rights reserved.

This file was generated from file(s) of the Babel system.
---------------------------------------------------------

It may be distributed and/or modified under the
conditions of the LaTeX Project Public License, either version 1.3
of this license or (at your option) any later version.
The latest version of this license is in
  http://www.latex-project.org/lppl.txt
and version 1.3 or later is part of all distributions of LaTeX
version 2003/12/01 or later.

This work has the LPPL maintenance status "maintained".

The Current Maintainer of this work is Johannes Braams.

This file may only be distributed together with a copy of the Babel
system. You may however distribute the Babel system without
such generated files.

The list of all files belonging to the Babel distribution is
given in the file `manifest.bbl'. See also `legal.bbl for additional
information.

In particular, permission is granted to customize the declarations in
this file to serve the needs of your installation.

However, NO PERMISSION is granted to distribute a modified version
of this file under its original name.

\endpreamble

\keepsilent

\usedir{tex/generic/babel} 

\usepreamble\mainpreamble
\generate{\file{english.ldf}{\from{english.dtx}{code}}
          }
\usepreamble\fdpreamble

\ifToplevel{
\Msg{***********************************************************}
\Msg{*}
\Msg{* To finish the installation you have to move the following}
\Msg{* files into a directory searched by TeX:}
\Msg{*}
\Msg{* \space\space All *.def, *.fd, *.ldf, *.sty}
\Msg{*}
\Msg{* To produce the documentation run the files ending with}
\Msg{* '.dtx' and `.fdd' through LaTeX.}
\Msg{*}
\Msg{* Happy TeXing}
\Msg{***********************************************************}
}
 
\endinput
}
\DeclareOption{bulgarian}{%%
%% This file will generate fast loadable files and documentation
%% driver files from the doc files in this package when run through
%% LaTeX or TeX.
%%
%% Copyright 1989-2008 Johannes L. Braams and any individual authors
%% listed elsewhere in this file.  All rights reserved.
%% 
%% This file is part of the Babel system.
%% --------------------------------------
%% 
%% It may be distributed and/or modified under the
%% conditions of the LaTeX Project Public License, either version 1.3
%% of this license or (at your option) any later version.
%% The latest version of this license is in
%%   http://www.latex-project.org/lppl.txt
%% and version 1.3 or later is part of all distributions of LaTeX
%% version 2003/12/01 or later.
%% 
%% This work has the LPPL maintenance status "maintained".
%% 
%% The Current Maintainer of this work is Johannes Braams.
%% 
%% The list of all files belonging to the LaTeX base distribution is
%% given in the file `manifest.bbl. See also `legal.bbl' for additional
%% information.
%% 
%% The list of derived (unpacked) files belonging to the distribution
%% and covered by LPPL is defined by the unpacking scripts (with
%% extension .ins) which are part of the distribution.
%%
%% --------------- start of docstrip commands ------------------
%%
\def\filedate{2000/09/21}
\def\batchfile{bulgarian.ins}
\input docstrip.tex

{\ifx\generate\undefined
\Msg{**********************************************}
\Msg{*}
\Msg{* This installation requires docstrip}
\Msg{* version 2.3c or later.}
\Msg{*}
\Msg{* An older version of docstrip has been input}
\Msg{*}
\Msg{**********************************************}
\errhelp{Move or rename old docstrip.tex.}
\errmessage{Old docstrip in input path}
\batchmode
\csname @@end\endcsname
\fi}

\declarepreamble\mainpreamble
This is a generated file.

Copyright 1989-2008 Johannes L. Braams and any individual authors
listed elsewhere in this file.  All rights reserved.

This file was generated from file(s) of the Babel system.
---------------------------------------------------------

It may be distributed and/or modified under the
conditions of the LaTeX Project Public License, either version 1.3
of this license or (at your option) any later version.
The latest version of this license is in
  http://www.latex-project.org/lppl.txt
and version 1.3 or later is part of all distributions of LaTeX
version 2003/12/01 or later.

This work has the LPPL maintenance status "maintained".

The Current Maintainer of this work is Johannes Braams.

This file may only be distributed together with a copy of the Babel
system. You may however distribute the Babel system without
such generated files.

The list of all files belonging to the Babel distribution is
given in the file `manifest.bbl'. See also `legal.bbl for additional
information.

The list of derived (unpacked) files belonging to the distribution
and covered by LPPL is defined by the unpacking scripts (with
extension .ins) which are part of the distribution.
\endpreamble

\declarepreamble\fdpreamble
This is a generated file.

Copyright 1989-2008 Johannes L. Braams and any individual authors
listed elsewhere in this file.  All rights reserved.

This file was generated from file(s) of the Babel system.
---------------------------------------------------------

It may be distributed and/or modified under the
conditions of the LaTeX Project Public License, either version 1.3
of this license or (at your option) any later version.
The latest version of this license is in
  http://www.latex-project.org/lppl.txt
and version 1.3 or later is part of all distributions of LaTeX
version 2003/12/01 or later.

This work has the LPPL maintenance status "maintained".

The Current Maintainer of this work is Johannes Braams.

This file may only be distributed together with a copy of the Babel
system. You may however distribute the Babel system without
such generated files.

The list of all files belonging to the Babel distribution is
given in the file `manifest.bbl'. See also `legal.bbl for additional
information.

In particular, permission is granted to customize the declarations in
this file to serve the needs of your installation.

However, NO PERMISSION is granted to distribute a modified version
of this file under its original name.

\endpreamble

\keepsilent

\usedir{tex/generic/babel} 

\usepreamble\mainpreamble
\generate{\file{bulgarian.ldf}{\from{bulgarian.dtx}{code}}
          }
\usepreamble\fdpreamble

\ifToplevel{
\Msg{***********************************************************}
\Msg{*}
\Msg{* To finish the installation you have to move the following}
\Msg{* files into a directory searched by TeX:}
\Msg{*}
\Msg{* \space\space All *.def, *.fd, *.ldf, *.sty}
\Msg{*}
\Msg{* To produce the documentation run the files ending with}
\Msg{* '.dtx' and `.fdd' through LaTeX.}
\Msg{*}
\Msg{* Happy TeXing}
\Msg{***********************************************************}
}
 
\endinput}
\DeclareOption{canadian}{%%
%% This file will generate fast loadable files and documentation
%% driver files from the doc files in this package when run through
%% LaTeX or TeX.
%%
%% Copyright 1989-2005 Johannes L. Braams and any individual authors
%% listed elsewhere in this file.  All rights reserved.
%% 
%% This file is part of the Babel system.
%% --------------------------------------
%% 
%% It may be distributed and/or modified under the
%% conditions of the LaTeX Project Public License, either version 1.3
%% of this license or (at your option) any later version.
%% The latest version of this license is in
%%   http://www.latex-project.org/lppl.txt
%% and version 1.3 or later is part of all distributions of LaTeX
%% version 2003/12/01 or later.
%% 
%% This work has the LPPL maintenance status "maintained".
%% 
%% The Current Maintainer of this work is Johannes Braams.
%% 
%% The list of all files belonging to the LaTeX base distribution is
%% given in the file `manifest.bbl. See also `legal.bbl' for additional
%% information.
%% 
%% The list of derived (unpacked) files belonging to the distribution
%% and covered by LPPL is defined by the unpacking scripts (with
%% extension .ins) which are part of the distribution.
%%
%% --------------- start of docstrip commands ------------------
%%
\def\filedate{1999/04/11}
\def\batchfile{english.ins}
\input docstrip.tex

{\ifx\generate\undefined
\Msg{**********************************************}
\Msg{*}
\Msg{* This installation requires docstrip}
\Msg{* version 2.3c or later.}
\Msg{*}
\Msg{* An older version of docstrip has been input}
\Msg{*}
\Msg{**********************************************}
\errhelp{Move or rename old docstrip.tex.}
\errmessage{Old docstrip in input path}
\batchmode
\csname @@end\endcsname
\fi}

\declarepreamble\mainpreamble
This is a generated file.

Copyright 1989-2005 Johannes L. Braams and any individual authors
listed elsewhere in this file.  All rights reserved.

This file was generated from file(s) of the Babel system.
---------------------------------------------------------

It may be distributed and/or modified under the
conditions of the LaTeX Project Public License, either version 1.3
of this license or (at your option) any later version.
The latest version of this license is in
  http://www.latex-project.org/lppl.txt
and version 1.3 or later is part of all distributions of LaTeX
version 2003/12/01 or later.

This work has the LPPL maintenance status "maintained".

The Current Maintainer of this work is Johannes Braams.

This file may only be distributed together with a copy of the Babel
system. You may however distribute the Babel system without
such generated files.

The list of all files belonging to the Babel distribution is
given in the file `manifest.bbl'. See also `legal.bbl for additional
information.

The list of derived (unpacked) files belonging to the distribution
and covered by LPPL is defined by the unpacking scripts (with
extension .ins) which are part of the distribution.
\endpreamble

\declarepreamble\fdpreamble
This is a generated file.

Copyright 1989-2005 Johannes L. Braams and any individual authors
listed elsewhere in this file.  All rights reserved.

This file was generated from file(s) of the Babel system.
---------------------------------------------------------

It may be distributed and/or modified under the
conditions of the LaTeX Project Public License, either version 1.3
of this license or (at your option) any later version.
The latest version of this license is in
  http://www.latex-project.org/lppl.txt
and version 1.3 or later is part of all distributions of LaTeX
version 2003/12/01 or later.

This work has the LPPL maintenance status "maintained".

The Current Maintainer of this work is Johannes Braams.

This file may only be distributed together with a copy of the Babel
system. You may however distribute the Babel system without
such generated files.

The list of all files belonging to the Babel distribution is
given in the file `manifest.bbl'. See also `legal.bbl for additional
information.

In particular, permission is granted to customize the declarations in
this file to serve the needs of your installation.

However, NO PERMISSION is granted to distribute a modified version
of this file under its original name.

\endpreamble

\keepsilent

\usedir{tex/generic/babel} 

\usepreamble\mainpreamble
\generate{\file{english.ldf}{\from{english.dtx}{code}}
          }
\usepreamble\fdpreamble

\ifToplevel{
\Msg{***********************************************************}
\Msg{*}
\Msg{* To finish the installation you have to move the following}
\Msg{* files into a directory searched by TeX:}
\Msg{*}
\Msg{* \space\space All *.def, *.fd, *.ldf, *.sty}
\Msg{*}
\Msg{* To produce the documentation run the files ending with}
\Msg{* '.dtx' and `.fdd' through LaTeX.}
\Msg{*}
\Msg{* Happy TeXing}
\Msg{***********************************************************}
}
 
\endinput
}
\DeclareOption{canadien}{%%
%% This file will generate fast loadable files and documentation
%% driver files from the doc files in this package when run through
%% LaTeX or TeX.
%%
%% Copyright 1989-2011 Johannes L. Braams and any individual authors
%% listed elsewhere in this file.  All rights reserved.
%% 
%% This file is part of the Babel system.
%% --------------------------------------
%% 
%% It may be distributed and/or modified under the
%% conditions of the LaTeX Project Public License, either version 1.3
%% of this license or (at your option) any later version.
%% The latest version of this license is in
%%   http://www.latex-project.org/lppl.txt
%% and version 1.3 or later is part of all distributions of LaTeX
%% version 2003/12/01 or later.
%% 
%% This work has the LPPL maintenance status "maintained".
%% 
%% The Current Maintainer of this work is Johannes Braams.
%% 
%% The list of all files belonging to the LaTeX base distribution is
%% given in the file `manifest.bbl. See also `legal.bbl' for additional
%% information.
%% 
%% The list of derived (unpacked) files belonging to the distribution
%% and covered by LPPL is defined by the unpacking scripts (with
%% extension .ins) which are part of the distribution.
%%
%% --------------- start of docstrip commands ------------------
%%
\def\filedate{1999/10/30}
\def\batchfile{frenchb.ins}
\input docstrip.tex

{\ifx\generate\undefined
\Msg{**********************************************}
\Msg{*}
\Msg{* This installation requires docstrip}
\Msg{* version 2.3c or later.}
\Msg{*}
\Msg{* An older version of docstrip has been input}
\Msg{*}
\Msg{**********************************************}
\errhelp{Move or rename old docstrip.tex.}
\errmessage{Old docstrip in input path}
\batchmode
\csname @@end\endcsname
\fi}

\declarepreamble\mainpreamble
This is a generated file.

Copyright 1989-2011 Johannes L. Braams and any individual authors
listed elsewhere in this file.  All rights reserved.

This file was generated from file(s) of the Babel system.
---------------------------------------------------------

It may be distributed and/or modified under the
conditions of the LaTeX Project Public License, either version 1.3
of this license or (at your option) any later version.
The latest version of this license is in
  http://www.latex-project.org/lppl.txt
and version 1.3 or later is part of all distributions of LaTeX
version 2003/12/01 or later.

This work has the LPPL maintenance status "maintained".

The Current Maintainer of this work is Johannes Braams.

This file may only be distributed together with a copy of the Babel
system. You may however distribute the Babel system without
such generated files.

The list of all files belonging to the Babel distribution is
given in the file `manifest.bbl'. See also `legal.bbl for additional
information.

The list of derived (unpacked) files belonging to the distribution
and covered by LPPL is defined by the unpacking scripts (with
extension .ins) which are part of the distribution.
\endpreamble

\declarepreamble\fdpreamble
This is a generated file.

Copyright 1989-2011 Johannes L. Braams and any individual authors
listed elsewhere in this file.  All rights reserved.

This file was generated from file(s) of the Babel system.
---------------------------------------------------------

It may be distributed and/or modified under the
conditions of the LaTeX Project Public License, either version 1.3
of this license or (at your option) any later version.
The latest version of this license is in
  http://www.latex-project.org/lppl.txt
and version 1.3 or later is part of all distributions of LaTeX
version 2003/12/01 or later.

This work has the LPPL maintenance status "maintained".

The Current Maintainer of this work is Johannes Braams.

This file may only be distributed together with a copy of the Babel
system. You may however distribute the Babel system without
such generated files.

The list of all files belonging to the Babel distribution is
given in the file `manifest.bbl'. See also `legal.bbl for additional
information.

In particular, permission is granted to customize the declarations in
this file to serve the needs of your installation.

However, NO PERMISSION is granted to distribute a modified version
of this file under its original name.

\endpreamble

\keepsilent

\usedir{tex/generic/babel} 

\usepreamble\mainpreamble
\generate{\file{frenchb.ldf}{\from{frenchb.dtx}{code}}
          }
\nopreamble
\nopostamble
\generate{\file{frenchb.cfg}{\from{frenchb.dtx}{cfg}}
          }

\ifToplevel{
\Msg{***********************************************************}
\Msg{*}
\Msg{* To finish the installation you have to move the following}
\Msg{* files into a directory searched by TeX:}
\Msg{*}
\Msg{* \space\space All *.def, *.fd, *.ldf, *.sty}
\Msg{*}
\Msg{* To produce the documentation run the files ending with}
\Msg{* '.dtx' and `.fdd' through LaTeX.}
\Msg{*}
\Msg{* Happy TeXing}
\Msg{***********************************************************}
}
 
\endinput




}
\DeclareOption{catalan}{%%
%% This file will generate fast loadable files and documentation
%% driver files from the doc files in this package when run through
%% LaTeX or TeX.
%%
%% Copyright 1989-2005 Johannes L. Braams and any individual authors
%% listed elsewhere in this file.  All rights reserved.
%% 
%% This file is part of the Babel system.
%% --------------------------------------
%% 
%% It may be distributed and/or modified under the
%% conditions of the LaTeX Project Public License, either version 1.3
%% of this license or (at your option) any later version.
%% The latest version of this license is in
%%   http://www.latex-project.org/lppl.txt
%% and version 1.3 or later is part of all distributions of LaTeX
%% version 2003/12/01 or later.
%% 
%% This work has the LPPL maintenance status "maintained".
%% 
%% The Current Maintainer of this work is Johannes Braams.
%% 
%% The list of all files belonging to the LaTeX base distribution is
%% given in the file `manifest.bbl. See also `legal.bbl' for additional
%% information.
%% 
%% The list of derived (unpacked) files belonging to the distribution
%% and covered by LPPL is defined by the unpacking scripts (with
%% extension .ins) which are part of the distribution.
%%
%% --------------- start of docstrip commands ------------------
%%
\def\filedate{1999/04/11}
\def\batchfile{catalan.ins}
\input docstrip.tex

{\ifx\generate\undefined
\Msg{**********************************************}
\Msg{*}
\Msg{* This installation requires docstrip}
\Msg{* version 2.3c or later.}
\Msg{*}
\Msg{* An older version of docstrip has been input}
\Msg{*}
\Msg{**********************************************}
\errhelp{Move or rename old docstrip.tex.}
\errmessage{Old docstrip in input path}
\batchmode
\csname @@end\endcsname
\fi}

\declarepreamble\mainpreamble
This is a generated file.

Copyright 1989-2005 Johannes L. Braams and any individual authors
listed elsewhere in this file.  All rights reserved.

This file was generated from file(s) of the Babel system.
---------------------------------------------------------

It may be distributed and/or modified under the
conditions of the LaTeX Project Public License, either version 1.3
of this license or (at your option) any later version.
The latest version of this license is in
  http://www.latex-project.org/lppl.txt
and version 1.3 or later is part of all distributions of LaTeX
version 2003/12/01 or later.

This work has the LPPL maintenance status "maintained".

The Current Maintainer of this work is Johannes Braams.

This file may only be distributed together with a copy of the Babel
system. You may however distribute the Babel system without
such generated files.

The list of all files belonging to the Babel distribution is
given in the file `manifest.bbl'. See also `legal.bbl for additional
information.

The list of derived (unpacked) files belonging to the distribution
and covered by LPPL is defined by the unpacking scripts (with
extension .ins) which are part of the distribution.
\endpreamble

\declarepreamble\fdpreamble
This is a generated file.

Copyright 1989-2005 Johannes L. Braams and any individual authors
listed elsewhere in this file.  All rights reserved.

This file was generated from file(s) of the Babel system.
---------------------------------------------------------

It may be distributed and/or modified under the
conditions of the LaTeX Project Public License, either version 1.3
of this license or (at your option) any later version.
The latest version of this license is in
  http://www.latex-project.org/lppl.txt
and version 1.3 or later is part of all distributions of LaTeX
version 2003/12/01 or later.

This work has the LPPL maintenance status "maintained".

The Current Maintainer of this work is Johannes Braams.

This file may only be distributed together with a copy of the Babel
system. You may however distribute the Babel system without
such generated files.

The list of all files belonging to the Babel distribution is
given in the file `manifest.bbl'. See also `legal.bbl for additional
information.

In particular, permission is granted to customize the declarations in
this file to serve the needs of your installation.

However, NO PERMISSION is granted to distribute a modified version
of this file under its original name.

\endpreamble

\keepsilent

\usedir{tex/generic/babel} 

\usepreamble\mainpreamble
\generate{\file{catalan.ldf}{\from{catalan.dtx}{code}}
          }
\usepreamble\fdpreamble

\ifToplevel{
\Msg{***********************************************************}
\Msg{*}
\Msg{* To finish the installation you have to move the following}
\Msg{* files into a directory searched by TeX:}
\Msg{*}
\Msg{* \space\space All *.def, *.fd, *.ldf, *.sty}
\Msg{*}
\Msg{* To produce the documentation run the files ending with}
\Msg{* '.dtx' and `.fdd' through LaTeX.}
\Msg{*}
\Msg{* Happy TeXing}
\Msg{***********************************************************}
}
 
\endinput
}
\DeclareOption{croatian}{%%
%% This file will generate fast loadable files and documentation
%% driver files from the doc files in this package when run through
%% LaTeX or TeX.
%%
%% Copyright 1989-2005 Johannes L. Braams and any individual authors
%% listed elsewhere in this file.  All rights reserved.
%% 
%% This file is part of the Babel system.
%% --------------------------------------
%% 
%% It may be distributed and/or modified under the
%% conditions of the LaTeX Project Public License, either version 1.3
%% of this license or (at your option) any later version.
%% The latest version of this license is in
%%   http://www.latex-project.org/lppl.txt
%% and version 1.3 or later is part of all distributions of LaTeX
%% version 2003/12/01 or later.
%% 
%% This work has the LPPL maintenance status "maintained".
%% 
%% The Current Maintainer of this work is Johannes Braams.
%% 
%% The list of all files belonging to the LaTeX base distribution is
%% given in the file `manifest.bbl. See also `legal.bbl' for additional
%% information.
%% 
%% The list of derived (unpacked) files belonging to the distribution
%% and covered by LPPL is defined by the unpacking scripts (with
%% extension .ins) which are part of the distribution.
%%
%% --------------- start of docstrip commands ------------------
%%
\def\filedate{1999/04/11}
\def\batchfile{croatian.ins}
\input docstrip.tex

{\ifx\generate\undefined
\Msg{**********************************************}
\Msg{*}
\Msg{* This installation requires docstrip}
\Msg{* version 2.3c or later.}
\Msg{*}
\Msg{* An older version of docstrip has been input}
\Msg{*}
\Msg{**********************************************}
\errhelp{Move or rename old docstrip.tex.}
\errmessage{Old docstrip in input path}
\batchmode
\csname @@end\endcsname
\fi}

\declarepreamble\mainpreamble
This is a generated file.

Copyright 1989-2005 Johannes L. Braams and any individual authors
listed elsewhere in this file.  All rights reserved.

This file was generated from file(s) of the Babel system.
---------------------------------------------------------

It may be distributed and/or modified under the
conditions of the LaTeX Project Public License, either version 1.3
of this license or (at your option) any later version.
The latest version of this license is in
  http://www.latex-project.org/lppl.txt
and version 1.3 or later is part of all distributions of LaTeX
version 2003/12/01 or later.

This work has the LPPL maintenance status "maintained".

The Current Maintainer of this work is Johannes Braams.

This file may only be distributed together with a copy of the Babel
system. You may however distribute the Babel system without
such generated files.

The list of all files belonging to the Babel distribution is
given in the file `manifest.bbl'. See also `legal.bbl for additional
information.

The list of derived (unpacked) files belonging to the distribution
and covered by LPPL is defined by the unpacking scripts (with
extension .ins) which are part of the distribution.
\endpreamble

\declarepreamble\fdpreamble
This is a generated file.

Copyright 1989-2005 Johannes L. Braams and any individual authors
listed elsewhere in this file.  All rights reserved.

This file was generated from file(s) of the Babel system.
---------------------------------------------------------

It may be distributed and/or modified under the
conditions of the LaTeX Project Public License, either version 1.3
of this license or (at your option) any later version.
The latest version of this license is in
  http://www.latex-project.org/lppl.txt
and version 1.3 or later is part of all distributions of LaTeX
version 2003/12/01 or later.

This work has the LPPL maintenance status "maintained".

The Current Maintainer of this work is Johannes Braams.

This file may only be distributed together with a copy of the Babel
system. You may however distribute the Babel system without
such generated files.

The list of all files belonging to the Babel distribution is
given in the file `manifest.bbl'. See also `legal.bbl for additional
information.

In particular, permission is granted to customize the declarations in
this file to serve the needs of your installation.

However, NO PERMISSION is granted to distribute a modified version
of this file under its original name.

\endpreamble

\keepsilent

\usedir{tex/generic/babel} 

\usepreamble\mainpreamble
\generate{\file{croatian.ldf}{\from{croatian.dtx}{code}}
          }
\usepreamble\fdpreamble

\ifToplevel{
\Msg{***********************************************************}
\Msg{*}
\Msg{* To finish the installation you have to move the following}
\Msg{* files into a directory searched by TeX:}
\Msg{*}
\Msg{* \space\space All *.def, *.fd, *.ldf, *.sty}
\Msg{*}
\Msg{* To produce the documentation run the files ending with}
\Msg{* '.dtx' and `.fdd' through LaTeX.}
\Msg{*}
\Msg{* Happy TeXing}
\Msg{***********************************************************}
}
 
\endinput
}
\DeclareOption{czech}{% \iffalse meta-comment
%
% Copyright 1989-2008 Johannes L. Braams and any individual authors
% listed elsewhere in this file.  All rights reserved.
% 
% This file is part of the Babel system.
% --------------------------------------
% 
% It may be distributed and/or modified under the
% conditions of the LaTeX Project Public License, either version 1.3
% of this license or (at your option) any later version.
% The latest version of this license is in
%   http://www.latex-project.org/lppl.txt
% and version 1.3 or later is part of all distributions of LaTeX
% version 2003/12/01 or later.
% 
% This work has the LPPL maintenance status "maintained".
% 
% The Current Maintainer of this work is Johannes Braams.
% 
% The list of all files belonging to the Babel system is
% given in the file `manifest.bbl. See also `legal.bbl' for additional
% information.
% 
% The list of derived (unpacked) files belonging to the distribution
% and covered by LPPL is defined by the unpacking scripts (with
% extension .ins) which are part of the distribution.
% \fi
% \CheckSum{1142}
%
% \iffalse
%    Tell the \LaTeX\ system who we are and write an entry on the
%    transcript.
%<*dtx>
\ProvidesFile{czech.dtx}
%</dtx>
%<+code>\ProvidesLanguage{czech}
%\fi
%\ProvidesFile{czech.dtx}
        [2008/07/06 v3.1a Czech support from the babel system]
%\iffalse
%% File `czech.dtx'
%% Babel package for LaTeX version 2e
%% Copyright (C) 1989 - 2008
%%           by Johannes Braams, TeXniek
%
%% Copyright (C) 2005, 2008
%%           by Petr Tesa\v r\'ik (babel at tesarici.cz)
%%
%% Czech Language Definition File
% This file is also based on CSLaTeX
%                       by Ji\v r\'i Zlatu\v ska, Zden\v ek Wagner,
%                          Jaroslav \v Snajdr and Petr Ol\v s\'ak.
%
%% Please report errors to: Petr Tesa\v r\'ik
%%                          babel at tesarici.cz
%
%    This file is part of the babel system, it provides the source
%    code for the Czech language definition file.
%<*filedriver>
\documentclass{ltxdoc}
\newcommand*\TeXhax{\TeX hax}
\newcommand*\babel{\textsf{babel}}
\newcommand*\langvar{$\langle \it lang \rangle$}
\newcommand*\note[1]{}
\newcommand*\Lopt[1]{\textsf{#1}}
\newcommand*\file[1]{\texttt{#1}}
\begin{document}
 \DocInput{czech.dtx}
\end{document}
%</filedriver>
%\fi
%
% \def\CS{$\cal C\kern-.1667em
%   \lower.5ex\hbox{$\cal S$}\kern-.075em$}
%
% \GetFileInfo{czech.dtx}
%
% \changes{czech-1.0a}{1991/07/15}{Renamed babel.sty in babel.com}
% \changes{czech-1.1}{1992/02/15}{Brought up-to-date with babel 3.2a}
% \changes{czech-1.2}{1993/07/11}{Included some features from Kasal's
%    czech.sty}
% \changes{czech-1.3}{1994/02/27}{Update for \LaTeXe}
% \changes{czech-1.3d}{1994/06/26}{Removed the use of \cs{filedate}
%    and moved identification after the loading of \file{babel.def}}
% \changes{czech-1.3h}{1996/10/10}{Replaced \cs{undefined} with
%    \cs{@undefined} and \cs{empty} with \cs{@empty} for consistency
%    with \LaTeX, moved the definition of \cs{atcatcode} right to the
%    beginning.}
% \changes{czech-3.0}{2005/09/10}{Implemented the functionality of
%    \CS\LaTeX's czech.sty.  The version number was bumped to 3.0
%    to minimize confusion by being higher than the last version
%    of \CS\LaTeX.}
%
%  \section{The Czech Language}
%
%    The file \file{\filename}\footnote{The file described in this
%    section has version number \fileversion\ and was last revised on
%    \filedate.  It was rewritten by Petr Tesa\v r\'ik
%    (\texttt{babel@tesarici.cz}).} defines all the language definition
%    macros for the Czech language.  It is meant as a replacement of
%    \CS\LaTeX, the most-widely used standard for typesetting Czech
%    documents in \LaTeX.
%
%  \subsection{Usage}
%    For this language |\frenchspacing| is set.
%
%    Additionally, two macros are defined  |\q| and |\w| for easy
%    access to two accents are defined.
%
%    The command |\q| is used with the letters (\texttt{t},
%    \texttt{d}, \texttt{l}, and \texttt{L}) and adds a \texttt{'} to
%    them to simulate a `hook' that should be there.  The result looks
%    like t\kern-2pt\char'47. The command |\w| is used to put the
%    ring-accent which appears in \aa ngstr\o m over the letters
%    \texttt{u} and \texttt{U}.
%
%  \subsection{Compatibility}
%
%    Great care has been taken to ensure backward compatibility with
%    \CS\LaTeX.  In particular, documents which load this file with
%    |\usepackage{czech}| should produce identical output with no
%    modifications to the source.  Additionally, all the \CS\LaTeX{}
%    options are recognized:
%
%    \label{tab:czech-options}
%    \begin{list}{}
%     {\def\makelabel#1{\sbox0{\Lopt{#1}}%
%        \ifdim\wd0>\labelwidth
%          \setbox0\vbox{\box0\hbox{}} \wd0=0pt \fi
%        \box0\hfil}
%      \setlength{\labelwidth}{2\parindent}
%      \setlength{\leftmargin}{2\parindent}
%      \setlength{\rightmargin}{\parindent}}
%     \item[IL2, T1, OT1]
%       These options set the default font encoding.  Please note
%       that their use is deprecated. You should use the |fontenc|
%       package to select font encoding.
%
%     \item[split, nosplit]
%       These options control whether hyphenated words are
%       automatically split according to Czech typesetting rules.
%       With the \Lopt{split} option ``je-li'' is hyphenated as
%       ``je-/-li''. The \Lopt{nosplit} option disables this behavior.
%
%       The use of this option is strongly discouraged, as it breaks
%       too many common things---hyphens cannot be used in labels,
%       negative arguments to \TeX{} primitives will not work in
%       horizontal mode (use \cs{minus} as a workaround), and there are
%       a few other peculiarities with using this mode.
%
%     \item[nocaptions]
%
%       This option was used in \CS\LaTeX{} to set up Czech/Slovak
%       typesetting rules, but leave the original captions and dates.
%       The recommended way to achieve this is to use English as the main
%       language of the document and use the environment |otherlanguage*|
%       for Czech text.
%
%     \item[olduv]
%       There are two version of \cs{uv}.  The older one allows the use
%       of \cs{verb} inside the quotes but breaks any respective kerning
%       with the quotes (like that in \CS{} fonts).  The newer one honors
%       the kerning in the font but does not allow \cs{verb} inside the
%       quotes.
%
%       The new version is used by default in \LaTeXe{} and the old version
%       is used with plain \TeX.  You may use \Lopt{olduv} to override the
%       default in \LaTeXe.
%       
%     \item[cstex]
%       This option was used to include the commands \cs{csprimeson} and
%       \cs{csprimesoff}.  Since these commands are always included now,
%       it has been removed and the empty definition lasts for compatibility.
%    \end{list}
%
% \StopEventually{}
%
%  \subsection{Implementation}
%
%    The macro |\LdfInit| takes care of preventing that this file is
%    loaded more than once, checking the category code of the
%    \texttt{@} sign, etc.
% \changes{czech-1.3h}{1996/11/02}{Now use \cs{LdfInit} to perform
%    initial checks} 
%    \begin{macrocode}
%<*code>
\LdfInit\CurrentOption{date\CurrentOption}
%    \end{macrocode}
%
%    When this file is read as an option, i.e. by the |\usepackage|
%    command, \texttt{czech} might be an `unknown' language in which
%    case we have to make it known. So we check for the existence of
%    |\l@czech| to see whether we have to do something here.
%
% \changes{czech-1.0b}{1991/10/27}{Removed use of \cs{@ifundefined}}
% \changes{czech-1.1}{1992/02/15}{Added a warning when no hyphenation
%    patterns were loaded.}
% \changes{czech-1.3d}{1994/06/26}{Now use \cs{@nopatterns} to produce
%    the warning}
%    \begin{macrocode}
\ifx\l@czech\@undefined
    \@nopatterns{Czech}
    \adddialect\l@czech0\fi
%    \end{macrocode}
%
%    We need to define these macros early in the process.
%
%    \begin{macrocode}
\def\cs@iltw@{IL2}
\newif\ifcs@splithyphens
\cs@splithyphensfalse
%    \end{macrocode}
%
%    If Babel is not loaded, we provide compatibility with \CS\LaTeX.
%    However, if macro \cs{@ifpackageloaded} is not defined, we assume
%    to be loaded from plain and provide compatibility with csplain.
%    Of course, this does not work well with \LaTeX$\:$2.09, but I
%    doubt anyone will ever want to use this file with \LaTeX$\:$2.09.
%
%    \begin{macrocode}
\ifx\@ifpackageloaded\@undefined
  \let\cs@compat@plain\relax
  \message{csplain compatibility mode}
\else
  \@ifpackageloaded{babel}{}{%
    \let\cs@compat@latex\relax
    \message{cslatex compatibility mode}}
\fi
\ifx\cs@compat@latex\relax
  \ProvidesPackage{czech}[2008/07/06 v3.1a CSTeX Czech style]
%    \end{macrocode}
%
%    Declare \CS\LaTeX{} options (see also the descriptions on page
%    \pageref{tab:czech-options}).
%
%    \begin{macrocode}
  \DeclareOption{IL2}{\def\encodingdefault{IL2}}
  \DeclareOption {T1}{\def\encodingdefault {T1}}
  \DeclareOption{OT1}{\def\encodingdefault{OT1}}
  \DeclareOption{nosplit}{\cs@splithyphensfalse}
  \DeclareOption{split}{\cs@splithyphenstrue}
  \DeclareOption{nocaptions}{\let\cs@nocaptions=\relax}
  \DeclareOption{olduv}{\let\cs@olduv=\relax}
  \DeclareOption{cstex}{\relax}
%    \end{macrocode}
%
%    Make |IL2| encoding the default.  This can be overriden with
%    the other font encoding options.
%    \begin{macrocode}
  \ExecuteOptions{\cs@iltw@}
%    \end{macrocode}
%
%    Now, process the user-supplied options.
%    \begin{macrocode}
  \ProcessOptions
%    \end{macrocode}
%
%    Standard \LaTeXe{} does not include the IL2 encoding in the format.
%    The encoding can be loaded later using the fontenc package, but
%    \CS\LaTeX{} included IL2 by default.  This means existing documents
%    for \CS\LaTeX{} do not load that package, so load the encoding
%    ourselves in compatibility mode.
%
%    \begin{macrocode}
  \ifx\encodingdefault\cs@iltw@
    \input il2enc.def
  \fi
%    \end{macrocode}
%
%    Restore the definition of \cs{CurrentOption}, clobbered by processing
%    the options.
%
%    \begin{macrocode}
  \def\CurrentOption{czech}
\fi
%    \end{macrocode}
%
%    The next step consists of defining commands to switch to (and
%    from) the Czech language.
%
%  \begin{macro}{\captionsczech}
%    The macro \cs{captionsczech} defines all strings used in the four
%    standard documentclasses provided with \LaTeX.
%
% \changes{czech-1.1}{1992/02/15}{Added \cs{seename}, \cs{alsoname}
%    and \cs{prefacename}}
% \changes{czech-1.3f}{1995/07/04}{Added \cs{proofname} for AMS-\LaTeX}
% \changes{czech-1.3g}{1996/02/12}{Fixed two errors and provided
%    translation for `proof'} 
% \changes{czech-1.3j}{2000/09/19}{Added \cs{glossaryname}}
% \changes{czech-1.3k}{2004/02/18}{Added translation for Glossary}
% \changes{czech-3.0}{2005/11/25}{Updated some translations.  Former
%    translations were: `Dodatek' for \cs{appendixname} and `Index'
%    for \cs{indexname}. Also removed spurious colon at the end of
%    \cs{ccname}.}
%
%    \begin{macrocode}
\@namedef{captions\CurrentOption}{%
  \def\prefacename{P\v{r}edmluva}%
  \def\refname{Reference}%
  \def\abstractname{Abstrakt}%
  \def\bibname{Literatura}%
  \def\chaptername{Kapitola}%
  \def\appendixname{P\v{r}\'{\i}loha}%
  \def\contentsname{Obsah}%
  \def\listfigurename{Seznam obr\'azk\r{u}}%
  \def\listtablename{Seznam tabulek}%
  \def\indexname{Rejst\v{r}\'{\i}k}%
  \def\figurename{Obr\'azek}%
  \def\tablename{Tabulka}%
  \def\partname{\v{C}\'ast}%
  \def\enclname{P\v{r}\'{\i}loha}%
  \def\ccname{Na v\v{e}dom\'{\i}}%
  \def\headtoname{Komu}%
  \def\pagename{Strana}%
  \def\seename{viz}%
  \def\alsoname{viz tak\'e}%
  \def\proofname{D\r{u}kaz}%
  \def\glossaryname{Slovn\'{\i}k}%
  }%
%    \end{macrocode}
%  \end{macro}
%
%  \begin{macro}{\dateczech}
%    The macro \cs{dateczech} redefines the command \cs{today}
%    to produce Czech dates.
%
%    \CS\LaTeX{} allows line break between the day and the month.
%    However, this behavior has been agreed upon to be a bad thing by
%    the csTeX mailing list in December 2005 and has not been adopted.
%    
% \changes{czech-1.3i}{1997/10/01}{Use \cs{edef} to define \cs{today}
%    to save memory}
% \changes{czech-1.3i}{1998/03/28}{Use \cs{def} instead of \cs{edef}}
%    \begin{macrocode}
\@namedef{date\CurrentOption}{%
  \def\today{\number\day.~\ifcase\month\or ledna\or \'unora\or
    b\v{r}ezna\or dubna\or kv\v{e}tna\or \v{c}ervna\or \v{c}ervence\or
    srpna\or z\'a\v{r}\'\i\or \v{r}\'{\i}jna\or listopadu\or
    prosince\fi \space\number\year}}
%    \end{macrocode}
%  \end{macro}
%
%  \begin{macro}{\extrasczech}
%  \begin{macro}{\noextrasczech}
%    The macro |\extrasczech| will perform all the extra definitions
%    needed for the Czech language. The macro |\noextrasczech| is used
%    to cancel the actions of |\extrasczech|.  This means saving the
%    meaning of two one-letter control sequences before defining them.
%
%    For Czech texts \cs{frenchspacing} should be in effect.  Language
%    group for shorthands is also set here.
% \changes{czech-1.3e}{1995/03/14}{now use \cs{bbl@frenchspacing} and
%    \cs{bbl@nonfrenchspacing}}
% \changes{czech-3.1}{2006/10/07}{move \cs{languageshorthands} here,
%    so that the language group is always initialized correctly}
%    \begin{macrocode}
\expandafter\addto\csname extras\CurrentOption\endcsname{%
  \bbl@frenchspacing
  \languageshorthands{czech}}
\expandafter\addto\csname noextras\CurrentOption\endcsname{%
  \bbl@nonfrenchspacing}
%    \end{macrocode}
%
% \changes{czech-1.1a}{1992/07/07}{Removed typo, \cs{q} was restored
%    twice, once too many.}
% \changes{czech-1.3e}{1995/03/15}{Use \LaTeX's \cs{v} and \cs{r}
%    accent commands}
%    \begin{macrocode}
\expandafter\addto\csname extras\CurrentOption\endcsname{%
  \babel@save\q\let\q\v
  \babel@save\w\let\w\r}
%    \end{macrocode}
%  \end{macro}
%  \end{macro}
%
%  \begin{macro}{\sq}
%  \begin{macro}{\dq}
%    We save the original single and double quote characters in
%    \cs{sq} and \cs{dq} to make them available later.
%    \begin{macrocode}
\begingroup\catcode`\"=12\catcode`\'=12
\def\x{\endgroup
  \def\sq{'}
  \def\dq{"}}
\x
%    \end{macrocode}
%  \end{macro}
%  \end{macro}
%
% \changes{czech-3.0}{2005/12/10}{Added default for setting hyphenmin
%   parameters.  Values taken from \CS\LaTeX.}
%    This macro is used to store the correct values of the hyphenation
%    parameters |\lefthyphenmin| and |\righthyphenmin|.
%    \begin{macrocode}
\providehyphenmins{\CurrentOption}{\tw@\thr@@}
%    \end{macrocode}
%
%  \begin{macro}{\v}
%    \LaTeX's normal |\v| accent places a caron over the letter that
%    follows it (\v{o}). This is not what we want for the letters d,
%    t, l and L; for those the accent should change shape. This is
%    acheived by the following.
%    \begin{macrocode}
\AtBeginDocument{%
  \DeclareTextCompositeCommand{\v}{OT1}{t}{%
    t\kern-.23em\raise.24ex\hbox{'}}
  \DeclareTextCompositeCommand{\v}{OT1}{d}{%
    d\kern-.13em\raise.24ex\hbox{'}}
  \DeclareTextCompositeCommand{\v}{OT1}{l}{\lcaron{}}
  \DeclareTextCompositeCommand{\v}{OT1}{L}{\Lcaron{}}}
%    \end{macrocode}
%
%  \begin{macro}{\lcaron}
%  \begin{macro}{\Lcaron}
%    For the letters \texttt{l} and \texttt{L} we want to disinguish
%    between normal fonts and monospaced fonts.
%    \begin{macrocode}
\def\lcaron{%
  \setbox0\hbox{M}\setbox\tw@\hbox{i}%
  \ifdim\wd0>\wd\tw@\relax
    l\kern-.13em\raise.24ex\hbox{'}\kern-.11em%
  \else
    l\raise.45ex\hbox to\z@{\kern-.35em '\hss}%
  \fi}
\def\Lcaron{%
  \setbox0\hbox{M}\setbox\tw@\hbox{i}%
  \ifdim\wd0>\wd\tw@\relax
    L\raise.24ex\hbox to\z@{\kern-.28em'\hss}%
  \else
    L\raise.45ex\hbox to\z@{\kern-.40em '\hss}%
  \fi}
%    \end{macrocode}
%  \end{macro}
%  \end{macro}
%  \end{macro}
%
%    Initialize active quotes.  \CS\LaTeX{} provides a way of
%    converting English-style quotes into Czech-style ones.  Both
%    single and double quotes are affected, i.e. |``text''| is
%    converted to something like |,,text``| and |`text'| is converted
%    to |,text`|.  This conversion can be switched on and off with
%    \cs{csprimeson} and \cs{csprimesoff}.\footnote{By the way, the
%    names of these macros are misleading, because the handling of
%    primes in math mode is rather marginal, the most important thing
%    being the handling of quotes\ldots}
%
%    These quotes present various troubles, e.g. the kerning is broken,
%    apostrophes are converted to closing single quote, some primitives
%    are broken (most notably the |\catcode`\|\meta{char} syntax will
%    not work any more), and writing them to \file{.aux} files cannot
%    be handled correctly.  For these reasons, these commands are only
%    available in \CS\LaTeX{} compatibility mode.
%
%    \begin{macrocode}
\ifx\cs@compat@latex\relax
  \let\cs@ltxprim@s\prim@s
  \def\csprimeson{%
    \catcode`\`\active \catcode`\'\active \let\prim@s\bbl@prim@s}
  \def\csprimesoff{%
    \catcode`\`12 \catcode`\'12 \let\prim@s\cs@ltxprim@s}
  \begingroup\catcode`\`\active
  \def\x{\endgroup
    \def`{\futurelet\cs@next\cs@openquote}
    \def\cs@openquote{%
      \ifx`\cs@next \expandafter\cs@opendq
      \else \expandafter\clq
      \fi}%
  }\x
  \begingroup\catcode`\'\active
  \def\x{\endgroup
    \def'{\textormath{\futurelet\cs@next\cs@closequote}
                     {^\bgroup\prim@s}}
    \def\cs@closequote{%
      \ifx'\cs@next \expandafter\cs@closedq
      \else \expandafter\crq
      \fi}%
  }\x
  \def\cs@opendq{\clqq\let\cs@next= }
  \def\cs@closedq{\crqq\let\cs@next= }
%    \end{macrocode}
%
%    The way I recommend for typesetting quotes in Czech documents
%    is to use shorthands similar to those used in German.
%    
%    \begin{macrocode}
\else
  \initiate@active@char{"}
  \expandafter\addto\csname extras\CurrentOption\endcsname{%
    \bbl@activate{"}}
  \expandafter\addto\csname noextras\CurrentOption\endcsname{%
    \bbl@deactivate{"}}
  \declare@shorthand{czech}{"`}{\clqq}
  \declare@shorthand{czech}{"'}{\crqq}
  \declare@shorthand{czech}{"<}{\flqq}
  \declare@shorthand{czech}{">}{\frqq}
  \declare@shorthand{czech}{"=}{\cs@splithyphen}
\fi
%    \end{macrocode}
%
%  \begin{macro}{\clqq}
%    This is the CS opening quote, which is similar to the German
%    quote (\cs{glqq}) but the kerning is different.
%
%    For the OT1 encoding, the quote is constructed from the right
%    double quote (i.e. the ``Opening quotes'' character) by moving
%    it down to the baseline and shifting it to the right, or to the
%    left if italic correction is positive.
%
%    For T1, the ``German Opening quotes'' is used.  It is moved to
%    the right and the total width is enlarged.  This is done in an
%    attempt to minimize the difference between the OT1 and T1
%    versions.
%
% \changes{3.0}{2006/04/20}{Added \cs{leavevmode} to allow an opening
%   quote at the beginning of a paragraph}
%    \begin{macrocode}
\ProvideTextCommand{\clqq}{OT1}{%
  \set@low@box{\textquotedblright}%
  \setbox\@ne=\hbox{l\/}\dimen\@ne=\wd\@ne
  \setbox\@ne=\hbox{l}\advance\dimen\@ne-\wd\@ne
  \leavevmode
  \ifdim\dimen\@ne>\z@\kern-.1em\box\z@\kern.1em
    \else\kern.1em\box\z@\kern-.1em\fi\allowhyphens}
\ProvideTextCommand{\clqq}{T1}
  {\kern.1em\quotedblbase\kern-.0158em\relax}
\ProvideTextCommandDefault{\clqq}{\UseTextSymbol{OT1}\clqq}
%    \end{macrocode}
%  \end{macro}
%
%  \begin{macro}{\crqq}
%    For OT1, the CS closing quote is basically the same as
%    \cs{grqq}, only the \cs{textormath} macro is not used, because
%    as far as I know, \cs{grqq} does not work in math mode anyway.
%
%    For T1, the character is slightly wider and shifted to the
%    right to match its OT1 counterpart.
%
%    \begin{macrocode}
\ProvideTextCommand{\crqq}{OT1}
  {\save@sf@q{\nobreak\kern-.07em\textquotedblleft\kern.07em}}
\ProvideTextCommand{\crqq}{T1}
  {\save@sf@q{\nobreak\kern.06em\textquotedblleft\kern.024em}}
\ProvideTextCommandDefault{\crqq}{\UseTextSymbol{OT1}\crqq}
%    \end{macrocode}
%  \end{macro}
%
%  \begin{macro}{\clq}
%  \begin{macro}{\crq}
%
%    Single CS quotes are similar to double quotes (see the
%    discussion above).
%
%    \begin{macrocode}
\ProvideTextCommand{\clq}{OT1}
  {\set@low@box{\textquoteright}\box\z@\kern.04em\allowhyphens}
\ProvideTextCommand{\clq}{T1}
  {\quotesinglbase\kern-.0428em\relax}
\ProvideTextCommandDefault{\clq}{\UseTextSymbol{OT1}\clq}
\ProvideTextCommand{\crq}{OT1}
  {\save@sf@q{\nobreak\textquoteleft\kern.17em}}
\ProvideTextCommand{\crq}{T1}
  {\save@sf@q{\nobreak\textquoteleft\kern.17em}}
\ProvideTextCommandDefault{\crq}{\UseTextSymbol{OT1}\crq}
%    \end{macrocode}
%  \end{macro}
%  \end{macro}
%
%  \begin{macro}{\uv}
%    There are two versions of \cs{uv}.  The older one opens a group
%    and uses \cs{aftergroup} to typeset the closing quotes.  This
%    version allows using \cs{verb} inside the quotes, because the
%    enclosed text is not passed as an argument, but unfortunately
%    it breaks any kerning with the quotes.  Although the kerning
%    with the opening quote could be fixed, the kerning with the
%    closing quote cannot.
%
%    The newer version is defined as a command with one parameter.
%    It preserves kerning but since the quoted text is passed as an
%    argument, it cannot contain \cs{verb}.
%
%    Decide which version of \cs{uv} should be used.  For sake
%    of compatibility, we use the older version with plain \TeX{}
%    and the newer version with \LaTeXe.
%    \begin{macrocode}
\ifx\cs@compat@plain\@undefined\else\let\cs@olduv=\relax\fi
\ifx\cs@olduv\@undefined
  \DeclareRobustCommand\uv[1]{{\leavevmode\clqq#1\crqq}}
\else
  \DeclareRobustCommand\uv{\bgroup\aftergroup\closequotes
    \leavevmode\clqq\let\cs@next=}
  \def\closequotes{\unskip\crqq\relax}
\fi
%    \end{macrocode}
%  \end{macro}
%
%
%  \begin{macro}{\cs@wordlen}
%    Declare a counter to hold the length of the word after the
%    hyphen.
%
%    \begin{macrocode}
\newcount\cs@wordlen
%    \end{macrocode}
%  \end{macro}
%
%  \begin{macro}{\cs@hyphen}
%  \begin{macro}{\cs@endash}
%  \begin{macro}{\cs@emdash}
%    Store the original hyphen in a macro. Ditto for the ligatures.
%
% \changes{czech-3.1}{2006/10/07}{ensure correct catcode for the
%    saved hyphen}
%    \begin{macrocode}
\begingroup\catcode`\-12
\def\x{\endgroup
  \def\cs@hyphen{-}
  \def\cs@endash{--}
  \def\cs@emdash{---}
%    \end{macrocode}
%  \end{macro}
%  \end{macro}
%  \end{macro}
%
%  \begin{macro}{\cs@boxhyphen}
%    Provide a non-breakable hyphen to be used when a compound word
%    is too short to be split, i.e. the second part is shorter than
%    \cs{righthyphenmin}.
%
%    \begin{macrocode}
  \def\cs@boxhyphen{\hbox{-}}
%    \end{macrocode}
%  \end{macro}
%
%  \begin{macro}{\cs@splithyphen}
%    The macro \cs{cs@splithyphen} inserts a split hyphen, while
%    allowing both parts of the compound word to be hyphenated at
%    other places too.
%
%    \begin{macrocode}
  \def\cs@splithyphen{\kern\z@
    \discretionary{-}{\char\hyphenchar\the\font}{-}\nobreak\hskip\z@}
}\x
%    \end{macrocode}
%  \end{macro}
%
%  \begin{macro}{-}
%    To minimize the effects of activating the hyphen character,
%    the active definition expands to the non-active character
%    in all cases where hyphenation cannot occur, i.e. if not
%    typesetting (check \cs{protect}), not in horizontal mode,
%    or in inner horizontal mode.
%
%    \begin{macrocode}
\initiate@active@char{-}
\declare@shorthand{czech}{-}{%
  \ifx\protect\@typeset@protect
    \ifhmode
      \ifinner
        \bbl@afterelse\bbl@afterelse\bbl@afterelse\cs@hyphen
      \else
        \bbl@afterfi\bbl@afterelse\bbl@afterelse\cs@firsthyphen
      \fi
    \else
      \bbl@afterfi\bbl@afterelse\cs@hyphen
    \fi
  \else
    \bbl@afterfi\cs@hyphen
  \fi}
%    \end{macrocode}
%  \end{macro}
%
%  \begin{macro}{\cs@firsthyphen}
%  \begin{macro}{\cs@firsthyph@n}
%  \begin{macro}{\cs@secondhyphen}
%  \begin{macro}{\cs@secondhyph@n}
%    If we encounter a hyphen, check whether it is followed
%    by a second or a third hyphen and if so, insert the
%    corresponding ligature.
%
%    If we don't find a hyphen, the token found will be placed
%    in \cs{cs@token} for further analysis, and it will also stay
%    in the input.
%
%    \begin{macrocode}
\begingroup\catcode`\-\active
\def\x{\endgroup
  \def\cs@firsthyphen{\futurelet\cs@token\cs@firsthyph@n}
  \def\cs@firsthyph@n{%
    \ifx -\cs@token
      \bbl@afterelse\cs@secondhyphen
    \else
      \bbl@afterfi\cs@checkhyphen
    \fi}
  \def\cs@secondhyphen ##1{%
    \futurelet\cs@token\cs@secondhyph@n}
  \def\cs@secondhyph@n{%
    \ifx -\cs@token
      \bbl@afterelse\cs@emdash\@gobble
    \else
      \bbl@afterfi\cs@endash
    \fi}
}\x
%    \end{macrocode}
%  \end{macro}
%  \end{macro}
%  \end{macro}
%  \end{macro}
%
%  \begin{macro}{\cs@checkhyphen}
%    Check that hyphenation is enabled, and if so, start analyzing
%    the rest of the word, i.e. initialize \cs{cs@word} and \cs{cs@wordlen}
%    and start processing input with \cs{cs@scanword}.
%
%    \begin{macrocode}
\def\cs@checkhyphen{%
  \ifnum\expandafter\hyphenchar\the\font=`\-
    \def\cs@word{}\cs@wordlen\z@
    \bbl@afterelse\cs@scanword
  \else
    \cs@hyphen
  \fi}
%    \end{macrocode}
%  \end{macro}
%
%  \begin{macro}{\cs@scanword}
%  \begin{macro}{\cs@continuescan}
%  \begin{macro}{\cs@gettoken}
%  \begin{macro}{\cs@gett@ken}
%    Each token is first analyzed with \cs{cs@scanword}, which expands
%    the token and passes the first token of the result to
%    \cs{cs@gett@ken}. If the expanded token is not identical to the
%    unexpanded one, presume that it might be expanded further and
%    pass it back to \cs{cs@scanword} until you get an unexpandable
%    token. Then analyze it in \cs{cs@examinetoken}.
%
%    The \cs{cs@continuescan} macro does the same thing as
%    \cs{cs@scanword}, but it does not require the first token to be
%    in \cs{cs@token} already.
%
%    \begin{macrocode}
\def\cs@scanword{\let\cs@lasttoken= \cs@token\expandafter\cs@gettoken}
\def\cs@continuescan{\let\cs@lasttoken\@undefined\expandafter\cs@gettoken}
\def\cs@gettoken{\futurelet\cs@token\cs@gett@ken}
\def\cs@gett@ken{%
  \ifx\cs@token\cs@lasttoken \def\cs@next{\cs@examinetoken}%
  \else \def\cs@next{\cs@scanword}%
  \fi \cs@next}
%    \end{macrocode}
%  \end{macro}
%  \end{macro}
%  \end{macro}
%  \end{macro}
%
%  \begin{macro}{cs@examinetoken}
%    Examine the token in \cs{cs@token}:
%
%    \begin{itemize}
%    \item
%      If it is a letter (catcode 11) or other (catcode 12), add it
%      to \cs{cs@word} with \cs{cs@addparam}.
%
%    \item
%      If it is the \cs{char} primitive, add it with \cs{cs@expandchar}.
%
%    \item
%      If the token starts or ends a group, ignore it with
%      \cs{cs@ignoretoken}.
%
%    \item
%      Otherwise analyze the meaning of the token with
%      \cs{cs@checkchardef} to detect primitives defined with
%      \cs{chardef}.
%
%    \end{itemize}
%
%    \begin{macrocode}
\def\cs@examinetoken{%
  \ifcat A\cs@token
    \def\cs@next{\cs@addparam}%
  \else\ifcat 0\cs@token
    \def\cs@next{\cs@addparam}%
  \else\ifx\char\cs@token
    \def\cs@next{\afterassignment\cs@expandchar\let\cs@token= }%
  \else\ifx\bgroup\cs@token
    \def\cs@next{\cs@ignoretoken\bgroup}%
  \else\ifx\egroup\cs@token
    \def\cs@next{\cs@ignoretoken\egroup}%
  \else\ifx\begingroup\cs@token
    \def\cs@next{\cs@ignoretoken\begingroup}%
  \else\ifx\endgroup\cs@token
    \def\cs@next{\cs@ignoretoken\endgroup}%
  \else
    \def\cs@next{\expandafter\expandafter\expandafter\cs@checkchardef
      \expandafter\meaning\expandafter\cs@token\string\char\end}%
  \fi\fi\fi\fi\fi\fi\fi\cs@next}
%    \end{macrocode}
%  \end{macro}
%
%  \begin{macro}{\cs@checkchardef}
%    Check the meaning of a token and if it is a primitive defined
%    with \cs{chardef}, pass it to \cs{\cs@examinechar} as if it were
%    a \cs{char} sequence. Otherwise, there are no more word characters,
%    so do the final actions in \cs{cs@nosplit}.
%
%    \begin{macrocode}
\expandafter\def\expandafter\cs@checkchardef
  \expandafter#\expandafter1\string\char#2\end{%
    \def\cs@token{#1}%
    \ifx\cs@token\@empty
      \def\cs@next{\afterassignment\cs@examinechar\let\cs@token= }%
    \else
      \def\cs@next{\cs@nosplit}%
    \fi \cs@next}
%    \end{macrocode}
%  \end{macro}
%
%  \begin{macro}{\cs@ignoretoken}
%    Add a token to \cs{cs@word} but do not update the \cs{cs@wordlen}
%    counter. This is mainly useful for group starting and ending
%    primitives, which need to be preserved, but do not affect the word
%    boundary.
%
%    \begin{macrocode}
\def\cs@ignoretoken#1{%
  \edef\cs@word{\cs@word#1}%
  \afterassignment\cs@continuescan\let\cs@token= }
%    \end{macrocode}
%  \end{macro}
%
%  \begin{macro}{cs@addparam}
%    Add a token to \cs{cs@word} and check its lccode. Note that
%    this macro can only be used for tokens which can be passed as
%    a parameter.
%
%    \begin{macrocode}
\def\cs@addparam#1{%
  \edef\cs@word{\cs@word#1}%
  \cs@checkcode{\lccode`#1}}
%    \end{macrocode}
%  \end{macro}
%
%  \begin{macro}{\cs@expandchar}
%  \begin{macro}{\cs@examinechar}
%    Add a \cs{char} sequence to \cs{cs@word} and check its lccode.
%    The charcode is first parsed in \cs{cs@expandchar} and then the
%    resulting \cs{chardef}-defined sequence is analyzed in
%    \cs{cs@examinechar}.
%
%    \begin{macrocode}
\def\cs@expandchar{\afterassignment\cs@examinechar\chardef\cs@token=}
\def\cs@examinechar{%
  \edef\cs@word{\cs@word\char\the\cs@token\space}%
  \cs@checkcode{\lccode\cs@token}}
%    \end{macrocode}
%  \end{macro}
%  \end{macro}
%
%  \begin{macro}{\cs@checkcode}
%    Check the lccode of a character. If it is zero, it does not count
%    to the current word, so finish it with \cs{cs@nosplit}. Otherwise
%    update the \cs{cs@wordlen} counter and go on scanning the word
%    with \cs{cs@continuescan}. When enough characters are gathered in
%    \cs{cs@word} to allow word break, insert the split hyphen and
%    finish.
%
%    \begin{macrocode}
\def\cs@checkcode#1{%
  \ifnum0=#1
    \def\cs@next{\cs@nosplit}%
  \else
    \advance\cs@wordlen\@ne
    \ifnum\righthyphenmin>\the\cs@wordlen
      \def\cs@next{\cs@continuescan}%
    \else
      \cs@splithyphen
      \def\cs@next{\cs@word}%
    \fi
  \fi \cs@next}
%    \end{macrocode}
%  \end{macro}
%
%  \begin{macro}{\cs@nosplit}
%    Insert a non-breakable hyphen followed by the saved word.
%
%    \begin{macrocode}
\def\cs@nosplit{\cs@boxhyphen\cs@word}
%    \end{macrocode}
%  \end{macro}
%
%  \begin{macro}{\cs@hyphen}
%    The \cs{minus} sequence can be used where the active hyphen
%    does not work, e.g. in arguments to \TeX{} primitives in outer
%    horizontal mode.
%
%    \begin{macrocode}
\let\minus\cs@hyphen
%    \end{macrocode}
%  \end{macro}

%  \begin{macro}{\standardhyphens}
%  \begin{macro}{\splithyphens}
%    These macros control whether split hyphens are allowed in Czech
%    and/or Slovak texts. You may use them in any language, but the
%    split hyphen is only activated for Czech and Slovak.
%
% \changes{czech-3.1}{2006/10/07}{activate with split hyphens and
%    deactivate with standard hyphens, not vice versa}
%    \begin{macrocode}
\def\standardhyphens{\cs@splithyphensfalse\cs@deactivatehyphens}
\def\splithyphens{\cs@splithyphenstrue\cs@activatehyphens}
%    \end{macrocode}
%  \end{macro}
%  \end{macro}
%
%  \begin{macro}{\cs@splitattr}
%    Now we declare the |split| language attribute.  This is
%    similar to the |split| package option of cslatex, but it
%    only affects Czech, not Slovak.
%
% \changes{czech-3.1}{2006/10/07}{attribute added}
%    \begin{macrocode}
\def\cs@splitattr{\babel@save\ifcs@splithyphens\splithyphens}
\bbl@declare@ttribute{czech}{split}{%
  \addto\extrasczech{\cs@splitattr}}
%    \end{macrocode}
%  \end{macro}
%
%  \begin{macro}{\cs@activatehyphens}
%  \begin{macro}{\cs@deactivatehyphens}
%    These macros are defined as \cs{relax} by default to prevent
%    activating/deactivating the hyphen character. They are redefined
%    when the language is switched to Czech/Slovak. At that moment
%    the hyphen is also activated if split hyphens were requested with
%    \cs{splithyphens}.
%
%    When the language is de-activated, de-activate the hyphen and
%    restore the bogus definitions of these macros.
%
%    \begin{macrocode}
\let\cs@activatehyphens\relax
\let\cs@deactivatehyphens\relax
\expandafter\addto\csname extras\CurrentOption\endcsname{%
  \def\cs@activatehyphens{\bbl@activate{-}}%
  \def\cs@deactivatehyphens{\bbl@deactivate{-}}%
  \ifcs@splithyphens\cs@activatehyphens\fi}
\expandafter\addto\csname noextras\CurrentOption\endcsname{%
  \cs@deactivatehyphens
  \let\cs@activatehyphens\relax
  \let\cs@deactivatehyphens\relax}
%    \end{macrocode}
%  \end{macro}
%  \end{macro}
%
%  \begin{macro}{\cs@looseness}
%  \begin{macro}{\looseness}
%    One of the most common situations where an active hyphen will not
%    work properly is the \cs{looseness} primitive. Change its definition
%    so that it deactivates the hyphen if needed.
%
%    \begin{macrocode}
\let\cs@looseness\looseness
\def\looseness{%
  \ifcs@splithyphens
    \cs@deactivatehyphens\afterassignment\cs@activatehyphens \fi
  \cs@looseness}
%    \end{macrocode}
%  \end{macro}
%  \end{macro}
%
%  \begin{macro}{\cs@selectlanguage}
%  \begin{macro}{\cs@main@language}
%    Specifying the |nocaptions| option means that captions and dates
%    are not redefined by default, but they can be switched on later
%    with \cs{captionsczech} and/or \cs{dateczech}. 
%
%    We mimic this behavior by redefining \cs{selectlanguage}.  This
%    macro is called once at the beginning of the document to set the
%    main language of the document.  If this is \cs{cs@main@language},
%    it disables the macros for setting captions and date.  In any
%    case, it restores the original definition of \cs{selectlanguage}
%    and expands it.
%
%    The definition of \cs{selectlanguage} can be shared between Czech
%    and Slovak; the actual language is stored in \cs{cs@main@language}.
%
%    \begin{macrocode}
\ifx\cs@nocaptions\@undefined\else
  \edef\cs@main@language{\CurrentOption}
  \ifx\cs@origselect\@undefined
    \let\cs@origselect=\selectlanguage
    \def\selectlanguage{%
      \let\selectlanguage\cs@origselect
      \ifx\bbl@main@language\cs@main@language
        \expandafter\cs@selectlanguage
      \else
        \expandafter\selectlanguage
      \fi}
    \def\cs@selectlanguage{%
      \cs@tempdisable{captions}%
      \cs@tempdisable{date}%
      \selectlanguage}
%    \end{macrocode}
%
%  \begin{macro}{\cs@tempdisable}
%    \cs{cs@tempdisable} disables a language setup macro temporarily,
%    i.e. the macro with the name of \meta{\#1}|\bbl@main@language|
%    just restores the original definition and purges the saved macro
%    from memory.
%
%    \begin{macrocode}
    \def\cs@tempdisable#1{%
      \def\@tempa{cs@#1}%
      \def\@tempb{#1\bbl@main@language}%
      \expandafter\expandafter\expandafter\let
        \expandafter \csname\expandafter \@tempa \expandafter\endcsname
        \csname \@tempb \endcsname
      \expandafter\edef\csname \@tempb \endcsname{%
        \let \expandafter\noexpand \csname \@tempb \endcsname
          \expandafter\noexpand \csname \@tempa \endcsname
        \let \expandafter\noexpand\csname \@tempa \endcsname
          \noexpand\@undefined}}
%    \end{macrocode}
%  \end{macro}
%
%    These macros are not needed, once the initialization is over.
%
%    \begin{macrocode}
    \@onlypreamble\cs@main@language
    \@onlypreamble\cs@origselect
    \@onlypreamble\cs@selectlanguage
    \@onlypreamble\cs@tempdisable
  \fi
\fi
%    \end{macrocode}
%  \end{macro}
%  \end{macro}
%
%    The encoding of mathematical fonts should be changed to IL2.  This
%    allows to use accented letter in some font families.  Besides,
%    documents do not use CM fonts if there are equivalents in CS-fonts,
%    so there is no need to have both bitmaps of CM-font and CS-font.
%
%    \cs{@font@warning} and \cs{@font@info} are temporarily redefined
%    to avoid annoying font warnings.
%
%    \begin{macrocode}
\ifx\cs@compat@plain\@undefined
\ifx\cs@check@enc\@undefined\else
  \def\cs@check@enc{
    \ifx\encodingdefault\cs@iltw@
      \let\cs@warn\@font@warning \let\@font@warning\@gobble
      \let\cs@info\@font@info    \let\@font@info\@gobble
      \SetSymbolFont{operators}{normal}{\cs@iltw@}{cmr}{m}{n}
      \SetSymbolFont{operators}{bold}{\cs@iltw@}{cmr}{bx}{n}
      \SetMathAlphabet\mathbf{normal}{\cs@iltw@}{cmr}{bx}{n}
      \SetMathAlphabet\mathit{normal}{\cs@iltw@}{cmr}{m}{it}
      \SetMathAlphabet\mathrm{normal}{\cs@iltw@}{cmr}{m}{n}
      \SetMathAlphabet\mathsf{normal}{\cs@iltw@}{cmss}{m}{n}
      \SetMathAlphabet\mathtt{normal}{\cs@iltw@}{cmtt}{m}{n}
      \SetMathAlphabet\mathbf{bold}{\cs@iltw@}{cmr}{bx}{n}
      \SetMathAlphabet\mathit{bold}{\cs@iltw@}{cmr}{bx}{it}
      \SetMathAlphabet\mathrm{bold}{\cs@iltw@}{cmr}{bx}{n}
      \SetMathAlphabet\mathsf{bold}{\cs@iltw@}{cmss}{bx}{n}
      \SetMathAlphabet\mathtt{bold}{\cs@iltw@}{cmtt}{m}{n}
      \let\@font@warning\cs@warn \let\cs@warn\@undefined
      \let\@font@info\cs@info    \let\cs@info\@undefined
    \fi
    \let\cs@check@enc\@undefined}
  \AtBeginDocument{\cs@check@enc}
\fi
\fi
%    \end{macrocode}
%
%  \begin{macro}{cs@undoiltw@}
%
%    The thing is that \LaTeXe{} core only supports the T1 encoding
%    and does not bother changing the uc/lc/sfcodes when encoding
%    is switched. :( However, the IL2 encoding \emph{does} change
%    these codes, so if encoding is switched back from IL2, we must
%    also undo the effect of this change to be compatible with
%    \LaTeXe.  OK, this is not the right\textsuperscript{TM} solution
%    but it works.  Cheers to Petr Ol\v s\'ak.
%
%    \begin{macrocode}
\def\cs@undoiltw@{%
  \uccode158=208 \lccode158=158 \sfcode158=1000
  \sfcode159=1000
  \uccode165=133 \lccode165=165 \sfcode165=1000
  \uccode169=137 \lccode169=169 \sfcode169=1000
  \uccode171=139 \lccode171=171 \sfcode171=1000
  \uccode174=142 \lccode174=174 \sfcode174=1000
  \uccode181=149
  \uccode185=153
  \uccode187=155
  \uccode190=0   \lccode190=0
  \uccode254=222 \lccode254=254 \sfcode254=1000
  \uccode255=223 \lccode255=255 \sfcode255=1000}
%    \end{macrocode}
%  \end{macro}
%
%  \begin{macro}{@@enc@update}
%
%    Redefine the \LaTeXe{} internal function \cs{@@enc@update} to
%    change the encodings correctly.
%
%    \begin{macrocode}
\ifx\cs@enc@update\@undefined
\ifx\@@enc@update\@undefined\else
  \let\cs@enc@update\@@enc@update
  \def\@@enc@update{\ifx\cf@encoding\cs@iltw@\cs@undoiltw@\fi
    \cs@enc@update
    \expandafter\ifnum\csname l@\languagename\endcsname=\the\language
      \expandafter\ifx
      \csname l@\languagename:\f@encoding\endcsname\relax
      \else
        \expandafter\expandafter\expandafter\let
          \expandafter\csname
          \expandafter l\expandafter @\expandafter\languagename
          \expandafter\endcsname\csname l@\languagename:\f@encoding\endcsname
      \fi
      \language=\csname l@\languagename\endcsname\relax
    \fi}
\fi\fi
%    \end{macrocode}
%  \end{macro}
%
%    The macro |\ldf@finish| takes care of looking for a
%    configuration file, setting the main language to be switched on
%    at |\begin{document}| and resetting the category code of
%    \texttt{@} to its original value.
% \changes{czech-1.3h}{1996/11/02}{Now use \cs{ldf@finish} to wrap up}
%    \begin{macrocode}
\ldf@finish\CurrentOption
%</code>
%    \end{macrocode}
%
% \Finale
%%
%% \CharacterTable
%%  {Upper-case    \A\B\C\D\E\F\G\H\I\J\K\L\M\N\O\P\Q\R\S\T\U\V\W\X\Y\Z
%%   Lower-case    \a\b\c\d\e\f\g\h\i\j\k\l\m\n\o\p\q\r\s\t\u\v\w\x\y\z
%%   Digits        \0\1\2\3\4\5\6\7\8\9
%%   Exclamation   \!     Double quote  \"     Hash (number) \#
%%   Dollar        \$     Percent       \%     Ampersand     \&
%%   Acute accent  \'     Left paren    \(     Right paren   \)
%%   Asterisk      \*     Plus          \+     Comma         \,
%%   Minus         \-     Point         \.     Solidus       \/
%%   Colon         \:     Semicolon     \;     Less than     \<
%%   Equals        \=     Greater than  \>     Question mark \?
%%   Commercial at \@     Left bracket  \[     Backslash     \\
%%   Right bracket \]     Circumflex    \^     Underscore    \_
%%   Grave accent  \`     Left brace    \{     Vertical bar  \|
%%   Right brace   \}     Tilde         \~}
%%
\endinput
}
\DeclareOption{danish}{%%
%% This file will generate fast loadable files and documentation
%% driver files from the doc files in this package when run through
%% LaTeX or TeX.
%%
%% Copyright 1989-2009 Johannes L. Braams and any individual authors
%% listed elsewhere in this file.  All rights reserved.
%% 
%% This file is part of the Babel system.
%% --------------------------------------
%% 
%% It may be distributed and/or modified under the
%% conditions of the LaTeX Project Public License, either version 1.3
%% of this license or (at your option) any later version.
%% The latest version of this license is in
%%   http://www.latex-project.org/lppl.txt
%% and version 1.3 or later is part of all distributions of LaTeX
%% version 2003/12/01 or later.
%% 
%% This work has the LPPL maintenance status "maintained".
%% 
%% The Current Maintainer of this work is Johannes Braams.
%% 
%% The list of all files belonging to the LaTeX base distribution is
%% given in the file `manifest.bbl. See also `legal.bbl' for additional
%% information.
%% 
%% The list of derived (unpacked) files belonging to the distribution
%% and covered by LPPL is defined by the unpacking scripts (with
%% extension .ins) which are part of the distribution.
%%
%% --------------- start of docstrip commands ------------------
%%
\def\filedate{1999/04/11}
\def\batchfile{danish.ins}
\input docstrip.tex

{\ifx\generate\undefined
\Msg{**********************************************}
\Msg{*}
\Msg{* This installation requires docstrip}
\Msg{* version 2.3c or later.}
\Msg{*}
\Msg{* An older version of docstrip has been input}
\Msg{*}
\Msg{**********************************************}
\errhelp{Move or rename old docstrip.tex.}
\errmessage{Old docstrip in input path}
\batchmode
\csname @@end\endcsname
\fi}

\declarepreamble\mainpreamble
This is a generated file.

Copyright 1989-2009 Johannes L. Braams and any individual authors
listed elsewhere in this file.  All rights reserved.

This file was generated from file(s) of the Babel system.
---------------------------------------------------------

It may be distributed and/or modified under the
conditions of the LaTeX Project Public License, either version 1.3
of this license or (at your option) any later version.
The latest version of this license is in
  http://www.latex-project.org/lppl.txt
and version 1.3 or later is part of all distributions of LaTeX
version 2003/12/01 or later.

This work has the LPPL maintenance status "maintained".

The Current Maintainer of this work is Johannes Braams.

This file may only be distributed together with a copy of the Babel
system. You may however distribute the Babel system without
such generated files.

The list of all files belonging to the Babel distribution is
given in the file `manifest.bbl'. See also `legal.bbl for additional
information.

The list of derived (unpacked) files belonging to the distribution
and covered by LPPL is defined by the unpacking scripts (with
extension .ins) which are part of the distribution.
\endpreamble

\declarepreamble\fdpreamble
This is a generated file.

Copyright 1989-2009 Johannes L. Braams and any individual authors
listed elsewhere in this file.  All rights reserved.

This file was generated from file(s) of the Babel system.
---------------------------------------------------------

It may be distributed and/or modified under the
conditions of the LaTeX Project Public License, either version 1.3
of this license or (at your option) any later version.
The latest version of this license is in
  http://www.latex-project.org/lppl.txt
and version 1.3 or later is part of all distributions of LaTeX
version 2003/12/01 or later.

This work has the LPPL maintenance status "maintained".

The Current Maintainer of this work is Johannes Braams.

This file may only be distributed together with a copy of the Babel
system. You may however distribute the Babel system without
such generated files.

The list of all files belonging to the Babel distribution is
given in the file `manifest.bbl'. See also `legal.bbl for additional
information.

In particular, permission is granted to customize the declarations in
this file to serve the needs of your installation.

However, NO PERMISSION is granted to distribute a modified version
of this file under its original name.

\endpreamble

\keepsilent

\usedir{tex/generic/babel} 

\usepreamble\mainpreamble
\generate{\file{danish.ldf}{\from{danish.dtx}{code}}
          }
\usepreamble\fdpreamble

\ifToplevel{
\Msg{***********************************************************}
\Msg{*}
\Msg{* To finish the installation you have to move the following}
\Msg{* files into a directory searched by TeX:}
\Msg{*}
\Msg{* \space\space All *.def, *.fd, *.ldf, *.sty}
\Msg{*}
\Msg{* To produce the documentation run the files ending with}
\Msg{* '.dtx' and `.fdd' through LaTeX.}
\Msg{*}
\Msg{* Happy TeXing}
\Msg{***********************************************************}
}
 
\endinput
}
\DeclareOption{dutch}{%%
%% This file will generate fast loadable files and documentation
%% driver files from the doc files in this package when run through
%% LaTeX or TeX.
%%
%% Copyright 1989-2005 Johannes L. Braams and any individual authors
%% listed elsewhere in this file.  All rights reserved.
%% 
%% This file is part of the Babel system.
%% --------------------------------------
%% 
%% It may be distributed and/or modified under the
%% conditions of the LaTeX Project Public License, either version 1.3
%% of this license or (at your option) any later version.
%% The latest version of this license is in
%%   http://www.latex-project.org/lppl.txt
%% and version 1.3 or later is part of all distributions of LaTeX
%% version 2003/12/01 or later.
%% 
%% This work has the LPPL maintenance status "maintained".
%% 
%% The Current Maintainer of this work is Johannes Braams.
%% 
%% The list of all files belonging to the LaTeX base distribution is
%% given in the file `manifest.bbl. See also `legal.bbl' for additional
%% information.
%% 
%% The list of derived (unpacked) files belonging to the distribution
%% and covered by LPPL is defined by the unpacking scripts (with
%% extension .ins) which are part of the distribution.
%%
%% --------------- start of docstrip commands ------------------
%%
\def\filedate{1999/04/11}
\def\batchfile{dutch.ins}
\input docstrip.tex

{\ifx\generate\undefined
\Msg{**********************************************}
\Msg{*}
\Msg{* This installation requires docstrip}
\Msg{* version 2.3c or later.}
\Msg{*}
\Msg{* An older version of docstrip has been input}
\Msg{*}
\Msg{**********************************************}
\errhelp{Move or rename old docstrip.tex.}
\errmessage{Old docstrip in input path}
\batchmode
\csname @@end\endcsname
\fi}

\declarepreamble\mainpreamble
This is a generated file.

Copyright 1989-2005 Johannes L. Braams and any individual authors
listed elsewhere in this file.  All rights reserved.

This file was generated from file(s) of the Babel system.
---------------------------------------------------------

It may be distributed and/or modified under the
conditions of the LaTeX Project Public License, either version 1.3
of this license or (at your option) any later version.
The latest version of this license is in
  http://www.latex-project.org/lppl.txt
and version 1.3 or later is part of all distributions of LaTeX
version 2003/12/01 or later.

This work has the LPPL maintenance status "maintained".

The Current Maintainer of this work is Johannes Braams.

This file may only be distributed together with a copy of the Babel
system. You may however distribute the Babel system without
such generated files.

The list of all files belonging to the Babel distribution is
given in the file `manifest.bbl'. See also `legal.bbl for additional
information.

The list of derived (unpacked) files belonging to the distribution
and covered by LPPL is defined by the unpacking scripts (with
extension .ins) which are part of the distribution.
\endpreamble

\declarepreamble\fdpreamble
This is a generated file.

Copyright 1989-2005 Johannes L. Braams and any individual authors
listed elsewhere in this file.  All rights reserved.

This file was generated from file(s) of the Babel system.
---------------------------------------------------------

It may be distributed and/or modified under the
conditions of the LaTeX Project Public License, either version 1.3
of this license or (at your option) any later version.
The latest version of this license is in
  http://www.latex-project.org/lppl.txt
and version 1.3 or later is part of all distributions of LaTeX
version 2003/12/01 or later.

This work has the LPPL maintenance status "maintained".

The Current Maintainer of this work is Johannes Braams.

This file may only be distributed together with a copy of the Babel
system. You may however distribute the Babel system without
such generated files.

The list of all files belonging to the Babel distribution is
given in the file `manifest.bbl'. See also `legal.bbl for additional
information.

In particular, permission is granted to customize the declarations in
this file to serve the needs of your installation.

However, NO PERMISSION is granted to distribute a modified version
of this file under its original name.

\endpreamble

\keepsilent

\usedir{tex/generic/babel} 

\usepreamble\mainpreamble
\generate{\file{dutch.ldf}{\from{dutch.dtx}{code}}
          }
\usepreamble\fdpreamble

\ifToplevel{
\Msg{***********************************************************}
\Msg{*}
\Msg{* To finish the installation you have to move the following}
\Msg{* files into a directory searched by TeX:}
\Msg{*}
\Msg{* \space\space All *.def, *.fd, *.ldf, *.sty}
\Msg{*}
\Msg{* To produce the documentation run the files ending with}
\Msg{* '.dtx' and `.fdd' through LaTeX.}
\Msg{*}
\Msg{* Happy TeXing}
\Msg{***********************************************************}
}
 
\endinput
}
%    \end{macrocode}
% \changes{babel~3.5b}{1995/06/06}{Added the \Lopt{estonian} option}
%    \begin{macrocode}
\DeclareOption{english}{%%
%% This file will generate fast loadable files and documentation
%% driver files from the doc files in this package when run through
%% LaTeX or TeX.
%%
%% Copyright 1989-2005 Johannes L. Braams and any individual authors
%% listed elsewhere in this file.  All rights reserved.
%% 
%% This file is part of the Babel system.
%% --------------------------------------
%% 
%% It may be distributed and/or modified under the
%% conditions of the LaTeX Project Public License, either version 1.3
%% of this license or (at your option) any later version.
%% The latest version of this license is in
%%   http://www.latex-project.org/lppl.txt
%% and version 1.3 or later is part of all distributions of LaTeX
%% version 2003/12/01 or later.
%% 
%% This work has the LPPL maintenance status "maintained".
%% 
%% The Current Maintainer of this work is Johannes Braams.
%% 
%% The list of all files belonging to the LaTeX base distribution is
%% given in the file `manifest.bbl. See also `legal.bbl' for additional
%% information.
%% 
%% The list of derived (unpacked) files belonging to the distribution
%% and covered by LPPL is defined by the unpacking scripts (with
%% extension .ins) which are part of the distribution.
%%
%% --------------- start of docstrip commands ------------------
%%
\def\filedate{1999/04/11}
\def\batchfile{english.ins}
\input docstrip.tex

{\ifx\generate\undefined
\Msg{**********************************************}
\Msg{*}
\Msg{* This installation requires docstrip}
\Msg{* version 2.3c or later.}
\Msg{*}
\Msg{* An older version of docstrip has been input}
\Msg{*}
\Msg{**********************************************}
\errhelp{Move or rename old docstrip.tex.}
\errmessage{Old docstrip in input path}
\batchmode
\csname @@end\endcsname
\fi}

\declarepreamble\mainpreamble
This is a generated file.

Copyright 1989-2005 Johannes L. Braams and any individual authors
listed elsewhere in this file.  All rights reserved.

This file was generated from file(s) of the Babel system.
---------------------------------------------------------

It may be distributed and/or modified under the
conditions of the LaTeX Project Public License, either version 1.3
of this license or (at your option) any later version.
The latest version of this license is in
  http://www.latex-project.org/lppl.txt
and version 1.3 or later is part of all distributions of LaTeX
version 2003/12/01 or later.

This work has the LPPL maintenance status "maintained".

The Current Maintainer of this work is Johannes Braams.

This file may only be distributed together with a copy of the Babel
system. You may however distribute the Babel system without
such generated files.

The list of all files belonging to the Babel distribution is
given in the file `manifest.bbl'. See also `legal.bbl for additional
information.

The list of derived (unpacked) files belonging to the distribution
and covered by LPPL is defined by the unpacking scripts (with
extension .ins) which are part of the distribution.
\endpreamble

\declarepreamble\fdpreamble
This is a generated file.

Copyright 1989-2005 Johannes L. Braams and any individual authors
listed elsewhere in this file.  All rights reserved.

This file was generated from file(s) of the Babel system.
---------------------------------------------------------

It may be distributed and/or modified under the
conditions of the LaTeX Project Public License, either version 1.3
of this license or (at your option) any later version.
The latest version of this license is in
  http://www.latex-project.org/lppl.txt
and version 1.3 or later is part of all distributions of LaTeX
version 2003/12/01 or later.

This work has the LPPL maintenance status "maintained".

The Current Maintainer of this work is Johannes Braams.

This file may only be distributed together with a copy of the Babel
system. You may however distribute the Babel system without
such generated files.

The list of all files belonging to the Babel distribution is
given in the file `manifest.bbl'. See also `legal.bbl for additional
information.

In particular, permission is granted to customize the declarations in
this file to serve the needs of your installation.

However, NO PERMISSION is granted to distribute a modified version
of this file under its original name.

\endpreamble

\keepsilent

\usedir{tex/generic/babel} 

\usepreamble\mainpreamble
\generate{\file{english.ldf}{\from{english.dtx}{code}}
          }
\usepreamble\fdpreamble

\ifToplevel{
\Msg{***********************************************************}
\Msg{*}
\Msg{* To finish the installation you have to move the following}
\Msg{* files into a directory searched by TeX:}
\Msg{*}
\Msg{* \space\space All *.def, *.fd, *.ldf, *.sty}
\Msg{*}
\Msg{* To produce the documentation run the files ending with}
\Msg{* '.dtx' and `.fdd' through LaTeX.}
\Msg{*}
\Msg{* Happy TeXing}
\Msg{***********************************************************}
}
 
\endinput
}
\DeclareOption{esperanto}{% \iffalse meta-comment
%
% Copyright 1989-2007 Johannes L. Braams and any individual authors
% listed elsewhere in this file.  All rights reserved.
% 
% This file is part of the Babel system.
% --------------------------------------
% 
% It may be distributed and/or modified under the
% conditions of the LaTeX Project Public License, either version 1.3
% of this license or (at your option) any later version.
% The latest version of this license is in
%   http://www.latex-project.org/lppl.txt
% and version 1.3 or later is part of all distributions of LaTeX
% version 2003/12/01 or later.
% 
% This work has the LPPL maintenance status "maintained".
% 
% The Current Maintainer of this work is Johannes Braams.
% 
% The list of all files belonging to the Babel system is
% given in the file `manifest.bbl. See also `legal.bbl' for additional
% information.
% 
% The list of derived (unpacked) files belonging to the distribution
% and covered by LPPL is defined by the unpacking scripts (with
% extension .ins) which are part of the distribution.
% \fi
% \CheckSum{261}
%\iffalse
%    Tell the \LaTeX\ system who we are and write an entry on the
%    transcript.
%<*dtx>
\ProvidesFile{esperanto.dtx}
%</dtx>
%<code>\ProvidesLanguage{esperanto}
%\fi
%\ProvidesFile{esperant.dtx}
        [2007/10/20 v1.4t Esperanto support from the babel system]
%\iffalse
%% File 'esperanto.dtx'
%% Babel package for LaTeX version 2e
%% Copyright (C) 1989 - 2007
%%           by Johannes Braams, TeXniek
%
%% Please report errors to: J.L. Braams
%%                          babel at braams dot xs4all dot nl
%
%    This file is part of the babel system, it provides the source
%    code for the Esperanto language definition file.  A contribution
%    was made by Ruiz-Altaba Marti (ruizaltb@cernvm.cern.ch) Code from
%    esperant.sty version 1.1 by Joerg Knappen
%    (\texttt{knappen@vkpmzd.kph.uni-mainz.de}) was included in
%    version 1.2.
%<*filedriver>
\documentclass{ltxdoc}
\newcommand*\TeXhax{\TeX hax}
\newcommand*\babel{\textsf{babel}}
\newcommand*\langvar{$\langle \it lang \rangle$}
\newcommand*\note[1]{}
\newcommand*\Lopt[1]{\textsf{#1}}
\newcommand*\file[1]{\texttt{#1}}
\begin{document}
 \DocInput{esperanto.dtx}
\end{document}
%</filedriver>
%\fi
% \GetFileInfo{esperanto.dtx}
%
% \changes{esperanto-1.0a}{1991/07/15}{Renamed \file{babel.sty} in
%    \file{babel.com}}
% \changes{esperanto-1.1}{1992/02/15}{Brought up-to-date with
%    babel~3.2a}
% \changes{esperanto-1.2}{1992/02/18}{Included code from
%    \texttt{esperant.sty}}
% \changes{esperanto-1.4a}{1994/02/04}{Updated for \LaTeXe}
% \changes{esperanto-1.4d}{1994/06/25}{Removed the use of
%    \cs{filedate}, moved Identification after loading of
%    \file{babel.def}}
% \changes{esperanto-1.4e}{1995/02/09}{Moved identification code to
%    the top of the file}
% \changes{esperanto-1.4f}{1995/06/14}{Corrected typos (PR1652)}
% \changes{esperanto-1.4i}{1996/07/10}{Replaced \cs{undefined} with
%    \cs{@undefined} and \cs{empty} with \cs{@empty} for consistency
%    with \LaTeX} 
% \changes{esperanto-1.4i}{1996/10/10}{Moved the definition of
%    \cs{atcatcode} right to the beginning.}
%
%  \section{The Esperanto language}
%
%    The file \file{\filename}\footnote{The file described in this
%    section has version number \fileversion\ and was last revised on
%    \filedate. A contribution was made by Ruiz-Altaba Marti
%    (\texttt{ruizaltb@cernvm.cern.ch}). Code from the file
%    \texttt{esperant.sty} by J\"org Knappen
%    (\texttt{knappen@vkpmzd.kph.uni-mainz.de}) was included.} defines
%    all the language-specific macros for the Esperanto language.
%
%    For this language the character |^| is made active.
%    In table~\ref{tab:esp-act} an overview is given of its purpose.
% \changes{esperanto-1.4j}{1997/01/06}{fixed typo in table caption
%    (funtion instead of function)} 
% \begin{table}[htb]
%    \centering
%     \begin{tabular}{lp{8cm}}
%      |^c| & gives \^c with hyphenation in the rest of the word
%             allowed, this works for c, C, g, G, H, J, s, S, z, Z\\
%      |^h| & prevents h\llap{\^{}} from becoming too tall\\
%      |^j| & gives \^\j\\
%      |^u| & gives \u u, with hyphenation in the rest of the word
%                   allowed\\
%      |^U| & gives \u U, with hyphenation in the rest of the word
%                   allowed\\
%      \verb=^|= & inserts a |\discretionary{-}{}{}|\\
%      \end{tabular}
%      \caption{The functions of the active character for Esperanto.}
%    \label{tab:esp-act}
% \end{table}
%
%  \StopEventually{}
%
%    The macro |\LdfInit| takes care of preventing that this file is
%    loaded more than once, checking the category code of the
%    \texttt{@} sign, etc.
% \changes{esperanto-1.4i}{1996/11/02}{Now use \cs{LdfInit} to perform
%    initial checks}
%    \begin{macrocode}
%<*code>
\LdfInit{esperanto}\captionsesperanto
%    \end{macrocode}
%
%    When this file is read as an option, i.e. by the |\usepackage|
%    command, \texttt{esperanto} will be an `unknown' language in
%    which case we have to make it known. So we check for the
%    existence of |\l@esperanto| to see whether we have to do
%    something here.
%
% \changes{esperanto-1.0b}{1991/10/29}{Removed use of
%    \cs{makeatletter}}
% \changes{esperanto-1.1}{1992/02/15}{Added a warning when no
%    hyphenation patterns were loaded.}
% \changes{esperanto-1.4d}{1994/06/25}{Use \cs{@nopatterns} for the
%    warning}
%    \begin{macrocode}
\ifx\l@esperanto\@undefined
  \@nopatterns{Esperanto}
  \adddialect\l@esperanto0\fi
%    \end{macrocode}
%
%    The next step consists of defining commands to switch to the
%    Esperanto language. The reason for this is that a user might want
%    to switch back and forth between languages.
%
% \begin{macro}{\captionsesperanto}
%    The macro |\captionsesperanto| defines all strings used
%    in the four standard documentclasses provided with \LaTeX.
% \changes{esperanto-1.1}{1992/02/15}{Added \cs{seename},
%    \cs{alsoname} and \cs{prefacename}}
% \changes{esperanto-1.3}{1993/07/10}{Repaired a number of mistakes,
%    indicated by D. Ederveen}
% \changes{esperanto-1.3}{1993/07/15}{\cs{headpagename} should be
%    \cs{pagename}}
% \changes{esperanto-1.4a}{1994/02/04}{added missing closing brace}
% \changes{esperanto-1.4g}{1995/07/04}{Added \cs{proofname} for
%    AMS-\LaTeX}
% \changes{esperanto-1.4i}{1996/07/06}{Replaced `Proof' by `Pruvo' 
%    PR 2207} 
% \changes{esperanto-1.4p}{2000/09/19}{Added \cs{glossaryname}}
% \changes{esperanto-1.4q}{2002/01/07}{Added translation for Glossary}
%    \begin{macrocode}
\addto\captionsesperanto{%
  \def\prefacename{Anta\u{u}parolo}%
  \def\refname{Cita\^\j{}oj}%
  \def\abstractname{Resumo}%
  \def\bibname{Bibliografio}%
  \def\chaptername{{\^C}apitro}%
  \def\appendixname{Apendico}%
  \def\contentsname{Enhavo}%
  \def\listfigurename{Listo de figuroj}%
  \def\listtablename{Listo de tabeloj}%
  \def\indexname{Indekso}%
  \def\figurename{Figuro}%
  \def\tablename{Tabelo}%
  \def\partname{Parto}%
  \def\enclname{Aldono(j)}%
  \def\ccname{Kopie al}%
  \def\headtoname{Al}%
  \def\pagename{Pa\^go}%
  \def\subjectname{Temo}%
  \def\seename{vidu}%   a^u: vd.
  \def\alsoname{vidu anka\u{u}}% a^u vd. anka\u{u}
  \def\proofname{Pruvo}%
  \def\glossaryname{Glosaro}%
  }
%    \end{macrocode}
% \end{macro}
%
% \begin{macro}{\dateesperanto}
%    The macro |\dateesperanto| redefines the command |\today| to
%    produce Esperanto dates.
% \changes{esperanto-1.3}{1993/07/10}{Removed the capitals from
%    \cs{today}}
% \changes{esperanto-1.4k}{1997/10/01}{Use \cs{edef} to define
%    \cs{today} to save memory}
% \changes{esperanto-1.4k}{1998/03/27}{Removed Rthe use of \cs{edef}
%    again} 
%    \begin{macrocode}
\def\dateesperanto{%
  \def\today{\number\day{--a}~de~\ifcase\month\or
    januaro\or februaro\or marto\or aprilo\or majo\or junio\or
    julio\or a\u{u}gusto\or septembro\or oktobro\or novembro\or
    decembro\fi,\space \number\year}}
%    \end{macrocode}
% \end{macro}
%
% \begin{macro}{\extrasesperanto}
% \begin{macro}{\noextrasesperanto}
%    The macro |\extrasesperanto| performs all the extra definitions
%    needed for the Esperanto language. The macro |\noextrasesperanto|
%    is used to cancel the actions of |\extrasesperanto|.
%
%    For Esperanto the |^| character is made active. This is done
%    once, later on its definition may vary.
%
%    \begin{macrocode}
\initiate@active@char{^}
%    \end{macrocode}
%    Because the character |^| is used in math mode with quite a
%    different purpose we need to add an extra level of evaluation to
%    the definition of the active |^|. It checks whether math mode is
%    active; if so the shorthand mechanism is bypassed by a direct
%    call of |\normal@char^|.
% \changes{esperanto-1.4n}{1999/09/30}{Added a check for math mode to
%    the definition of the shorthand character}
% \changes{esperant0-1.4o}{1999/12/18}{Moved the check for math to
%    babel.def}
%    \begin{macrocode}
\addto\extrasesperanto{\languageshorthands{esperanto}}
\addto\extrasesperanto{\bbl@activate{^}}
\addto\noextrasesperanto{\bbl@deactivate{^}}
%    \end{macrocode}
% \end{macro}
% \end{macro}
%
%    In order to prevent problems with the active |^| we add a
%    shorthand on system level which expands to a `normal |^|.
% \changes{esperanto-1.4l}{1999/04/11}{Added a shorthand definition on
%    system level}
%    \begin{macrocode}
\declare@shorthand{system}{^}{\csname normal@char\string^\endcsname}
%    \end{macrocode}
%    And here are the uses of the active |^|:
% \changes{esperanto-1.4h}{1995/07/27}{Added a few shorthands}
%    \begin{macrocode}
\declare@shorthand{esperanto}{^c}{\^{c}\allowhyphens}
\declare@shorthand{esperanto}{^C}{\^{C}\allowhyphens}
\declare@shorthand{esperanto}{^g}{\^{g}\allowhyphens}
\declare@shorthand{esperanto}{^G}{\^{G}\allowhyphens}
\declare@shorthand{esperanto}{^h}{h\llap{\^{}}\allowhyphens}
\declare@shorthand{esperanto}{^H}{\^{H}\allowhyphens}
\declare@shorthand{esperanto}{^j}{\^{\j}\allowhyphens}
\declare@shorthand{esperanto}{^J}{\^{J}\allowhyphens}
\declare@shorthand{esperanto}{^s}{\^{s}\allowhyphens}
\declare@shorthand{esperanto}{^S}{\^{S}\allowhyphens}
\declare@shorthand{esperanto}{^u}{\u u\allowhyphens}
\declare@shorthand{esperanto}{^U}{\u U\allowhyphens}
\declare@shorthand{esperanto}{^|}{\discretionary{-}{}{}\allowhyphens}
%    \end{macrocode}
%
% \begin{macro}{\Esper}
% \begin{macro}{\esper}
%    In \file{esperant.sty} J\"org Knappen provides the macros
%    |\esper| and |\Esper| that can be used instead of |\alph| and
%    |\Alph|. These macros are available in this file as well.
%
%    Their definition takes place in two steps. First the toplevel.
%    \begin{macrocode}
\def\esper#1{\@esper{\@nameuse{c@#1}}}
\def\Esper#1{\@Esper{\@nameuse{c@#1}}}
%    \end{macrocode}
%    Then the second level.
% \changes{esperanto-1.4q}{2002/01/08}{Removed the extra level of
%    expansion for more than five items, as was done in \LaTeX}
% \changes{esperanto-1.4t}{2006/06/05}{Added the missing `r' in these
%    macros} 
%    \begin{macrocode}
\def\@esper#1{%
  \ifcase#1\or a\or b\or c\or \^c\or d\or e\or f\or g\or \^g\or
    h\or h\llap{\^{}}\or i\or j\or \^\j\or k\or l\or m\or n\or o\or
    p\or r\or s\or \^s\or t\or u\or \u{u}\or v\or z\else\@ctrerr\fi}
\def\@Esper#1{%
  \ifcase#1\or A\or B\or C\or \^C\or D\or E\or F\or G\or \^G\or
    H\or \^H\or I\or J\or \^J\or K\or L\or M\or N\or O\or
    P\or R\or S\or \^S\or T\or U\or \u{U}\or V\or Z\else\@ctrerr\fi}
%    \end{macrocode}
% \end{macro}
% \end{macro}
%
% \begin{macro}{\hodiau}
% \begin{macro}{\hodiaun}
%    In \file{esperant.sty} J\"org Knappen provides two alternative
%    macros for |\today|, |\hodiau| and |\hodiaun|. The second macro
%    produces an accusative version of the date in Esperanto.
%    \begin{macrocode}
\addto\dateesperanto{\def\hodiau{la \today}}
\def\hodiaun{la \number\day --an~de~\ifcase\month\or
  januaro\or februaro\or marto\or aprilo\or majo\or junio\or
  julio\or a\u{u}gusto\or septembro\or oktobro\or novembro\or
  decembro\fi, \space \number\year}
%    \end{macrocode}
% \end{macro}
% \end{macro}
%
%    The macro |\ldf@finish| takes care of looking for a
%    configuration file, setting the main language to be switched on
%    at |\begin{document}| and resetting the category code of
%    \texttt{@} to its original value.
% \changes{esperanto-1.4i}{1996/11/02}{Now use \cs{ldf@finish} to wrap
%    up} 
%    \begin{macrocode}
\ldf@finish{esperanto}
%</code>
%    \end{macrocode}
%
% \Finale
%%
%% \CharacterTable
%%  {Upper-case    \A\B\C\D\E\F\G\H\I\J\K\L\M\N\O\P\Q\R\S\T\U\V\W\X\Y\Z
%%   Lower-case    \a\b\c\d\e\f\g\h\i\j\k\l\m\n\o\p\q\r\s\t\u\v\w\x\y\z
%%   Digits        \0\1\2\3\4\5\6\7\8\9
%%   Exclamation   \!     Double quote  \"     Hash (number) \#
%%   Dollar        \$     Percent       \%     Ampersand     \&
%%   Acute accent  \'     Left paren    \(     Right paren   \)
%%   Asterisk      \*     Plus          \+     Comma         \,
%%   Minus         \-     Point         \.     Solidus       \/
%%   Colon         \:     Semicolon     \;     Less than     \<
%%   Equals        \=     Greater than  \>     Question mark \?
%%   Commercial at \@     Left bracket  \[     Backslash     \\
%%   Right bracket \]     Circumflex    \^     Underscore    \_
%%   Grave accent  \`     Left brace    \{     Vertical bar  \|
%%   Right brace   \}     Tilde         \~}
%%
\endinput
}
\DeclareOption{estonian}{% \iffalse meta-comment
%
% Copyright 1989-2008 Johannes L. Braams and any individual authors
% listed elsewhere in this file.  All rights reserved.
% 
% This file is part of the Babel system.
% --------------------------------------
% 
% It may be distributed and/or modified under the
% conditions of the LaTeX Project Public License, either version 1.3
% of this license or (at your option) any later version.
% The latest version of this license is in
%   http://www.latex-project.org/lppl.txt
% and version 1.3 or later is part of all distributions of LaTeX
% version 2003/12/01 or later.
% 
% This work has the LPPL maintenance status "maintained".
% 
% The Current Maintainer of this work is Johannes Braams.
% 
% The list of all files belonging to the Babel system is
% given in the file `manifest.bbl. See also `legal.bbl' for additional
% information.
% 
% The list of derived (unpacked) files belonging to the distribution
% and covered by LPPL is defined by the unpacking scripts (with
% extension .ins) which are part of the distribution.
% \fi
% \CheckSum{330}
% \iffalse
%    Tell the \LaTeX\ system who we are and write an entry on the
%    transcript.
%<*dtx>
\ProvidesFile{estonian.dtx}
%</dtx>
%<code>\ProvidesLanguage{estonian}
%\fi
%\ProvidesFile{estonian.dtx}
        [2008/07/07 v1.0j Estonian support from the babel system]
%\iffalse
%% File `estonian.dtx'
%% Babel package for LaTeX version 2e
%% Copyright (C) 1989 - 2005
%%           by Johannes Braams, TeXniek
%
%% Estonian language Definition File
%% Copyright (C) 1991 - 2005
%%           by Enn Saar, Tartu Astrophysical Observatory
%              Tartu Astrophysical Observatory
%              EE-2444 T\~oravere
%              Estonia
%              tel: +372 7 410 267
%              fax: +372 7 410 205
%              saar@aai.ee
%
%              Johannes Braams, TeXniek
%
%% Please report errors to: Enn Saar saar at aai.ee
%%                          (or J.L. Braams babel atbraams.cistron.nl
%
%    This file is part of the babel system, it provides the source
%    code for the Estonian language definition file.  The original
%    version of this file was written by Enn Saar,
%    (saar@aai.ee).
%<*filedriver>
\documentclass{ltxdoc}
\newcommand*\TeXhax{\TeX hax}
\newcommand*\babel{\textsf{babel}}
\newcommand*\langvar{$\langle \it lang \rangle$}
\newcommand*\note[1]{}
\newcommand*\Lopt[1]{\textsf{#1}}
\newcommand*\file[1]{\texttt{#1}}
\begin{document}
 \DocInput{estonian.dtx}
\end{document}
%</filedriver>
%\fi
%
% \changes{estonian-1.0b}{1995/06/16}{corrected typos}
% \changes{estonian-1.0e}{1996/10/10}{Replaced \cs{undefined} with
%    \cs{@undefined} and \cs{empty} with \cs{@empty} for consistency
%    with \LaTeX, moved the definition of \cs{atcatcode} right to the
%    beginning.}
%
% \GetFileInfo{estonian.dtx}
%
% \section{The Estonian language}
%
%    The file \file{\filename}\footnote{The file described in this
%    section has version number \fileversion\ and was last revised on
%    \filedate. The original author is Enn Saar,
%    (\texttt{saar@aai.ee}).}  defines the language definition macro's
%    for the Estonian language.
%
%    This file was written as part of the TWGML project, and borrows
%    heavily from the \babel\ German and Spanish language files
%    \file{germanb.ldf} and \file{spanish.ldf}.
%
%    Estonian has the same umlauts as German (\"a, \"o, \"u), but in
%    addition to this, we have also \~o, and two recent characters
%    \v s and \v z, so we need at least two active characters.
%    We shall use |"| and |~| to type Estonian accents on ASCII
%    keyboards (in the 7-bit character world). Their use is given in
%    table~\ref{tab:estonian-quote}.
%    \begin{table}[htb]
%     \begin{center}
%     \begin{tabular}{lp{8cm}}
%      |~o| & |\~o|, (and uppercase); \\
%      |"a| & |\"a|, (and uppercase); \\
%      |"o| & |\"o|, (and uppercase); \\
%      |"u| & |\"u|, (and uppercase); \\
%      |~s| & |\v s|, (and uppercase); \\
%      |~z| & |\v z|, (and uppercase); \\
%      \verb="|= & disable ligature at this position;\\
%      |"-| & an explicit hyphen sign, allowing hyphenation
%                  in the rest of the word;\\
%      |\-| & like the old |\-|, but allowing hyphenation
%             in the rest of the word; \\
%      |"`| & for Estonian low left double quotes (same as German);\\
%      |"'| & for Estonian right double quotes;\\
%      |"<| & for French left double quotes (also rather popular)\\
%      |">| & for French right double quotes.\\
%     \end{tabular}
%     \caption{The extra definitions made
%              by \file{estonian.ldf}}\label{tab:estonian-quote}
%     \end{center}
%    \end{table}
%    These active accent characters behave according to their original
%    definitions if not followed by one of the characters indicated in
%    that table; the original quote character can be typed using the
%    macro |\dq|.
%
%    We support also the T1 output encoding (and Cork-encoded text
%    input).  You can choose the T1 encoding by the command
%    |\usepackage[T1]{fontenc}|.  This package must be loaded before
%    \babel. As the standard Estonian hyphenation file
%    \file{eehyph.tex} is in the Cork encoding, choosing this encoding
%    will give you better hyphenation.
%
%    As mentioned in the Spanish style file, it may happen that some
%    packages fail (usually in a \cs{message}). In this case you
%    should change the order of the \cs{usepackage} declarations
%    or the order of the style options in \cs{documentclass}.
%
% \StopEventually{}
%
% \subsection{Implementation}
%
%    The macro |\LdfInit| takes care of preventing that this file is
%    loaded more than once, checking the category code of the
%    \texttt{@} sign, etc.
% \changes{estonian-1.0e}{1996/10/30}{Now use \cs{LdfInit} to perform
%    initial checks} 
%    \begin{macrocode}
%<*code>
\LdfInit{estonian}\captionsestonian
%    \end{macrocode}
%
%    If Estonian is not included in the format file (does not have
%    hyphenation patterns), we shall use English hyphenation.
%
%    \begin{macrocode}
\ifx\l@estonian\@undefined
  \@nopatterns{Estonian}
  \adddialect\l@estonian0
\fi
%    \end{macrocode}
%
%    Now come the commands to switch to (and from) Estonian.
%
%  \begin{macro}{\captionsestonian}
%    The macro |\captionsestonian| defines all strings used in the
%    four standard documentclasses provided with \LaTeX.
%
% \changes{estonian-1.0c}{1995/07/04}{Added \cs{proofname} for
%    AMS-\LaTeX}
% \changes{estonian-1.0d}{1995/07/27}{Added translation of `Proof'}
% \changes{estonian-1.0h}{2000/09/20}{Added \cs{glossaryname}}
% \changes{estonian-1.0j}{2008/07/07}{Replaced the translation of
%    `Proof'}
%    \begin{macrocode}
\addto\captionsestonian{%
  \def\prefacename{Sissejuhatus}%
  \def\refname{Viited}%
  \def\bibname{Kirjandus}%
  \def\appendixname{Lisa}%
  \def\contentsname{Sisukord}%
  \def\listfigurename{Joonised}%
  \def\listtablename{Tabelid}%
  \def\indexname{Indeks}%
  \def\figurename{Joonis}%
  \def\tablename{Tabel}%
  \def\partname{Osa}%
  \def\enclname{Lisa(d)}%
  \def\ccname{Koopia(d)}%
  \def\headtoname{}%
  \def\pagename{Lk.}%
  \def\seename{vt.}%
  \def\alsoname{vt. ka}%
  \def\proofname{T~oestus}%
  \def\glossaryname{Glossary}% <-- Needs translation
  }
%    \end{macrocode}
%
%    These captions contain accented characters.
%
%    \begin{macrocode}
\begingroup \catcode`\"\active
\def\x{\endgroup
\addto\captionsestonian{%
  \def\abstractname{Kokkuv~ote}%
  \def\chaptername{Peat"ukk}}}
\x
%    \end{macrocode}
%  \end{macro}
%
%  \begin{macro}{\dateestonian}
%    The macro |\dateestonian| redefines the command |\today| to
%    produce Estonian dates.
% \changes{estonian-1.0f}{1997/10/01}{Use \cs{edef} to define
%    \cs{today} to save memory}
% \changes{estonian-1.0f}{1998/03/28}{use \cs{def} instead of
%    \cs{edef}} 
%    \begin{macrocode}
\begingroup \catcode`\"\active
\def\x{\endgroup
  \def\month@estonian{\ifcase\month\or
    jaanuar\or veebruar\or m"arts\or aprill\or mai\or juuni\or
    juuli\or august\or september\or oktoober\or november\or
    detsember\fi}}
\x
\def\dateestonian{%
  \def\today{\number\day.\space\month@estonian
    \space\number\year.\space a.}}
%    \end{macrocode}
%  \end{macro}
%
%  \begin{macro}{\extrasestonian}
%  \begin{macro}{\noextrasestonian}
%    The macro |\extrasestonian| will perform all the extra
%    definitions needed for Estonian. The macro |\noextrasestonian| is
%    used to cancel the actions of |\extrasestonian|. For Estonian,
%    |"| is made active and has to be treated as `special' (|~| is
%    active already).
%
%    \begin{macrocode}
\initiate@active@char{"}
\initiate@active@char{~}
\addto\extrasestonian{\languageshorthands{estonian}}
\addto\extrasestonian{\bbl@activate{"}\bbl@activate{~}}
%    \end{macrocode}
%    Store the original macros, and redefine accents.
%
%    \begin{macrocode}
\addto\extrasestonian{\babel@save\"\umlautlow\babel@save\~\tildelow}
%    \end{macrocode}
%
% \changes{estonian-1.0d}{1995/07/27}{Removed the code that changes
%    category, lower case, uper case and space factor codes }
%
%    Estonian does not use extra spaces after sentences.
%
%    \begin{macrocode}
\addto\extrasestonian{\bbl@frenchspacing}
\addto\noextrasestonian{\bbl@nonfrenchspacing}
%    \end{macrocode}
%  \end{macro}
%  \end{macro}
%
%  \begin{macro}{\estonianhyphenmins}
%     For Estonian, |\lefthyphenmin| and |\righthyphenmin| are
%    both~2. 
% \changes{estonian-1.0h}{2000/09/22}{Now use \cs{providehyphenmins} to
%    provide a default value}
%    \begin{macrocode}
\providehyphenmins{\CurrentOption}{\tw@\tw@}
%    \end{macrocode}
%  \end{macro}
%
%  \begin{macro}{\tildelow}
%  \begin{macro}{\gentilde}
%  \begin{macro}{\newtilde}
%  \begin{macro}{\newcheck}
%    The standard \TeX\ accents are too high for Estonian typography,
%    we have to lower them (following the \babel\ German style).  For
%    a detailed explanation see the file \file{glyphs.dtx}.
%
%    \begin{macrocode}
\def\tildelow{\def\~{\protect\gentilde}}
\def\gentilde#1{\if#1o\newtilde{#1}\else\if#1O\newtilde{#1}%
    \else\newcheck{#1}%
    \fi\fi}
\def\newtilde#1{\leavevmode\allowhyphens
  {\U@D 1ex%
  {\setbox\z@\hbox{\char126}\dimen@ -.45ex\advance\dimen@\ht\z@
  \ifdim 1ex<\dimen@ \fontdimen5\font\dimen@ \fi}%
  \accent126\fontdimen5\font\U@D #1}\allowhyphens}
\def\newcheck#1{\leavevmode\allowhyphens
  {\U@D 1ex%
  {\setbox\z@\hbox{\char20}\dimen@ -.45ex\advance\dimen@\ht\z@
  \ifdim 1ex<\dimen@ \fontdimen5\font\dimen@ \fi}%
  \accent20\fontdimen5\font\U@D #1}\allowhyphens}
%    \end{macrocode}
%  \end{macro}
%  \end{macro}
%  \end{macro}
%  \end{macro}
%
%    We save the double quote character in |\dq|, and  tilde in |\til|,
%    and store the original definitions of |\"| and |\~| as |\dieresis|
%    and |\texttilde|.
%
%    \begin{macrocode}
\begingroup \catcode`\"12
\edef\x{\endgroup
  \def\noexpand\dq{"}
  \def\noexpand\til{~}}
\x
\let\dieresis\"
\let\texttilde\~
%    \end{macrocode}
%
%    This part follows closely \file{spanish.ldf}. We check the
%    encoding and if it is T1, we have to tell \TeX\ about our
%    redefined accents.
%
% \changes{estonian-1.0g}{1999/04/11}{use \cs{bbl@t@one} instead of
%    \cs{bbl@next}} 
%    \begin{macrocode}
\ifx\f@encoding\bbl@t@one
  \let\@umlaut\dieresis
  \let\@tilde\texttilde
  \DeclareTextComposite{\~}{T1}{s}{178}
  \DeclareTextComposite{\~}{T1}{S}{146}
  \DeclareTextComposite{\~}{T1}{z}{186}
  \DeclareTextComposite{\~}{T1}{Z}{154}
  \DeclareTextComposite{\"}{T1}{'}{17}
  \DeclareTextComposite{\"}{T1}{`}{18}
  \DeclareTextComposite{\"}{T1}{<}{19}
  \DeclareTextComposite{\"}{T1}{>}{20}
%    \end{macrocode}
%
%    If the encoding differs from T1, we expand the accents, enabling
%    hyphenation beyond the accent. In this case \TeX\ will not find
%    all possible breaks, and we have to warn people.
%
%    \begin{macrocode}
\else
  \wlog{Warning: Hyphenation would work better for the T1 encoding.}
  \let\@umlaut\newumlaut
  \let\@tilde\gentilde
\fi
%    \end{macrocode}
%
%     Now we define the shorthands.
%
% \changes{estonian-1.0d}{1995/07/27}{The second argument was missing
%    in the definition of some of the double-quote shorthands}
%    \begin{macrocode}
\declare@shorthand{estonian}{"a}{\textormath{\"{a}}{\ddot a}}
\declare@shorthand{estonian}{"A}{\textormath{\"{A}}{\ddot A}}
\declare@shorthand{estonian}{"o}{\textormath{\"{o}}{\ddot o}}
\declare@shorthand{estonian}{"O}{\textormath{\"{O}}{\ddot O}}
\declare@shorthand{estonian}{"u}{\textormath{\"{u}}{\ddot u}}
\declare@shorthand{estonian}{"U}{\textormath{\"{U}}{\ddot U}}
%    \end{macrocode}
%    german and french quotes,
% \changes{estonian-1.0f}{1997/04/03}{Removed empty groups after
%    double quote and guillemot characters}
%    \begin{macrocode}
\declare@shorthand{estonian}{"`}{%
  \textormath{\quotedblbase}{\mbox{\quotedblbase}}}
\declare@shorthand{estonian}{"'}{%
  \textormath{\textquotedblleft}{\mbox{\textquotedblleft}}}
\declare@shorthand{estonian}{"<}{%
  \textormath{\guillemotleft}{\mbox{\guillemotleft}}}
\declare@shorthand{estonian}{">}{%
  \textormath{\guillemotright}{\mbox{\guillemotright}}}
%    \end{macrocode}
%    
%    \begin{macrocode}
\declare@shorthand{estonian}{~o}{\textormath{\@tilde o}{\tilde o}}
\declare@shorthand{estonian}{~O}{\textormath{\@tilde O}{\tilde O}}
\declare@shorthand{estonian}{~s}{\textormath{\@tilde s}{\check s}}
\declare@shorthand{estonian}{~S}{\textormath{\@tilde S}{\check S}}
\declare@shorthand{estonian}{~z}{\textormath{\@tilde z}{\check z}}
\declare@shorthand{estonian}{~Z}{\textormath{\@tilde Z}{\check Z}}
%    \end{macrocode}
%    and some additional commands:
%    \begin{macrocode}
\declare@shorthand{estonian}{"-}{\nobreak\-\bbl@allowhyphens}
\declare@shorthand{estonian}{"|}{%
  \textormath{\nobreak\discretionary{-}{}{\kern.03em}%
              \allowhyphens}{}}
\declare@shorthand{estonian}{""}{\dq}
\declare@shorthand{estonian}{~~}{\til}
%    \end{macrocode}
%
%    The macro |\ldf@finish| takes care of looking for a
%    configuration file, setting the main language to be switched on
%    at |\begin{document}| and resetting the category code of
%    \texttt{@} to its original value.
% \changes{estonian-1.0e}{1996/11/02}{Now use \cs{ldf@finish} to wrap
%    up} 
%    \begin{macrocode}
\ldf@finish{estonian}
%</code>
%    \end{macrocode}
%
% \Finale
%%
%% \CharacterTable
%%  {Upper-case    \A\B\C\D\E\F\G\H\I\J\K\L\M\N\O\P\Q\R\S\T\U\V\W\X\Y\Z
%%   Lower-case    \a\b\c\d\e\f\g\h\i\j\k\l\m\n\o\p\q\r\s\t\u\v\w\x\y\z
%%   Digits        \0\1\2\3\4\5\6\7\8\9
%%   Exclamation   \!     Double quote  \"     Hash (number) \#
%%   Dollar        \$     Percent       \%     Ampersand     \&
%%   Acute accent  \'     Left paren    \(     Right paren   \)
%%   Asterisk      \*     Plus          \+     Comma         \,
%%   Minus         \-     Point         \.     Solidus       \/
%%   Colon         \:     Semicolon     \;     Less than     \<
%%   Equals        \=     Greater than  \>     Question mark \?
%%   Commercial at \@     Left bracket  \[     Backslash     \\
%%   Right bracket \]     Circumflex    \^     Underscore    \_
%%   Grave accent  \`     Left brace    \{     Vertical bar  \|
%%   Right brace   \}     Tilde         \~}
%%
\endinput
}
\DeclareOption{finnish}{% \iffalse meta-comment
%
% Copyright 1989-2007 Johannes L. Braams and any individual authors
% listed elsewhere in this file.  All rights reserved.
% 
% This file is part of the Babel system.
% --------------------------------------
% 
% It may be distributed and/or modified under the
% conditions of the LaTeX Project Public License, either version 1.3
% of this license or (at your option) any later version.
% The latest version of this license is in
%   http://www.latex-project.org/lppl.txt
% and version 1.3 or later is part of all distributions of LaTeX
% version 2003/12/01 or later.
% 
% This work has the LPPL maintenance status "maintained".
% 
% The Current Maintainer of this work is Johannes Braams.
% 
% The list of all files belonging to the Babel system is
% given in the file `manifest.bbl. See also `legal.bbl' for additional
% information.
% 
% The list of derived (unpacked) files belonging to the distribution
% and covered by LPPL is defined by the unpacking scripts (with
% extension .ins) which are part of the distribution.
% \fi
% \CheckSum{172}
% \iffalse
%    Tell the \LaTeX\ system who we are and write an entry on the
%    transcript.
%<*dtx>
\ProvidesFile{finnish.dtx}
%</dtx>
%<code>\ProvidesLanguage{finnish}
%\fi
%\ProvidesFile{finnish.dtx}
        [2007/10/20 v1.3q Finnish support from the babel system]
%\iffalse
%% File `finnish.dtx'
%% Babel package for LaTeX version 2e
%% Copyright (C) 1989 - 2007
%%           by Johannes Braams, TeXniek
%
%% Please report errors to: J.L. Braams
%%                          babel at braams dot xs4all dot nl
%
%    This file is part of the babel system, it provides the source
%    code for the Finnish language definition file.  A contribution
%    was made by Mikko KANERVA (KANERVA@CERNVM) and Keranen Reino
%    (KERANEN@CERNVM).
%<*filedriver>
\documentclass{ltxdoc}
\newcommand*\TeXhax{\TeX hax}
\newcommand*\babel{\textsf{babel}}
\newcommand*\langvar{$\langle \it lang \rangle$}
\newcommand*\note[1]{}
\newcommand*\Lopt[1]{\textsf{#1}}
\newcommand*\file[1]{\texttt{#1}}
\begin{document}
 \DocInput{finnish.dtx}
\end{document}
%</filedriver>
%\fi
%
% \GetFileInfo{finnish.dtx}
%
% \changes{finnish-1.0a}{1991/07/15}{Renamed \file{babel.sty} in
%    \file{babel.com}}
% \changes{finnish-1.1}{1991/02/15}{Brought up-to-date with babel 3.2a}
% \changes{finnish-1.2}{1994/02/27}{Update for \LaTeXe}
% \changes{finnish-1.3c}{1994/06/26}{Removed the use of \cs{filedate}
%    and moved identification after the loading of \file{babel.def}}
% \changes{finnish-1.3d}{1994/06/30}{Removed a few references to
%    \file{babel.com}}
% \changes{finnish-1.3i}{1996/10/10}{Replaced \cs{undefined} with
%    \cs{@undefined} and \cs{empty} with \cs{@empty} for consistency
%    with \LaTeX, moved the definition of \cs{atcatcode} right to the
%    beginning.}
% \changes{finnish-1.3q}{2006/06/05}{Small documentation fix}
%
%  \section{The Finnish language}
%
%    The file \file{\filename}\footnote{The file described in this
%    section has version number \fileversion\ and was last revised on
%    \filedate.  A contribution was made by Mikko KANERVA
%    (\texttt{KANERVA@CERNVM}) and Keranen Reino
%    (\texttt{KERANEN@CERNVM}).}  defines all the language definition
%    macros for the Finnish language.
%
%    For this language the character |"| is made active. In
%    table~\ref{tab:finnish-quote} an overview is given of its purpose.
%
%    \begin{table}[htb]
%     \centering
%     \begin{tabular}{lp{8cm}}
%       \verb="|= & disable ligature at this position.\\
%        |"-| & an explicit hyphen sign, allowing hyphenation
%               in the rest of the word.\\
%        |"=| & an explicit hyphen sign for expressions such as
%               ``pakastekaapit ja -arkut''.\\
%        |""| & like \verb="-=, but producing no hyphen sign (for
%              words that should break at some sign such as
%              ``entrada/salida.''\\
%        |"`| & lowered double left quotes (looks like ,,)\\
%        |"'| & normal double right quotes\\
%        |"<| & for French left double quotes (similar to $<<$).\\
%        |">| & for French right double quotes (similar to $>>$).\\
%        |\-| & like the old |\-|, but allowing hyphenation
%               in the rest of the word.
%     \end{tabular}
%     \caption{The extra definitions made by \file{finnish.ldf}}
%     \label{tab:finnish-quote}
%    \end{table}
%
% \StopEventually{}
%
%    The macro |\LdfInit| takes care of preventing that this file is
%    loaded more than once, checking the category code of the
%    \texttt{@} sign, etc.
% \changes{finnish-1.3i}{1996/11/02}{Now use \cs{LdfInit} to perform
%    initial checks} 
%    \begin{macrocode}
%<*code>
\LdfInit{finnish}\captionsfinnish
%    \end{macrocode}
%
%    When this file is read as an option, i.e. by the |\usepackage|
%    command, \texttt{finnish} will be an `unknown' language in which
%    case we have to make it known.  So we check for the existence of
%    |\l@finnish| to see whether we have to do something here.
%
% \changes{finnish-1.0b}{1991/10/29}{Removed use of \cs{@ifundefined}}
% \changes{finnish-1.1}{1992/02/15}{Added a warning when no hyphenation
%    patterns were loaded.}
% \changes{finnish-1.3c}{1994/06/26}{Now use \cs{@nopatterns} to
%    produce the warning}
%    \begin{macrocode}
\ifx\l@finnish\@undefined
    \@nopatterns{Finnish}
    \adddialect\l@finnish0\fi
%    \end{macrocode}
%
%    The next step consists of defining commands to switch to the
%    Finnish language. The reason for this is that a user might want
%    to switch back and forth between languages.
%
% \begin{macro}{\captionsfinnish}
%    The macro |\captionsfinnish| defines all strings used in the four
%    standard documentclasses provided with \LaTeX.
% \changes{finnish-1.1}{1992/02/15}{Added \cs{seename}, \cs{alsoname}
%    and \cs{prefacename}}
% \changes{finnish-1.1}{1993/07/15}{\cs{headpagename} should be
%    \cs{pagename}}
% \changes{finnish-1.1.2}{1993/09/16}{Added translations}
% \changes{finnish-1.3g}{1995/07/02}{Added \cs{proofname} for
%    AMS-\LaTeX}
% \changes{finnish-1.3h}{1995/07/25}{Added finnish word for `Proof'}
% \changes{finnish-1.3n}{2000/09/19}{Added \cs{glossaryname}}
% \changes{finnish-1.3o}{2001/03/12}{Provided translation for
%    Glossary}
%    \begin{macrocode}
\addto\captionsfinnish{%
  \def\prefacename{Esipuhe}%
  \def\refname{Viitteet}%
  \def\abstractname{Tiivistelm\"a}
  \def\bibname{Kirjallisuutta}%
  \def\chaptername{Luku}%
  \def\appendixname{Liite}%
  \def\contentsname{Sis\"alt\"o}%   /* Could be "Sis\"allys" as well */
  \def\listfigurename{Kuvat}%
  \def\listtablename{Taulukot}%
  \def\indexname{Hakemisto}%
  \def\figurename{Kuva}%
  \def\tablename{Taulukko}%
  \def\partname{Osa}%
  \def\enclname{Liitteet}%
  \def\ccname{Jakelu}%
  \def\headtoname{Vastaanottaja}%
  \def\pagename{Sivu}%
  \def\seename{katso}%
  \def\alsoname{katso my\"os}%
  \def\proofname{Todistus}%
  \def\glossaryname{Sanasto}%
  }%
%    \end{macrocode}
% \end{macro}
%
% \begin{macro}{\datefinnish}
%    The macro |\datefinnish| redefines the command |\today| to
%    produce Finnish dates.
% \changes{finnish-1.3e}{1994/07/12}{Added a`.' after the number of
%    the day}
% \changes{finnish-1.3k}{1997/10/01}{Use \cs{edef} to define \cs{today}
%    to save memory}
% \changes{finnish-1.3k}{1998/03/28}{use \cs{def} instead of \cs{edef}}
%    \begin{macrocode}
\def\datefinnish{%
  \def\today{\number\day.~\ifcase\month\or
    tammikuuta\or helmikuuta\or maaliskuuta\or huhtikuuta\or
    toukokuuta\or kes\"akuuta\or hein\"akuuta\or elokuuta\or
    syyskuuta\or lokakuuta\or marraskuuta\or joulukuuta\fi
    \space\number\year}}
%    \end{macrocode}
% \end{macro}
%
% \begin{macro}{\extrasfinnish}
% \begin{macro}{\noextrasfinnish}
%    Finnish has many long words (some of them compound, some not).
%    For this reason hyphenation is very often the only solution in
%    line breaking. For this reason the values of |\hyphenpenalty|,
%    |\exhyphenpenalty| and |\doublehyphendemerits| should be
%    decreased. (In one of the manuals of style Matti Rintala noticed
%    a paragraph with ten lines, eight of which ended in a hyphen!)
%
%    Matti Rintala noticed that with these changes \TeX\ handles
%    Finnish very well, although sometimes the values of |\tolerance|
%    and |\emergencystretch| must be increased. However, I don't think
%    changing these values in \file{finnish.ldf} is appropriate, as
%    the looseness of the font (and the line width) affect the correct
%    choice of these parameters.
% \changes{finnish-1.3f}{1995/05/13}{Added the setting of more
%    hyphenation parameters, according to PR1027}
%    \begin{macrocode}
\addto\extrasfinnish{%
  \babel@savevariable\hyphenpenalty\hyphenpenalty=30%
  \babel@savevariable\exhyphenpenalty\exhyphenpenalty=30%
  \babel@savevariable\doublehyphendemerits\doublehyphendemerits=5000%
  \babel@savevariable\finalhyphendemerits\finalhyphendemerits=5000%
  }
\addto\noextrasfinnish{}
%    \end{macrocode}
%
%    Another thing |\extrasfinnish| needs to do is to ensure that
%    |\frenchspacing| is in effect.  If this is not the case the
%    execution of |\noextrasfinnish| will switch it of again.
% \changes{finnish-1.3f}{1995/05/15}{Added the setting of
%    \cs{frenchspacing}}
%    \begin{macrocode}
\addto\extrasfinnish{\bbl@frenchspacing}
\addto\noextrasfinnish{\bbl@nonfrenchspacing}
%    \end{macrocode}
%
%    For Finnish the \texttt{"} character is made active. This is
%    done once, later on its definition may vary. Other languages in
%    the same document may also use the \texttt{"} character for
%    shorthands; we specify that the finnish group of shorthands
%    should be used.
% \changes{finnish-1.3g}{1995/07/02}{Added the active double quote}
%    \begin{macrocode}
\initiate@active@char{"}
\addto\extrasfinnish{\languageshorthands{finnish}}
%    \end{macrocode}
%    Don't forget to turn the shorthands off again.
% \changes{finnish-1.3n}{1999/12/16}{Deactive shorthands ouside of Finnish}
%    \begin{macrocode}
\addto\extrasfinnish{\bbl@activate{"}}
\addto\noextrasfinnish{\bbl@deactivate{"}}
%    \end{macrocode}
%
%
%    The `umlaut' character should be positioned lower on \emph{all}
%    vowels in Finnish texts.
%    \begin{macrocode}
\addto\extrasfinnish{\umlautlow\umlautelow}
\addto\noextrasfinnish{\umlauthigh}
%    \end{macrocode}
%
%    First we define access to the low opening double quote and
%    guillemets for quotations,
% \changes{finnish-1.3k}{1997/04/03}{Removed empty groups after
%    double quote and guillemot characters}
% \changes{finnish-1.3m}{1999/04/13}{Added misisng closing brace}
%    \begin{macrocode}
\declare@shorthand{finnish}{"`}{%
  \textormath{\quotedblbase}{\mbox{\quotedblbase}}}
\declare@shorthand{finnish}{"'}{%
  \textormath{\textquotedblright}{\mbox{\textquotedblright}}}
\declare@shorthand{finnish}{"<}{%
  \textormath{\guillemotleft}{\mbox{\guillemotleft}}}
\declare@shorthand{finnish}{">}{%
  \textormath{\guillemotright}{\mbox{\guillemotright}}}
%    \end{macrocode}
%    then we define two shorthands to be able to specify hyphenation
%    breakpoints that behave a little different from |\-|.
% \changes{finnish-1.3p}{2001/11/13}{\texttt{\symbol{34}\symbol{61}}
%    should also use \cs{bbl@allowhyphens}}
%    \begin{macrocode}
\declare@shorthand{finnish}{"-}{\nobreak-\bbl@allowhyphens}
\declare@shorthand{finnish}{""}{\hskip\z@skip}
\declare@shorthand{finnish}{"=}{\hbox{-}\bbl@allowhyphens}
%    \end{macrocode}
%    And we want to have a shorthand for disabling a ligature.
%    \begin{macrocode}
\declare@shorthand{finnish}{"|}{%
  \textormath{\discretionary{-}{}{\kern.03em}}{}}
%    \end{macrocode}
% \end{macro}
% \end{macro}
%
%  \begin{macro}{\-}
%
%    All that is left now is the redefinition of |\-|. The new version
%    of |\-| should indicate an extra hyphenation position, while
%    allowing other hyphenation positions to be generated
%    automatically. The standard behaviour of \TeX\ in this respect is
%    very unfortunate for languages such as Dutch, Finnish and German,
%    where long compound words are quite normal and all one needs is a
%    means to indicate an extra hyphenation position on top of the
%    ones that \TeX\ can generate from the hyphenation patterns.
% \changes{finnish-1.3g}{1995/07/02}{Added change of \cs{-}}
% \changes{finnish-1.3n}{2000/09/19}{\cs{allowhyphens} should have
%    been \cs{bbl@allowhyphens}}
%    \begin{macrocode}
\addto\extrasfinnish{\babel@save\-}
\addto\extrasfinnish{\def\-{\bbl@allowhyphens
                          \discretionary{-}{}{}\bbl@allowhyphens}}
%    \end{macrocode}
%  \end{macro}
%
%  \begin{macro}{\finishhyphenmins}
%    The finnish hyphenation patterns can be used with |\lefthyphenmin|
%    set to~2 and |\righthyphenmin| set to~2.
% \changes{finnish-1.3f}{1995/05/13}{use \cs{finnishhyphenmins} to
%    store the correct values}
% \changes{finnish-1.3n}{2000/09/22}{Now use \cs{providehyphenmins} to
%    provide a default value}
%    \begin{macrocode}
\providehyphenmins{\CurrentOption}{\tw@\tw@}
%    \end{macrocode}
%  \end{macro}
%
%    The macro |\ldf@finish| takes care of looking for a
%    configuration file, setting the main language to be switched on
%    at |\begin{document}| and resetting the category code of
%    \texttt{@} to its original value.
% \changes{finnish-1.3i}{1996/11/02}{Now use \cs{ldf@finish} to wrap up}
%    \begin{macrocode}
\ldf@finish{finnish}
%</code>
%    \end{macrocode}
%
% \Finale
%%
%% \CharacterTable
%%  {Upper-case    \A\B\C\D\E\F\G\H\I\J\K\L\M\N\O\P\Q\R\S\T\U\V\W\X\Y\Z
%%   Lower-case    \a\b\c\d\e\f\g\h\i\j\k\l\m\n\o\p\q\r\s\t\u\v\w\x\y\z
%%   Digits        \0\1\2\3\4\5\6\7\8\9
%%   Exclamation   \!     Double quote  \"     Hash (number) \#
%%   Dollar        \$     Percent       \%     Ampersand     \&
%%   Acute accent  \'     Left paren    \(     Right paren   \)
%%   Asterisk      \*     Plus          \+     Comma         \,
%%   Minus         \-     Point         \.     Solidus       \/
%%   Colon         \:     Semicolon     \;     Less than     \<
%%   Equals        \=     Greater than  \>     Question mark \?
%%   Commercial at \@     Left bracket  \[     Backslash     \\
%%   Right bracket \]     Circumflex    \^     Underscore    \_
%%   Grave accent  \`     Left brace    \{     Vertical bar  \|
%%   Right brace   \}     Tilde         \~}
%%
\endinput

}
%    \end{macrocode}
%    The \babel\ support or French used to be stored in
%    \file{francais.ldf}; therefor the \LaTeX2.09 option used to be
%    \Lopt{francais}. The hyphenation patterns may be loaded as either
%    `french' or as `francais'.
% \changes{babel~3.5f}{1996/01/10}{Now use the file \file{frenchb.ldf}
%    from Daniel Flipo for french support}
% \changes{babel~3.6e}{1997/01/08}{Added option \Lopt{frenchb} an
%    alias for \Lopt{francais}}
%    \begin{macrocode}
\DeclareOption{francais}{%%
%% This file will generate fast loadable files and documentation
%% driver files from the doc files in this package when run through
%% LaTeX or TeX.
%%
%% Copyright 1989-2011 Johannes L. Braams and any individual authors
%% listed elsewhere in this file.  All rights reserved.
%% 
%% This file is part of the Babel system.
%% --------------------------------------
%% 
%% It may be distributed and/or modified under the
%% conditions of the LaTeX Project Public License, either version 1.3
%% of this license or (at your option) any later version.
%% The latest version of this license is in
%%   http://www.latex-project.org/lppl.txt
%% and version 1.3 or later is part of all distributions of LaTeX
%% version 2003/12/01 or later.
%% 
%% This work has the LPPL maintenance status "maintained".
%% 
%% The Current Maintainer of this work is Johannes Braams.
%% 
%% The list of all files belonging to the LaTeX base distribution is
%% given in the file `manifest.bbl. See also `legal.bbl' for additional
%% information.
%% 
%% The list of derived (unpacked) files belonging to the distribution
%% and covered by LPPL is defined by the unpacking scripts (with
%% extension .ins) which are part of the distribution.
%%
%% --------------- start of docstrip commands ------------------
%%
\def\filedate{1999/10/30}
\def\batchfile{frenchb.ins}
\input docstrip.tex

{\ifx\generate\undefined
\Msg{**********************************************}
\Msg{*}
\Msg{* This installation requires docstrip}
\Msg{* version 2.3c or later.}
\Msg{*}
\Msg{* An older version of docstrip has been input}
\Msg{*}
\Msg{**********************************************}
\errhelp{Move or rename old docstrip.tex.}
\errmessage{Old docstrip in input path}
\batchmode
\csname @@end\endcsname
\fi}

\declarepreamble\mainpreamble
This is a generated file.

Copyright 1989-2011 Johannes L. Braams and any individual authors
listed elsewhere in this file.  All rights reserved.

This file was generated from file(s) of the Babel system.
---------------------------------------------------------

It may be distributed and/or modified under the
conditions of the LaTeX Project Public License, either version 1.3
of this license or (at your option) any later version.
The latest version of this license is in
  http://www.latex-project.org/lppl.txt
and version 1.3 or later is part of all distributions of LaTeX
version 2003/12/01 or later.

This work has the LPPL maintenance status "maintained".

The Current Maintainer of this work is Johannes Braams.

This file may only be distributed together with a copy of the Babel
system. You may however distribute the Babel system without
such generated files.

The list of all files belonging to the Babel distribution is
given in the file `manifest.bbl'. See also `legal.bbl for additional
information.

The list of derived (unpacked) files belonging to the distribution
and covered by LPPL is defined by the unpacking scripts (with
extension .ins) which are part of the distribution.
\endpreamble

\declarepreamble\fdpreamble
This is a generated file.

Copyright 1989-2011 Johannes L. Braams and any individual authors
listed elsewhere in this file.  All rights reserved.

This file was generated from file(s) of the Babel system.
---------------------------------------------------------

It may be distributed and/or modified under the
conditions of the LaTeX Project Public License, either version 1.3
of this license or (at your option) any later version.
The latest version of this license is in
  http://www.latex-project.org/lppl.txt
and version 1.3 or later is part of all distributions of LaTeX
version 2003/12/01 or later.

This work has the LPPL maintenance status "maintained".

The Current Maintainer of this work is Johannes Braams.

This file may only be distributed together with a copy of the Babel
system. You may however distribute the Babel system without
such generated files.

The list of all files belonging to the Babel distribution is
given in the file `manifest.bbl'. See also `legal.bbl for additional
information.

In particular, permission is granted to customize the declarations in
this file to serve the needs of your installation.

However, NO PERMISSION is granted to distribute a modified version
of this file under its original name.

\endpreamble

\keepsilent

\usedir{tex/generic/babel} 

\usepreamble\mainpreamble
\generate{\file{frenchb.ldf}{\from{frenchb.dtx}{code}}
          }
\nopreamble
\nopostamble
\generate{\file{frenchb.cfg}{\from{frenchb.dtx}{cfg}}
          }

\ifToplevel{
\Msg{***********************************************************}
\Msg{*}
\Msg{* To finish the installation you have to move the following}
\Msg{* files into a directory searched by TeX:}
\Msg{*}
\Msg{* \space\space All *.def, *.fd, *.ldf, *.sty}
\Msg{*}
\Msg{* To produce the documentation run the files ending with}
\Msg{* '.dtx' and `.fdd' through LaTeX.}
\Msg{*}
\Msg{* Happy TeXing}
\Msg{***********************************************************}
}
 
\endinput




}
\DeclareOption{frenchb}{%%
%% This file will generate fast loadable files and documentation
%% driver files from the doc files in this package when run through
%% LaTeX or TeX.
%%
%% Copyright 1989-2011 Johannes L. Braams and any individual authors
%% listed elsewhere in this file.  All rights reserved.
%% 
%% This file is part of the Babel system.
%% --------------------------------------
%% 
%% It may be distributed and/or modified under the
%% conditions of the LaTeX Project Public License, either version 1.3
%% of this license or (at your option) any later version.
%% The latest version of this license is in
%%   http://www.latex-project.org/lppl.txt
%% and version 1.3 or later is part of all distributions of LaTeX
%% version 2003/12/01 or later.
%% 
%% This work has the LPPL maintenance status "maintained".
%% 
%% The Current Maintainer of this work is Johannes Braams.
%% 
%% The list of all files belonging to the LaTeX base distribution is
%% given in the file `manifest.bbl. See also `legal.bbl' for additional
%% information.
%% 
%% The list of derived (unpacked) files belonging to the distribution
%% and covered by LPPL is defined by the unpacking scripts (with
%% extension .ins) which are part of the distribution.
%%
%% --------------- start of docstrip commands ------------------
%%
\def\filedate{1999/10/30}
\def\batchfile{frenchb.ins}
\input docstrip.tex

{\ifx\generate\undefined
\Msg{**********************************************}
\Msg{*}
\Msg{* This installation requires docstrip}
\Msg{* version 2.3c or later.}
\Msg{*}
\Msg{* An older version of docstrip has been input}
\Msg{*}
\Msg{**********************************************}
\errhelp{Move or rename old docstrip.tex.}
\errmessage{Old docstrip in input path}
\batchmode
\csname @@end\endcsname
\fi}

\declarepreamble\mainpreamble
This is a generated file.

Copyright 1989-2011 Johannes L. Braams and any individual authors
listed elsewhere in this file.  All rights reserved.

This file was generated from file(s) of the Babel system.
---------------------------------------------------------

It may be distributed and/or modified under the
conditions of the LaTeX Project Public License, either version 1.3
of this license or (at your option) any later version.
The latest version of this license is in
  http://www.latex-project.org/lppl.txt
and version 1.3 or later is part of all distributions of LaTeX
version 2003/12/01 or later.

This work has the LPPL maintenance status "maintained".

The Current Maintainer of this work is Johannes Braams.

This file may only be distributed together with a copy of the Babel
system. You may however distribute the Babel system without
such generated files.

The list of all files belonging to the Babel distribution is
given in the file `manifest.bbl'. See also `legal.bbl for additional
information.

The list of derived (unpacked) files belonging to the distribution
and covered by LPPL is defined by the unpacking scripts (with
extension .ins) which are part of the distribution.
\endpreamble

\declarepreamble\fdpreamble
This is a generated file.

Copyright 1989-2011 Johannes L. Braams and any individual authors
listed elsewhere in this file.  All rights reserved.

This file was generated from file(s) of the Babel system.
---------------------------------------------------------

It may be distributed and/or modified under the
conditions of the LaTeX Project Public License, either version 1.3
of this license or (at your option) any later version.
The latest version of this license is in
  http://www.latex-project.org/lppl.txt
and version 1.3 or later is part of all distributions of LaTeX
version 2003/12/01 or later.

This work has the LPPL maintenance status "maintained".

The Current Maintainer of this work is Johannes Braams.

This file may only be distributed together with a copy of the Babel
system. You may however distribute the Babel system without
such generated files.

The list of all files belonging to the Babel distribution is
given in the file `manifest.bbl'. See also `legal.bbl for additional
information.

In particular, permission is granted to customize the declarations in
this file to serve the needs of your installation.

However, NO PERMISSION is granted to distribute a modified version
of this file under its original name.

\endpreamble

\keepsilent

\usedir{tex/generic/babel} 

\usepreamble\mainpreamble
\generate{\file{frenchb.ldf}{\from{frenchb.dtx}{code}}
          }
\nopreamble
\nopostamble
\generate{\file{frenchb.cfg}{\from{frenchb.dtx}{cfg}}
          }

\ifToplevel{
\Msg{***********************************************************}
\Msg{*}
\Msg{* To finish the installation you have to move the following}
\Msg{* files into a directory searched by TeX:}
\Msg{*}
\Msg{* \space\space All *.def, *.fd, *.ldf, *.sty}
\Msg{*}
\Msg{* To produce the documentation run the files ending with}
\Msg{* '.dtx' and `.fdd' through LaTeX.}
\Msg{*}
\Msg{* Happy TeXing}
\Msg{***********************************************************}
}
 
\endinput




}
%    \end{macrocode}
%    With \LaTeXe\ we can now also use the option \Lopt{french} and
%    still call the file \file{frenchb.ldf}.
% \changes{babel~3.5d}{1995/07/02}{Load \file{french.ldf} when it is
%    found instead of \file{frenchb.ldf}}
% \changes{babel~3.7j}{2003/06/07}{\emph{only} load
%    \file{frenchb.ldf}}
%    \begin{macrocode}
\DeclareOption{french}{%%
%% This file will generate fast loadable files and documentation
%% driver files from the doc files in this package when run through
%% LaTeX or TeX.
%%
%% Copyright 1989-2011 Johannes L. Braams and any individual authors
%% listed elsewhere in this file.  All rights reserved.
%% 
%% This file is part of the Babel system.
%% --------------------------------------
%% 
%% It may be distributed and/or modified under the
%% conditions of the LaTeX Project Public License, either version 1.3
%% of this license or (at your option) any later version.
%% The latest version of this license is in
%%   http://www.latex-project.org/lppl.txt
%% and version 1.3 or later is part of all distributions of LaTeX
%% version 2003/12/01 or later.
%% 
%% This work has the LPPL maintenance status "maintained".
%% 
%% The Current Maintainer of this work is Johannes Braams.
%% 
%% The list of all files belonging to the LaTeX base distribution is
%% given in the file `manifest.bbl. See also `legal.bbl' for additional
%% information.
%% 
%% The list of derived (unpacked) files belonging to the distribution
%% and covered by LPPL is defined by the unpacking scripts (with
%% extension .ins) which are part of the distribution.
%%
%% --------------- start of docstrip commands ------------------
%%
\def\filedate{1999/10/30}
\def\batchfile{frenchb.ins}
\input docstrip.tex

{\ifx\generate\undefined
\Msg{**********************************************}
\Msg{*}
\Msg{* This installation requires docstrip}
\Msg{* version 2.3c or later.}
\Msg{*}
\Msg{* An older version of docstrip has been input}
\Msg{*}
\Msg{**********************************************}
\errhelp{Move or rename old docstrip.tex.}
\errmessage{Old docstrip in input path}
\batchmode
\csname @@end\endcsname
\fi}

\declarepreamble\mainpreamble
This is a generated file.

Copyright 1989-2011 Johannes L. Braams and any individual authors
listed elsewhere in this file.  All rights reserved.

This file was generated from file(s) of the Babel system.
---------------------------------------------------------

It may be distributed and/or modified under the
conditions of the LaTeX Project Public License, either version 1.3
of this license or (at your option) any later version.
The latest version of this license is in
  http://www.latex-project.org/lppl.txt
and version 1.3 or later is part of all distributions of LaTeX
version 2003/12/01 or later.

This work has the LPPL maintenance status "maintained".

The Current Maintainer of this work is Johannes Braams.

This file may only be distributed together with a copy of the Babel
system. You may however distribute the Babel system without
such generated files.

The list of all files belonging to the Babel distribution is
given in the file `manifest.bbl'. See also `legal.bbl for additional
information.

The list of derived (unpacked) files belonging to the distribution
and covered by LPPL is defined by the unpacking scripts (with
extension .ins) which are part of the distribution.
\endpreamble

\declarepreamble\fdpreamble
This is a generated file.

Copyright 1989-2011 Johannes L. Braams and any individual authors
listed elsewhere in this file.  All rights reserved.

This file was generated from file(s) of the Babel system.
---------------------------------------------------------

It may be distributed and/or modified under the
conditions of the LaTeX Project Public License, either version 1.3
of this license or (at your option) any later version.
The latest version of this license is in
  http://www.latex-project.org/lppl.txt
and version 1.3 or later is part of all distributions of LaTeX
version 2003/12/01 or later.

This work has the LPPL maintenance status "maintained".

The Current Maintainer of this work is Johannes Braams.

This file may only be distributed together with a copy of the Babel
system. You may however distribute the Babel system without
such generated files.

The list of all files belonging to the Babel distribution is
given in the file `manifest.bbl'. See also `legal.bbl for additional
information.

In particular, permission is granted to customize the declarations in
this file to serve the needs of your installation.

However, NO PERMISSION is granted to distribute a modified version
of this file under its original name.

\endpreamble

\keepsilent

\usedir{tex/generic/babel} 

\usepreamble\mainpreamble
\generate{\file{frenchb.ldf}{\from{frenchb.dtx}{code}}
          }
\nopreamble
\nopostamble
\generate{\file{frenchb.cfg}{\from{frenchb.dtx}{cfg}}
          }

\ifToplevel{
\Msg{***********************************************************}
\Msg{*}
\Msg{* To finish the installation you have to move the following}
\Msg{* files into a directory searched by TeX:}
\Msg{*}
\Msg{* \space\space All *.def, *.fd, *.ldf, *.sty}
\Msg{*}
\Msg{* To produce the documentation run the files ending with}
\Msg{* '.dtx' and `.fdd' through LaTeX.}
\Msg{*}
\Msg{* Happy TeXing}
\Msg{***********************************************************}
}
 
\endinput




}%
\DeclareOption{galician}{%%
%% This file will generate fast loadable files and documentation
%% driver files from the doc files in this package when run through
%% LaTeX or TeX.
%%
%% Copyright 1989-2008 Johannes L. Braams and any individual authors
%% listed elsewhere in this file.  All rights reserved.
%% 
%% This file is part of the Babel system.
%% --------------------------------------
%% 
%% It may be distributed and/or modified under the
%% conditions of the LaTeX Project Public License, either version 1.3
%% of this license or (at your option) any later version.
%% The latest version of this license is in
%%   http://www.latex-project.org/lppl.txt
%% and version 1.3 or later is part of all distributions of LaTeX
%% version 2003/12/01 or later.
%% 
%% This work has the LPPL maintenance status "maintained".
%% 
%% The Current Maintainer of this work is Johannes Braams.
%% 
%% The list of all files belonging to the LaTeX base distribution is
%% given in the file `manifest.bbl. See also `legal.bbl' for additional
%% information.
%% 
%% The list of derived (unpacked) files belonging to the distribution
%% and covered by LPPL is defined by the unpacking scripts (with
%% extension .ins) which are part of the distribution.
%%
%% --------------- start of docstrip commands ------------------
%%
\def\batchfile{galician.ins}
\def\filedate{2007/01/29}
\input docstrip.tex

{\ifx\generate\undefined
  \Msg{**********************************************}
  \Msg{*}
  \Msg{* This installation requires docstrip}
  \Msg{* version 2.3c or later.}
  \Msg{*}
  \Msg{* An older version of docstrip has been input}
  \Msg{*}
  \Msg{**********************************************}
  \errhelp{Move or rename old docstrip.tex.}
  \errmessage{Old docstrip in input path}
  \batchmode
  \csname @@end\endcsname
\fi}

\declarepreamble\mainpreamble
This is a generated file.

Copyright 1989-2008 Johannes L. Braams and any individual authors
listed elsewhere in this file.  All rights reserved.

This file was generated from file(s) of the Babel system.
---------------------------------------------------------

It may be distributed and/or modified under the
conditions of the LaTeX Project Public License, either version 1.3
of this license or (at your option) any later version.
The latest version of this license is in
  http://www.latex-project.org/lppl.txt
and version 1.3 or later is part of all distributions of LaTeX
version 2003/12/01 or later.

This work has the LPPL maintenance status "maintained".

The Current Maintainer of this work is Johannes Braams.

This file may only be distributed together with a copy of the Babel
system. You may however distribute the Babel system without
such generated files.

The list of all files belonging to the Babel distribution is
given in the file `manifest.bbl'. See also `legal.bbl for additional
information.

The list of derived (unpacked) files belonging to the distribution
and covered by LPPL is defined by the unpacking scripts (with
extension .ins) which are part of the distribution.
\endpreamble

\declarepreamble\fdpreamble
This is a generated file.

Copyright 1989-2008 Johannes L. Braams and any individual authors
listed elsewhere in this file.  All rights reserved.

This file was generated from file(s) of the Babel system.
---------------------------------------------------------

It may be distributed and/or modified under the
conditions of the LaTeX Project Public License, either version 1.3
of this license or (at your option) any later version.
The latest version of this license is in
  http://www.latex-project.org/lppl.txt
and version 1.3 or later is part of all distributions of LaTeX
version 2003/12/01 or later.

This work has the LPPL maintenance status "maintained".

The Current Maintainer of this work is Johannes Braams.

This file may only be distributed together with a copy of the Babel
system. You may however distribute the Babel system without
such generated files.

The list of all files belonging to the Babel distribution is
given in the file `manifest.bbl'. See also `legal.bbl for additional
information.

In particular, permission is granted to customize the declarations in
this file to serve the needs of your installation.

However, NO PERMISSION is granted to distribute a modified version
of this file under its original name.

\endpreamble

\keepsilent

\usedir{tex/generic/babel} 

\keepsilent

\usepreamble\mainpreamble
\generate{\file{galician.ldf}{\from{galician.dtx}{code}}%
          \file{glbst.tex}{\from{galician.dtx}{bblbst}}%
          \file{glromidx.tex}{\from{galician.dtx}{indexgl}}}
\usepreamble\fdpreamble

\ifToplevel{
\Msg{***********************************************************}
\Msg{*}
\Msg{* To finish the installation you have to move the following}
\Msg{* files into a directory searched by TeX:}
\Msg{*}
\Msg{* \space\space All *.def, *.fd, *.ldf, *.sty}
\Msg{*}
\Msg{* To produce the documentation run the files ending with}
\Msg{* '.dtx' and `.fdd' through LaTeX.}
\Msg{*}
\Msg{* Happy TeXing}
\Msg{***********************************************************}
}
 
\endinput
}
\DeclareOption{german}{% \iffalse meta-comm

% Copyright 1989-2008 Johannes L. Braams and any individual auth
% listed elsewhere in this file.  All rights reserv

% This file is part of the Babel syst
% -----------------------------------

% It may be distributed and/or modified under
% conditions of the LaTeX Project Public License, either version
% of this license or (at your option) any later versi
% The latest version of this license is
%   http://www.latex-project.org/lppl.
% and version 1.3 or later is part of all distributions of La
% version 2003/12/01 or lat

% This work has the LPPL maintenance status "maintaine

% The Current Maintainer of this work is Johannes Braa

% The list of all files belonging to the Babel system
% given in the file `manifest.bbl. See also `legal.bbl' for additio
% informati

% The list of derived (unpacked) files belonging to the distribut
% and covered by LPPL is defined by the unpacking scripts (w
% extension .ins) which are part of the distributi
%
% \CheckSum{3

% \iffa
%    Tell the \LaTeX\ system who we are and write an entry on
%    transcri
%<*d
\ProvidesFile{germanb.d
%</d
%<code>\ProvidesLanguage{germa
%
%\ProvidesFile{germanb.d
        [2008/06/01 v2.6m German support from the babel syst
%\iffa
%% File `germanb.d
%% Babel package for LaTeX version
%% Copyright (C) 1989 - 2
%%           by Johannes Braams, TeXn

%% Germanb Language Definition F
%% Copyright (C) 1989 - 2
%%           by Bernd Raichle raichle at azu.Informatik.Uni-Stuttgart
%%              Johannes Braams, TeXn
% This file is based on german.tex version 2.
%                       by Bernd Raichle, Hubert Partl et.

%% Please report errors to: J.L. Bra
%%                          babel at braams.xs4all

%<*filedriv
\documentclass{ltxd
\font\manual=logo10 % font used for the METAFONT logo, e
\newcommand*\MF{{\manual META}\-{\manual FON
\newcommand*\TeXhax{\TeX h
\newcommand*\babel{\textsf{babe
\newcommand*\langvar{$\langle \it lang \rangl
\newcommand*\note[1
\newcommand*\Lopt[1]{\textsf{#
\newcommand*\file[1]{\texttt{#
\begin{docume
 \DocInput{germanb.d
\end{docume
%</filedriv
%
% \GetFileInfo{germanb.d

% \changes{germanb-1.0a}{1990/05/14}{Incorporated Nico's commen
% \changes{germanb-1.0b}{1990/05/22}{fixed typo in definition
%    austrian language found by Werenfried S
%    \texttt{nspit@fys.ruu.n
% \changes{germanb-1.0c}{1990/07/16}{Fixed some typ
% \changes{germanb-1.1}{1990/07/30}{When using PostScript fonts w
%    the Adobe fontencoding, the dieresis-accent is located elsewhe
%    modified co
% \changes{germanb-1.1a}{1990/08/27}{Modified the documentat
%    somewh
% \changes{germanb-2.0}{1991/04/23}{Modified for babel 3
% \changes{germanb-2.0a}{1991/05/25}{Removed some problems in cha
%    l
% \changes{germanb-2.1}{1991/05/29}{Removed bug found by van der Me
% \changes{germanb-2.2}{1991/06/11}{Removed global assignmen
%    brought uptodate with \file{german.tex} v2.
% \changes{germanb-2.2a}{1991/07/15}{Renamed \file{babel.sty}
%    \file{babel.co
% \changes{germanb-2.3}{1991/11/05}{Rewritten parts of the code to
%    the new features of babel version 3
% \changes{germanb-2.3e}{1991/11/10}{Brought up-to-date w
%    \file{german.tex} v2.3e (plus some bug fixes) [b
% \changes{germanb-2.5}{1994/02/08}{Update or \LaTe
% \changes{germanb-2.5c}{1994/06/26}{Removed the use of \cs{fileda
%    and moved the identification after the loading
%    \file{babel.de
% \changes{germanb-2.6a}{1995/02/15}{Moved the identification to
%    top of the fi
% \changes{germanb-2.6a}{1995/02/15}{Rewrote the code that handles
%    active double quote charact
% \changes{germanb-2.6d}{1996/07/10}{Replaced \cs{undefined} w
%    \cs{@undefined} and \cs{empty} with \cs{@empty} for consiste
%    with \LaTe
% \changes{germanb-2.6d}{1996/10/10}{Moved the definition
%    \cs{atcatcode} right to the beginning

%  \section{The German langua

%    The file \file{\filename}\footnote{The file described in t
%    section has version number \fileversion\ and was last revised
%    \filedate.}  defines all the language definition macros for
%    German language as well as for the Austrian dialect of t
%    language\footnote{This file is a re-implementation of Hub
%    Partl's \file{german.sty} version 2.5b, see~\cite{HP}

%    For this language the character |"| is made active.
%    table~\ref{tab:german-quote} an overview is given of
%    purpose. One of the reasons for this is that in the Ger
%    language some character combinations change when a word is bro
%    between the combination. Also the vertical placement of
%    umlaut can be controlled this w
%    \begin{table}[h
%     \begin{cent
%     \begin{tabular}{lp{8c
%      |"a| & |\"a|, also implemented for the ot
%                  lowercase and uppercase vowels.
%      |"s| & to produce the German \ss{} (like |\ss{}|).
%      |"z| & to produce the German \ss{} (like |\ss{}|).
%      |"ck|& for |ck| to be hyphenated as |k-k|.
%      |"ff|& for |ff| to be hyphenated as |ff-
%                  this is also implemented for l, m, n, p, r and
%      |"S| & for |SS| to be |\uppercase{"s}|.
%      |"Z| & for |SZ| to be |\uppercase{"z}|.
%      \verb="|= & disable ligature at this position.
%      |"-| & an explicit hyphen sign, allowing hyphenat
%             in the rest of the word.
%      |""| & like |"-|, but producing no hyphen s
%             (for compund words with hyphen, e.g.\ |x-""y|).
%      |"~| & for a compound word mark without a breakpoint.
%      |"=| & for a compound word mark with a breakpoint, allow
%             hyphenation in the composing words.
%      |"`| & for German left double quotes (looks like ,,).
%      |"'| & for German right double quotes.
%      |"<| & for French left double quotes (similar to $<<$).
%      |">| & for French right double quotes (similar to $>>$).
%     \end{tabul
%     \caption{The extra definitions m
%              by \file{german.ldf}}\label{tab:german-quo
%     \end{cent
%    \end{tab
%    The quotes in table~\ref{tab:german-quote} can also be typeset
%    using the commands in table~\ref{tab:more-quot
%    \begin{table}[h
%     \begin{cent
%     \begin{tabular}{lp{8c
%      |\glqq| & for German left double quotes (looks like ,,).
%      |\grqq| & for German right double quotes (looks like ``).
%      |\glq|  & for German left single quotes (looks like ,).
%      |\grq|  & for German right single quotes (looks like `).
%      |\flqq| & for French left double quotes (similar to $<<$).
%      |\frqq| & for French right double quotes (similar to $>>$)
%      |\flq|  & for (French) left single quotes (similar to $<$).
%      |\frq|  & for (French) right single quotes (similar to $>$).
%      |\dq|   & the original quotes character (|"|).
%     \end{tabul
%     \caption{More commands which produce quotes, defi
%              by \file{german.ldf}}\label{tab:more-quo
%     \end{cent
%    \end{tab

% \StopEventuall

%    When this file was read through the option \Lopt{germanb} we m
%    it behave as if \Lopt{german} was specifi
% \changes{german-2.6l}{2008/03/17}{Making germanb behave like ger
%    needs some more work besides defining \cs{CurrentOptio
% \changes{germanb-2.6m}{2008/06/01}{Correted a ty
%    \begin{macroco
\def\bbl@tempa{germa
\ifx\CurrentOption\bbl@te
  \def\CurrentOption{germ
  \ifx\l@german\@undefi
    \@nopatterns{Germ
    \adddialect\l@germ

  \let\l@germanb\l@ger
  \AtBeginDocumen
    \let\captionsgermanb\captionsger
    \let\dategermanb\dateger
    \let\extrasgermanb\extrasger
    \let\noextrasgermanb\noextrasger


%    \end{macroco

%    The macro |\LdfInit| takes care of preventing that this file
%    loaded more than once, checking the category code of
%    \texttt{@} sign, e
% \changes{germanb-2.6d}{1996/11/02}{Now use \cs{LdfInit} to perf
%    initial check
%    \begin{macroco
%<*co
\LdfInit\CurrentOption{captions\CurrentOpti
%    \end{macroco

%    When this file is read as an option, i.e., by the |\usepacka
%    command, \texttt{german} will be an `unknown' language, so
%    have to make it known.  So we check for the existence
%    |\l@german| to see whether we have to do something he

% \changes{germanb-2.0}{1991/04/23}{Now use \cs{adddialect}
%    language undefin
% \changes{germanb-2.2d}{1991/10/27}{Removed use of \cs{@ifundefine
% \changes{germanb-2.3e}{1991/11/10}{Added warning, if no ger
%    patterns load
% \changes{germanb-2.5c}{1994/06/26}{Now use \cs{@nopatterns}
%    produce the warni
%    \begin{macroco
\ifx\l@german\@undefi
  \@nopatterns{Germ
  \adddialect\l@germ

%    \end{macroco

%    For the Austrian version of these definitions we just add anot
%    languag
% \changes{germanb-2.0}{1991/04/23}{Now use \cs{adddialect}
%    austri
%    \begin{macroco
\adddialect\l@austrian\l@ger
%    \end{macroco

%    The next step consists of defining commands to switch to (
%    from) the German langua

%  \begin{macro}{\captionsgerm
%  \begin{macro}{\captionsaustri
%    Either the macro |\captionsgerman| or the ma
%    |\captionsaustrian| will define all strings used in the f
%    standard document classes provided with \LaT

% \changes{germanb-2.2}{1991/06/06}{Removed \cs{global} definitio
% \changes{germanb-2.2}{1991/06/06}{\cs{pagename} should
%    \cs{headpagenam
% \changes{germanb-2.3e}{1991/11/10}{Added \cs{prefacenam
%    \cs{seename} and \cs{alsonam
% \changes{germanb-2.4}{1993/07/15}{\cs{headpagename} should
%    \cs{pagenam
% \changes{germanb-2.6b}{1995/07/04}{Added \cs{proofname}
%    AMS-\LaT
% \changes{germanb-2.6d}{1996/07/10}{Construct control sequence on
%    f
% \changes{germanb-2.6j}{2000/09/20}{Added \cs{glossarynam
%    \begin{macroco
\@namedef{captions\CurrentOption
  \def\prefacename{Vorwor
  \def\refname{Literatu
  \def\abstractname{Zusammenfassun
  \def\bibname{Literaturverzeichni
  \def\chaptername{Kapite
  \def\appendixname{Anhan
  \def\contentsname{Inhaltsverzeichnis}%    % oder nur: Inh
  \def\listfigurename{Abbildungsverzeichni
  \def\listtablename{Tabellenverzeichni
  \def\indexname{Inde
  \def\figurename{Abbildun
  \def\tablename{Tabelle}%                  % oder: Ta
  \def\partname{Tei
  \def\enclname{Anlage(n)}%                 % oder: Beilage
  \def\ccname{Verteiler}%                   % oder: Kopien
  \def\headtoname{A
  \def\pagename{Seit
  \def\seename{sieh
  \def\alsoname{siehe auc
  \def\proofname{Bewei
  \def\glossaryname{Glossa

%    \end{macroco
%  \end{mac
%  \end{mac

%  \begin{macro}{\dategerm
%    The macro |\dategerman| redefines the comm
%    |\today| to produce German dat
% \changes{germanb-2.3e}{1991/11/10}{Added \cs{month@germa
% \changes{germanb-2.6f}{1997/10/01}{Use \cs{edef} to def
%    \cs{today} to save memo
% \changes{germanb-2.6f}{1998/03/28}{use \cs{def} instead
%    \cs{ede
%    \begin{macroco
\def\month@german{\ifcase\month
  Januar\or Februar\or M\"arz\or April\or Mai\or Juni
  Juli\or August\or September\or Oktober\or November\or Dezember\
\def\dategerman{\def\today{\number\day.~\month@ger
    \space\number\yea
%    \end{macroco
%  \end{mac

%  \begin{macro}{\dateaustri
%    The macro |\dateaustrian| redefines the comm
%    |\today| to produce Austrian version of the German dat
% \changes{germanb-2.6f}{1997/10/01}{Use \cs{edef} to def
%    \cs{today} to save memo
% \changes{germanb-2.6f}{1998/03/28}{use \cs{def} instead
%    \cs{ede
%    \begin{macroco
\def\dateaustrian{\def\today{\number\day.~\ifnum1=\mo
  J\"anner\else \month@german\fi \space\number\yea
%    \end{macroco
%  \end{mac

%  \begin{macro}{\extrasgerm
%  \begin{macro}{\extrasaustri
% \changes{germanb-2.0b}{1991/05/29}{added some comment chars
%    prevent white spa
% \changes{germanb-2.2}{1991/06/11}{Save all redefined macr
%  \begin{macro}{\noextrasgerm
%  \begin{macro}{\noextrasaustri
% \changes{germanb-1.1}{1990/07/30}{Added \cs{dieresi
% \changes{germanb-2.0b}{1991/05/29}{added some comment chars
%    prevent white spa
% \changes{germanb-2.2}{1991/06/11}{Try to restore everything to
%    former sta
% \changes{germanb-2.6d}{1996/07/10}{Construct control seque
%    \cs{extrasgerman} or \cs{extrasaustrian} on the f

%    Either the macro |\extrasgerman| or the macros |\extrasaustri
%    will perform all the extra definitions needed for the Ger
%    language. The macro |\noextrasgerman| is used to cancel
%    actions of |\extrasgerman

%    For German (as well as for Dutch) the \texttt{"} character
%    made active. This is done once, later on its definition may va
%    \begin{macroco
\initiate@active@char
\@namedef{extras\CurrentOption
  \languageshorthands{germa
\expandafter\addto\csname extras\CurrentOption\endcsnam
  \bbl@activate{
%    \end{macroco
%    Don't forget to turn the shorthands off aga
% \changes{germanb-2.6i}{1999/12/16}{Deactivate shorthands ouside
%    Germ
%    \begin{macroco
\addto\noextrasgerman{\bbl@deactivate{
%    \end{macroco

% \changes{germanb-2.6a}{1995/02/15}{All the code to handle the act
%    double quote has been moved to \file{babel.de

%    In order for \TeX\ to be able to hyphenate German words wh
%    contain `\ss' (in the \texttt{OT1} position |^^Y|) we have
%    give the character a nonzero |\lccode| (see Appendix H, the \
%    boo
% \changes{germanb-2.6c}{1996/04/08}{Use decimal number instead
%    hat-notation as the hat may be activat
%    \begin{macroco
\expandafter\addto\csname extras\CurrentOption\endcsnam
  \babel@savevariable{\lccode2
  \lccode25=
%    \end{macroco
% \changes{germanb-2.6a}{1995/02/15}{Removeed \cs{3} as it is
%    longer in \file{german.ld

%    The umlaut accent macro |\"| is changed to lower the umlaut do
%    The redefinition is done with the help of |\umlautlo
%    \begin{macroco
\expandafter\addto\csname extras\CurrentOption\endcsnam
  \babel@save\"\umlautl
\@namedef{noextras\CurrentOption}{\umlauthi
%    \end{macroco
%    The german hyphenation patterns can be used with |\lefthyphenm
%    and |\righthyphenmin| set to
% \changes{germanb-2.6a}{1995/05/13}{use \cs{germanhyphenmins} to st
%    the correct valu
% \changes{germanb-2.6j}{2000/09/22}{Now use \cs{providehyphenmins}
%    provide a default val
%    \begin{macroco
\providehyphenmins{\CurrentOption}{\tw@\t
%    \end{macroco
%    For German texts we need to make sure that |\frenchspacing|
%    turned
% \changes{germanb-2.6k}{2001/01/26}{Turn frenchspacing on, as
%    \texttt{german.st
%    \begin{macroco
\expandafter\addto\csname extras\CurrentOption\endcsnam
  \bbl@frenchspaci
\expandafter\addto\csname noextras\CurrentOption\endcsnam
  \bbl@nonfrenchspaci
%    \end{macroco
%  \end{mac
%  \end{mac
%  \end{mac
%  \end{mac

% \changes{germanb-2.6a}{1995/02/15}{\cs{umlautlow}
%    \cs{umlauthigh} moved to \file{glyphs.dtx}, as well
%    \cs{newumlaut} (now \cs{lower@umlau

%    The code above is necessary because we need an extra act
%    character. This character is then used as indicated
%    table~\ref{tab:german-quot

%    To be able to define the function of |"|, we first defin
%    couple of `support' macr

% \changes{germanb-2.3e}{1991/11/10}{Added \cs{save@sf@q} macro
%    rewrote all quote macros to use
% \changes{germanb-2.3h}{1991/02/16}{moved definition
%    \cs{allowhyphens}, \cs{set@low@box} and \cs{save@sf@q}
%    \file{babel.co
% \changes{germanb-2.6a}{1995/02/15}{Moved all quotation characters
%    \file{glyphs.dt

%  \begin{macro}{\
%    We save the original double quote character in |\dq| to k
%    it available, the math accent |\"| can now be typed as |
%    \begin{macroco
\begingroup \catcode`\
\def\x{\endgr
  \def\@SS{\mathchar"701
  \def\dq{

%    \end{macroco
%  \end{mac
% \changes{germanb-2.6c}{1996/01/24}{Moved \cs{german@dq@disc}
%    babel.def, calling it \cs{bbl@dis

% \changes{germanb-2.6a}{1995/02/15}{Use \cs{ddot} instead
%    \cs{@MATHUMLAU

%    Now we can define the doublequote macros: the umlau
% \changes{germanb-2.6c}{1996/05/30}{added the \cs{allowhyphen
%    \begin{macroco
\declare@shorthand{german}{"a}{\textormath{\"{a}\allowhyphens}{\ddot
\declare@shorthand{german}{"o}{\textormath{\"{o}\allowhyphens}{\ddot
\declare@shorthand{german}{"u}{\textormath{\"{u}\allowhyphens}{\ddot
\declare@shorthand{german}{"A}{\textormath{\"{A}\allowhyphens}{\ddot
\declare@shorthand{german}{"O}{\textormath{\"{O}\allowhyphens}{\ddot
\declare@shorthand{german}{"U}{\textormath{\"{U}\allowhyphens}{\ddot
%    \end{macroco
%    trem
%    \begin{macroco
\declare@shorthand{german}{"e}{\textormath{\"{e}}{\ddot
\declare@shorthand{german}{"E}{\textormath{\"{E}}{\ddot
\declare@shorthand{german}{"i}{\textormath{\"{\i
                              {\ddot\imat
\declare@shorthand{german}{"I}{\textormath{\"{I}}{\ddot
%    \end{macroco
%    german es-zet (sharp
% \changes{germanb-2.6f}{1997/05/08}{use \cs{SS} instead
%    \texttt{SS}, removed braces after \cs{ss
%    \begin{macroco
\declare@shorthand{german}{"s}{\textormath{\ss}{\@SS{
\declare@shorthand{german}{"S}{\
\declare@shorthand{german}{"z}{\textormath{\ss}{\@SS{
\declare@shorthand{german}{"Z}{
%    \end{macroco
%    german and french quot
%    \begin{macroco
\declare@shorthand{german}{"`}{\gl
\declare@shorthand{german}{"'}{\gr
\declare@shorthand{german}{"<}{\fl
\declare@shorthand{german}{">}{\fr
%    \end{macroco
%    discretionary comma
%    \begin{macroco
\declare@shorthand{german}{"c}{\textormath{\bbl@disc ck}{
\declare@shorthand{german}{"C}{\textormath{\bbl@disc CK}{
\declare@shorthand{german}{"F}{\textormath{\bbl@disc F{FF}}{
\declare@shorthand{german}{"l}{\textormath{\bbl@disc l{ll}}{
\declare@shorthand{german}{"L}{\textormath{\bbl@disc L{LL}}{
\declare@shorthand{german}{"m}{\textormath{\bbl@disc m{mm}}{
\declare@shorthand{german}{"M}{\textormath{\bbl@disc M{MM}}{
\declare@shorthand{german}{"n}{\textormath{\bbl@disc n{nn}}{
\declare@shorthand{german}{"N}{\textormath{\bbl@disc N{NN}}{
\declare@shorthand{german}{"p}{\textormath{\bbl@disc p{pp}}{
\declare@shorthand{german}{"P}{\textormath{\bbl@disc P{PP}}{
\declare@shorthand{german}{"r}{\textormath{\bbl@disc r{rr}}{
\declare@shorthand{german}{"R}{\textormath{\bbl@disc R{RR}}{
\declare@shorthand{german}{"t}{\textormath{\bbl@disc t{tt}}{
\declare@shorthand{german}{"T}{\textormath{\bbl@disc T{TT}}{
%    \end{macroco
%    We need to treat |"f| a bit differently in order to preserve
%    ff-ligatur
% \changes{germanb-2.6f}{1998/06/15}{Copied the coding for \texttt{
%    from german.dtx version 2.5
%    \begin{macroco
\declare@shorthand{german}{"f}{\textormath{\bbl@discff}{
\def\bbl@discff{\penalty
  \afterassignment\bbl@insertff \let\bbl@nextff
\def\bbl@insertf
  \if f\bbl@nex
    \expandafter\@firstoftwo\else\expandafter\@secondoftwo
  {\relax\discretionary{ff-}{f}{ff}\allowhyphens}{f\bbl@nextf
\let\bbl@nextf
%    \end{macroco
%    and some additional comman
%    \begin{macroco
\declare@shorthand{german}{"-}{\nobreak\-\bbl@allowhyphe
\declare@shorthand{german}{"|
  \textormath{\penalty\@M\discretionary{-}{}{\kern.03e
              \allowhyphens}
\declare@shorthand{german}{""}{\hskip\z@sk
\declare@shorthand{german}{"~}{\textormath{\leavevmode\hbox{-}}{
\declare@shorthand{german}{"=}{\penalty\@M-\hskip\z@sk
%    \end{macroco

%  \begin{macro}{\mdq
%  \begin{macro}{\mdqo
%  \begin{macro}{\
%    All that's left to do now is to  define a couple of comma
%    for reasons of compatibility with \file{german.st
% \changes{germanb-2.6f}{1998/06/07}{Now use \cs{shorthandon}
%    \cs{shorthandoff
%    \begin{macroco
\def\mdqon{\shorthandon{
\def\mdqoff{\shorthandoff{
\def\ck{\allowhyphens\discretionary{k-}{k}{ck}\allowhyphe
%    \end{macroco
%  \end{mac
%  \end{mac
%  \end{mac

%    The macro |\ldf@finish| takes care of looking fo
%    configuration file, setting the main language to be switched
%    at |\begin{document}| and resetting the category code
%    \texttt{@} to its original val
% \changes{germanb-2.6d}{1996/11/02}{Now use \cs{ldf@finish} to w
%    u
%    \begin{macroco
\ldf@finish\CurrentOpt
%</co
%    \end{macroco

% \Fin

%% \CharacterTa
%%  {Upper-case    \A\B\C\D\E\F\G\H\I\J\K\L\M\N\O\P\Q\R\S\T\U\V\W\X\
%%   Lower-case    \a\b\c\d\e\f\g\h\i\j\k\l\m\n\o\p\q\r\s\t\u\v\w\x\
%%   Digits        \0\1\2\3\4\5\6\7\
%%   Exclamation   \!     Double quote  \"     Hash (number)
%%   Dollar        \$     Percent       \%     Ampersand
%%   Acute accent  \'     Left paren    \(     Right paren
%%   Asterisk      \*     Plus          \+     Comma
%%   Minus         \-     Point         \.     Solidus
%%   Colon         \:     Semicolon     \;     Less than
%%   Equals        \=     Greater than  \>     Question mark
%%   Commercial at \@     Left bracket  \[     Backslash
%%   Right bracket \]     Circumflex    \^     Underscore
%%   Grave accent  \`     Left brace    \{     Vertical bar
%%   Right brace   \}     Tilde

\endin
}
\DeclareOption{germanb}{% \iffalse meta-comm

% Copyright 1989-2008 Johannes L. Braams and any individual auth
% listed elsewhere in this file.  All rights reserv

% This file is part of the Babel syst
% -----------------------------------

% It may be distributed and/or modified under
% conditions of the LaTeX Project Public License, either version
% of this license or (at your option) any later versi
% The latest version of this license is
%   http://www.latex-project.org/lppl.
% and version 1.3 or later is part of all distributions of La
% version 2003/12/01 or lat

% This work has the LPPL maintenance status "maintaine

% The Current Maintainer of this work is Johannes Braa

% The list of all files belonging to the Babel system
% given in the file `manifest.bbl. See also `legal.bbl' for additio
% informati

% The list of derived (unpacked) files belonging to the distribut
% and covered by LPPL is defined by the unpacking scripts (w
% extension .ins) which are part of the distributi
%
% \CheckSum{3

% \iffa
%    Tell the \LaTeX\ system who we are and write an entry on
%    transcri
%<*d
\ProvidesFile{germanb.d
%</d
%<code>\ProvidesLanguage{germa
%
%\ProvidesFile{germanb.d
        [2008/06/01 v2.6m German support from the babel syst
%\iffa
%% File `germanb.d
%% Babel package for LaTeX version
%% Copyright (C) 1989 - 2
%%           by Johannes Braams, TeXn

%% Germanb Language Definition F
%% Copyright (C) 1989 - 2
%%           by Bernd Raichle raichle at azu.Informatik.Uni-Stuttgart
%%              Johannes Braams, TeXn
% This file is based on german.tex version 2.
%                       by Bernd Raichle, Hubert Partl et.

%% Please report errors to: J.L. Bra
%%                          babel at braams.xs4all

%<*filedriv
\documentclass{ltxd
\font\manual=logo10 % font used for the METAFONT logo, e
\newcommand*\MF{{\manual META}\-{\manual FON
\newcommand*\TeXhax{\TeX h
\newcommand*\babel{\textsf{babe
\newcommand*\langvar{$\langle \it lang \rangl
\newcommand*\note[1
\newcommand*\Lopt[1]{\textsf{#
\newcommand*\file[1]{\texttt{#
\begin{docume
 \DocInput{germanb.d
\end{docume
%</filedriv
%
% \GetFileInfo{germanb.d

% \changes{germanb-1.0a}{1990/05/14}{Incorporated Nico's commen
% \changes{germanb-1.0b}{1990/05/22}{fixed typo in definition
%    austrian language found by Werenfried S
%    \texttt{nspit@fys.ruu.n
% \changes{germanb-1.0c}{1990/07/16}{Fixed some typ
% \changes{germanb-1.1}{1990/07/30}{When using PostScript fonts w
%    the Adobe fontencoding, the dieresis-accent is located elsewhe
%    modified co
% \changes{germanb-1.1a}{1990/08/27}{Modified the documentat
%    somewh
% \changes{germanb-2.0}{1991/04/23}{Modified for babel 3
% \changes{germanb-2.0a}{1991/05/25}{Removed some problems in cha
%    l
% \changes{germanb-2.1}{1991/05/29}{Removed bug found by van der Me
% \changes{germanb-2.2}{1991/06/11}{Removed global assignmen
%    brought uptodate with \file{german.tex} v2.
% \changes{germanb-2.2a}{1991/07/15}{Renamed \file{babel.sty}
%    \file{babel.co
% \changes{germanb-2.3}{1991/11/05}{Rewritten parts of the code to
%    the new features of babel version 3
% \changes{germanb-2.3e}{1991/11/10}{Brought up-to-date w
%    \file{german.tex} v2.3e (plus some bug fixes) [b
% \changes{germanb-2.5}{1994/02/08}{Update or \LaTe
% \changes{germanb-2.5c}{1994/06/26}{Removed the use of \cs{fileda
%    and moved the identification after the loading
%    \file{babel.de
% \changes{germanb-2.6a}{1995/02/15}{Moved the identification to
%    top of the fi
% \changes{germanb-2.6a}{1995/02/15}{Rewrote the code that handles
%    active double quote charact
% \changes{germanb-2.6d}{1996/07/10}{Replaced \cs{undefined} w
%    \cs{@undefined} and \cs{empty} with \cs{@empty} for consiste
%    with \LaTe
% \changes{germanb-2.6d}{1996/10/10}{Moved the definition
%    \cs{atcatcode} right to the beginning

%  \section{The German langua

%    The file \file{\filename}\footnote{The file described in t
%    section has version number \fileversion\ and was last revised
%    \filedate.}  defines all the language definition macros for
%    German language as well as for the Austrian dialect of t
%    language\footnote{This file is a re-implementation of Hub
%    Partl's \file{german.sty} version 2.5b, see~\cite{HP}

%    For this language the character |"| is made active.
%    table~\ref{tab:german-quote} an overview is given of
%    purpose. One of the reasons for this is that in the Ger
%    language some character combinations change when a word is bro
%    between the combination. Also the vertical placement of
%    umlaut can be controlled this w
%    \begin{table}[h
%     \begin{cent
%     \begin{tabular}{lp{8c
%      |"a| & |\"a|, also implemented for the ot
%                  lowercase and uppercase vowels.
%      |"s| & to produce the German \ss{} (like |\ss{}|).
%      |"z| & to produce the German \ss{} (like |\ss{}|).
%      |"ck|& for |ck| to be hyphenated as |k-k|.
%      |"ff|& for |ff| to be hyphenated as |ff-
%                  this is also implemented for l, m, n, p, r and
%      |"S| & for |SS| to be |\uppercase{"s}|.
%      |"Z| & for |SZ| to be |\uppercase{"z}|.
%      \verb="|= & disable ligature at this position.
%      |"-| & an explicit hyphen sign, allowing hyphenat
%             in the rest of the word.
%      |""| & like |"-|, but producing no hyphen s
%             (for compund words with hyphen, e.g.\ |x-""y|).
%      |"~| & for a compound word mark without a breakpoint.
%      |"=| & for a compound word mark with a breakpoint, allow
%             hyphenation in the composing words.
%      |"`| & for German left double quotes (looks like ,,).
%      |"'| & for German right double quotes.
%      |"<| & for French left double quotes (similar to $<<$).
%      |">| & for French right double quotes (similar to $>>$).
%     \end{tabul
%     \caption{The extra definitions m
%              by \file{german.ldf}}\label{tab:german-quo
%     \end{cent
%    \end{tab
%    The quotes in table~\ref{tab:german-quote} can also be typeset
%    using the commands in table~\ref{tab:more-quot
%    \begin{table}[h
%     \begin{cent
%     \begin{tabular}{lp{8c
%      |\glqq| & for German left double quotes (looks like ,,).
%      |\grqq| & for German right double quotes (looks like ``).
%      |\glq|  & for German left single quotes (looks like ,).
%      |\grq|  & for German right single quotes (looks like `).
%      |\flqq| & for French left double quotes (similar to $<<$).
%      |\frqq| & for French right double quotes (similar to $>>$)
%      |\flq|  & for (French) left single quotes (similar to $<$).
%      |\frq|  & for (French) right single quotes (similar to $>$).
%      |\dq|   & the original quotes character (|"|).
%     \end{tabul
%     \caption{More commands which produce quotes, defi
%              by \file{german.ldf}}\label{tab:more-quo
%     \end{cent
%    \end{tab

% \StopEventuall

%    When this file was read through the option \Lopt{germanb} we m
%    it behave as if \Lopt{german} was specifi
% \changes{german-2.6l}{2008/03/17}{Making germanb behave like ger
%    needs some more work besides defining \cs{CurrentOptio
% \changes{germanb-2.6m}{2008/06/01}{Correted a ty
%    \begin{macroco
\def\bbl@tempa{germa
\ifx\CurrentOption\bbl@te
  \def\CurrentOption{germ
  \ifx\l@german\@undefi
    \@nopatterns{Germ
    \adddialect\l@germ

  \let\l@germanb\l@ger
  \AtBeginDocumen
    \let\captionsgermanb\captionsger
    \let\dategermanb\dateger
    \let\extrasgermanb\extrasger
    \let\noextrasgermanb\noextrasger


%    \end{macroco

%    The macro |\LdfInit| takes care of preventing that this file
%    loaded more than once, checking the category code of
%    \texttt{@} sign, e
% \changes{germanb-2.6d}{1996/11/02}{Now use \cs{LdfInit} to perf
%    initial check
%    \begin{macroco
%<*co
\LdfInit\CurrentOption{captions\CurrentOpti
%    \end{macroco

%    When this file is read as an option, i.e., by the |\usepacka
%    command, \texttt{german} will be an `unknown' language, so
%    have to make it known.  So we check for the existence
%    |\l@german| to see whether we have to do something he

% \changes{germanb-2.0}{1991/04/23}{Now use \cs{adddialect}
%    language undefin
% \changes{germanb-2.2d}{1991/10/27}{Removed use of \cs{@ifundefine
% \changes{germanb-2.3e}{1991/11/10}{Added warning, if no ger
%    patterns load
% \changes{germanb-2.5c}{1994/06/26}{Now use \cs{@nopatterns}
%    produce the warni
%    \begin{macroco
\ifx\l@german\@undefi
  \@nopatterns{Germ
  \adddialect\l@germ

%    \end{macroco

%    For the Austrian version of these definitions we just add anot
%    languag
% \changes{germanb-2.0}{1991/04/23}{Now use \cs{adddialect}
%    austri
%    \begin{macroco
\adddialect\l@austrian\l@ger
%    \end{macroco

%    The next step consists of defining commands to switch to (
%    from) the German langua

%  \begin{macro}{\captionsgerm
%  \begin{macro}{\captionsaustri
%    Either the macro |\captionsgerman| or the ma
%    |\captionsaustrian| will define all strings used in the f
%    standard document classes provided with \LaT

% \changes{germanb-2.2}{1991/06/06}{Removed \cs{global} definitio
% \changes{germanb-2.2}{1991/06/06}{\cs{pagename} should
%    \cs{headpagenam
% \changes{germanb-2.3e}{1991/11/10}{Added \cs{prefacenam
%    \cs{seename} and \cs{alsonam
% \changes{germanb-2.4}{1993/07/15}{\cs{headpagename} should
%    \cs{pagenam
% \changes{germanb-2.6b}{1995/07/04}{Added \cs{proofname}
%    AMS-\LaT
% \changes{germanb-2.6d}{1996/07/10}{Construct control sequence on
%    f
% \changes{germanb-2.6j}{2000/09/20}{Added \cs{glossarynam
%    \begin{macroco
\@namedef{captions\CurrentOption
  \def\prefacename{Vorwor
  \def\refname{Literatu
  \def\abstractname{Zusammenfassun
  \def\bibname{Literaturverzeichni
  \def\chaptername{Kapite
  \def\appendixname{Anhan
  \def\contentsname{Inhaltsverzeichnis}%    % oder nur: Inh
  \def\listfigurename{Abbildungsverzeichni
  \def\listtablename{Tabellenverzeichni
  \def\indexname{Inde
  \def\figurename{Abbildun
  \def\tablename{Tabelle}%                  % oder: Ta
  \def\partname{Tei
  \def\enclname{Anlage(n)}%                 % oder: Beilage
  \def\ccname{Verteiler}%                   % oder: Kopien
  \def\headtoname{A
  \def\pagename{Seit
  \def\seename{sieh
  \def\alsoname{siehe auc
  \def\proofname{Bewei
  \def\glossaryname{Glossa

%    \end{macroco
%  \end{mac
%  \end{mac

%  \begin{macro}{\dategerm
%    The macro |\dategerman| redefines the comm
%    |\today| to produce German dat
% \changes{germanb-2.3e}{1991/11/10}{Added \cs{month@germa
% \changes{germanb-2.6f}{1997/10/01}{Use \cs{edef} to def
%    \cs{today} to save memo
% \changes{germanb-2.6f}{1998/03/28}{use \cs{def} instead
%    \cs{ede
%    \begin{macroco
\def\month@german{\ifcase\month
  Januar\or Februar\or M\"arz\or April\or Mai\or Juni
  Juli\or August\or September\or Oktober\or November\or Dezember\
\def\dategerman{\def\today{\number\day.~\month@ger
    \space\number\yea
%    \end{macroco
%  \end{mac

%  \begin{macro}{\dateaustri
%    The macro |\dateaustrian| redefines the comm
%    |\today| to produce Austrian version of the German dat
% \changes{germanb-2.6f}{1997/10/01}{Use \cs{edef} to def
%    \cs{today} to save memo
% \changes{germanb-2.6f}{1998/03/28}{use \cs{def} instead
%    \cs{ede
%    \begin{macroco
\def\dateaustrian{\def\today{\number\day.~\ifnum1=\mo
  J\"anner\else \month@german\fi \space\number\yea
%    \end{macroco
%  \end{mac

%  \begin{macro}{\extrasgerm
%  \begin{macro}{\extrasaustri
% \changes{germanb-2.0b}{1991/05/29}{added some comment chars
%    prevent white spa
% \changes{germanb-2.2}{1991/06/11}{Save all redefined macr
%  \begin{macro}{\noextrasgerm
%  \begin{macro}{\noextrasaustri
% \changes{germanb-1.1}{1990/07/30}{Added \cs{dieresi
% \changes{germanb-2.0b}{1991/05/29}{added some comment chars
%    prevent white spa
% \changes{germanb-2.2}{1991/06/11}{Try to restore everything to
%    former sta
% \changes{germanb-2.6d}{1996/07/10}{Construct control seque
%    \cs{extrasgerman} or \cs{extrasaustrian} on the f

%    Either the macro |\extrasgerman| or the macros |\extrasaustri
%    will perform all the extra definitions needed for the Ger
%    language. The macro |\noextrasgerman| is used to cancel
%    actions of |\extrasgerman

%    For German (as well as for Dutch) the \texttt{"} character
%    made active. This is done once, later on its definition may va
%    \begin{macroco
\initiate@active@char
\@namedef{extras\CurrentOption
  \languageshorthands{germa
\expandafter\addto\csname extras\CurrentOption\endcsnam
  \bbl@activate{
%    \end{macroco
%    Don't forget to turn the shorthands off aga
% \changes{germanb-2.6i}{1999/12/16}{Deactivate shorthands ouside
%    Germ
%    \begin{macroco
\addto\noextrasgerman{\bbl@deactivate{
%    \end{macroco

% \changes{germanb-2.6a}{1995/02/15}{All the code to handle the act
%    double quote has been moved to \file{babel.de

%    In order for \TeX\ to be able to hyphenate German words wh
%    contain `\ss' (in the \texttt{OT1} position |^^Y|) we have
%    give the character a nonzero |\lccode| (see Appendix H, the \
%    boo
% \changes{germanb-2.6c}{1996/04/08}{Use decimal number instead
%    hat-notation as the hat may be activat
%    \begin{macroco
\expandafter\addto\csname extras\CurrentOption\endcsnam
  \babel@savevariable{\lccode2
  \lccode25=
%    \end{macroco
% \changes{germanb-2.6a}{1995/02/15}{Removeed \cs{3} as it is
%    longer in \file{german.ld

%    The umlaut accent macro |\"| is changed to lower the umlaut do
%    The redefinition is done with the help of |\umlautlo
%    \begin{macroco
\expandafter\addto\csname extras\CurrentOption\endcsnam
  \babel@save\"\umlautl
\@namedef{noextras\CurrentOption}{\umlauthi
%    \end{macroco
%    The german hyphenation patterns can be used with |\lefthyphenm
%    and |\righthyphenmin| set to
% \changes{germanb-2.6a}{1995/05/13}{use \cs{germanhyphenmins} to st
%    the correct valu
% \changes{germanb-2.6j}{2000/09/22}{Now use \cs{providehyphenmins}
%    provide a default val
%    \begin{macroco
\providehyphenmins{\CurrentOption}{\tw@\t
%    \end{macroco
%    For German texts we need to make sure that |\frenchspacing|
%    turned
% \changes{germanb-2.6k}{2001/01/26}{Turn frenchspacing on, as
%    \texttt{german.st
%    \begin{macroco
\expandafter\addto\csname extras\CurrentOption\endcsnam
  \bbl@frenchspaci
\expandafter\addto\csname noextras\CurrentOption\endcsnam
  \bbl@nonfrenchspaci
%    \end{macroco
%  \end{mac
%  \end{mac
%  \end{mac
%  \end{mac

% \changes{germanb-2.6a}{1995/02/15}{\cs{umlautlow}
%    \cs{umlauthigh} moved to \file{glyphs.dtx}, as well
%    \cs{newumlaut} (now \cs{lower@umlau

%    The code above is necessary because we need an extra act
%    character. This character is then used as indicated
%    table~\ref{tab:german-quot

%    To be able to define the function of |"|, we first defin
%    couple of `support' macr

% \changes{germanb-2.3e}{1991/11/10}{Added \cs{save@sf@q} macro
%    rewrote all quote macros to use
% \changes{germanb-2.3h}{1991/02/16}{moved definition
%    \cs{allowhyphens}, \cs{set@low@box} and \cs{save@sf@q}
%    \file{babel.co
% \changes{germanb-2.6a}{1995/02/15}{Moved all quotation characters
%    \file{glyphs.dt

%  \begin{macro}{\
%    We save the original double quote character in |\dq| to k
%    it available, the math accent |\"| can now be typed as |
%    \begin{macroco
\begingroup \catcode`\
\def\x{\endgr
  \def\@SS{\mathchar"701
  \def\dq{

%    \end{macroco
%  \end{mac
% \changes{germanb-2.6c}{1996/01/24}{Moved \cs{german@dq@disc}
%    babel.def, calling it \cs{bbl@dis

% \changes{germanb-2.6a}{1995/02/15}{Use \cs{ddot} instead
%    \cs{@MATHUMLAU

%    Now we can define the doublequote macros: the umlau
% \changes{germanb-2.6c}{1996/05/30}{added the \cs{allowhyphen
%    \begin{macroco
\declare@shorthand{german}{"a}{\textormath{\"{a}\allowhyphens}{\ddot
\declare@shorthand{german}{"o}{\textormath{\"{o}\allowhyphens}{\ddot
\declare@shorthand{german}{"u}{\textormath{\"{u}\allowhyphens}{\ddot
\declare@shorthand{german}{"A}{\textormath{\"{A}\allowhyphens}{\ddot
\declare@shorthand{german}{"O}{\textormath{\"{O}\allowhyphens}{\ddot
\declare@shorthand{german}{"U}{\textormath{\"{U}\allowhyphens}{\ddot
%    \end{macroco
%    trem
%    \begin{macroco
\declare@shorthand{german}{"e}{\textormath{\"{e}}{\ddot
\declare@shorthand{german}{"E}{\textormath{\"{E}}{\ddot
\declare@shorthand{german}{"i}{\textormath{\"{\i
                              {\ddot\imat
\declare@shorthand{german}{"I}{\textormath{\"{I}}{\ddot
%    \end{macroco
%    german es-zet (sharp
% \changes{germanb-2.6f}{1997/05/08}{use \cs{SS} instead
%    \texttt{SS}, removed braces after \cs{ss
%    \begin{macroco
\declare@shorthand{german}{"s}{\textormath{\ss}{\@SS{
\declare@shorthand{german}{"S}{\
\declare@shorthand{german}{"z}{\textormath{\ss}{\@SS{
\declare@shorthand{german}{"Z}{
%    \end{macroco
%    german and french quot
%    \begin{macroco
\declare@shorthand{german}{"`}{\gl
\declare@shorthand{german}{"'}{\gr
\declare@shorthand{german}{"<}{\fl
\declare@shorthand{german}{">}{\fr
%    \end{macroco
%    discretionary comma
%    \begin{macroco
\declare@shorthand{german}{"c}{\textormath{\bbl@disc ck}{
\declare@shorthand{german}{"C}{\textormath{\bbl@disc CK}{
\declare@shorthand{german}{"F}{\textormath{\bbl@disc F{FF}}{
\declare@shorthand{german}{"l}{\textormath{\bbl@disc l{ll}}{
\declare@shorthand{german}{"L}{\textormath{\bbl@disc L{LL}}{
\declare@shorthand{german}{"m}{\textormath{\bbl@disc m{mm}}{
\declare@shorthand{german}{"M}{\textormath{\bbl@disc M{MM}}{
\declare@shorthand{german}{"n}{\textormath{\bbl@disc n{nn}}{
\declare@shorthand{german}{"N}{\textormath{\bbl@disc N{NN}}{
\declare@shorthand{german}{"p}{\textormath{\bbl@disc p{pp}}{
\declare@shorthand{german}{"P}{\textormath{\bbl@disc P{PP}}{
\declare@shorthand{german}{"r}{\textormath{\bbl@disc r{rr}}{
\declare@shorthand{german}{"R}{\textormath{\bbl@disc R{RR}}{
\declare@shorthand{german}{"t}{\textormath{\bbl@disc t{tt}}{
\declare@shorthand{german}{"T}{\textormath{\bbl@disc T{TT}}{
%    \end{macroco
%    We need to treat |"f| a bit differently in order to preserve
%    ff-ligatur
% \changes{germanb-2.6f}{1998/06/15}{Copied the coding for \texttt{
%    from german.dtx version 2.5
%    \begin{macroco
\declare@shorthand{german}{"f}{\textormath{\bbl@discff}{
\def\bbl@discff{\penalty
  \afterassignment\bbl@insertff \let\bbl@nextff
\def\bbl@insertf
  \if f\bbl@nex
    \expandafter\@firstoftwo\else\expandafter\@secondoftwo
  {\relax\discretionary{ff-}{f}{ff}\allowhyphens}{f\bbl@nextf
\let\bbl@nextf
%    \end{macroco
%    and some additional comman
%    \begin{macroco
\declare@shorthand{german}{"-}{\nobreak\-\bbl@allowhyphe
\declare@shorthand{german}{"|
  \textormath{\penalty\@M\discretionary{-}{}{\kern.03e
              \allowhyphens}
\declare@shorthand{german}{""}{\hskip\z@sk
\declare@shorthand{german}{"~}{\textormath{\leavevmode\hbox{-}}{
\declare@shorthand{german}{"=}{\penalty\@M-\hskip\z@sk
%    \end{macroco

%  \begin{macro}{\mdq
%  \begin{macro}{\mdqo
%  \begin{macro}{\
%    All that's left to do now is to  define a couple of comma
%    for reasons of compatibility with \file{german.st
% \changes{germanb-2.6f}{1998/06/07}{Now use \cs{shorthandon}
%    \cs{shorthandoff
%    \begin{macroco
\def\mdqon{\shorthandon{
\def\mdqoff{\shorthandoff{
\def\ck{\allowhyphens\discretionary{k-}{k}{ck}\allowhyphe
%    \end{macroco
%  \end{mac
%  \end{mac
%  \end{mac

%    The macro |\ldf@finish| takes care of looking fo
%    configuration file, setting the main language to be switched
%    at |\begin{document}| and resetting the category code
%    \texttt{@} to its original val
% \changes{germanb-2.6d}{1996/11/02}{Now use \cs{ldf@finish} to w
%    u
%    \begin{macroco
\ldf@finish\CurrentOpt
%</co
%    \end{macroco

% \Fin

%% \CharacterTa
%%  {Upper-case    \A\B\C\D\E\F\G\H\I\J\K\L\M\N\O\P\Q\R\S\T\U\V\W\X\
%%   Lower-case    \a\b\c\d\e\f\g\h\i\j\k\l\m\n\o\p\q\r\s\t\u\v\w\x\
%%   Digits        \0\1\2\3\4\5\6\7\
%%   Exclamation   \!     Double quote  \"     Hash (number)
%%   Dollar        \$     Percent       \%     Ampersand
%%   Acute accent  \'     Left paren    \(     Right paren
%%   Asterisk      \*     Plus          \+     Comma
%%   Minus         \-     Point         \.     Solidus
%%   Colon         \:     Semicolon     \;     Less than
%%   Equals        \=     Greater than  \>     Question mark
%%   Commercial at \@     Left bracket  \[     Backslash
%%   Right bracket \]     Circumflex    \^     Underscore
%%   Grave accent  \`     Left brace    \{     Vertical bar
%%   Right brace   \}     Tilde

\endin
}
%    \end{macrocode}
% \changes{babel~3.5f}{1996/05/31}{Added the \Lopt{greek} option}
% \changes{babel~3.7a}{1997/11/13}{Added the \Lopt{polutonikogreek}
%    option}
% \changes{babel~3.7c}{1999/04/22}{set the correct language attribute
%    for polutoniko greek}
%    \begin{macrocode}
\DeclareOption{greek}{% \iffalse meta-comment
%
% Copyright 1989-2008 Johannes L. Braams and any individual authors
% listed elsewhere in this file.  All rights reserved.
% 
% This file is part of the Babel system.
% --------------------------------------
% 
% It may be distributed and/or modified under the
% conditions of the LaTeX Project Public License, either version 1.3
% of this license or (at your option) any later version.
% The latest version of this license is in
%   http://www.latex-project.org/lppl.txt
% and version 1.3 or later is part of all distributions of LaTeX
% version 2003/12/01 or later.
% 
% This work has the LPPL maintenance status "maintained".
% 
% The Current Maintainer of this work is Johannes Braams.
% 
% The list of all files belonging to the Babel system is
% given in the file `manifest.bbl. See also `legal.bbl' for additional
% information.
% 
% The list of derived (unpacked) files belonging to the distribution
% and covered by LPPL is defined by the unpacking scripts (with
% extension .ins) which are part of the distribution.
% \fi
% \CheckSum{636}
%
% \iffalse
%    Tell the \LaTeX\ system who we are and write an entry on the
%    transcript.
%<*dtx>
\ProvidesFile{greek.dtx}
%</dtx>
%<code>\ProvidesLanguage{greek}
%\fi
%\ProvidesFile{greek.dtx}
        [2005/03/30 v1.3l Greek support from the babel system]
%\iffalse
%% File `greek.dtx'
%% Babel package for LaTeX version 2e
%% Copyright (C) 1989 -- 2005
%%           by Johannes Braams, TeXniek
%
%% Greek language Definition File
%% Copyright (C) 1997, 2005
%%           by Apostolos Syropoulos
%%              Johannes Braams, TeXniek
%
%% Please report errors to: Apostolos Syropoulos
%%                          apostolo at platon.ee.duth.gr or
%%                          apostolo at obelix.ee.duth.gr
%%                          (or J.L. Braams <babel at braams.cistron.nl)
%
%    This file is part of the babel system, it provides the source
%    code for the greek language definition file. The original
%    version of this file was written by Apostolos Syropoulos.
%    It was then enhanced by adding code from kdgreek.sty from David
%    Kastrup <dak@neuroinformatik.ruhr-uni-bochum.de> with his
%    consent. 
%<*filedriver>
\documentclass{ltxdoc}
\newcommand*{\TeXhax}{\TeX hax}
\newcommand*{\babel}{\textsf{babel}}
\newcommand*{\langvar}{$\langle \mathit lang \rangle$}
\newcommand*{\note}[1]{}
\newcommand*{\Lopt}[1]{\textsf{#1}}
\newcommand*{\file}[1]{\texttt{#1}}
\newcommand*{\pkg}[1]{\texttt{#1}}
\begin{document}
 \DocInput{greek.dtx}
\end{document}
%</filedriver>
%\fi
% \GetFileInfo{greek.dtx}
%
% \changes{greek-1.0b}{1996/07/10}{Replaced \cs{undefined} with
%    \cs{@undefined} and \cs{empty} with \cs{@empty} for consistency
%    with \LaTeX} 
% \changes{greek-1.0b}{1996/10/10}{Moved the definition of
%    \cs{atcatcode} right to the beginning}
% \changes{greek-1.2}{1997/10/28}{Classical Greek is now a dialect}
% \changes{greek-1.2b}{1997/11/01}{Classical Greek is now called 
%   ``Polutoniko'' Greek. The previous name was at least misleading}
% \changes{greek-1.2c}{1998/06/26}{This version conforms to version
%   2.0 of the CB fonts and consequently we added a few new 
%   symbol-producing commands} 
% \changes{greek-1.3a}{1998/07/04}{polutoniko is now an attribute to
%    Greek, no longer a `dialect'}
%
%  \section{The Greek language}
%
%    The file \file{\filename}\footnote{The file described in this
%    section has version number \fileversion\ and was last revised on
%    \filedate. The original author is Apostolos Syropoulos
%    (\texttt{apostolo@platon.ee.duth.gr}), code from
%    \file{kdgreek.sty} by David Kastrup
%    \texttt{dak@neuroinformatik.ruhr-uni-bochum.de} was used to
%    enhance the support for typesetting greek texts.} defines all the
%    language definition macros for the Greek language, i.e.,
%    as it used today with only one accent, and the attribute
%    $\pi o\lambda\upsilon\tau o\nu\kappa\acute{o}$ (``Polutoniko'')
%    for typesetting greek text with all accents. This separation
%    arose out of the need to simplify things, for only very few
%    people will be really interested to typeset polytonic Greek
%    text.
%
%  \DescribeMacro\greektext
%  \DescribeMacro\latintext
%    The commands |\greektext| and |\latintext| can be used to switch
%    to greek or latin fonts. These are declarations.
%
%  \DescribeMacro\textgreek
%  \DescribeMacro\textlatin
%    The commands |\textgreek| and |\textlatin| both take one argument
%    which is then typeset using the requested font encoding.
%  \DescribeMacro\textol
%    The command |\greekol| switches to the greek outline font family,
%    while the command |\textol| typests a short text in outline font.
%    A number of extra greek characters are made available through the
%    added text commands |\stigma|, |\qoppa|, |\sampi|, |\ddigamma|,
%    |\Digamma|, |\euro|, |\permill|, and |\vardigamma|. 
%
%  \subsection{Typing conventions}
%
%    Entering greek text can be quite difficult because of the many
%    diacritical signs that need to be added for various purposes.
%    The fonts that are used to typeset Greek make this a lot
%    easier by offering a lot of ligatures. But in order for this to
%    work, some characters need to be considered as letters. These
%    characters are |<|, |>|, |~|, |`|, |'|, |"| and
%    \verb=|=. Therefore their |\lccode| is changed when Greek is in
%    effect. In order to let |\uppercase| give correct results, the
%    |\uccode| of these characters is set to a non-existing character
%    to make them disappear. Of course not all characters are needed
%    when typesetting ``modern'' $\mu o\nu o\tau o\nu
%    \iota\kappa\acute{o}$. In that case we only need the |'| and |"|
%    symbols which are treated in the proper way.
%
%  \subsection{Greek numbering}
%
%    The Greek alphabetical numbering system, like the Roman one, is 
%    still used in everyday life for short enumerations. Unfortunately 
%    most Greeks don't know how to write Greek numbers bigger than 20 or
%    30. Nevertheless, in official editions of the last century and
%    beginning of this century this numbering system was also used  for
%    dates and numbers in the range of several thousands. Nowadays
%    this numbering system is primary used by the Eastern Orthodox
%    Church and by certain scholars. It is hence necessary to be able
%    to typeset any Greek numeral up to \hbox{999\,999}. Here are the 
%    conventions:
%    \begin{itemize}
%    \item There is no Greek numeral for any number less than or equal
%      to $0$. 
%    \item Numbers from $1$ to $9$ are denoted by letters alpha, beta,
%      gamma,  delta, epsilon, stigma, zeta, eta, theta, followed by a
%      mark similar to the mathematical symbol ``prime''. (Nowadays
%      instead of letter stigma the digraph sigma tau is used for number
%      $6$. Mainly because the letter stigma is not always available, so
%      people opt to write down the first two letters of its name as an
%      alternative. In our implementation we produce the letter stigma,
%      not the digraph sigma tau.) 
%    \item Decades from $10$ to $90$ are denoted by letters iota,
%      kappa, lambda, mu, nu, xi, omikron, pi, qoppa, again followed by
%      the numeric mark. The qoppa used for this purpose has a special
%      zig-zag form, which doesn't resemble at all the original
%      `q'-like qoppa.
%    \item Hundreds from $100$ to $900$ are denoted by letters rho,
%      sigma, tau, upsilon, phi, chi, psi, omega, sampi, followed by the
%      numeric mark. 
%    \item Any number between $1$ and $999$ is obtained by a group of
%      letters denoting the hundreds decades and units, followed by a
%      numeric mark.
%    \item To denote thousands one uses the same method, but this time
%      the mark is placed in front of the letter, and under the baseline
%      (it is inverted by 180 degrees). When a group of letters denoting
%      thousands is followed by a group of letters denoting a number
%      under $1000$, then both marks are used. 
%    \end{itemize}
%
%    Using these conventions one obtains numbers up to \hbox{999\,999}. 
%  \DescribeMacro{\greeknumeral}
%    The command |\greeknumeral| makes it possible to typeset Greek
%    numerals. There is also an
%  \DescribeMacro{\Greeknumeral} 
%    ``uppercase'' version of this macro: |\Greeknumeral|.
%
%    Another system which was in wide use only in Athens, could
%    express any positive number. This system is implemented in 
%    package |athnum|.  
%
% \StopEventually{}
%
%    The macro |\LdfInit| takes care of preventing that this file is
%    loaded more than once, checking the category code of the
%    \texttt{@} sign, etc.
% \changes{greek-1.0b}{1996/11/02}{Now use \cs{LdfInit} to perform
%    initial checks} 
%    \begin{macrocode}
%<*code>
\LdfInit\CurrentOption{captions\CurrentOption}
%    \end{macrocode}
%    When the option \Lopt{polutonikogreek} was used, redefine
%    |\CurrentOption| to prevent problems later on.
%    \begin{macrocode}
\gdef\CurrentOption{greek}%
%    \end{macrocode}
%
%    When this file is read as an option, i.e. by the |\usepackage|
%    command, \texttt{greek} could be an `unknown' language in
%    which case we have to make it known.  So we check for the
%    existence of |\l@greek| to see whether we have to do
%    something here.
%
%    \begin{macrocode}
\ifx\l@greek\@undefined
  \@nopatterns{greek}
  \adddialect\l@greek0\fi
%    \end{macrocode}
%
%    Now we declare the |polutoniko| language attribute.
%    \begin{macrocode}
\bbl@declare@ttribute{greek}{polutoniko}{%
%    \end{macrocode}
%    This code adds the expansion of |\extraspolutonikogreek| to
%    |\extrasgreek| and changes the definition of |\today| for Greek
%    to produce polytonic month names.
%    \begin{macrocode}
  \expandafter\addto\expandafter\extrasgreek
  \expandafter{\extraspolutonikogreek}%
  \let\captionsgreek\captionspolutonikogreek
  \let\gr@month\gr@c@month
%    \end{macrocode}
%    We need to take some extra precautions in order not to break
%    older documents which still use the old \Lopt{polutonikogreek}
%    option.
% \changes{greek-1.3f}{1999/09/29}{Added some code to make older
%    documents work}
% \changes{greek-1.3g}{2000/02/04}{\cs{noextraspolutonikogreek} was
%    missing}
%    \begin{macrocode}
  \let\l@polutonikogreek\l@greek
  \let\datepolutonikogreek\dategreek
  \let\extraspolutonikogreek\extrasgreek
  \let\noextraspolutonikogreek\noextrasgreek
  }
%    \end{macrocode}
%
%    Typesetting Greek texts implies that a special set of fonts needs
%    to be used. The current support for greek uses the |cb| fonts
%    created by Claudio Beccari\footnote{Apostolos Syropoulos wishes
%    to thank him for his patience, collaboration, cooments and
%    suggestions.}. The |cb| fonts provide all  sorts of \textit{font
%    combinations}. In order to use these fonts we define the Local
%    GReek encoding (LGR, see the file \file{greek.fdd}). We make sure
%    that this encoding is known to \LaTeX, and if it isn't we abort.
% \changes{greek-1.2a}{1997/10/31}{filename \file{lgrenc.def} now
%    lowercase}
%    \begin{macrocode}
\InputIfFileExists{lgrenc.def}{%
  \message{Loading the definitions for the Greek font encoding}}{%
  \errhelp{I can't find the lgrenc.def file for the Greek fonts}%
  \errmessage{Since I do not know what the LGR encoding means^^J
    I can't typeset Greek.^^J
    I stop here, while you get a suitable lgrenc.def file}\@@end
 }
%    \end{macrocode}
%
%    Now we define two commands that offer the possibility to switch
%    between Greek and Roman encodings.
%
%  \begin{macro}{\greektext}
%    The command |\greektext| will switch from Latin font encoding to
%    the Greek font encoding. This assumes that the `normal' font
%    encoding is a Latin one. This command is a \emph{declaration},
%    for shorter pieces of text the command |\textgreek| should be
%    used.
%    \begin{macrocode}
\DeclareRobustCommand{\greektext}{%
  \fontencoding{LGR}\selectfont
  \def\encodingdefault{LGR}}
%    \end{macrocode}
%  \end{macro}
%
%  \begin{macro}{\textgreek}
%    This command takes an argument which is then typeset using the
%    requested font encoding. In order to avoid many encoding switches
%    it operates in a local scope.
% \changes{greek-1.0b}{1996/09/23}{Added a level of braces to keep
%    encoding change local} 
% \changes{greek-1.3k}{2003/03/19}{Added \cs{leavevmode} as was done
%    with \cs{latintext}} 
%    \begin{macrocode}
\DeclareRobustCommand{\textgreek}[1]{\leavevmode{\greektext #1}}
%    \end{macrocode}
%  \end{macro}
%
%  \begin{macro}{\textol}
%    A last aspect of the set of fonts provided with this version of
%    support for typesetting Greek texts is that it contains an
%    outline family. In order to make it available we define the command
%    |\textol|.
%    \begin{macrocode}
\def\outlfamily{\usefont{LGR}{cmro}{m}{n}}
\DeclareTextFontCommand{\textol}{\outlfamily}
%    \end{macrocode}
%  \end{macro}
%
%    The next step consists in defining commands to switch to (and
%    from) the Greek language.
%
%  \begin{macro}{\greekhyphenmins}
%    This macro is used to store the correct values of the hyphenation
%    parameters |\lefthyphenmin| and |\righthyphenmin|.
% \changes{greek-1.3h}{2000/09/22}{Now use \cs{providehyphenmins} to
%    provide a default value}
%    \begin{macrocode}
% Yannis Haralambous has suggested this value
\providehyphenmins{\CurrentOption}{\@ne\@ne}
%    \end{macrocode}
%  \end{macro}
% 
% \changes{greek-1.1e}{1997/10/12}{Added caption name for proof}
% \changes{greek-1.3d}{1999/08/28}{Fixed typo, \texttt{bl'epe ep'ishc}
%    instead of \texttt{bl'pe ep'ishc}}
%
%  \begin{macro}{\captionsgreek}
%    The macro |\captionsgreek| defines all strings used in the
%    four standard document classes provided with \LaTeX.
% \changes{greek-1.3h}{2000/09/20}{Added \cs{glossaryname}}
% \changes{greek-1.3i}{2000/10/02}{The final sigma in all names appears
%    as `s' instead of `c'.}
%    \begin{macrocode}
\addto\captionsgreek{%
  \def\prefacename{Pr'ologos}%
  \def\refname{Anafor'es}%
  \def\abstractname{Per'ilhyh}%
  \def\bibname{Bibliograf'ia}%
  \def\chaptername{Kef'alaio}%
  \def\appendixname{Par'arthma}%
  \def\contentsname{Perieq'omena}%
  \def\listfigurename{Kat'alogos Sqhm'atwn}%
  \def\listtablename{Kat'alogos Pin'akwn}%
  \def\indexname{Euret'hrio}%
  \def\figurename{Sq'hma}%
  \def\tablename{P'inakas}%
  \def\partname{M'eros}%
  \def\enclname{Sunhmm'ena}%
  \def\ccname{Koinopo'ihsh}%
  \def\headtoname{Pros}%
  \def\pagename{Sel'ida}%
  \def\seename{bl'epe}%
  \def\alsoname{bl'epe ep'ishs}%
  \def\proofname{Ap'odeixh}%
  \def\glossaryname{Glwss'ari}% 
  }
%    \end{macrocode}
%  \end{macro}
% \changes{greek-1.2}{1997/10/28}{Added caption names for
%    \cs{polutonikogreek}}
% \changes{greek-1.3d}{1999/08/28}{Fixed typo, \texttt{bl'epe >ep'ishc}
%    instead of \texttt{bl'pe >ep'ishc}}
%
%  \begin{macro}{\captionspolutonikogreek}
%    For texts written in the $\pi o\lambda\upsilon\tau
%    o\nu\kappa\acute{o}$ (polytonic greek) the translations are
%    the same as above, but some words are spelled differently. For
%    now we just add extra definitions to |\captionsgreek| in order to
%    override the earlier definitions.
%    \begin{macrocode}
\let\captionspolutonikogreek\captionsgreek
\addto\captionspolutonikogreek{%
  \def\refname{>Anafor`es}%
  \def\indexname{E<uret'hrio}%
  \def\figurename{Sq~hma}%
  \def\headtoname{Pr`os}%
  \def\alsoname{bl'epe >ep'ishs}%
  \def\proofname{>Ap'odeixh}%
}
%    \end{macrocode}
%  \end{macro}
%
%  \begin{macro}{\gr@month}
% \changes{greek-1.1e}{1997/10/12}{Macro added}
%  \begin{macro}{\dategreek}
%    The macro |\dategreek| redefines the command |\today| to
%    produce greek dates. The name of the month is now produced
%    by the macro |\gr@month| since it is needed in the definition
%    of the macro |\Grtoday|.
% \changes{greek-1.1a}{1997/03/03}{Fixed typo, \texttt{Oktwbr'iou}
%    instead of \texttt{Oktobr'iou}}
% \changes{greek-1.1d}{1997/10/12}{Macro \cs{gr@month} now produces
%    the  name of the month} 
% \changes{greek-1.2a}{1997/10/31}{Use \cs{edef} to define \cs{today}}
% \changes{greek-1.2b}{1998/03/28}{use \cs{def} instead of \cs{edef}}
%    \begin{macrocode}
\def\gr@month{%
  \ifcase\month\or
    Ianouar'iou\or Febrouar'iou\or Mart'iou\or April'iou\or
    Ma'"iou\or Ioun'iou\or Ioul'iou\or Augo'ustou\or
    Septembr'iou\or Oktwbr'iou\or Noembr'iou\or Dekembr'iou\fi}
\def\dategreek{%
  \def\today{\number\day \space \gr@month\space \number\year}}
%    \end{macrocode}
%  \end{macro}
%  \end{macro}
%
%  \begin{macro}{\gr@c@greek}
% \changes{greek-1.2}{1997/10/28}{Added macro \cs{gr@cl@month}}
% \changes{greek-1.2}{1997/10/28}{Added macro
%    \cs{datepolutonikogreek}}
% \changes{greek-1.3a}{1997/10/28}{removed macro
%    \cs{datepolutonikogreek}}
%    \begin{macrocode}
\def\gr@c@month{%
  \ifcase\month\or >Ianouar'iou\or
    Febrouar'iou\or Mart'iou\or >April'iou\or Ma"'iou\or
    >Ioun'iou\or  >Ioul'iou\or A>ugo'ustou\or Septembr'iou\or
    >Oktwbr'iou\or Noembr'iou\or Dekembr'iou\fi}
%    \end{macrocode}
%  \end{macro}
%
%  \begin{macro}{\Grtoday}
% \changes{greek-1.1}{1996/10/28}{Added macro \cs{Grtoday}}
%    The macro |\Grtoday| produces the current date, only that the
%    month and the day are shown as greek numerals instead of arabic
%    as it is usually the case.
%    \begin{macrocode}
\def\Grtoday{%
  \expandafter\Greeknumeral\expandafter{\the\day}\space
  \gr@c@month \space
  \expandafter\Greeknumeral\expandafter{\the\year}}
%    \end{macrocode}
%  \end{macro}
%
%  \begin{macro}{\extrasgreek}
%  \begin{macro}{\noextrasgreek}
%    The macro |\extrasgreek| will perform all the extra definitions
%    needed for the Greek language. The macro |\noextrasgreek| is used
%    to cancel the actions of |\extrasgreek|. For the moment these
%    macros switch the fontencoding used and the definition of the
%    internal macros |\@alph| and |\@Alph| because in Greek we do use
%    the Greek numerals. 
%
%    \begin{macrocode}
\addto\extrasgreek{\greektext}
\addto\noextrasgreek{\latintext}
%    \end{macrocode}
%
%  \begin{macro}{\gr@ill@value}
%    When the argument of |\greeknumeral| has a value outside of the
%    acceptable bounds ($0 < x < 999999$) a warning will be issued
%    (and nothing will be printed).
%    \begin{macrocode}
\def\gr@ill@value#1{%
  \PackageWarning{babel}{Illegal value (#1) for greeknumeral}}
%    \end{macrocode}
%  \end{macro}
%
%  \begin{macro}{\anw@true}
%  \begin{macro}{\anw@false}
%  \begin{macro}{\anw@print}
%    When a a large number with three \emph{trailing} zero's is to be
%    printed those zeros \emph{and} the numeric mark need to be
%    discarded. As each `digit' is processed by a separate macro
%    \emph{and} because the processing needs to be expandable we need
%    some helper macros that help remember to \emph{not} print the
%    numeric mark (|\anwtonos|).
%
%    The command |\anw@false| switches the printing of the numeric
%    mark off by making |\anw@print| expand to nothing. The command
%    |\anw@true| (re)enables the printing of the numeric marc. These
%    macro's need to be robust in order to prevent improper expansion
%    during writing to files or during |\uppercase|.
%    \begin{macrocode}
\DeclareRobustCommand\anw@false{%
  \DeclareRobustCommand\anw@print{}}
\DeclareRobustCommand\anw@true{%
  \DeclareRobustCommand\anw@print{\anwtonos}}
\anw@true
%    \end{macrocode}
%  \end{macro}
%  \end{macro}
%  \end{macro}
%
%  \begin{macro}{\greeknumeral}
%    The command |\greeknumeral| needs to be \emph{fully} expandable
%    in order to get the right information in auxiliary
%    files. Therefore we use a big |\if|-construction to check the
%    value of the argument and start the parsing at the right level.
%    \begin{macrocode}
\def\greeknumeral#1{%
%    \end{macrocode}
%    If the value is negative or zero nothing is printed and a warning
%    is issued.
% \changes{greek-1.3b}{1999/04/03}{Added \cs{expandafter} and
%    \cs{number} (PR3000) in order to make a counter an acceptable
%    argument}
%    \begin{macrocode}
  \ifnum#1<\@ne\space\gr@ill@value{#1}%
  \else
    \ifnum#1<10\expandafter\gr@num@i\number#1%
    \else
      \ifnum#1<100\expandafter\gr@num@ii\number#1%
      \else
%    \end{macrocode}
%    We use the available shorthands for 1.000 (|\@m|) and 10.000
%    (|\@M|) to save a few tokens.
%    \begin{macrocode}
        \ifnum#1<\@m\expandafter\gr@num@iii\number#1%
        \else
          \ifnum#1<\@M\expandafter\gr@num@iv\number#1%
          \else
            \ifnum#1<100000\expandafter\gr@num@v\number#1%
            \else
              \ifnum#1<1000000\expandafter\gr@num@vi\number#1%
              \else
%    \end{macrocode}
%    If the value is too large, nothing is printed and a warning
%    is issued.
%    \begin{macrocode}
                \space\gr@ill@value{#1}%
              \fi
            \fi
          \fi
        \fi
      \fi
    \fi
  \fi
}
%    \end{macrocode}
%  \end{macro}
%
%  \begin{macro}{\Greeknumeral}
%    The command |\Greeknumeral| prints uppercase greek numerals. 
%    The parsing is performed by the macro |\greeknumeral|.
%    \begin{macrocode}
\def\Greeknumeral#1{%
  \expandafter\MakeUppercase\expandafter{\greeknumeral{#1}}}
%    \end{macrocode}
%  \end{macro}
%
%  \begin{macro}{\greek@alph}
%  \begin{macro}{\greek@Alph}
%    In the previous release of this language definition the
%    commands |\greek@aplh| and |\greek@Alph| were kept just for
%    reasons of compatibility. Here again they become meaningful macros.
%    They are definited in a way that even page numbering with greek
%    numerals is possible. Since the macros |\@alph| and |\@Alph| will
%    lose their original meaning while the Greek option is active, we
%    must save their original value.
%    macros |\@alph|
%    \begin{macrocode}
\let\latin@alph\@alph
\let\latin@Alph\@Alph
%    \end{macrocode}
%    Then we define the Greek versions; the additional |\expandafter|s
%    are needed in order to make sure the table of contents will be
%    correct, e.g., when we have appendixes. 
%    \begin{macrocode}
\def\greek@alph#1{\expandafter\greeknumeral\expandafter{\the#1}}
\def\greek@Alph#1{\expandafter\Greeknumeral\expandafter{\the#1}}
%    \end{macrocode}
%
%    Now we can set up the switching.
% \changes{greek-1.1a}{1997/03/03}{removed two superfluous @'s which
%    made \cs{@alph} undefined}
%    \begin{macrocode}
\addto\extrasgreek{%
  \let\@alph\greek@alph
  \let\@Alph\greek@Alph}
\addto\noextrasgreek{%
  \let\@alph\latin@alph
  \let\@Alph\latin@Alph}
%    \end{macrocode}
%  \end{macro}
%  \end{macro}
%
%  \begin{macro}{\greek@roman}
%  \begin{macro}{\greek@Roman}
% \changes{greek-1.2e}{1999/04/16}{Moved redefinition of \cs{@roman}
%    back to the language specific file}
% \changes{greek-1.3d}{1999/08/27}{\cs{@roman} and \cs{@Roman} need to
%    be added to \cs{extraspolutonikogreek}} 
% \changes{greek-1.3e}{1999/09/24}{\cs{@roman} and \cs{@Roman} need
%    \emph{not} be in \cs{extraspolutonikogreek} when they are already
%    in \cs{extrasgreek}}
%
%    To prevent roman numerals being typeset in greek letters we need
%    to adopt the internal \LaTeX\ commands |\@roman| and
%    |\@Roman|. \textbf{Note that this may cause errors where
%    |\@roman| ends up in a situation where it needs to be expanded;
%    problems are known to exist with the AMS document classes.}
%    \begin{macrocode}
\let\latin@roman\@roman
\let\latin@Roman\@Roman
\def\greek@roman#1{\textlatin{\latin@roman{#1}}}
\def\greek@Roman#1{\textlatin{\latin@Roman{#1}}}
\addto\extrasgreek{%
  \let\@roman\greek@roman
  \let\@Roman\greek@Roman}
\addto\noextrasgreek{%
  \let\@roman\latin@roman
  \let\@Roman\latin@Roman}
%    \end{macrocode}
%  \end{macro}
%  \end{macro}
%
%  \begin{macro}{\greek@amp}
%  \begin{macro}{\ltx@amp}
%    The greek fonts do not contain an ampersand, so the \LaTeX\
%    command |\&| dosn't give the expected result if we do not do
%    something about it.
% \changes{greek-1.2f}{1999/04/25}{Now switch the definition of
%    \cs{\&} from \cs{extrasgreek}} 
% \changes{greek-1.3c}{1999/05/17}{Added a missing opening brace}
%    \begin{macrocode}
\let\ltx@amp\&
\def\greek@amp{\textlatin{\ltx@amp}}
\addto\extrasgreek{\let\&\greek@amp}
\addto\noextrasgreek{\let\&\ltx@amp}
%    \end{macrocode}
%  \end{macro}
%  \end{macro}
%
%    What is left now is the definition of a set of macros to produce
%    the various digits.
%  \begin{macro}{\gr@num@i}
%  \begin{macro}{\gr@num@ii}
%  \begin{macro}{\gr@num@iii}
% \changes{greek-1.2b}{1997/11/13}{No longer use \cs{\let} in the
%    expansion of the \cs{gr@num@x} macros as they ned to be
%    expandable}
%    As there is no representation for $0$ in this system the zeros
%    are simply discarded. When we have a large number with three
%    \emph{trailing} zero's also the numeric mark is discarded. 
%    Therefore these macros need to pass the information to each other
%    about the (non-)translation of a zero.
%    \begin{macrocode}
\def\gr@num@i#1{%
  \ifcase#1\or a\or b\or g\or d\or e\or \stigma\or z\or h\or j\fi
  \ifnum#1=\z@\else\anw@true\fi\anw@print}
\def\gr@num@ii#1{%
  \ifcase#1\or i\or k\or l\or m\or n\or x\or o\or p\or \qoppa\fi
  \ifnum#1=\z@\else\anw@true\fi\gr@num@i}
\def\gr@num@iii#1{%
  \ifcase#1\or r\or sv\or t\or u\or f\or q\or y\or w\or \sampi\fi
  \ifnum#1=\z@\anw@false\else\anw@true\fi\gr@num@ii}
%    \end{macrocode}
%  \end{macro}
%  \end{macro}
%  \end{macro}
%
%  \begin{macro}{\gr@num@iv}
%  \begin{macro}{\gr@num@v}
%  \begin{macro}{\gr@num@vi}
%    The first three `digits' always have the numeric mark, except
%    when one is discarded because it's value is zero.
%    \begin{macrocode}
\def\gr@num@iv#1{%
  \ifnum#1=\z@\else\katwtonos\fi
  \ifcase#1\or a\or b\or g\or d\or e\or \stigma\or z\or h\or j\fi
  \gr@num@iii}
\def\gr@num@v#1{%
  \ifnum#1=\z@\else\katwtonos\fi
  \ifcase#1\or i\or k\or l\or m\or n\or x\or o\or p\or \qoppa\fi
  \gr@num@iv}
\def\gr@num@vi#1{%
  \katwtonos
  \ifcase#1\or r\or sv\or t\or u\or f\or q\or y\or w\or \sampi\fi
  \gr@num@v}
%    \end{macrocode}
%  \end{macro}
%  \end{macro}
%  \end{macro}
%
%  \begin{macro}{\greek@tilde}
% \changes{greek-1.0c}{1997/02/19}{Added command}
%    In greek typesetting we need a number of characters with more
%    than one accent. In the underlying family of fonts (the
%    |cb| fonts) this is solved using Knuth's ligature
%    mechanism. 
%    Characters we need to have ligatures with are the tilde, the
%    acute and grave accent characters, the rough and smooth breathings,
%    the subscript, and the double quote character.
%    In text input the |~| is normaly used to produce an
%    unbreakable space. The command |\~| normally produces a tilde
%    accent. For  polytonic Greek we change the definition of |\~| 
%    to produce the tilde character itself, making sure it has category 
%    code 12.
% \changes{greek-1.3k}{2003/04/10}{Make sure the character `!' is not
%    active during the definition of \cs{greek@tilde}} 
%    \begin{macrocode}
\begingroup
  \@ifundefined{active@char\string!}{}{\catcode`!=12\relax}
  \catcode`\~=12
  \lccode`\!=`\~
  \lowercase{\def\x{\endgroup
      \def\greek@tilde{!}}\x}
\addto\extraspolutonikogreek{%
  \babel@save\~\let\~\greek@tilde}
%    \end{macrocode}
%  \end{macro}
%    In order to get correct hyphenation we need to set the lower case
%    code of a number of characters. The `v' character has a special
%    usage for the |cb| fonts: in fact this ligature mechanism detects
%    the end of a word and assures that a final sigma is typeset with
%    the proper sign wich is different from that of an initial or
%    medial sigma;  the  `v  'after  an  \textit{isolated} sigma fools
%    the ligature mechanism in order to typeset $\sigma$ in place of
%    $\varsigma$. Because of this we make sure its lowercase code is
%    not changed. For ``modern'' greek we have to deal only with |'|
%    and |"| and so things are easy.
% \changes{greek-1.1c}{1997/04/30}{fixed two typos}
% \changes{greek-1.1e}{1997/10/12}{Added lowercase code for v}
% \changes{greek-1.2}{1997/10/28}{Definitions for ``modern'' Greek are
%    now the definitions of ``Polutoniko'' Greek} 
% \changes{greek-1.2}{1997/10/28}{Added lowercase codes for ``modern''
%    greek} 
% \changes{greek-1.3e}{1999/09/24}{\cs{extrasgreek} and
%    \cs{extraspolutonikogreek} should be complementary}
%    \begin{macrocode}
\addto\extrasgreek{%
  \babel@savevariable{\lccode`v}\lccode`v=`v%
  \babel@savevariable{\lccode`\'}\lccode`\'=`\'%
  \babel@savevariable{\lccode`\"}\lccode`\"=`\"}
\addto\extraspolutonikogreek{%
  \babel@savevariable{\lccode`\<}\lccode`\<=`\<%
  \babel@savevariable{\lccode`\>}\lccode`\>=`\>%
  \babel@savevariable{\lccode`\~}\lccode`\~=`\~%
  \babel@savevariable{\lccode`\|}\lccode`\|=`\|%
  \babel@savevariable{\lccode`\`}\lccode`\`=`\`}
%    \end{macrocode}
%    And in order to get rid of all accents and breathings when a
%    string is |\uppercase|d we also change a number of uppercase
%    codes. 
% \changes{greek-1.1b}{1997/03/06}{Added setting of \cs{uccode}s
%    (after \file{kdgreek.sty})}
% \changes{greek-1.1e}{1997/10/12}{Added uppercase code for special
%    letter ``v''. Uppercase code for accents is now \texttt{9f},
%    instead of \texttt{ff}}
% \changes{greek-1.2}{1997/10/28}{Added uppercase codes for ``modern''
%    Greek. The old codes are now for ``Polutoniko'' Greek} 
% \changes{greek-1.3e}{1999/09/24}{\cs{extrasgreek} and
%    \cs{extraspolutonikogreek} should be complementary}
% {\catcode`|=12\relax\gdef\indexbar{\cs{|}}}
% \changes{greek-1.3g}{1999/11/17}{uc code of \indexbar{} is now just
%    \indexbar{} to reflect recent changes in the cb fonts}
% \changes{greek-1.3i}{2000/10/02}{uc code of `v' is switched to V
%    so that mixed text appears correctly in headers.}
% \changes{greek-1.3j}{2001/02/03}{Because other languages might
%    make the caret active, we can't use the double caret notation
%    here}
%    \begin{macrocode}
\addto\extrasgreek{%
  \babel@savevariable{\uccode`\"}\uccode`\"=`\"%
  \babel@savevariable{\uccode`\'}\uccode`\'=159} %% 159 == ^^9f
\addto\extraspolutonikogreek{%
  \babel@savevariable{\uccode`\~}\uccode`\~=159%
  \babel@savevariable{\uccode`\>}\uccode`\>=159%
  \babel@savevariable{\uccode`\<}\uccode`\<=159%
  \babel@savevariable{\uccode`\|}\uccode`\|=`\|%
  \babel@savevariable{\uccode`\`}\uccode`\`=159}
%    \end{macrocode}
%    For this to work we make the character |^^9f| a shorthand that
%    expands to nothing. In order for this to work we need to make a
%    character look like |^^9f| in \TeX's eyes. The trick is to have
%    another character and assign it a different lowercase code. The
%    execute the macros needed in a |\lowercase| environment. Usually
%    the tile |~| character is used for such purposes. Before we do
%    this we save it's original lowercase code to restore it once
%    we're done.
% \changes{greek-1.1b}{1997/03/06}{Added shorthand for \cs{char255}}
% \changes{greek-1.1e}{1997/10/12}{Shorthand is changed. Active
%    character is now \cs{char159}} 
% \changes{greek-1.2a}{1997/10/31}{Need shorthand to exist for
%    ``monotoniko'' Greek, not ``polutoniko'' Greek}
% \changes{greek-1.3j}{2001/02/03}{Ues the tilde as an alias for
%    character 159}
%    \begin{macrocode}
\@tempcnta=\lccode`\~
\lccode`\~=159
\lowercase{%
  \initiate@active@char{~}%
  \declare@shorthand{greek}{~}{}}
\lccode`\~=\@tempcnta
%    \end{macrocode}
%    We can also make the tilde character itself expand to a tilde with
%    category code 12 to make the typing of texts easier.
% \changes{greek-1.1b}{1997/03/06}{Made tilde expand to a tilde with
%    \cs{catcode 12}} 
%    \begin{macrocode}
\addto\extraspolutonikogreek{\languageshorthands{greek}}%
\declare@shorthand{greek}{~}{\greek@tilde}
%    \end{macrocode}
%  \end{macro}
%  \end{macro}
%  
%  
%
% \changes{greek-1.1c}{1997/03/10}{Added a couple of symbols, needed
%    for \cs{greeknumeral}}
% \changes{greek-1.1e}{1997/10/12}{Most symbols are removed and are
%    now defined in package grsymb} 
% \changes{greek-1.2c}{1998/06/26}{Package grsymb has been eliminated
%   because the CB fonts v2.0 do not inlcude certain symbols and so
%   the remaining symbol definitions have been moved here}
%    We now define a few symbols which are used in the typesetting of
%    greek numerals, as well as some other symbols which are usefull,
%    such as the $\epsilon\upsilon\rho\omega$ symbol, etc.
%    \begin{macrocode}
\DeclareTextCommand{\anwtonos}{LGR}{\char"FE\relax}
\DeclareTextCommand{\katwtonos}{LGR}{\char"FF\relax}
\DeclareTextCommand{\qoppa}{LGR}{\char"12\relax}
\DeclareTextCommand{\stigma}{LGR}{\char"06\relax}
\DeclareTextCommand{\sampi}{LGR}{\char"1B\relax}
\DeclareTextCommand{\Digamma}{LGR}{\char"C3\relax}
\DeclareTextCommand{\ddigamma}{LGR}{\char"93\relax}
\DeclareTextCommand{\vardigamma}{LGR}{\char"07\relax}
\DeclareTextCommand{\euro}{LGR}{\char"18\relax}
\DeclareTextCommand{\permill}{LGR}{\char"19\relax}
%    \end{macrocode}
%
%    Since the |~| cannot be used to produce an unbreakable white
%    space we must redefine at least the commands |\fnum@figure| and
%    |\fnum@table| so they do not produce a |~| instead of white
%    space. 
% \changes{greek-1.3l}{2004/02/19}{Commented these lines out as this
%    change has made it into \LaTeX{} itself.} 
%    \begin{macrocode} 
%\def\fnum@figure{\figurename\nobreakspace\thefigure}
%\def\fnum@table{\tablename\nobreakspace\thetable}
%    \end{macrocode}
%
%    The macro |\ldf@finish| takes care of looking for a
%    configuration file, setting the main language to be switched on
%    at |\begin{document}| and resetting the category code of
%    \texttt{@} to its original value.
% \changes{greek-1.0b}{1996/11/02}{Now use \cs{ldf@finish} to wrap up}
%    \begin{macrocode}
\ldf@finish{\CurrentOption}
%</code>
%    \end{macrocode}
%
% \Finale
%\endinput
%% \CharacterTable
%%  {Upper-case    \A\B\C\D\E\F\G\H\I\J\K\L\M\N\O\P\Q\R\S\T\U\V\W\X\Y\Z
%%   Lower-case    \a\b\c\d\e\f\g\h\i\j\k\l\m\n\o\p\q\r\s\t\u\v\w\x\y\z
%%   Digits        \0\1\2\3\4\5\6\7\8\9
%%   Exclamation   \!     Double quote  \"     Hash (number) \#
%%   Dollar        \$     Percent       \%     Ampersand     \&
%%   Acute accent  \'     Left paren    \(     Right paren   \)
%%   Asterisk      \*     Plus          \+     Comma         \,
%%   Minus         \-     Point         \.     Solidus       \/
%%   Colon         \:     Semicolon     \;     Less than     \<
%%   Equals        \=     Greater than  \>     Question mark \?
%%   Commercial at \@     Left bracket  \[     Backslash     \\
%%   Right bracket \]     Circumflex    \^     Underscore    \_
%%   Grave accent  \`     Left brace    \{     Vertical bar  \|
%%   Right brace   \}     Tilde         \~}
%%
}
\DeclareOption{polutonikogreek}{%
  % \iffalse meta-comment
%
% Copyright 1989-2008 Johannes L. Braams and any individual authors
% listed elsewhere in this file.  All rights reserved.
% 
% This file is part of the Babel system.
% --------------------------------------
% 
% It may be distributed and/or modified under the
% conditions of the LaTeX Project Public License, either version 1.3
% of this license or (at your option) any later version.
% The latest version of this license is in
%   http://www.latex-project.org/lppl.txt
% and version 1.3 or later is part of all distributions of LaTeX
% version 2003/12/01 or later.
% 
% This work has the LPPL maintenance status "maintained".
% 
% The Current Maintainer of this work is Johannes Braams.
% 
% The list of all files belonging to the Babel system is
% given in the file `manifest.bbl. See also `legal.bbl' for additional
% information.
% 
% The list of derived (unpacked) files belonging to the distribution
% and covered by LPPL is defined by the unpacking scripts (with
% extension .ins) which are part of the distribution.
% \fi
% \CheckSum{636}
%
% \iffalse
%    Tell the \LaTeX\ system who we are and write an entry on the
%    transcript.
%<*dtx>
\ProvidesFile{greek.dtx}
%</dtx>
%<code>\ProvidesLanguage{greek}
%\fi
%\ProvidesFile{greek.dtx}
        [2005/03/30 v1.3l Greek support from the babel system]
%\iffalse
%% File `greek.dtx'
%% Babel package for LaTeX version 2e
%% Copyright (C) 1989 -- 2005
%%           by Johannes Braams, TeXniek
%
%% Greek language Definition File
%% Copyright (C) 1997, 2005
%%           by Apostolos Syropoulos
%%              Johannes Braams, TeXniek
%
%% Please report errors to: Apostolos Syropoulos
%%                          apostolo at platon.ee.duth.gr or
%%                          apostolo at obelix.ee.duth.gr
%%                          (or J.L. Braams <babel at braams.cistron.nl)
%
%    This file is part of the babel system, it provides the source
%    code for the greek language definition file. The original
%    version of this file was written by Apostolos Syropoulos.
%    It was then enhanced by adding code from kdgreek.sty from David
%    Kastrup <dak@neuroinformatik.ruhr-uni-bochum.de> with his
%    consent. 
%<*filedriver>
\documentclass{ltxdoc}
\newcommand*{\TeXhax}{\TeX hax}
\newcommand*{\babel}{\textsf{babel}}
\newcommand*{\langvar}{$\langle \mathit lang \rangle$}
\newcommand*{\note}[1]{}
\newcommand*{\Lopt}[1]{\textsf{#1}}
\newcommand*{\file}[1]{\texttt{#1}}
\newcommand*{\pkg}[1]{\texttt{#1}}
\begin{document}
 \DocInput{greek.dtx}
\end{document}
%</filedriver>
%\fi
% \GetFileInfo{greek.dtx}
%
% \changes{greek-1.0b}{1996/07/10}{Replaced \cs{undefined} with
%    \cs{@undefined} and \cs{empty} with \cs{@empty} for consistency
%    with \LaTeX} 
% \changes{greek-1.0b}{1996/10/10}{Moved the definition of
%    \cs{atcatcode} right to the beginning}
% \changes{greek-1.2}{1997/10/28}{Classical Greek is now a dialect}
% \changes{greek-1.2b}{1997/11/01}{Classical Greek is now called 
%   ``Polutoniko'' Greek. The previous name was at least misleading}
% \changes{greek-1.2c}{1998/06/26}{This version conforms to version
%   2.0 of the CB fonts and consequently we added a few new 
%   symbol-producing commands} 
% \changes{greek-1.3a}{1998/07/04}{polutoniko is now an attribute to
%    Greek, no longer a `dialect'}
%
%  \section{The Greek language}
%
%    The file \file{\filename}\footnote{The file described in this
%    section has version number \fileversion\ and was last revised on
%    \filedate. The original author is Apostolos Syropoulos
%    (\texttt{apostolo@platon.ee.duth.gr}), code from
%    \file{kdgreek.sty} by David Kastrup
%    \texttt{dak@neuroinformatik.ruhr-uni-bochum.de} was used to
%    enhance the support for typesetting greek texts.} defines all the
%    language definition macros for the Greek language, i.e.,
%    as it used today with only one accent, and the attribute
%    $\pi o\lambda\upsilon\tau o\nu\kappa\acute{o}$ (``Polutoniko'')
%    for typesetting greek text with all accents. This separation
%    arose out of the need to simplify things, for only very few
%    people will be really interested to typeset polytonic Greek
%    text.
%
%  \DescribeMacro\greektext
%  \DescribeMacro\latintext
%    The commands |\greektext| and |\latintext| can be used to switch
%    to greek or latin fonts. These are declarations.
%
%  \DescribeMacro\textgreek
%  \DescribeMacro\textlatin
%    The commands |\textgreek| and |\textlatin| both take one argument
%    which is then typeset using the requested font encoding.
%  \DescribeMacro\textol
%    The command |\greekol| switches to the greek outline font family,
%    while the command |\textol| typests a short text in outline font.
%    A number of extra greek characters are made available through the
%    added text commands |\stigma|, |\qoppa|, |\sampi|, |\ddigamma|,
%    |\Digamma|, |\euro|, |\permill|, and |\vardigamma|. 
%
%  \subsection{Typing conventions}
%
%    Entering greek text can be quite difficult because of the many
%    diacritical signs that need to be added for various purposes.
%    The fonts that are used to typeset Greek make this a lot
%    easier by offering a lot of ligatures. But in order for this to
%    work, some characters need to be considered as letters. These
%    characters are |<|, |>|, |~|, |`|, |'|, |"| and
%    \verb=|=. Therefore their |\lccode| is changed when Greek is in
%    effect. In order to let |\uppercase| give correct results, the
%    |\uccode| of these characters is set to a non-existing character
%    to make them disappear. Of course not all characters are needed
%    when typesetting ``modern'' $\mu o\nu o\tau o\nu
%    \iota\kappa\acute{o}$. In that case we only need the |'| and |"|
%    symbols which are treated in the proper way.
%
%  \subsection{Greek numbering}
%
%    The Greek alphabetical numbering system, like the Roman one, is 
%    still used in everyday life for short enumerations. Unfortunately 
%    most Greeks don't know how to write Greek numbers bigger than 20 or
%    30. Nevertheless, in official editions of the last century and
%    beginning of this century this numbering system was also used  for
%    dates and numbers in the range of several thousands. Nowadays
%    this numbering system is primary used by the Eastern Orthodox
%    Church and by certain scholars. It is hence necessary to be able
%    to typeset any Greek numeral up to \hbox{999\,999}. Here are the 
%    conventions:
%    \begin{itemize}
%    \item There is no Greek numeral for any number less than or equal
%      to $0$. 
%    \item Numbers from $1$ to $9$ are denoted by letters alpha, beta,
%      gamma,  delta, epsilon, stigma, zeta, eta, theta, followed by a
%      mark similar to the mathematical symbol ``prime''. (Nowadays
%      instead of letter stigma the digraph sigma tau is used for number
%      $6$. Mainly because the letter stigma is not always available, so
%      people opt to write down the first two letters of its name as an
%      alternative. In our implementation we produce the letter stigma,
%      not the digraph sigma tau.) 
%    \item Decades from $10$ to $90$ are denoted by letters iota,
%      kappa, lambda, mu, nu, xi, omikron, pi, qoppa, again followed by
%      the numeric mark. The qoppa used for this purpose has a special
%      zig-zag form, which doesn't resemble at all the original
%      `q'-like qoppa.
%    \item Hundreds from $100$ to $900$ are denoted by letters rho,
%      sigma, tau, upsilon, phi, chi, psi, omega, sampi, followed by the
%      numeric mark. 
%    \item Any number between $1$ and $999$ is obtained by a group of
%      letters denoting the hundreds decades and units, followed by a
%      numeric mark.
%    \item To denote thousands one uses the same method, but this time
%      the mark is placed in front of the letter, and under the baseline
%      (it is inverted by 180 degrees). When a group of letters denoting
%      thousands is followed by a group of letters denoting a number
%      under $1000$, then both marks are used. 
%    \end{itemize}
%
%    Using these conventions one obtains numbers up to \hbox{999\,999}. 
%  \DescribeMacro{\greeknumeral}
%    The command |\greeknumeral| makes it possible to typeset Greek
%    numerals. There is also an
%  \DescribeMacro{\Greeknumeral} 
%    ``uppercase'' version of this macro: |\Greeknumeral|.
%
%    Another system which was in wide use only in Athens, could
%    express any positive number. This system is implemented in 
%    package |athnum|.  
%
% \StopEventually{}
%
%    The macro |\LdfInit| takes care of preventing that this file is
%    loaded more than once, checking the category code of the
%    \texttt{@} sign, etc.
% \changes{greek-1.0b}{1996/11/02}{Now use \cs{LdfInit} to perform
%    initial checks} 
%    \begin{macrocode}
%<*code>
\LdfInit\CurrentOption{captions\CurrentOption}
%    \end{macrocode}
%    When the option \Lopt{polutonikogreek} was used, redefine
%    |\CurrentOption| to prevent problems later on.
%    \begin{macrocode}
\gdef\CurrentOption{greek}%
%    \end{macrocode}
%
%    When this file is read as an option, i.e. by the |\usepackage|
%    command, \texttt{greek} could be an `unknown' language in
%    which case we have to make it known.  So we check for the
%    existence of |\l@greek| to see whether we have to do
%    something here.
%
%    \begin{macrocode}
\ifx\l@greek\@undefined
  \@nopatterns{greek}
  \adddialect\l@greek0\fi
%    \end{macrocode}
%
%    Now we declare the |polutoniko| language attribute.
%    \begin{macrocode}
\bbl@declare@ttribute{greek}{polutoniko}{%
%    \end{macrocode}
%    This code adds the expansion of |\extraspolutonikogreek| to
%    |\extrasgreek| and changes the definition of |\today| for Greek
%    to produce polytonic month names.
%    \begin{macrocode}
  \expandafter\addto\expandafter\extrasgreek
  \expandafter{\extraspolutonikogreek}%
  \let\captionsgreek\captionspolutonikogreek
  \let\gr@month\gr@c@month
%    \end{macrocode}
%    We need to take some extra precautions in order not to break
%    older documents which still use the old \Lopt{polutonikogreek}
%    option.
% \changes{greek-1.3f}{1999/09/29}{Added some code to make older
%    documents work}
% \changes{greek-1.3g}{2000/02/04}{\cs{noextraspolutonikogreek} was
%    missing}
%    \begin{macrocode}
  \let\l@polutonikogreek\l@greek
  \let\datepolutonikogreek\dategreek
  \let\extraspolutonikogreek\extrasgreek
  \let\noextraspolutonikogreek\noextrasgreek
  }
%    \end{macrocode}
%
%    Typesetting Greek texts implies that a special set of fonts needs
%    to be used. The current support for greek uses the |cb| fonts
%    created by Claudio Beccari\footnote{Apostolos Syropoulos wishes
%    to thank him for his patience, collaboration, cooments and
%    suggestions.}. The |cb| fonts provide all  sorts of \textit{font
%    combinations}. In order to use these fonts we define the Local
%    GReek encoding (LGR, see the file \file{greek.fdd}). We make sure
%    that this encoding is known to \LaTeX, and if it isn't we abort.
% \changes{greek-1.2a}{1997/10/31}{filename \file{lgrenc.def} now
%    lowercase}
%    \begin{macrocode}
\InputIfFileExists{lgrenc.def}{%
  \message{Loading the definitions for the Greek font encoding}}{%
  \errhelp{I can't find the lgrenc.def file for the Greek fonts}%
  \errmessage{Since I do not know what the LGR encoding means^^J
    I can't typeset Greek.^^J
    I stop here, while you get a suitable lgrenc.def file}\@@end
 }
%    \end{macrocode}
%
%    Now we define two commands that offer the possibility to switch
%    between Greek and Roman encodings.
%
%  \begin{macro}{\greektext}
%    The command |\greektext| will switch from Latin font encoding to
%    the Greek font encoding. This assumes that the `normal' font
%    encoding is a Latin one. This command is a \emph{declaration},
%    for shorter pieces of text the command |\textgreek| should be
%    used.
%    \begin{macrocode}
\DeclareRobustCommand{\greektext}{%
  \fontencoding{LGR}\selectfont
  \def\encodingdefault{LGR}}
%    \end{macrocode}
%  \end{macro}
%
%  \begin{macro}{\textgreek}
%    This command takes an argument which is then typeset using the
%    requested font encoding. In order to avoid many encoding switches
%    it operates in a local scope.
% \changes{greek-1.0b}{1996/09/23}{Added a level of braces to keep
%    encoding change local} 
% \changes{greek-1.3k}{2003/03/19}{Added \cs{leavevmode} as was done
%    with \cs{latintext}} 
%    \begin{macrocode}
\DeclareRobustCommand{\textgreek}[1]{\leavevmode{\greektext #1}}
%    \end{macrocode}
%  \end{macro}
%
%  \begin{macro}{\textol}
%    A last aspect of the set of fonts provided with this version of
%    support for typesetting Greek texts is that it contains an
%    outline family. In order to make it available we define the command
%    |\textol|.
%    \begin{macrocode}
\def\outlfamily{\usefont{LGR}{cmro}{m}{n}}
\DeclareTextFontCommand{\textol}{\outlfamily}
%    \end{macrocode}
%  \end{macro}
%
%    The next step consists in defining commands to switch to (and
%    from) the Greek language.
%
%  \begin{macro}{\greekhyphenmins}
%    This macro is used to store the correct values of the hyphenation
%    parameters |\lefthyphenmin| and |\righthyphenmin|.
% \changes{greek-1.3h}{2000/09/22}{Now use \cs{providehyphenmins} to
%    provide a default value}
%    \begin{macrocode}
% Yannis Haralambous has suggested this value
\providehyphenmins{\CurrentOption}{\@ne\@ne}
%    \end{macrocode}
%  \end{macro}
% 
% \changes{greek-1.1e}{1997/10/12}{Added caption name for proof}
% \changes{greek-1.3d}{1999/08/28}{Fixed typo, \texttt{bl'epe ep'ishc}
%    instead of \texttt{bl'pe ep'ishc}}
%
%  \begin{macro}{\captionsgreek}
%    The macro |\captionsgreek| defines all strings used in the
%    four standard document classes provided with \LaTeX.
% \changes{greek-1.3h}{2000/09/20}{Added \cs{glossaryname}}
% \changes{greek-1.3i}{2000/10/02}{The final sigma in all names appears
%    as `s' instead of `c'.}
%    \begin{macrocode}
\addto\captionsgreek{%
  \def\prefacename{Pr'ologos}%
  \def\refname{Anafor'es}%
  \def\abstractname{Per'ilhyh}%
  \def\bibname{Bibliograf'ia}%
  \def\chaptername{Kef'alaio}%
  \def\appendixname{Par'arthma}%
  \def\contentsname{Perieq'omena}%
  \def\listfigurename{Kat'alogos Sqhm'atwn}%
  \def\listtablename{Kat'alogos Pin'akwn}%
  \def\indexname{Euret'hrio}%
  \def\figurename{Sq'hma}%
  \def\tablename{P'inakas}%
  \def\partname{M'eros}%
  \def\enclname{Sunhmm'ena}%
  \def\ccname{Koinopo'ihsh}%
  \def\headtoname{Pros}%
  \def\pagename{Sel'ida}%
  \def\seename{bl'epe}%
  \def\alsoname{bl'epe ep'ishs}%
  \def\proofname{Ap'odeixh}%
  \def\glossaryname{Glwss'ari}% 
  }
%    \end{macrocode}
%  \end{macro}
% \changes{greek-1.2}{1997/10/28}{Added caption names for
%    \cs{polutonikogreek}}
% \changes{greek-1.3d}{1999/08/28}{Fixed typo, \texttt{bl'epe >ep'ishc}
%    instead of \texttt{bl'pe >ep'ishc}}
%
%  \begin{macro}{\captionspolutonikogreek}
%    For texts written in the $\pi o\lambda\upsilon\tau
%    o\nu\kappa\acute{o}$ (polytonic greek) the translations are
%    the same as above, but some words are spelled differently. For
%    now we just add extra definitions to |\captionsgreek| in order to
%    override the earlier definitions.
%    \begin{macrocode}
\let\captionspolutonikogreek\captionsgreek
\addto\captionspolutonikogreek{%
  \def\refname{>Anafor`es}%
  \def\indexname{E<uret'hrio}%
  \def\figurename{Sq~hma}%
  \def\headtoname{Pr`os}%
  \def\alsoname{bl'epe >ep'ishs}%
  \def\proofname{>Ap'odeixh}%
}
%    \end{macrocode}
%  \end{macro}
%
%  \begin{macro}{\gr@month}
% \changes{greek-1.1e}{1997/10/12}{Macro added}
%  \begin{macro}{\dategreek}
%    The macro |\dategreek| redefines the command |\today| to
%    produce greek dates. The name of the month is now produced
%    by the macro |\gr@month| since it is needed in the definition
%    of the macro |\Grtoday|.
% \changes{greek-1.1a}{1997/03/03}{Fixed typo, \texttt{Oktwbr'iou}
%    instead of \texttt{Oktobr'iou}}
% \changes{greek-1.1d}{1997/10/12}{Macro \cs{gr@month} now produces
%    the  name of the month} 
% \changes{greek-1.2a}{1997/10/31}{Use \cs{edef} to define \cs{today}}
% \changes{greek-1.2b}{1998/03/28}{use \cs{def} instead of \cs{edef}}
%    \begin{macrocode}
\def\gr@month{%
  \ifcase\month\or
    Ianouar'iou\or Febrouar'iou\or Mart'iou\or April'iou\or
    Ma'"iou\or Ioun'iou\or Ioul'iou\or Augo'ustou\or
    Septembr'iou\or Oktwbr'iou\or Noembr'iou\or Dekembr'iou\fi}
\def\dategreek{%
  \def\today{\number\day \space \gr@month\space \number\year}}
%    \end{macrocode}
%  \end{macro}
%  \end{macro}
%
%  \begin{macro}{\gr@c@greek}
% \changes{greek-1.2}{1997/10/28}{Added macro \cs{gr@cl@month}}
% \changes{greek-1.2}{1997/10/28}{Added macro
%    \cs{datepolutonikogreek}}
% \changes{greek-1.3a}{1997/10/28}{removed macro
%    \cs{datepolutonikogreek}}
%    \begin{macrocode}
\def\gr@c@month{%
  \ifcase\month\or >Ianouar'iou\or
    Febrouar'iou\or Mart'iou\or >April'iou\or Ma"'iou\or
    >Ioun'iou\or  >Ioul'iou\or A>ugo'ustou\or Septembr'iou\or
    >Oktwbr'iou\or Noembr'iou\or Dekembr'iou\fi}
%    \end{macrocode}
%  \end{macro}
%
%  \begin{macro}{\Grtoday}
% \changes{greek-1.1}{1996/10/28}{Added macro \cs{Grtoday}}
%    The macro |\Grtoday| produces the current date, only that the
%    month and the day are shown as greek numerals instead of arabic
%    as it is usually the case.
%    \begin{macrocode}
\def\Grtoday{%
  \expandafter\Greeknumeral\expandafter{\the\day}\space
  \gr@c@month \space
  \expandafter\Greeknumeral\expandafter{\the\year}}
%    \end{macrocode}
%  \end{macro}
%
%  \begin{macro}{\extrasgreek}
%  \begin{macro}{\noextrasgreek}
%    The macro |\extrasgreek| will perform all the extra definitions
%    needed for the Greek language. The macro |\noextrasgreek| is used
%    to cancel the actions of |\extrasgreek|. For the moment these
%    macros switch the fontencoding used and the definition of the
%    internal macros |\@alph| and |\@Alph| because in Greek we do use
%    the Greek numerals. 
%
%    \begin{macrocode}
\addto\extrasgreek{\greektext}
\addto\noextrasgreek{\latintext}
%    \end{macrocode}
%
%  \begin{macro}{\gr@ill@value}
%    When the argument of |\greeknumeral| has a value outside of the
%    acceptable bounds ($0 < x < 999999$) a warning will be issued
%    (and nothing will be printed).
%    \begin{macrocode}
\def\gr@ill@value#1{%
  \PackageWarning{babel}{Illegal value (#1) for greeknumeral}}
%    \end{macrocode}
%  \end{macro}
%
%  \begin{macro}{\anw@true}
%  \begin{macro}{\anw@false}
%  \begin{macro}{\anw@print}
%    When a a large number with three \emph{trailing} zero's is to be
%    printed those zeros \emph{and} the numeric mark need to be
%    discarded. As each `digit' is processed by a separate macro
%    \emph{and} because the processing needs to be expandable we need
%    some helper macros that help remember to \emph{not} print the
%    numeric mark (|\anwtonos|).
%
%    The command |\anw@false| switches the printing of the numeric
%    mark off by making |\anw@print| expand to nothing. The command
%    |\anw@true| (re)enables the printing of the numeric marc. These
%    macro's need to be robust in order to prevent improper expansion
%    during writing to files or during |\uppercase|.
%    \begin{macrocode}
\DeclareRobustCommand\anw@false{%
  \DeclareRobustCommand\anw@print{}}
\DeclareRobustCommand\anw@true{%
  \DeclareRobustCommand\anw@print{\anwtonos}}
\anw@true
%    \end{macrocode}
%  \end{macro}
%  \end{macro}
%  \end{macro}
%
%  \begin{macro}{\greeknumeral}
%    The command |\greeknumeral| needs to be \emph{fully} expandable
%    in order to get the right information in auxiliary
%    files. Therefore we use a big |\if|-construction to check the
%    value of the argument and start the parsing at the right level.
%    \begin{macrocode}
\def\greeknumeral#1{%
%    \end{macrocode}
%    If the value is negative or zero nothing is printed and a warning
%    is issued.
% \changes{greek-1.3b}{1999/04/03}{Added \cs{expandafter} and
%    \cs{number} (PR3000) in order to make a counter an acceptable
%    argument}
%    \begin{macrocode}
  \ifnum#1<\@ne\space\gr@ill@value{#1}%
  \else
    \ifnum#1<10\expandafter\gr@num@i\number#1%
    \else
      \ifnum#1<100\expandafter\gr@num@ii\number#1%
      \else
%    \end{macrocode}
%    We use the available shorthands for 1.000 (|\@m|) and 10.000
%    (|\@M|) to save a few tokens.
%    \begin{macrocode}
        \ifnum#1<\@m\expandafter\gr@num@iii\number#1%
        \else
          \ifnum#1<\@M\expandafter\gr@num@iv\number#1%
          \else
            \ifnum#1<100000\expandafter\gr@num@v\number#1%
            \else
              \ifnum#1<1000000\expandafter\gr@num@vi\number#1%
              \else
%    \end{macrocode}
%    If the value is too large, nothing is printed and a warning
%    is issued.
%    \begin{macrocode}
                \space\gr@ill@value{#1}%
              \fi
            \fi
          \fi
        \fi
      \fi
    \fi
  \fi
}
%    \end{macrocode}
%  \end{macro}
%
%  \begin{macro}{\Greeknumeral}
%    The command |\Greeknumeral| prints uppercase greek numerals. 
%    The parsing is performed by the macro |\greeknumeral|.
%    \begin{macrocode}
\def\Greeknumeral#1{%
  \expandafter\MakeUppercase\expandafter{\greeknumeral{#1}}}
%    \end{macrocode}
%  \end{macro}
%
%  \begin{macro}{\greek@alph}
%  \begin{macro}{\greek@Alph}
%    In the previous release of this language definition the
%    commands |\greek@aplh| and |\greek@Alph| were kept just for
%    reasons of compatibility. Here again they become meaningful macros.
%    They are definited in a way that even page numbering with greek
%    numerals is possible. Since the macros |\@alph| and |\@Alph| will
%    lose their original meaning while the Greek option is active, we
%    must save their original value.
%    macros |\@alph|
%    \begin{macrocode}
\let\latin@alph\@alph
\let\latin@Alph\@Alph
%    \end{macrocode}
%    Then we define the Greek versions; the additional |\expandafter|s
%    are needed in order to make sure the table of contents will be
%    correct, e.g., when we have appendixes. 
%    \begin{macrocode}
\def\greek@alph#1{\expandafter\greeknumeral\expandafter{\the#1}}
\def\greek@Alph#1{\expandafter\Greeknumeral\expandafter{\the#1}}
%    \end{macrocode}
%
%    Now we can set up the switching.
% \changes{greek-1.1a}{1997/03/03}{removed two superfluous @'s which
%    made \cs{@alph} undefined}
%    \begin{macrocode}
\addto\extrasgreek{%
  \let\@alph\greek@alph
  \let\@Alph\greek@Alph}
\addto\noextrasgreek{%
  \let\@alph\latin@alph
  \let\@Alph\latin@Alph}
%    \end{macrocode}
%  \end{macro}
%  \end{macro}
%
%  \begin{macro}{\greek@roman}
%  \begin{macro}{\greek@Roman}
% \changes{greek-1.2e}{1999/04/16}{Moved redefinition of \cs{@roman}
%    back to the language specific file}
% \changes{greek-1.3d}{1999/08/27}{\cs{@roman} and \cs{@Roman} need to
%    be added to \cs{extraspolutonikogreek}} 
% \changes{greek-1.3e}{1999/09/24}{\cs{@roman} and \cs{@Roman} need
%    \emph{not} be in \cs{extraspolutonikogreek} when they are already
%    in \cs{extrasgreek}}
%
%    To prevent roman numerals being typeset in greek letters we need
%    to adopt the internal \LaTeX\ commands |\@roman| and
%    |\@Roman|. \textbf{Note that this may cause errors where
%    |\@roman| ends up in a situation where it needs to be expanded;
%    problems are known to exist with the AMS document classes.}
%    \begin{macrocode}
\let\latin@roman\@roman
\let\latin@Roman\@Roman
\def\greek@roman#1{\textlatin{\latin@roman{#1}}}
\def\greek@Roman#1{\textlatin{\latin@Roman{#1}}}
\addto\extrasgreek{%
  \let\@roman\greek@roman
  \let\@Roman\greek@Roman}
\addto\noextrasgreek{%
  \let\@roman\latin@roman
  \let\@Roman\latin@Roman}
%    \end{macrocode}
%  \end{macro}
%  \end{macro}
%
%  \begin{macro}{\greek@amp}
%  \begin{macro}{\ltx@amp}
%    The greek fonts do not contain an ampersand, so the \LaTeX\
%    command |\&| dosn't give the expected result if we do not do
%    something about it.
% \changes{greek-1.2f}{1999/04/25}{Now switch the definition of
%    \cs{\&} from \cs{extrasgreek}} 
% \changes{greek-1.3c}{1999/05/17}{Added a missing opening brace}
%    \begin{macrocode}
\let\ltx@amp\&
\def\greek@amp{\textlatin{\ltx@amp}}
\addto\extrasgreek{\let\&\greek@amp}
\addto\noextrasgreek{\let\&\ltx@amp}
%    \end{macrocode}
%  \end{macro}
%  \end{macro}
%
%    What is left now is the definition of a set of macros to produce
%    the various digits.
%  \begin{macro}{\gr@num@i}
%  \begin{macro}{\gr@num@ii}
%  \begin{macro}{\gr@num@iii}
% \changes{greek-1.2b}{1997/11/13}{No longer use \cs{\let} in the
%    expansion of the \cs{gr@num@x} macros as they ned to be
%    expandable}
%    As there is no representation for $0$ in this system the zeros
%    are simply discarded. When we have a large number with three
%    \emph{trailing} zero's also the numeric mark is discarded. 
%    Therefore these macros need to pass the information to each other
%    about the (non-)translation of a zero.
%    \begin{macrocode}
\def\gr@num@i#1{%
  \ifcase#1\or a\or b\or g\or d\or e\or \stigma\or z\or h\or j\fi
  \ifnum#1=\z@\else\anw@true\fi\anw@print}
\def\gr@num@ii#1{%
  \ifcase#1\or i\or k\or l\or m\or n\or x\or o\or p\or \qoppa\fi
  \ifnum#1=\z@\else\anw@true\fi\gr@num@i}
\def\gr@num@iii#1{%
  \ifcase#1\or r\or sv\or t\or u\or f\or q\or y\or w\or \sampi\fi
  \ifnum#1=\z@\anw@false\else\anw@true\fi\gr@num@ii}
%    \end{macrocode}
%  \end{macro}
%  \end{macro}
%  \end{macro}
%
%  \begin{macro}{\gr@num@iv}
%  \begin{macro}{\gr@num@v}
%  \begin{macro}{\gr@num@vi}
%    The first three `digits' always have the numeric mark, except
%    when one is discarded because it's value is zero.
%    \begin{macrocode}
\def\gr@num@iv#1{%
  \ifnum#1=\z@\else\katwtonos\fi
  \ifcase#1\or a\or b\or g\or d\or e\or \stigma\or z\or h\or j\fi
  \gr@num@iii}
\def\gr@num@v#1{%
  \ifnum#1=\z@\else\katwtonos\fi
  \ifcase#1\or i\or k\or l\or m\or n\or x\or o\or p\or \qoppa\fi
  \gr@num@iv}
\def\gr@num@vi#1{%
  \katwtonos
  \ifcase#1\or r\or sv\or t\or u\or f\or q\or y\or w\or \sampi\fi
  \gr@num@v}
%    \end{macrocode}
%  \end{macro}
%  \end{macro}
%  \end{macro}
%
%  \begin{macro}{\greek@tilde}
% \changes{greek-1.0c}{1997/02/19}{Added command}
%    In greek typesetting we need a number of characters with more
%    than one accent. In the underlying family of fonts (the
%    |cb| fonts) this is solved using Knuth's ligature
%    mechanism. 
%    Characters we need to have ligatures with are the tilde, the
%    acute and grave accent characters, the rough and smooth breathings,
%    the subscript, and the double quote character.
%    In text input the |~| is normaly used to produce an
%    unbreakable space. The command |\~| normally produces a tilde
%    accent. For  polytonic Greek we change the definition of |\~| 
%    to produce the tilde character itself, making sure it has category 
%    code 12.
% \changes{greek-1.3k}{2003/04/10}{Make sure the character `!' is not
%    active during the definition of \cs{greek@tilde}} 
%    \begin{macrocode}
\begingroup
  \@ifundefined{active@char\string!}{}{\catcode`!=12\relax}
  \catcode`\~=12
  \lccode`\!=`\~
  \lowercase{\def\x{\endgroup
      \def\greek@tilde{!}}\x}
\addto\extraspolutonikogreek{%
  \babel@save\~\let\~\greek@tilde}
%    \end{macrocode}
%  \end{macro}
%    In order to get correct hyphenation we need to set the lower case
%    code of a number of characters. The `v' character has a special
%    usage for the |cb| fonts: in fact this ligature mechanism detects
%    the end of a word and assures that a final sigma is typeset with
%    the proper sign wich is different from that of an initial or
%    medial sigma;  the  `v  'after  an  \textit{isolated} sigma fools
%    the ligature mechanism in order to typeset $\sigma$ in place of
%    $\varsigma$. Because of this we make sure its lowercase code is
%    not changed. For ``modern'' greek we have to deal only with |'|
%    and |"| and so things are easy.
% \changes{greek-1.1c}{1997/04/30}{fixed two typos}
% \changes{greek-1.1e}{1997/10/12}{Added lowercase code for v}
% \changes{greek-1.2}{1997/10/28}{Definitions for ``modern'' Greek are
%    now the definitions of ``Polutoniko'' Greek} 
% \changes{greek-1.2}{1997/10/28}{Added lowercase codes for ``modern''
%    greek} 
% \changes{greek-1.3e}{1999/09/24}{\cs{extrasgreek} and
%    \cs{extraspolutonikogreek} should be complementary}
%    \begin{macrocode}
\addto\extrasgreek{%
  \babel@savevariable{\lccode`v}\lccode`v=`v%
  \babel@savevariable{\lccode`\'}\lccode`\'=`\'%
  \babel@savevariable{\lccode`\"}\lccode`\"=`\"}
\addto\extraspolutonikogreek{%
  \babel@savevariable{\lccode`\<}\lccode`\<=`\<%
  \babel@savevariable{\lccode`\>}\lccode`\>=`\>%
  \babel@savevariable{\lccode`\~}\lccode`\~=`\~%
  \babel@savevariable{\lccode`\|}\lccode`\|=`\|%
  \babel@savevariable{\lccode`\`}\lccode`\`=`\`}
%    \end{macrocode}
%    And in order to get rid of all accents and breathings when a
%    string is |\uppercase|d we also change a number of uppercase
%    codes. 
% \changes{greek-1.1b}{1997/03/06}{Added setting of \cs{uccode}s
%    (after \file{kdgreek.sty})}
% \changes{greek-1.1e}{1997/10/12}{Added uppercase code for special
%    letter ``v''. Uppercase code for accents is now \texttt{9f},
%    instead of \texttt{ff}}
% \changes{greek-1.2}{1997/10/28}{Added uppercase codes for ``modern''
%    Greek. The old codes are now for ``Polutoniko'' Greek} 
% \changes{greek-1.3e}{1999/09/24}{\cs{extrasgreek} and
%    \cs{extraspolutonikogreek} should be complementary}
% {\catcode`|=12\relax\gdef\indexbar{\cs{|}}}
% \changes{greek-1.3g}{1999/11/17}{uc code of \indexbar{} is now just
%    \indexbar{} to reflect recent changes in the cb fonts}
% \changes{greek-1.3i}{2000/10/02}{uc code of `v' is switched to V
%    so that mixed text appears correctly in headers.}
% \changes{greek-1.3j}{2001/02/03}{Because other languages might
%    make the caret active, we can't use the double caret notation
%    here}
%    \begin{macrocode}
\addto\extrasgreek{%
  \babel@savevariable{\uccode`\"}\uccode`\"=`\"%
  \babel@savevariable{\uccode`\'}\uccode`\'=159} %% 159 == ^^9f
\addto\extraspolutonikogreek{%
  \babel@savevariable{\uccode`\~}\uccode`\~=159%
  \babel@savevariable{\uccode`\>}\uccode`\>=159%
  \babel@savevariable{\uccode`\<}\uccode`\<=159%
  \babel@savevariable{\uccode`\|}\uccode`\|=`\|%
  \babel@savevariable{\uccode`\`}\uccode`\`=159}
%    \end{macrocode}
%    For this to work we make the character |^^9f| a shorthand that
%    expands to nothing. In order for this to work we need to make a
%    character look like |^^9f| in \TeX's eyes. The trick is to have
%    another character and assign it a different lowercase code. The
%    execute the macros needed in a |\lowercase| environment. Usually
%    the tile |~| character is used for such purposes. Before we do
%    this we save it's original lowercase code to restore it once
%    we're done.
% \changes{greek-1.1b}{1997/03/06}{Added shorthand for \cs{char255}}
% \changes{greek-1.1e}{1997/10/12}{Shorthand is changed. Active
%    character is now \cs{char159}} 
% \changes{greek-1.2a}{1997/10/31}{Need shorthand to exist for
%    ``monotoniko'' Greek, not ``polutoniko'' Greek}
% \changes{greek-1.3j}{2001/02/03}{Ues the tilde as an alias for
%    character 159}
%    \begin{macrocode}
\@tempcnta=\lccode`\~
\lccode`\~=159
\lowercase{%
  \initiate@active@char{~}%
  \declare@shorthand{greek}{~}{}}
\lccode`\~=\@tempcnta
%    \end{macrocode}
%    We can also make the tilde character itself expand to a tilde with
%    category code 12 to make the typing of texts easier.
% \changes{greek-1.1b}{1997/03/06}{Made tilde expand to a tilde with
%    \cs{catcode 12}} 
%    \begin{macrocode}
\addto\extraspolutonikogreek{\languageshorthands{greek}}%
\declare@shorthand{greek}{~}{\greek@tilde}
%    \end{macrocode}
%  \end{macro}
%  \end{macro}
%  
%  
%
% \changes{greek-1.1c}{1997/03/10}{Added a couple of symbols, needed
%    for \cs{greeknumeral}}
% \changes{greek-1.1e}{1997/10/12}{Most symbols are removed and are
%    now defined in package grsymb} 
% \changes{greek-1.2c}{1998/06/26}{Package grsymb has been eliminated
%   because the CB fonts v2.0 do not inlcude certain symbols and so
%   the remaining symbol definitions have been moved here}
%    We now define a few symbols which are used in the typesetting of
%    greek numerals, as well as some other symbols which are usefull,
%    such as the $\epsilon\upsilon\rho\omega$ symbol, etc.
%    \begin{macrocode}
\DeclareTextCommand{\anwtonos}{LGR}{\char"FE\relax}
\DeclareTextCommand{\katwtonos}{LGR}{\char"FF\relax}
\DeclareTextCommand{\qoppa}{LGR}{\char"12\relax}
\DeclareTextCommand{\stigma}{LGR}{\char"06\relax}
\DeclareTextCommand{\sampi}{LGR}{\char"1B\relax}
\DeclareTextCommand{\Digamma}{LGR}{\char"C3\relax}
\DeclareTextCommand{\ddigamma}{LGR}{\char"93\relax}
\DeclareTextCommand{\vardigamma}{LGR}{\char"07\relax}
\DeclareTextCommand{\euro}{LGR}{\char"18\relax}
\DeclareTextCommand{\permill}{LGR}{\char"19\relax}
%    \end{macrocode}
%
%    Since the |~| cannot be used to produce an unbreakable white
%    space we must redefine at least the commands |\fnum@figure| and
%    |\fnum@table| so they do not produce a |~| instead of white
%    space. 
% \changes{greek-1.3l}{2004/02/19}{Commented these lines out as this
%    change has made it into \LaTeX{} itself.} 
%    \begin{macrocode} 
%\def\fnum@figure{\figurename\nobreakspace\thefigure}
%\def\fnum@table{\tablename\nobreakspace\thetable}
%    \end{macrocode}
%
%    The macro |\ldf@finish| takes care of looking for a
%    configuration file, setting the main language to be switched on
%    at |\begin{document}| and resetting the category code of
%    \texttt{@} to its original value.
% \changes{greek-1.0b}{1996/11/02}{Now use \cs{ldf@finish} to wrap up}
%    \begin{macrocode}
\ldf@finish{\CurrentOption}
%</code>
%    \end{macrocode}
%
% \Finale
%\endinput
%% \CharacterTable
%%  {Upper-case    \A\B\C\D\E\F\G\H\I\J\K\L\M\N\O\P\Q\R\S\T\U\V\W\X\Y\Z
%%   Lower-case    \a\b\c\d\e\f\g\h\i\j\k\l\m\n\o\p\q\r\s\t\u\v\w\x\y\z
%%   Digits        \0\1\2\3\4\5\6\7\8\9
%%   Exclamation   \!     Double quote  \"     Hash (number) \#
%%   Dollar        \$     Percent       \%     Ampersand     \&
%%   Acute accent  \'     Left paren    \(     Right paren   \)
%%   Asterisk      \*     Plus          \+     Comma         \,
%%   Minus         \-     Point         \.     Solidus       \/
%%   Colon         \:     Semicolon     \;     Less than     \<
%%   Equals        \=     Greater than  \>     Question mark \?
%%   Commercial at \@     Left bracket  \[     Backslash     \\
%%   Right bracket \]     Circumflex    \^     Underscore    \_
%%   Grave accent  \`     Left brace    \{     Vertical bar  \|
%%   Right brace   \}     Tilde         \~}
%%
%
  \languageattribute{greek}{polutoniko}}
%    \end{macrocode}
% \changes{babel~3.7a}{1998/03/27}{Added the \Lopt{hebrew} option}
%    \begin{macrocode}
\DeclareOption{hebrew}{%
  \input{rlbabel.def}%
  %%
%% This file will generate fast loadable files and documentation
%% driver files from the doc files in this package when run through
%% LaTeX or TeX.
%%
%% Copyright 1989-2005 Johannes L. Braams and any individual authors
%% listed elsewhere in this file.  All rights reserved.
%% 
%% This file is part of the Babel system.
%% --------------------------------------
%% 
%% It may be distributed and/or modified under the
%% conditions of the LaTeX Project Public License, either version 1.3
%% of this license or (at your option) any later version.
%% The latest version of this license is in
%%   http://www.latex-project.org/lppl.txt
%% and version 1.3 or later is part of all distributions of LaTeX
%% version 2003/12/01 or later.
%% 
%% This work has the LPPL maintenance status "maintained".
%% 
%% The Current Maintainer of this work is Johannes Braams.
%% 
%% The list of all files belonging to the LaTeX base distribution is
%% given in the file `manifest.bbl. See also `legal.bbl' for additional
%% information.
%% 
%% The list of derived (unpacked) files belonging to the distribution
%% and covered by LPPL is defined by the unpacking scripts (with
%% extension .ins) which are part of the distribution.
%%
%% --------------- start of docstrip commands ------------------
%%
%%
%% Copyright (C) 1997 -- 1998 Boris Lavva.
%% Copyright (C) 1989 -- 2004 by Johannes Braams,
%%                            TeXniek
%%                            All rights reserved.
%%
%% This file is contributed to the `babel' system.
%%
%% You are allowed to distribute this file together with all files
%% mentioned in manifest.bbl.
%%
%% You are not allowed to modify its contents.
%%
\def\filedate{2004/02/20}
\def\batchfile{hebrew.ins}
\input docstrip

{\ifx\generate\undefined
\Msg{**********************************************}
\Msg{*}
\Msg{* This installation requires docstrip}
\Msg{* version 2.3c or later.}
\Msg{*}
\Msg{* An older version of docstrip has been input}
\Msg{*}
\Msg{**********************************************}
\errhelp{Move or rename old docstrip.tex.}
\errmessage{Old docstrip in input path}
\batchmode
\csname @@end\endcsname
\fi}

\declarepreamble\mainpreamble

This is a generated file.

Copyright 1997-2004 Boris Lavva and any individual authors
listed elsewhere in this file. All rights reserved.

This is a generated file.

Copyright 1989-2005 Johannes L. Braams and any individual authors
listed elsewhere in this file.  All rights reserved.

This file was generated from file(s) of the Babel system.
---------------------------------------------------------

It may be distributed and/or modified under the
conditions of the LaTeX Project Public License, either version 1.3
of this license or (at your option) any later version.
The latest version of this license is in
  http://www.latex-project.org/lppl.txt
and version 1.3 or later is part of all distributions of LaTeX
version 2003/12/01 or later.

This work has the LPPL maintenance status "maintained".

The Current Maintainer of this work is Johannes Braams.

This file may only be distributed together with a copy of the Babel
system. You may however distribute the Babel system without
such generated files.

The list of all files belonging to the Babel distribution is
given in the file `manifest.bbl'. See also `legal.bbl for additional
information.

The list of derived (unpacked) files belonging to the distribution
and covered by LPPL is defined by the unpacking scripts (with
extension .ins) which are part of the distribution.
\endpreamble

\declarepreamble\fdpreamble

This is a generated file.

Copyright 1997-2004 Boris Lavva and any individual authors
listed elsewhere in this file. All rights reserved.

This is a generated file.

Copyright 1989-2005 Johannes L. Braams and any individual authors
listed elsewhere in this file.  All rights reserved.

This file was generated from file(s) of the Babel system.
---------------------------------------------------------

It may be distributed and/or modified under the
conditions of the LaTeX Project Public License, either version 1.3
of this license or (at your option) any later version.
The latest version of this license is in
  http://www.latex-project.org/lppl.txt
and version 1.3 or later is part of all distributions of LaTeX
version 2003/12/01 or later.

This work has the LPPL maintenance status "maintained".

The Current Maintainer of this work is Johannes Braams.

This file may only be distributed together with a copy of the Babel
system. You may however distribute the Babel system without
such generated files.

The list of all files belonging to the Babel distribution is
given in the file `manifest.bbl'. See also `legal.bbl for additional
information.

In particular, permission is granted to customize the declarations in
this file to serve the needs of your installation.

However, NO PERMISSION is granted to distribute a modified version
of this file under its original name.

\endpreamble

\keepsilent

\usedir{tex/generic/babel}
 
\Msg{*** Generating hebrew font encoding files ***}
\usepreamble\fdpreamble
\generate{\file{lheenc.def}{\from{hebrew.fdd}{LHEenc}}
          \file{lhecmr.fd}{\from{hebrew.fdd}{LHEcmr,nowarn}}
          \file{lhecmss.fd}{\from{hebrew.fdd}{LHEcmss,nowarn}}
          \file{lhecmtt.fd}{\from{hebrew.fdd}{LHEcmtt,nowarn}}
          \file{lheclas.fd}{\from{hebrew.fdd}{LHEclas,nowarn}}
          \file{he8enc.def}{\from{hebrew.fdd}{HE8enc}}
          \file{he8cmr.fd}{\from{hebrew.fdd}{HE8cmr,nowarn}}
          \file{he8cmss.fd}{\from{hebrew.fdd}{HE8cmss,nowarn}}
          \file{he8cmtt.fd}{\from{hebrew.fdd}{HE8cmtt,nowarn}}
          \file{he8aharoni.fd}{\from{hebrew.fdd}{HE8aharoni,nowarn}}
          \file{he8david.fd}{\from{hebrew.fdd}{HE8david,nowarn}}
          \file{he8drugulin.fd}{\from{hebrew.fdd}{HE8drugulin,nowarn}}
          \file{he8frankruehl.fd}{\from{hebrew.fdd}{HE8frankruehl,nowarn}}
          \file{he8yad.fd}{\from{hebrew.fdd}{HE8yad,nowarn}}
          \file{he8miriam.fd}{\from{hebrew.fdd}{HE8miriam,nowarn}}
          \file{he8nachlieli.fd}{\from{hebrew.fdd}{HE8nachlieli,nowarn}}
          \file{he8OmegaHebrew.fd}{\from{hebrew.fdd}{HE8OmegaHebrew,nowarn}}
          \file{lheshold.fd}{\from{hebrew.fdd}{LHEshold,nowarn}}
          \file{lheshscr.fd}{\from{hebrew.fdd}{LHEshscr,nowarn}}
          \file{lheshstk.fd}{\from{hebrew.fdd}{LHEshstk,nowarn}}
          \file{lhefr.fd}{\from{hebrew.fdd}{LHEfr,nowarn}}
          \file{lhecrml.fd}{\from{hebrew.fdd}{LHEcrml,nowarn}}
          \file{lheredis.fd}{\from{hebrew.fdd}{LHEredis,nowarn}}
          \file{hebfont.sty}{\from{hebrew.fdd}{hebfont}}
          }

\Msg{*** Generating hebrew input encoding files ***}
\usepreamble\mainpreamble
\generate{\file{8859-8.def}{\from{hebinp.dtx}{8859-8}}
          \file{cp1255.def}{\from{hebinp.dtx}{cp1255}}
          \file{cp862.def}{\from{hebinp.dtx}{cp862}}
          \file{si960.def}{\from{hebinp.dtx}{si960}}
          }

\Msg{*** Generating hebrew language support files ***}
\generate{\file{hebrew.ldf}{\from{hebrew.dtx}{hebrew}}
          \file{rlbabel.def}{\from{hebrew.dtx}{rightleft}}
          \file{hebcal.sty}{\from{hebrew.dtx}{calendar}}
          }

\Msg{*** Generating hebrew 2.09 compatibility files ***}
\generate{\file{hebrew_newcode.sty}{\from{heb209.dtx}{newcode}}
          \file{hebrew_p.sty}{\from{heb209.dtx}{pccode}}
          \file{hebrew_oldcode.sty}{\from{heb209.dtx}{oldcode}}
          }

\ifToplevel{
\Msg{*************************************************************}
\Msg{*}
\Msg{* To finish the installation you have to move the following}
\Msg{* files into a directory searched by TeX:}
\Msg{*}
\Msg{* \space\space All *.cls, *.sty, *.ldf, *.fd and *.def}
\Msg{*}
\Msg{* To produce the documentation run `hebrew.dtx' through LaTeX}
\Msg{*}
\Msg{* Happy TeXing}
\Msg{*************************************************************}
}
 
\endbatchfile
}
%    \end{macrocode}
%    \Lopt{hungarian} is just a synonym for \Lopt{magyar}
%^^A \changes{babel~3.6i}{1997/02/21}{Added the \Lopt{kannada} option}
% \changes{babel~3.7a}{1997/05/21}{Added the \Lopt{icelandic} option}
%^^A \changes{babel~3.7a}{1998/03/30}{{Added the \Lopt{nagari} option}
% \changes{babel~3.7b}{1998/06/25}{Added the \Lopt{latin} option}
% \changes{babel~3.7m}{2003/11/13}{Added the \Lopt{interlingua}
%    option}
%    \begin{macrocode}
\DeclareOption{hungarian}{% \iffalse meta-comment
%
% Copyright 1989-2005 Johannes L. Braams and any individual authors
% listed elsewhere in this file.  All rights reserved.
% 
% This file is part of the Babel system.
% --------------------------------------
% 
% It may be distributed and/or modified under the
% conditions of the LaTeX Project Public License, either version 1.3
% of this license or (at your option) any later version.
% The latest version of this license is in
%   http://www.latex-project.org/lppl.txt
% and version 1.3 or later is part of all distributions of LaTeX
% version 2003/12/01 or later.
% 
% This work has the LPPL maintenance status "maintained".
% 
% The Current Maintainer of this work is Johannes Braams.
% 
% The list of all files belonging to the Babel system is
% given in the file `manifest.bbl. See also `legal.bbl' for additional
% information.
% 
% The list of derived (unpacked) files belonging to the distribution
% and covered by LPPL is defined by the unpacking scripts (with
% extension .ins) which are part of the distribution.
% \fi
% \CheckSum{1538}
% \iffalse
%    Tell the \LaTeX\ system who we are and write an entry on the
%    transcript.
%<*dtx>
\ProvidesFile{magyar.dtx}
%</dtx>
%<code>\ProvidesLanguage{magyar}
%\fi
%\ProvidesFile{magyar.dtx}
        [2005/03/30 v1.4j Magyar support from the babel system]
%\iffalse
%% File `magyar.dtx'
%% Babel package for LaTeX version 2e
%% Copyright (C) 1989 - 2005
%%           by Johannes Braams, TeXniek
%
%% Magyar Language Definition File
%% Copyright (C) 1989 - 2005
%%           by Johannes Braams, TeXniek
%%              \'Arp\'ad B\'IR\'O
%%              J\'ozsef B\'ERCES
%
% Please report errors to: J.L. Braams babel braams.cistron.nl
%
%    This file is part of the babel system, it provides the source
%    code for the Hungarian language definition file.  A contribution
%    was made by Attila Koppanyi (attila@cernvm.cern.ch).
%<*filedriver>
\documentclass{ltxdoc}
\newcommand*\TeXhax{\TeX hax}
\newcommand*\babel{\textsf{babel}}
\newcommand*\langvar{$\langle \it lang \rangle$}
\newcommand*\note[1]{}
\newcommand*\Lopt[1]{\textsf{#1}}
\newcommand*\file[1]{\texttt{#1}}
\begin{document}
 \DocInput{magyar.dtx}
\end{document}
%</filedriver>
%\fi
% \GetFileInfo{magyar.dtx}
%
% \changes{magyar-1.0a}{1991/07/15}{Renamed \file{babel.sty} in
%    \file{babel.com}}
% \changes{magyar-1.1}{1992/02/16}{Brought up-to-date with babel 3.2a}
% \changes{magyar-1.1d}{1994/02/08}{Further spelling corrections}
% \changes{magyar-1.1e}{1994/02/09}{Still more spelling corrections}
% \changes{magyar-1.2}{1994/02/27}{Update for \LaTeXe}
% \changes{magyar-1.3c}{1994/06/26}{Removed the use of \cs{filedate}
%    and moved identification after the loading of \file{babel.def}}
% \changes{magyar-1.3g}{1996/07/12}{Replaced \cs{undefined} with
%    \cs{@undefined} and \cs{empty} with \cs{@empty} for consistency
%    with \LaTeX}
% \changes{magyar-1.4a}{1998/06/04}{order inverting in
%    headings/titles/captions; definite article handling; active char
%    for special hyphenation}
% \changes{magyar-1.4d}{2001/11/15}{Corrected checksum}
% \changes{magyar-1.4j}{2004/11/17}{Added missing comment characters
%    in the redefinitions of \cs{ps@headings} to prevent spurious
%    spaces} 
%
%  \section{The Hungarian language}
%
%    The file option \file{\filename} defines all the
%    language definition macros for the Hungarian language.
%
%    The \babel{} support for the Hungarian language until file
%    version 1.3i was essentially changing the English document
%    elements to Hungarian ones, but because of the differences
%    between these too languages this was actually unusable (`Part I'
%    was transferred to `R\'esz I' which is not usable instead of `I.\
%    r\'esz'). To enhance the typesetting facilities for Hungarian
%    the following should be considered:
%    \begin{itemize}
%      \item In Hungarian documents there is a period after
%            the part, section, subsection \mbox{etc.} numbers.
%
%      \item In the part, chapter, appendix name the number
%            (or letter) goes before the name, so `Part I' translates to
%            `I.\ r\'esz'.
%
%       \item The same is true with captions
%             (`Table 2.1' goes to `2.1.\ t\'abl\'azat').
%
%       \item There is a period after the caption name instead of a colon.
%             (`Table 2.1:' goes to `2.1.\ t\'abl\'azat.')
%
%       \item There is a period at the end of the title in a run-in head
%             (when |afterskip<0| in |\@startsection|).
%
%       \item Special hyphenation rules must be applied
%             for the so-called long double consonants (ccs, ssz,\dots).
%
%       \item The opening quotation mark is like the German one
%             (the closing is the same as in English).
%
%       \item In Hungarian figure, table, \mbox{etc.}
%             referencing a definite article is also incorporated.
%             The Hungarian definite articles behave like the English
%             indefinite ones (`a/an'). `a' is used for words beginning
%             with a consonant and `az' goes for a vowel.
%             Since some numbers begin with a vowel some others
%             with a consonant some commands should be provided for
%             automatic definite article generation.
%    \end{itemize}
%
%    Until file version 1.3i\footnote{That file was
%    last revised on 1996/12/23 with a contribution by the next
%    authors: Attila Kopp\'anyi (\texttt{attila@cernvm.cern.ch}),
%    \'Arp\'ad B\'{\i}r\'o (\texttt{JZP1104@HUSZEG11.bitnet}),
%    Istv\'an Hamecz (\texttt{hami@ursus.bke.hu)} and
%    Dezs\H{o} Horv\'ath (\texttt{horvath@pisa.infn.it}).} 
%    the special typesetting rules of the Hungarian language
%    mentioned above were not taken into consideration.
%    This version (\fileversion)\footnote{It was written by J\'ozsef
%    B\'erces (\texttt{jozsi@docs4.mht.bme.hu}) with some help from
%    Ferenc Wettl (\texttt{wettl@math.bme.hu}) and an idea from David
%    Carlisle (\texttt{david@dcarlisle.demon.co.uk}).} 
%    enables \babel{} to typeset `good-looking' Hungarian texts.
%
% \DescribeMacro\ontoday
%    The |\ontoday| command works like |\today| but
%    produces a slightly different date format used in expressions such
%    as `on February 10th'.
%
% \DescribeMacro\Az
%    The commands |\Az#1| and |\az#1| write the correct
%    definite article for the argument and the argument itself
%    (separated with a |~|). The star-forms (|\Az*| and |\az*|)
%    produce the article only.
%
% \DescribeMacro\Azr
%    |\Azr#1| and |\azr#1| treat the argument as a label so expand
%    it then write the definite article for |\r@#1|, a non-breakable
%    space then the label expansion. The star-forms do not print
%    the label expansion. |\Azr(#1| and |\azr(#1| are used for
%    equation referencing with the syntax |\azr(|\textit{label}|)|.
%
% \DescribeMacro\Aref
%    There are two aliases |\Aref| and |\aref| for |\Azr| and |\azr|,
%    respectively. During the preparation of a document it is not
%    known in general, if the code `|a~\ref{|\textit{label}|}|' or the code
%    `|az~\ref{|\textit{label}|}|' is the grammatically correct one. Writing
%    `|\aref{|\textit{label}|}|' instead of the previous ones solves the 
%    problem.
%
% \DescribeMacro\Azp
%    |\Azp#1| and |\azp#1| also treat the argument as a label but
%    use the label's page for definite article determination.
%    There are star-forms giving only the definite article without
%    the page number. 
%
% \DescribeMacro\Apageref
%    There are aliases |\Apageref| and |\apageref| for |\Azp| and
%    |\azp|, respectively. The code |\apageref{|\textit{label}|}| 
%    is equivalent either to |a~\pageref{|\textit{label}|}| or 
%    to |az~\pageref{|\textit{label}|}|.
%
% \DescribeMacro\Azc
%    |\Azc| and |\azc| work like the |\cite| command but
%    (of course) they insert the definite article. There can be
%    several comma separated cite labels and in that case the
%    definite article is given for the first one.
%    They accept |\cite|'s optional argument.
%    There are star-forms giving the definite article only.
%
% \DescribeMacro\Acite
%    There are aliases |\Acite| and |\acite| for
%    |\Azc| and |\azc|, respectively. 
%
%    For this language the character |`| is made active.
%    Table~\ref{tab:hun-actives} shows the shortcuts.
%    The main reason for the activation of the |`| character
%    is to handle the special hyphenation of the long double
%    consonants.
%    \begin{table}
%      \begin{center}
%        \begin{tabular}{lp{45mm}p{47mm}}
%          shortcut & explanation & example \\
%          \hline
%          |``|        &  same as |\glqq| in \babel{}, or
%                         |\quotedblbase| in T1 
%                         (opening quotation mark, like ,,) 
%                        & |``id\'ezet''|$\longrightarrow$,,id\'ezet'{}' \\
%          |`c|, |`C|  &  ccs is hyphenated as cs-cs
%                        & |lo`ccsan|$\longrightarrow$locs-csan \\
%          |`d|, |`D|  &  ddz is hyphenated as dz-dz
%                        & |e`ddz\"unk|$\longrightarrow$edz-dz\"unk \\
%          |`g|, |`G|  &  ggy is hyphenated as gy-gy
%                        & |po`ggy\'asz|$\longrightarrow$pogy-gy\'asz \\
%          |`l|, |`L|  &  lly is hyphenated as ly-ly
%                        & |Kod\'a`llyal|$\longrightarrow$Kod\'aly-lyal \\
%          |`n|, |`N|  &  nny is hyphenated as ny-ny
%                        & |me`nnyei|$\longrightarrow$meny-nyei \\
%          |`s|, |`S|  &  ssz is hyphenated as sz-sz
%                        & |vi`ssza|$\longrightarrow$visz-sza \\
%          |`t|, |`T|  &  tty is hyphenated as ty-ty
%                        & |po`ttyan|$\longrightarrow$poty-tyan \\
%          |`z|, |`Z|  &  zzs is hyphenated as zs-zs
%                        & |ri`zzsel|$\longrightarrow$rizs-zsel \\
%        \end{tabular}
%        \caption{The shortcuts defined in \file{magyar.ldf}}
%        \label{tab:hun-actives}
%      \end{center}
%    \end{table}
%
%
% \StopEventually{}
%
%    The macro |\LdfInit| takes care of preventing that this file is
%    loaded more than once, checking the category code of the
%    \texttt{@} sign, etc.
% \changes{magyar-1.3h}{1996/11/03}{Now use \cs{LdfInit} to perform
%    initial checks}
%    \begin{macrocode}
%<*code>
\LdfInit{magyar}{caption\CurrentOption}
%    \end{macrocode}
%
%    When this file is read as an option, i.e. by the |\usepackage|
%    command, \texttt{magyar} will be an `unknown' language in which
%    case we have to make it known.  So we check for the existence of
%    |\l@magyar| or |\l@hungarian| to see whether we have to do
%    something here.
%
% \changes{magyar-1.0b}{1991/10/29}{Removed use of \cs{@ifundefined}}
% \changes{magyar-1.1}{1992/02/16}{Added a warning when no hyphenation
%    patterns were loaded.}
% \changes{magyar-1.3c}{1994/06/26}{Now use \cs{@nopatterns} to
%    produce the warning}
% \changes{magyar-1.4d}{2003/09/18}{The \cs{else} clause got outside
%    of the \cs{if} statement, breaking the Hungarian support}
% \changes{magyar-1.4g}{2003/10/07}{Further change to make it work
%    when neither \cs{l@magyar} nor \cs{l@hugarian} are defined}
%    \begin{macrocode}
\ifx\l@magyar\@undefined
  \ifx\l@hungarian\@undefined
    \@nopatterns{Magyar}
    \adddialect\l@magyar0
  \else
    \let\l@magyar\l@hungarian
  \fi
\fi
%    \end{macrocode}
%    The statement above makes sure that |\l@magyar| is always
%    defined; if |\l@hungarian| is still undefined we make it equal to
%    |\l@magyar|. 
%    \begin{macrocode}
\ifx\l@hungarian\@undefined
  \let\l@hungarian\l@magyar
\fi
%    \end{macrocode}
%
%    The next step consists of defining commands to switch to (and
%    from) the Hungarian language.
%
% \begin{macro}{\captionsmagyar}
%    The macro |\captionsmagyar| defines all strings used in the four
%    standard document classes provided with \LaTeX.
% \changes{magyar-1.1}{1992/02/16}{Added \cs{seename}, \cs{alsoname}
%    and \cs{prefacename}}
% \changes{magyar-1.1}{1993/07/15}{\cs{headpagename} should be
%    \cs{pagename}}
% \changes{magyar-1.1c}{1994/01/05}{Added translations, fixed typos}
% \changes{magyar-1.3e}{1995/07/04}{Added \cs{proofname} for
%    AMS-\LaTeX}
% \changes{magyar-1.3f}{1996/04/18}{translated Proof and replaced some
%    translations}
% \changes{magyar-1.4a}{1998/06/04}{the initial letter of fejezet,
%    t\'abl\'azat, r\'esz, l\'asd changed to lowercase}
%    \begin{macrocode}
\@namedef{captions\CurrentOption}{%
  \def\prefacename{El\H osz\'o}%
%    \end{macrocode}
%    For the list of references at the end of an article we have a
%    choice between two words, `Referenci\'ak' (a Hungarian version of
%    the English word) and `Hivatkoz\'asok'. The latter seems to
%    be in more widespread use.
%    \begin{macrocode}
  \def\refname{Hivatkoz\'asok}%
%    \end{macrocode}
%    If you have a document with a summary instead of an abstract you
%    might want to replace the word `Kivonat' with
%    `\"Osszefoglal\'o'.
%    \begin{macrocode}
  \def\abstractname{Kivonat}%
%    \end{macrocode}
%    The Hungarian version of `Bibliography' is `Bibliogr\'afia', but
%    a more natural word to use is `Irodalomjegyz\'ek'.
%    \begin{macrocode}
  \def\bibname{Irodalomjegyz\'ek}%
  \def\chaptername{fejezet}%
  \def\appendixname{F\"uggel\'ek}%
  \def\contentsname{Tartalomjegyz\'ek}%
  \def\listfigurename{\'Abr\'ak jegyz\'eke}%
  \def\listtablename{T\'abl\'azatok jegyz\'eke}%
  \def\indexname{T\'argymutat\'o}%
  \def\figurename{\'abra}%
  \def\tablename{t\'abl\'azat}%
  \def\partname{r\'esz}%
  \def\enclname{Mell\'eklet}%
  \def\ccname{K\"orlev\'el--c\'\i mzettek}%
  \def\headtoname{C\'\i mzett}%
  \def\pagename{oldal}%
  \def\seename{l\'asd}%
  \def\alsoname{l\'asd m\'eg}%
%    \end{macrocode}
%    Besides the Hungarian word for Proof, `Bizony\'\i t\'as' we can
%    also name Corollary (K\"ovetkezm\'eny), Theorem (T\'etel) and
%    Lemma (Lemma).
% \changes{magyar-1.4b}{2000/09/20}{Added \cs{glossaryname}}
% \changes{magyar-1.4h}{2003/11/20}{Inserted translation for Glossary} 
%    \begin{macrocode}
  \def\proofname{Bizony\'\i t\'as}%
  \def\glossaryname{Sz\'ojegyz\'ek}%
  }%
%    \end{macrocode}
% \end{macro}
%
% \begin{macro}{\datemagyar}
%    The macro |\datemagyar| redefines the command |\today| to produce
%    Hungarian dates.
% \changes{magyar-1.1d}{1994/02/08}{Rewritten to produce the correct
%    date format}
% \changes{magyar-1.4a}{1998/06/10}{Use \cs{number}\cs{day} instead of
%    \cs{ifcase} construct}
%    \begin{macrocode}
\@namedef{date\CurrentOption}{%
  \def\today{%
    \number\year.\nobreakspace\ifcase\month\or
    janu\'ar\or febru\'ar\or m\'arcius\or
    \'aprilis\or m\'ajus\or j\'unius\or
    j\'ulius\or augusztus\or szeptember\or
    okt\'ober\or november\or december\fi
    \space\number\day.}}
%    \end{macrocode}
% \end{macro}
%
% \begin{macro}{\ondatemagyar}
%    The macro |\ondatemagyar| produces Hungarian dates which have the
%    meaning `\emph{on this day}'.  It does not redefine the command
%    |\today|.
% \changes{magyar-1.1c}{1994/01/05}{The date number should not be
%    followed by a dot.}
% \changes{magyar-1.1d}{1994/02/08}{Renamed from \cs{datemagyar};
%    nolonger redefines \cs{today}.}
%    \begin{macrocode}
\@namedef{ondate\CurrentOption}{%
  \number\year.\nobreakspace\ifcase\month\or
  janu\'ar\or febru\'ar\or m\'arcius\or
  \'aprilis\or m\'ajus\or j\'unius\or
  j\'ulius\or augusztus\or szeptember\or
  okt\'ober\or november\or december\fi
    \space\ifcase\day\or
    1-j\'en\or  2-\'an\or  3-\'an\or  4-\'en\or  5-\'en\or
    6-\'an\or  7-\'en\or  8-\'an\or  9-\'en\or 10-\'en\or
   11-\'en\or 12-\'en\or 13-\'an\or 14-\'en\or 15-\'en\or
   16-\'an\or 17-\'en\or 18-\'an\or 19-\'en\or 20-\'an\or
   21-\'en\or 22-\'en\or 23-\'an\or 24-\'en\or 25-\'en\or
   26-\'an\or 27-\'en\or 28-\'an\or 29-\'en\or 30-\'an\or
   31-\'en\fi}
%    \end{macrocode}
% \end{macro}
%
% \begin{macro}{\extrasmagyar}
% \begin{macro}{\noextrasmagyar}
%    The macro |\extrasmagyar| will perform all the extra definitions
%    needed for the Hungarian language. The macro |\noextrasmagyar| is
%    used to cancel the actions of |\extrasmagyar|.
%
%    \begin{macrocode}
 \@namedef{extras\CurrentOption}{%
   \expandafter\let\expandafter\ontoday
     \csname ondate\CurrentOption\endcsname}
\@namedef{noextras\CurrentOption}{\let\ontoday\@undefined}
%    \end{macrocode}
% \end{macro}
% \end{macro}
%
%    Now we redefine some commands included into \file{latex.ltx}.
%    The original form of a command is always saved with
%    |\babel@save| and the changes are added to |\extrasmagyar|.
%    This ensures that the Hungarian version of a macro is alive
%    \emph{only} if the Hungarian language is active.
%
% \begin{macro}{\fnum@figure}
% \begin{macro}{\fnum@table}
%    In figure and table captions the order of the figure/table
%    number and |\figurename|\hspace{0pt}/|\tablename| must be changed.
%    To achieve this |\fnum@figure| and |\fnum@table| are
%    redefined and added to |\extrasmagyar|.
% \changes{magyar-1.4i}{2004/02/20}{Use \cs{nobreakspace} instead of
%    tilde}
%    \begin{macrocode}
\expandafter\addto\csname extras\CurrentOption\endcsname{%
  \babel@save\fnum@figure
  \def\fnum@figure{\thefigure.\nobreakspace\figurename}}
\expandafter\addto\csname extras\CurrentOption\endcsname{%
  \babel@save\fnum@table
  \def\fnum@table{\thetable.\nobreakspace\tablename}}
%    \end{macrocode}
% \end{macro}
% \end{macro}
%
% \begin{macro}{\@makecaption}
%    The colon in a figure/table caption must be replaced by a dot
%    by redefining |\@makecaption|.
%    \begin{macrocode}
\expandafter\addto\csname extras\CurrentOption\endcsname{%
  \babel@save\@makecaption
  \def\@makecaption#1#2{%
    \vskip\abovecaptionskip
    \sbox\@tempboxa{#1. #2}%
    \ifdim \wd\@tempboxa >\hsize
      {#1. #2\csname par\endcsname}
    \else
      \global \@minipagefalse
      \hb@xt@\hsize{\hfil\box\@tempboxa\hfil}%
    \fi
    \vskip\belowcaptionskip}}
%    \end{macrocode}
% \end{macro}
%
% \begin{macro}{\@caption}
%    There should be a dot after the figure/table number in lof/lot,
%    so |\@caption| is redefined.
%    \begin{macrocode}
\expandafter\addto\csname extras\CurrentOption\endcsname{%
  \babel@save\@caption
  \long\def\@caption#1[#2]#3{%
    \csname par\endcsname
    \addcontentsline{\csname ext@#1\endcsname}{#1}%
      {\protect\numberline{\csname the#1\endcsname.}{\ignorespaces #2}}%
    \begingroup
      \@parboxrestore
      \if@minipage
        \@setminipage
      \fi
      \normalsize
      \@makecaption{\csname fnum@#1\endcsname}%
          {\ignorespaces #3}\csname par\endcsname
    \endgroup}}
%    \end{macrocode}
% \end{macro}
%
% \begin{macro}{\@seccntformat}
%    In order to have a dot after the section number
%    |\@seccntformat| is redefined.
%    \begin{macrocode}
\expandafter\addto\csname extras\CurrentOption\endcsname{%
  \babel@save\@seccntformat
  \def\@seccntformat#1{\csname the#1\endcsname.\quad}}
%    \end{macrocode}
% \end{macro}
%
% \begin{macro}{\@sect}
%    Alas, |\@sect| must also be redefined to have that dot in toc too.
%    On the other hand, we include a dot after a run-in head.
%    \begin{macrocode}
\expandafter\addto\csname extras\CurrentOption\endcsname{%
  \babel@save\@sect
  \def\@sect#1#2#3#4#5#6[#7]#8{%
    \ifnum #2>\c@secnumdepth
      \let\@svsec\@empty
    \else
      \refstepcounter{#1}%
      \protected@edef\@svsec{\@seccntformat{#1}\relax}%
    \fi
    \@tempskipa #5\relax
    \ifdim \@tempskipa>\z@
      \begingroup
        #6{%
          \@hangfrom{\hskip #3\relax\@svsec}%
            \interlinepenalty \@M #8\@@par}%
      \endgroup
      \csname #1mark\endcsname{#7}%
      \addcontentsline{toc}{#1}{%
        \ifnum #2>\c@secnumdepth \else
          \protect\numberline{\csname the#1\endcsname.}%
        \fi
        #7}%
    \else
      \def\@svsechd{%
        #6{\hskip #3\relax
        \@svsec #8.}%
        \csname #1mark\endcsname{#7}%
        \addcontentsline{toc}{#1}{%
          \ifnum #2>\c@secnumdepth \else
            \protect\numberline{\csname the#1\endcsname.}%
          \fi
          #7}}%
    \fi
    \@xsect{#5}}}
%    \end{macrocode}
% \end{macro}
%
% \begin{macro}{\@ssect}
%    In order to have that dot after a run-in head when the star form of the
%    sectioning commands is used, we have to redefine |\@ssect|.
%    \begin{macrocode}
\expandafter\addto\csname extras\CurrentOption\endcsname{%
  \babel@save\@ssect
  \def\@ssect#1#2#3#4#5{%
    \@tempskipa #3\relax
    \ifdim \@tempskipa>\z@
      \begingroup
        #4{%
          \@hangfrom{\hskip #1}%
            \interlinepenalty \@M #5\@@par}%
      \endgroup
    \else
      \def\@svsechd{#4{\hskip #1\relax #5.}}%
    \fi
    \@xsect{#3}}}
%    \end{macrocode}
% \end{macro}
%
% \begin{macro}{\@begintheorem}
% \begin{macro}{\@opargbegintheorem}
%    Order changing and dot insertion in theorem by redefining
%    |\@begintheorem| and |\@opargbegintheorem|.
%    \begin{macrocode}
\expandafter\addto\csname extras\CurrentOption\endcsname{%
  \babel@save\@begintheorem
  \def\@begintheorem#1#2{\trivlist
    \item[\hskip \labelsep{\bfseries #2.\ #1.}]\itshape}%
  \babel@save\@opargbegintheorem
  \def\@opargbegintheorem#1#2#3{\trivlist
    \item[\hskip \labelsep{\bfseries #2.\ #1\ (#3).}]\itshape}}
%    \end{macrocode}
% \end{macro}
% \end{macro}
%
%    The next step is to redefine some macros included into the
%    class files. It is determined which class file is loaded
%    then the original form of the macro is saved and the changes
%    are added to |\extrasmagyar|.
%
%    First we check if the \file{book.cls} is loaded.
%    \begin{macrocode}
\@ifclassloaded{book}{%
%    \end{macrocode}
% \begin{macro}{\ps@headings}
%    The look of the headings is changed: we have to insert some dots
%    and change the order of chapter number and |\chaptername|.
%    \begin{macrocode}
  \expandafter\addto\csname extras\CurrentOption\endcsname{%
    \babel@save\ps@headings}
  \expandafter\addto\csname extras\CurrentOption\endcsname{%
    \if@twoside
      \def\ps@headings{%
          \let\@oddfoot\@empty\let\@evenfoot\@empty
          \def\@evenhead{\thepage\hfil\slshape\leftmark}%
          \def\@oddhead{{\slshape\rightmark}\hfil\thepage}%
          \let\@mkboth\markboth
        \def\chaptermark##1{%
          \markboth {\MakeUppercase{%
            \ifnum \c@secnumdepth >\m@ne
              \if@mainmatter
                \thechapter. \@chapapp. \ %
              \fi
            \fi
            ##1}}{}}%
        \def\sectionmark##1{%
          \markright {\MakeUppercase{%
            \ifnum \c@secnumdepth >\z@
              \thesection. \ %
            \fi
            ##1}}}}%
    \else
      \def\ps@headings{%
        \let\@oddfoot\@empty
        \def\@oddhead{{\slshape\rightmark}\hfil\thepage}%
        \let\@mkboth\markboth
        \def\chaptermark##1{%
          \markright {\MakeUppercase{%
            \ifnum \c@secnumdepth >\m@ne
              \if@mainmatter
                \thechapter. \@chapapp. \ %
              \fi
            \fi
            ##1}}}}%
    \fi}
%    \end{macrocode}
% \end{macro}
% \begin{macro}{\@part}
%    At the beginning of a part we need \mbox{eg.} `I.\ r\'esz'
%    instead of `Part I' (in toc too).
%    To achieve this |\@part| is redefined.
%    \begin{macrocode}
  \expandafter\addto\csname extras\CurrentOption\endcsname{%
    \babel@save\@part
    \def\@part[#1]#2{%
        \ifnum \c@secnumdepth >-2\relax
          \refstepcounter{part}%
          \addcontentsline{toc}{part}{\thepart.\hspace{1em}#1}%
        \else
          \addcontentsline{toc}{part}{#1}%
        \fi
        \markboth{}{}%
        {\centering
         \interlinepenalty \@M
         \normalfont
         \ifnum \c@secnumdepth >-2\relax
           \huge\bfseries \thepart.\nobreakspace\partname
           \csname par\endcsname
           \vskip 20\p@
         \fi
         \Huge \bfseries #2\csname par\endcsname}%
        \@endpart}}
%    \end{macrocode}
% \end{macro}
% \begin{macro}{\@chapter}
%    The same changes are made to chapter.
%    First the screen typeout and the toc are changed by
%    redefining |\@chapter|.
%    \begin{macrocode}
  \expandafter\addto\csname extras\CurrentOption\endcsname{%
    \babel@save\@chapter
    \def\@chapter[#1]#2{\ifnum \c@secnumdepth >\m@ne
                           \if@mainmatter
                             \refstepcounter{chapter}%
                             \typeout{\thechapter.\space\@chapapp.}%
                             \addcontentsline{toc}{chapter}%
                                       {\protect\numberline{\thechapter.}#1}%
                           \else
                             \addcontentsline{toc}{chapter}{#1}%
                           \fi
                        \else
                          \addcontentsline{toc}{chapter}{#1}%
                        \fi
                        \chaptermark{#1}%
                        \addtocontents{lof}{\protect\addvspace{10\p@}}%
                        \addtocontents{lot}{\protect\addvspace{10\p@}}%
                        \if@twocolumn
                          \@topnewpage[\@makechapterhead{#2}]%
                        \else
                          \@makechapterhead{#2}%
                          \@afterheading
                        \fi}}
%    \end{macrocode}
% \end{macro}
% \begin{macro}{\@makechapterhead}
%    Then the look of the chapter-start is modified by redefining
%    |\@makechapterhead|.
%    \begin{macrocode}
  \expandafter\addto\csname extras\CurrentOption\endcsname{%
    \babel@save\@makechapterhead
    \def\@makechapterhead#1{%
      \vspace*{50\p@}%
      {\parindent \z@ \raggedright \normalfont
        \ifnum \c@secnumdepth >\m@ne
          \if@mainmatter
            \huge\bfseries \thechapter.\nobreakspace\@chapapp{}
            \csname par\endcsname\nobreak
            \vskip 20\p@
          \fi
        \fi
        \interlinepenalty\@M
        \Huge \bfseries #1\csname par\endcsname\nobreak
        \vskip 40\p@
      }}}%
%    \end{macrocode}
% \end{macro}
%    This the end of the book class modification.
%    \begin{macrocode}
}{}
%    \end{macrocode}
%
%    Now we check if \file{report.cls} is loaded.
%    \begin{macrocode}
\@ifclassloaded{report}{%
%    \end{macrocode}
% \begin{macro}{\ps@headings}
%    First the headings are modified just in case of the book class.
%    \begin{macrocode}
  \expandafter\addto\csname extras\CurrentOption\endcsname{%
    \babel@save\ps@headings}
  \expandafter\addto\csname extras\CurrentOption\endcsname{%
    \if@twoside
      \def\ps@headings{%
          \let\@oddfoot\@empty\let\@evenfoot\@empty
          \def\@evenhead{\thepage\hfil\slshape\leftmark}%
          \def\@oddhead{{\slshape\rightmark}\hfil\thepage}%
          \let\@mkboth\markboth
        \def\chaptermark##1{%
          \markboth {\MakeUppercase{%
            \ifnum \c@secnumdepth >\m@ne
                \thechapter. \@chapapp. \ %
            \fi
            ##1}}{}}%
        \def\sectionmark##1{%
          \markright {\MakeUppercase{%
            \ifnum \c@secnumdepth >\z@
              \thesection. \ %
            \fi
            ##1}}}}%
    \else
      \def\ps@headings{%
        \let\@oddfoot\@empty
        \def\@oddhead{{\slshape\rightmark}\hfil\thepage}%
        \let\@mkboth\markboth
        \def\chaptermark##1{%
          \markright {\MakeUppercase{%
            \ifnum \c@secnumdepth >\m@ne
                \thechapter. \@chapapp. \ %
            \fi
            ##1}}}}%
    \fi}
%    \end{macrocode}
% \end{macro}
% \begin{macro}{\@chapter}
%    Chapter-start modification with |\@chapter|
%    \begin{macrocode}
  \expandafter\addto\csname extras\CurrentOption\endcsname{%
    \babel@save\@chapter
    \def\@chapter[#1]#2{\ifnum \c@secnumdepth >\m@ne
                             \refstepcounter{chapter}%
                             \typeout{\thechapter.\space\@chapapp.}%
                             \addcontentsline{toc}{chapter}%
                                       {\protect\numberline{\thechapter.}#1}%
                        \else
                          \addcontentsline{toc}{chapter}{#1}%
                        \fi
                        \chaptermark{#1}%
                        \addtocontents{lof}{\protect\addvspace{10\p@}}%
                        \addtocontents{lot}{\protect\addvspace{10\p@}}%
                        \if@twocolumn
                          \@topnewpage[\@makechapterhead{#2}]%
                        \else
                          \@makechapterhead{#2}%
                          \@afterheading
                        \fi}}
%    \end{macrocode}
% \end{macro}
% \begin{macro}{\@makechapterhead}
%    and |\@makechapterhead|.
%    \begin{macrocode}
  \expandafter\addto\csname extras\CurrentOption\endcsname{%
    \babel@save\@makechapterhead
    \def\@makechapterhead#1{%
      \vspace*{50\p@}%
      {\parindent \z@ \raggedright \normalfont
        \ifnum \c@secnumdepth >\m@ne
            \huge\bfseries \thechapter.\nobreakspace\@chapapp{}
            \csname par\endcsname\nobreak
            \vskip 20\p@
        \fi
        \interlinepenalty\@M
        \Huge \bfseries #1\csname par\endcsname\nobreak
        \vskip 40\p@
      }}}%
%    \end{macrocode}
% \end{macro}
%    End of report class modification.
%    \begin{macrocode}
}{}
%    \end{macrocode}
%
%    Checking if \file{article.cls} is loaded.
%    \begin{macrocode}
\@ifclassloaded{article}{%
%    \end{macrocode}
% \begin{macro}{\ps@headings}
%    Changing headings by redefining |\ps@headings|.
%    \begin{macrocode}
  \expandafter\addto\csname extras\CurrentOption\endcsname{%
    \babel@save\ps@headings}
  \expandafter\addto\csname extras\CurrentOption\endcsname{%
    \if@twoside
      \def\ps@headings{%
          \let\@oddfoot\@empty\let\@evenfoot\@empty
          \def\@evenhead{\thepage\hfil\slshape\leftmark}%
          \def\@oddhead{{\slshape\rightmark}\hfil\thepage}%
          \let\@mkboth\markboth
        \def\sectionmark##1{%
          \markboth {\MakeUppercase{%
            \ifnum \c@secnumdepth >\z@
              \thesection.\quad
            \fi
            ##1}}{}}%
        \def\subsectionmark##1{%
          \markright {%
            \ifnum \c@secnumdepth >\@ne
              \thesubsection.\quad
            \fi
            ##1}}}%
    \else
      \def\ps@headings{%
        \let\@oddfoot\@empty
        \def\@oddhead{{\slshape\rightmark}\hfil\thepage}%
        \let\@mkboth\markboth
        \def\sectionmark##1{%
          \markright {\MakeUppercase{%
            \ifnum \c@secnumdepth >\m@ne
              \thesection.\quad
            \fi
            ##1}}}}%
    \fi}%
%    \end{macrocode}
% \end{macro}
%    No more necessary changes specific to the article class.
%    \begin{macrocode}
}{}
%    \end{macrocode}
%
%    And now this is the turn of \file{letter.cls}.
%    \begin{macrocode}
\@ifclassloaded{letter}{%
%    \end{macrocode}
% \begin{macro}{\ps@headings}
%    In the headings the page number must be followed by a dot
%    and then |\pagename|.
%    \begin{macrocode}
  \expandafter\addto\csname extras\CurrentOption\endcsname{%
    \babel@save\ps@headings}
  \expandafter\addto\csname extras\CurrentOption\endcsname{%
    \if@twoside
      \def\ps@headings{%
          \let\@oddfoot\@empty\let\@evenfoot\@empty
          \def\@oddhead{\slshape\headtoname{:} \ignorespaces\toname
                        \hfil \@date
                        \hfil \thepage.\nobreakspace\pagename}%
          \let\@evenhead\@oddhead}
    \else
      \def\ps@headings{%
          \let\@oddfoot\@empty
          \def\@oddhead{\slshape\headtoname{:} \ignorespaces\toname
                        \hfil \@date
                        \hfil \thepage.\nobreakspace\pagename}}
    \fi}%
%    \end{macrocode}
% \end{macro}
%    End of letter class.
%    \begin{macrocode}
}{}
%    \end{macrocode}
%
%    After making the changes to the \LaTeX{}
%    macros we define some new ones to handle the problem
%    with definite articles.
% \begin{macro}{\az}
%    |\az| is a user-level command which decides if the next
%    character is a star. |\@az| is called for |\az*| and
%    |\az@| for |\az|.
%    \begin{macrocode}
\def\az{a\@ifstar{\@az}{\az@}}
%    \end{macrocode}
% \end{macro}
% \begin{macro}{\Az}
%    |\Az| is used at the beginning of a sentence. Otherwise it behaves
%    the same as |\az|.
%    \begin{macrocode}
\def\Az{A\@ifstar{\@az}{\az@}}
%    \end{macrocode}
% \end{macro}
% \begin{macro}{\az@}
%    |\az@| is called if there is no star after |\az| or |\Az|.
%    It calls |\@az| and writes |#1| separating with a non-breakable
%    space.
%    \begin{macrocode}
\def\az@#1{\@az{#1}\nobreakspace#1}
%    \end{macrocode}
% \end{macro}
% \begin{macro}{\@az}
%    This macro calls |\hun@tempadef| to remove the accents from
%    the argument then calls |\@@az| that determines if a `z'
%    should be written after a/A (written by |\az|/|\Az|).
%    \begin{macrocode}
\def\@az#1{%
  \hun@tempadef{relax}{relax}{#1}%
  \edef\@tempb{\noexpand\@@az\@tempa\hbox!}%
  \@tempb}
%    \end{macrocode}
% \end{macro}
%
% \begin{macro}{\hun@tempadef}
%    The macro |\hun@tempadef| has three tasks:
%    \begin{itemize}
%      \item Accent removal. Accented letters confuse
%            |\@@az|, the main definite article determinator
%            macro, so they must be changed to their non-accented
%            counterparts. Special letters must also be changed,
%            \mbox{eg.} \oe$\,\rightarrow\,$o.
%      \item Labels must be expanded.
%      \item To handle Roman numerals correctly, commands starting
%            with |\hun@| are defined for labels containing Roman
%            numbers with the Roman numerals
%            replaced by their Arabic representation.
%            This macro can check if there is a |\hun@| command.
%    \end{itemize}
%    There are three arguments:
%    \begin{enumerate}
%      \item The primary command that should be expanded if it exists.
%            This is usually the |\hun@| command for a label.
%      \item The secondary command which is used if the first one
%            is |\relax|. This is usually the original \LaTeX{}
%            command for a label.
%      \item This is used if the first two is |\relax|. For this one
%            no expansion is carried out but the accents are still
%            removed and special letters are changed.
%    \end{enumerate}
% \changes{magyar-1.4f}{2003/09/29}{Added
%    \cs{def}\cs{safe@activesfalse{}} as a fix for PR3426} 
%    \begin{macrocode}
\def\hun@tempadef#1#2#3{%
  \begingroup
    \def\@safe@activesfalse{}%
    \def\setbox ##1{}% to get rid of accents and special letters
    \def\hbox ##1{}%
    \def\accent ##1 ##2{##2}%
    \def\add@accent ##1##2{##2}%
    \def\@text@composite@x ##1##2{##2}%
    \def\i{i}\def\j{j}%
    \def\ae{a}\def\AE{A}\def\oe{o}\def\OE{O}%
    \def\ss{s}\def\L{L}%
    \def\d{}\def\b{}\def\c{}\def\t{}%
    \expandafter\ifx\csname #1\endcsname\relax
      \expandafter\ifx\csname #2\endcsname\relax
        \xdef\@tempa{#3}%
      \else
        \xdef\@tempa{\csname #2\endcsname}%
      \fi
    \else
      \xdef\@tempa{\csname #1\endcsname}%
    \fi
  \endgroup}
%    \end{macrocode}
% \end{macro}
%
%    The following macros are used to determine the definite
%    article for a label's expansion.
% \begin{macro}{\aref}
%    |\aref| is an alias for |\azr|.
%    \begin{macrocode}
\def\aref{\azr}
%    \end{macrocode}
% \end{macro}
% \begin{macro}{\Aref}
%    |\Aref| is an alias for |\Azr|.
%    \begin{macrocode}
\def\Aref{\Azr}
%    \end{macrocode}
% \end{macro}
% \begin{macro}{\azr}
%    |\azr| calls |\@azr| if the next character is a star,
%    otherwise it calls |\azr@|.
%    \begin{macrocode}
\def\azr{a\@ifstar{\@azr}{\azr@}}
%    \end{macrocode}
% \end{macro}
% \begin{macro}{\Azr}
%    |\Azr| is the same as |\azr| except that it writes `A'
%    instead of `a'.
%    \begin{macrocode}
\def\Azr{A\@ifstar{\@azr}{\azr@}}
%    \end{macrocode}
% \end{macro}
% \begin{macro}{\azr@}
%    |\azr@| decides if the next character is |(| and
%    in that case it calls |\azr@@@| which writes an extra |(|
%    for equation referencing. Otherwise |\azr@@| is called.
%    \begin{macrocode}
\def\azr@{\@ifnextchar ({\azr@@@}{\azr@@}}
%    \end{macrocode}
% \end{macro}
% \begin{macro}{\azr@@}
%    Calls |\@azr| then writes the label's expansion preceded by
%    a non-breakable space.
%    \begin{macrocode}
\def\azr@@#1{\@azr{#1}\nobreakspace\ref{#1}}
%    \end{macrocode}
% \end{macro}
% \begin{macro}{\azr@@@}
%    Same as |\azr@@| but inserts a |(| between the
%    non-breakable space and the label expansion.
%    \begin{macrocode}
\def\azr@@@(#1{\@azr{#1}\nobreakspace(\ref{#1}}
%    \end{macrocode}
% \end{macro}
% \begin{macro}{\@azr}
%    Calls |\hun@tempadef| to choose between the label's
%    |\hun@| or original \LaTeX{} command and to expand
%    it with accent removal and special letter substitution.
%    Then calls |\@@az|, the core macro of definite article handling.
%    \begin{macrocode}
\def\@azr#1{%
  \hun@tempadef{hun@r@#1}{r@#1}{}%
  \ifx\@tempa\empty
  \else
    \edef\@tempb{\noexpand\@@az\expandafter\@firstoftwo\@tempa\hbox!}%
    \@tempb
  \fi
}
%    \end{macrocode}
% \end{macro}
%
%    The following commands are used to generate the definite article
%    for the page number of a label.
% \begin{macro}{\apageref}
%    |\apageref| is an alias for |\azp|.
%    \begin{macrocode}
\def\apageref{\azp}
%    \end{macrocode}
% \end{macro}
% \begin{macro}{\Apageref}
%    |\Apageref| is an alias for |\Azp|.
%    \begin{macrocode}
\def\Apageref{\Azp}
%    \end{macrocode}
% \end{macro}
% \begin{macro}{\azp}
%    Checks if the next character is |*| and calls
%    |\@azp| or |\azp@|.
%    \begin{macrocode}
\def\azp{a\@ifstar{\@azp}{\azp@}}
%    \end{macrocode}
% \end{macro}
% \begin{macro}{\Azp}
%    Same as |\azp| except that it writes `A' instead of `a'.
%    \begin{macrocode}
\def\Azp{A\@ifstar{\@azp}{\azp@}}
%    \end{macrocode}
% \end{macro}
% \begin{macro}{\azp@}
%    Calls |\@azp| then writes the label's page preceded by
%    a non-breakable space.
%    \begin{macrocode}
\def\azp@#1{\@azp{#1}\nobreakspace\pageref{#1}}
%    \end{macrocode}
% \end{macro}
% \begin{macro}{\@azp}
%    Calls |\hun@tempadef| then takes the label's page and passes
%    it to |\@@az|.
%    \begin{macrocode}
\def\@azp#1{%
  \hun@tempadef{hun@r@#1}{r@#1}{}%
  \ifx\@tempa\empty
  \else
    \edef\@tempb{\noexpand\@@az\expandafter\@secondoftwo\@tempa\hbox!}%
    \@tempb
  \fi
}
%    \end{macrocode}
% \end{macro}
%
%    The following macros are used to give the definite article to
%    citations.
% \begin{macro}{\acite}
%    This is an alias for |\azc|.
%    \begin{macrocode}
\def\acite{\azc}
%    \end{macrocode}
% \end{macro}
% \begin{macro}{\Acite}
%    This is an alias for |\Azc|.
%    \begin{macrocode}
\def\Acite{\Azc}
%    \end{macrocode}
% \end{macro}
% \begin{macro}{\azc}
%    Checks if the next character is a star and
%    calls |\@azc| or |\azc@|.
%    \begin{macrocode}
\def\azc{a\@ifstar{\@azc}{\azc@}}
%    \end{macrocode}
% \end{macro}
% \begin{macro}{\Azc}
%    Same as |\azc| but used at the beginning of sentences.
%    \begin{macrocode}
\def\Azc{A\@ifstar{\@azc}{\azc@}}
%    \end{macrocode}
% \end{macro}
% \begin{macro}{\azc@}
%    If there is no star we accept an optional argument,
%    just like the |\cite| command.
%    \begin{macrocode}
\def\azc@{\@ifnextchar [{\azc@@}{\azc@@[]}}
%    \end{macrocode}
% \end{macro}
% \begin{macro}{\azc@@}
%    First calls |\@azc| then |\cite|.
%    \begin{macrocode}
\def\azc@@[#1]#2{%
  \@azc{#2}\nobreakspace\def\@tempa{#1}%
    \ifx\@tempa\@empty\cite{#2}\else\cite[#1]{#2}\fi}
%    \end{macrocode}
% \end{macro}
% \begin{macro}{\@azc}
%    This is an auxiliary macro to get the first cite label
%    from a comma-separated list.
%    \begin{macrocode}
\def\@azc#1{\@@azc#1,\hbox!}
%    \end{macrocode}
% \end{macro}
% \begin{macro}{\@@azc}
%    This one uses only the first argument, that is the first
%    element of the comma-separated list of cite labels.
%    Calls |\hun@tempadef| to expand the cite label with accent
%    removal and special letter replacement.
%    Then |\@@az|, the core macro, is called.
%    \begin{macrocode}
\def\@@azc#1,#2\hbox#3!{%
  \hun@tempadef{hun@b@#1}{b@#1}{}%
  \ifx\@tempa\empty
  \else
    \edef\@tempb{\noexpand\@@az\@tempa\hbox!}%
    \@tempb
  \fi}
%    \end{macrocode}
% \end{macro}
%
% \begin{macro}{\hun@number@lehgth}
%    This macro is used to count the number of digits in its
%    argument until a non-digit character is found or
%    the end of the argument is reached.
%    It must be called as
%    |\hun@number@lehgth|\textit{arg}|\hbox\hbox!| and
%    |\count@| must be zeroed.
%    It is called by |\@@az|.
%    \begin{macrocode}
\def\hun@number@lehgth#1#2\hbox#3!{%
  \ifcat\noexpand#11%
    \ifnum\expandafter`\csname#1\endcsname>47
      \ifnum\expandafter`\csname#1\endcsname<58
        \advance\count@ by \@ne
        \hun@number@lehgth#2\hbox\hbox!\fi\fi\fi}
%    \end{macrocode}
% \end{macro}
%
% \begin{macro}{\hun@alph@lehgth}
%    This is used to count the number of letters until a
%    non-letter is found or the end of the argument is reached.
%    It must be called as
%    |\hun@alph@lehgth|\textit{arg}|\hbox\hbox!| and
%    |\count@| must be set to zero.
%    It is called by |\@@az@string|.
%    \begin{macrocode}
\def\hun@alph@lehgth#1#2\hbox#3!{%
  \ifcat\noexpand#1A%
    \advance\count@ by \@ne
    \hun@alph@lehgth#2\hbox\hbox!\fi}
%    \end{macrocode}
% \end{macro}
%
% \begin{macro}{\@@az@string}
%    This macro is called by |\@@az| if the argument begins
%    with a letter.
%    The task of |\@@az@string| is to determine if the argument
%    starts with a vowel and in that case |\let\@tempa\@tempb|.
%    After checking if the first letter is A, E, I, O, or U,
%    |\hun@alph@lehgth| is called
%    to determine the length of the argument. If it gives 1
%    (that is the argument is a single-letter one or the second
%    character is not letter) then the letters L, M, N, R, S, X, and Y
%    are also considered as a vowel since their Hungarian pronounced
%    name starts with a vowel.
%    \begin{macrocode}
\def\@@az@string#1#2{%
  \ifx#1A%
    \let\@tempa\@tempb
  \else\ifx#1E%
    \let\@tempa\@tempb
  \else\ifx#1I%
    \let\@tempa\@tempb
  \else\ifx#1O%
    \let\@tempa\@tempb
  \else\ifx#1U%
    \let\@tempa\@tempb
  \fi\fi\fi\fi\fi
  \ifx\@tempa\@tempb
  \else
    \count@\z@
    \hun@alph@lehgth#1#2\hbox\hbox!%
    \ifnum\count@=\@ne
      \ifx#1F%
        \let\@tempa\@tempb
      \else\ifx#1L%
        \let\@tempa\@tempb
      \else\ifx#1M%
        \let\@tempa\@tempb
      \else\ifx#1N%
        \let\@tempa\@tempb
      \else\ifx#1R%
        \let\@tempa\@tempb
      \else\ifx#1S%
        \let\@tempa\@tempb
      \else\ifx#1X%
        \let\@tempa\@tempb
      \else\ifx#1Y%
        \let\@tempa\@tempb
      \fi\fi\fi\fi\fi\fi\fi\fi
    \fi
  \fi}
%    \end{macrocode}
% \end{macro}
%
% \begin{macro}{\@@az}
%    This macro is the core of definite article handling.
%    It determines if the argument needs `az' or `a' definite
%    article by setting |\@tempa| to `z' or |\@empty|.
%    It sets |\@tempa| to `z' if
%    \begin{itemize}
%      \item the first character of the argument is 5; or
%      \item the first character of the argument is 1 and
%            the $\mathit{length\ of\ the\ number} \pmod 3 = 1$
%            (one--egy, thousand--ezer, million--egymilli\'o,\dots); or
%      \item the first character of the argument is
%            a, A, e, E, i, I, o, O, u, or U; or
%      \item the first character of the argument is
%            l, L, m, M, n, N, r, R, s, S, x, X, y, or Y
%            and the length of the argument is 1 or the second
%            character is a non-letter.
%    \end{itemize}
%    At the end it calls |\@tempa|, that is, it either typesets a `z'
%    or nothing.
%    \begin{macrocode}
\def\@@az#1#2\hbox#3!{%
  \let\@tempa\@empty
  \def\@tempb{z}%
  \uppercase{%
    \ifx5#1%
      \let\@tempa\@tempb
    \else\ifx1#1%
      \count@\@ne
      \hun@number@lehgth#2\hbox\hbox!%
      \loop
      \ifnum\count@>\thr@@
        \advance\count@-\thr@@
      \repeat
      \ifnum\count@=\@ne
        \let\@tempa\@tempb
      \fi
    \else
      \@@az@string{#1}{#2}%
    \fi\fi
  }%
  \@tempa}
%    \end{macrocode}
% \end{macro}
%
% \begin{macro}{\refstepcounter}
%    |\refstepcounter| must be redefined in order to keep
%    |\@currentlabel| unexpanded. This is necessary to enable
%    the |\label| command to write a |\hunnewlabel| command
%    to the aux file with the Roman numerals substituted by
%    their Arabic representations.
%    Of course, the original definition of |\refstepcounter| is
%    saved and restored if the Hungarian language is switched off.
%    \begin{macrocode}
\expandafter\addto\csname extras\CurrentOption\endcsname{%
  \babel@save\refstepcounter
  \def\refstepcounter#1{\stepcounter{#1}%
    \def\@currentlabel{\csname p@#1\endcsname\csname the#1\endcsname}}%
}
%    \end{macrocode}
% \end{macro}
%
% \begin{macro}{\label}
%    |\label| is redefined to write another line into the aux
%    file: |\hunnewlabel{ }{ }| where the Roman numerals
%    are replaced their Arabic representations.
%    The original definition of |\label| is saved into
%    |\old@label| and it is also called by |\label|.
%    On leaving the Hungarian typesetting mode |\label|'s
%    original is restored since it is added to |\noextrasmagyar|.
%    \begin{macrocode}
\expandafter\addto\csname extras\CurrentOption\endcsname{%
  \let\old@label\label
  \def\label#1{\@bsphack
    \old@label{#1}%
    \begingroup
      \let\romannumeral\number
      \def\@roman##1{\number ##1}%
      \def\@Roman##1{\number ##1}%
      {\toks0={\noexpand\noexpand\noexpand\number}%
        \def\number##1{\the\toks0 ##1}\xdef\tempb@{\thepage}}%
      \edef\@tempa##1{\noexpand\protected@write\@auxout{}%
           {\noexpand\string\noexpand\hunnewlabel
           {##1}{{\@currentlabel}{\tempb@}}}}%
      \@tempa{#1}%
    \endgroup
  \@esphack}%
}
\expandafter\addto\csname noextras\CurrentOption\endcsname{%
  \let\label\old@label
}
%    \end{macrocode}
% \end{macro}
%
% \begin{macro}{\hunnewlabel}
%    Finally, |\hunnewlabel| is defined.
%    It checks if the label's expansion (|#2|) differs from that
%    one given in the |\newlabel| command. If yes
%    (that is, the label contains some Roman numerals),
%    it defines the macro |\hun@r@|\textit{label},
%    otherwise it does nothing.
%    \begin{macrocode}
\def\hunnewlabel#1#2{%
  \def\@tempa{#2}%
  \expandafter\ifx\csname r@#1\endcsname\@tempa
    \relax% \message{No need for def: #1}%
  \else
    \global\expandafter\let\csname hun@r@#1\endcsname\@tempa%
  \fi
}
%    \end{macrocode}
% \end{macro}
%
%    For Hungarian the |`| character is made active.
% \changes{magyar-1.4c}{2001/03/05}{Make sure that the grave accent
%    has catcode 12 \emph{before} it is made \cs{active}}
%    \begin{macrocode}
\AtBeginDocument{%
  \if@filesw\immediate\write\@auxout{\catcode096=12}\fi}
\initiate@active@char{`}
\expandafter\addto\csname extras\CurrentOption\endcsname{%
  \languageshorthands{magyar}%
  \bbl@activate{`}}
\expandafter\addto\csname noextras\CurrentOption\endcsname{%
  \bbl@deactivate{`}}
%    \end{macrocode}
%
%    The character sequence |``| is declared as a shorthand
%    in order to produce
%    the opening quotation sign appropriate for Hungarian.
%    \begin{macrocode}
\declare@shorthand{magyar}{``}{\glqq}
%    \end{macrocode}
%
% In Hungarian there are some long double consonants which
% must be hyphenated specially.
% For all these long double consonants (except dzzs, that is
% extremely very-very rare) a shortcut is defined.
%    \begin{macrocode}
\declare@shorthand{magyar}{`c}{\textormath{\bbl@disc{c}{cs}}{c}}
\declare@shorthand{magyar}{`C}{\textormath{\bbl@disc{C}{CS}}{C}}
\declare@shorthand{magyar}{`d}{\textormath{\bbl@disc{d}{dz}}{d}}
\declare@shorthand{magyar}{`D}{\textormath{\bbl@disc{D}{DZ}}{D}}
\declare@shorthand{magyar}{`g}{\textormath{\bbl@disc{g}{gy}}{g}}
\declare@shorthand{magyar}{`G}{\textormath{\bbl@disc{G}{GY}}{G}}
\declare@shorthand{magyar}{`l}{\textormath{\bbl@disc{l}{ly}}{l}}
\declare@shorthand{magyar}{`L}{\textormath{\bbl@disc{L}{LY}}{L}}
\declare@shorthand{magyar}{`n}{\textormath{\bbl@disc{n}{ny}}{n}}
\declare@shorthand{magyar}{`N}{\textormath{\bbl@disc{N}{NY}}{N}}
\declare@shorthand{magyar}{`s}{\textormath{\bbl@disc{s}{sz}}{s}}
\declare@shorthand{magyar}{`S}{\textormath{\bbl@disc{S}{SZ}}{S}}
\declare@shorthand{magyar}{`t}{\textormath{\bbl@disc{t}{ty}}{t}}
\declare@shorthand{magyar}{`T}{\textormath{\bbl@disc{T}{TY}}{T}}
\declare@shorthand{magyar}{`z}{\textormath{\bbl@disc{z}{zs}}{z}}
\declare@shorthand{magyar}{`Z}{\textormath{\bbl@disc{Z}{ZS}}{Z}}
%    \end{macrocode}
%
%    The macro |\ldf@finish| takes care of looking for a
%    configuration file, setting the main language to be switched on
%    at |\begin{document}| and resetting the category code of
%    \texttt{@} to its original value.
% \changes{magyar-1.3h}{1996/10/30}{Now use \cs{ldf@finish} to wrap up}
%    \begin{macrocode}
\ldf@finish\CurrentOption
%</code>
%    \end{macrocode}
%
% \Finale
%%
%% \CharacterTable
%%  {Upper-case    \A\B\C\D\E\F\G\H\I\J\K\L\M\N\O\P\Q\R\S\T\U\V\W\X\Y\Z
%%   Lower-case    \a\b\c\d\e\f\g\h\i\j\k\l\m\n\o\p\q\r\s\t\u\v\w\x\y\z
%%   Digits        \0\1\2\3\4\5\6\7\8\9
%%   Exclamation   \!     Double quote  \"     Hash (number) \#
%%   Dollar        \$     Percent       \%     Ampersand     \&
%%   Acute accent  \'     Left paren    \(     Right paren   \)
%%   Asterisk      \*     Plus          \+     Comma         \,
%%   Minus         \-     Point         \.     Solidus       \/
%%   Colon         \:     Semicolon     \;     Less than     \<
%%   Equals        \=     Greater than  \>     Question mark \?
%%   Commercial at \@     Left bracket  \[     Backslash     \\
%%   Right bracket \]     Circumflex    \^     Underscore    \_
%%   Grave accent  \`     Left brace    \{     Vertical bar  \|
%%   Right brace   \}     Tilde         \~}
%%
\endinput
}
\DeclareOption{icelandic}{%%
%% This file will generate fast loadable files and documentation
%% driver files from the doc files in this package when run through
%% LaTeX or TeX.
%%
%% Copyright 1989-2005 Johannes L. Braams and any individual authors
%% listed elsewhere in this file.  All rights reserved.
%% 
%% This file is part of the Babel system.
%% --------------------------------------
%% 
%% It may be distributed and/or modified under the
%% conditions of the LaTeX Project Public License, either version 1.3
%% of this license or (at your option) any later version.
%% The latest version of this license is in
%%   http://www.latex-project.org/lppl.txt
%% and version 1.3 or later is part of all distributions of LaTeX
%% version 2003/12/01 or later.
%% 
%% This work has the LPPL maintenance status "maintained".
%% 
%% The Current Maintainer of this work is Johannes Braams.
%% 
%% The list of all files belonging to the LaTeX base distribution is
%% given in the file `manifest.bbl. See also `legal.bbl' for additional
%% information.
%% 
%% The list of derived (unpacked) files belonging to the distribution
%% and covered by LPPL is defined by the unpacking scripts (with
%% extension .ins) which are part of the distribution.
%%
%% --------------- start of docstrip commands ------------------
%%
\def\filedate{2000/09/26}
\def\batchfile{icelandic.ins}
\input docstrip.tex

{\ifx\generate\undefined
\Msg{**********************************************}
\Msg{*}
\Msg{* This installation requires docstrip}
\Msg{* version 2.3c or later.}
\Msg{*}
\Msg{* An older version of docstrip has been input}
\Msg{*}
\Msg{**********************************************}
\errhelp{Move or rename old docstrip.tex.}
\errmessage{Old docstrip in input path}
\batchmode
\csname @@end\endcsname
\fi}

\declarepreamble\mainpreamble
This is a generated file.

Copyright 1989-2005 Johannes L. Braams and any individual authors
listed elsewhere in this file.  All rights reserved.

This file was generated from file(s) of the Babel system.
---------------------------------------------------------

It may be distributed and/or modified under the
conditions of the LaTeX Project Public License, either version 1.3
of this license or (at your option) any later version.
The latest version of this license is in
  http://www.latex-project.org/lppl.txt
and version 1.3 or later is part of all distributions of LaTeX
version 2003/12/01 or later.

This work has the LPPL maintenance status "maintained".

The Current Maintainer of this work is Johannes Braams.

This file may only be distributed together with a copy of the Babel
system. You may however distribute the Babel system without
such generated files.

The list of all files belonging to the Babel distribution is
given in the file `manifest.bbl'. See also `legal.bbl for additional
information.

The list of derived (unpacked) files belonging to the distribution
and covered by LPPL is defined by the unpacking scripts (with
extension .ins) which are part of the distribution.
\endpreamble

\declarepreamble\fdpreamble
This is a generated file.

Copyright 1989-2005 Johannes L. Braams and any individual authors
listed elsewhere in this file.  All rights reserved.

This file was generated from file(s) of the Babel system.
---------------------------------------------------------

It may be distributed and/or modified under the
conditions of the LaTeX Project Public License, either version 1.3
of this license or (at your option) any later version.
The latest version of this license is in
  http://www.latex-project.org/lppl.txt
and version 1.3 or later is part of all distributions of LaTeX
version 2003/12/01 or later.

This work has the LPPL maintenance status "maintained".

The Current Maintainer of this work is Johannes Braams.

This file may only be distributed together with a copy of the Babel
system. You may however distribute the Babel system without
such generated files.

The list of all files belonging to the Babel distribution is
given in the file `manifest.bbl'. See also `legal.bbl for additional
information.

In particular, permission is granted to customize the declarations in
this file to serve the needs of your installation.

However, NO PERMISSION is granted to distribute a modified version
of this file under its original name.

\endpreamble

\keepsilent

\usedir{tex/generic/babel} 

\usepreamble\mainpreamble
\generate{\file{icelandic.ldf}{\from{icelandic.dtx}{code}}
          }
\usepreamble\fdpreamble

\ifToplevel{
\Msg{***********************************************************}
\Msg{*}
\Msg{* To finish the installation you have to move the following}
\Msg{* files into a directory searched by TeX:}
\Msg{*}
\Msg{* \space\space All *.def, *.fd, *.ldf, *.sty}
\Msg{*}
\Msg{* To produce the documentation run the files ending with}
\Msg{* '.dtx' and `.fdd' through LaTeX.}
\Msg{*}
\Msg{* Happy TeXing}
\Msg{***********************************************************}
}
 
\endinput
}
\DeclareOption{interlingua}{% \iffalse meta-comment
%
% Copyright 1989-2005 Johannes L. Braams and any individual authors
% listed elsewhere in this file.  All rights reserved.
% 
% This file is part of the Babel system.
% --------------------------------------
% 
% It may be distributed and/or modified under the
% conditions of the LaTeX Project Public License, either version 1.3
% of this license or (at your option) any later version.
% The latest version of this license is in
%   http://www.latex-project.org/lppl.txt
% and version 1.3 or later is part of all distributions of LaTeX
% version 2003/12/01 or later.
% 
% This work has the LPPL maintenance status "maintained".
% 
% The Current Maintainer of this work is Johannes Braams.
% 
% The list of all files belonging to the Babel system is
% given in the file `manifest.bbl. See also `legal.bbl' for additional
% information.
% 
% The list of derived (unpacked) files belonging to the distribution
% and covered by LPPL is defined by the unpacking scripts (with
% extension .ins) which are part of the distribution.
% \fi
% \CheckSum{85}
%
% \iffalse
%    Tell the \LaTeX\ system who we are and write an entry on the
%    transcript.
%<*dtx>
\ProvidesFile{interlingua.dtx}
%</dtx>
%<code>\ProvidesLanguage{interlingua}
%\fi
%\ProvidesFile{interlingua.dtx}
        [2005/03/30 v1.6 Interlingua support from the babel system]
%\iffalse
%% Babel package for LaTeX version 2e
%% Copyright (C) 1989 -- 2005
%%           by Johannes Braams, TeXniek
%
%% Please report errors to: J.L. Braams
%%                          babel at braams.cistron.nl
%
%    This file is part of the babel system, it provides the source code for
%    the Interlingua language definition file.
%<*filedriver>
\documentclass{ltxdoc}
\newcommand*{\TeXhax}{\TeX hax}
\newcommand*{\babel}{\textsf{babel}}
\newcommand*{\langvar}{$\langle \mathit lang \rangle$}
\newcommand*{\note}[1]{}
\newcommand*{\Lopt}[1]{\textsf{#1}}
\newcommand*{\file}[1]{\texttt{#1}}
\usepackage{url}
\begin{document}
 \DocInput{interlingua.dtx}
\end{document}
%</filedriver>
%\fi
% \GetFileInfo{interlingua.dtx}
%
%  \section{The Interlingua language}
%
%    The file \file{\filename}\footnote{The file described in this
%    section has version number \fileversion\ and was last revised on
%    \filedate.}  defines all the language definition macros for the
%    Interlingua language. This file was contributed by Peter
%    Kleiweg, kleiweg at let.rug.nl. 
%
%    Interlingua is an auxiliary language, built from the common
%    vocabulary of Spanish/Portuguese, English, Italian and French,
%    with some normalisation of spelling. The grammar is very easy,
%    more similar to English's than to neolatin languages. The site
%    \url{http://www.interlingua.com} is mostly written in interlingua
%    (as is \url{http://interlingua.altervista.org}), in case you want
%    to read some sample of it. 
% 
%    You can have a look at the grammar at
%    \url{http://www.geocities.com/linguablau} 
%
% \StopEventually{}
%
%    The macro |\LdfInit| takes care of preventing that this file is
%    loaded more than once, checking the category code of the
%    \texttt{@} sign, etc.
%    \begin{macrocode}
%<*code>
\LdfInit{interlingua}{captionsinterlingua}
%    \end{macrocode}
%
%    When this file is read as an option, i.e. by the |\usepackage|
%    command, \texttt{interlingua} could be an `unknown' language in
%    which case we have to make it known.  So we check for the
%    existence of |\l@interlingua| to see whether we have to do
%    something here.
%
%    \begin{macrocode}
\ifx\undefined\l@interlingua
  \@nopatterns{Interlingua}
  \adddialect\l@interlingua0\fi
%    \end{macrocode}

%    The next step consists of defining commands to switch to (and
%    from) the Interlingua language.
%
%
%  \begin{macro}{\interlinguahyphenmins}
%    This macro is used to store the correct values of the hyphenation
%    parameters |\lefthyphenmin| and |\righthyphenmin|.
%    \begin{macrocode}
\providehyphenmins{interlingua}{\tw@\tw@}
%    \end{macrocode}
%  \end{macro}
%
% \begin{macro}{\captionsinterlingua}
%    The macro |\captionsinterlingua| defines all strings used in the
%    four standard documentclasses provided with \LaTeX.
%    \begin{macrocode}
\def\captionsinterlingua{%
  \def\prefacename{Prefacio}%
  \def\refname{Referentias}%
  \def\abstractname{Summario}%
  \def\bibname{Bibliographia}%
  \def\chaptername{Capitulo}%
  \def\appendixname{Appendice}%
  \def\contentsname{Contento}%
  \def\listfigurename{Lista de figuras}%
  \def\listtablename{Lista de tabellas}%
  \def\indexname{Indice}%
  \def\figurename{Figura}%
  \def\tablename{Tabella}%
  \def\partname{Parte}%
  \def\enclname{Incluso}%
  \def\ccname{Copia}%
  \def\headtoname{A}%
  \def\pagename{Pagina}%
  \def\seename{vide}%
  \def\alsoname{vide etiam}%
  \def\proofname{Prova}%
  \def\glossaryname{Glossario}%
  }
%    \end{macrocode}
% \end{macro}


% \begin{macro}{\dateinterlingua}
%    The macro |\dateinterlingua| redefines the command |\today| to
%    produce Interlingua dates.
%    \begin{macrocode}
\def\dateinterlingua{%
  \def\today{le~\number\day\space de \ifcase\month\or
    januario\or februario\or martio\or april\or maio\or junio\or
    julio\or augusto\or septembre\or octobre\or novembre\or
    decembre\fi
    \space \number\year}}
%    \end{macrocode}
% \end{macro}
%

% \begin{macro}{\extrasinterlingua}
% \begin{macro}{\noextrasinterlingua}
%    The macro |\extrasinterlingua| will perform all the extra
%    definitions needed for the Interlingua language. The macro
%    |\noextrasinterlingua| is used to cancel the actions of
%    |\extrasinterlingua|.  For the moment these macros are empty but
%    they are defined for compatibility with the other
%    language definition files.
%
%    \begin{macrocode}
\addto\extrasinterlingua{}
\addto\noextrasinterlingua{}
%    \end{macrocode}
% \end{macro}
% \end{macro}
%
%    The macro |\ldf@finish| takes care of looking for a
%    configuration file, setting the main language to be switched on
%    at |\begin{document}| and resetting the category code of
%    \texttt{@} to its original value.
%    \begin{macrocode}
\ldf@finish{interlingua}
%</code>
%    \end{macrocode}
%
% \Finale
%\endinput
%% \CharacterTable
%%  {Upper-case    \A\B\C\D\E\F\G\H\I\J\K\L\M\N\O\P\Q\R\S\T\U\V\W\X\Y\Z
%%   Lower-case    \a\b\c\d\e\f\g\h\i\j\k\l\m\n\o\p\q\r\s\t\u\v\w\x\y\z
%%   Digits        \0\1\2\3\4\5\6\7\8\9
%%   Exclamation   \!     Double quote  \"     Hash (number) \#
%%   Dollar        \$     Percent       \%     Ampersand     \&
%%   Acute accent  \'     Left paren    \(     Right paren   \)
%%   Asterisk      \*     Plus          \+     Comma         \,
%%   Minus         \-     Point         \.     Solidus       \/
%%   Colon         \:     Semicolon     \;     Less than     \<
%%   Equals        \=     Greater than  \>     Question mark \?
%%   Commercial at \@     Left bracket  \[     Backslash     \\
%%   Right bracket \]     Circumflex    \^     Underscore    \_
%%   Grave accent  \`     Left brace    \{     Vertical bar  \|
%%   Right brace   \}     Tilde         \~}
%%
}
\DeclareOption{irish}{%%
%% This file will generate fast loadable files and documentation
%% driver files from the doc files in this package when run through
%% LaTeX or TeX.
%%
%% Copyright 1989-2005 Johannes L. Braams and any individual authors
%% listed elsewhere in this file.  All rights reserved.
%% 
%% This file is part of the Babel system.
%% --------------------------------------
%% 
%% It may be distributed and/or modified under the
%% conditions of the LaTeX Project Public License, either version 1.3
%% of this license or (at your option) any later version.
%% The latest version of this license is in
%%   http://www.latex-project.org/lppl.txt
%% and version 1.3 or later is part of all distributions of LaTeX
%% version 2003/12/01 or later.
%% 
%% This work has the LPPL maintenance status "maintained".
%% 
%% The Current Maintainer of this work is Johannes Braams.
%% 
%% The list of all files belonging to the LaTeX base distribution is
%% given in the file `manifest.bbl. See also `legal.bbl' for additional
%% information.
%% 
%% The list of derived (unpacked) files belonging to the distribution
%% and covered by LPPL is defined by the unpacking scripts (with
%% extension .ins) which are part of the distribution.
%%
%% --------------- start of docstrip commands ------------------
%%
\def\filedate{1999/04/11}
\def\batchfile{irish.ins}
\input docstrip.tex

{\ifx\generate\undefined
\Msg{**********************************************}
\Msg{*}
\Msg{* This installation requires docstrip}
\Msg{* version 2.3c or later.}
\Msg{*}
\Msg{* An older version of docstrip has been input}
\Msg{*}
\Msg{**********************************************}
\errhelp{Move or rename old docstrip.tex.}
\errmessage{Old docstrip in input path}
\batchmode
\csname @@end\endcsname
\fi}

\declarepreamble\mainpreamble
This is a generated file.

Copyright 1989-2005 Johannes L. Braams and any individual authors
listed elsewhere in this file.  All rights reserved.

This file was generated from file(s) of the Babel system.
---------------------------------------------------------

It may be distributed and/or modified under the
conditions of the LaTeX Project Public License, either version 1.3
of this license or (at your option) any later version.
The latest version of this license is in
  http://www.latex-project.org/lppl.txt
and version 1.3 or later is part of all distributions of LaTeX
version 2003/12/01 or later.

This work has the LPPL maintenance status "maintained".

The Current Maintainer of this work is Johannes Braams.

This file may only be distributed together with a copy of the Babel
system. You may however distribute the Babel system without
such generated files.

The list of all files belonging to the Babel distribution is
given in the file `manifest.bbl'. See also `legal.bbl for additional
information.

The list of derived (unpacked) files belonging to the distribution
and covered by LPPL is defined by the unpacking scripts (with
extension .ins) which are part of the distribution.
\endpreamble

\declarepreamble\fdpreamble
This is a generated file.

Copyright 1989-2005 Johannes L. Braams and any individual authors
listed elsewhere in this file.  All rights reserved.

This file was generated from file(s) of the Babel system.
---------------------------------------------------------

It may be distributed and/or modified under the
conditions of the LaTeX Project Public License, either version 1.3
of this license or (at your option) any later version.
The latest version of this license is in
  http://www.latex-project.org/lppl.txt
and version 1.3 or later is part of all distributions of LaTeX
version 2003/12/01 or later.

This work has the LPPL maintenance status "maintained".

The Current Maintainer of this work is Johannes Braams.

This file may only be distributed together with a copy of the Babel
system. You may however distribute the Babel system without
such generated files.

The list of all files belonging to the Babel distribution is
given in the file `manifest.bbl'. See also `legal.bbl for additional
information.

In particular, permission is granted to customize the declarations in
this file to serve the needs of your installation.

However, NO PERMISSION is granted to distribute a modified version
of this file under its original name.

\endpreamble

\keepsilent

\usedir{tex/generic/babel} 

\usepreamble\mainpreamble
\generate{\file{irish.ldf}{\from{irish.dtx}{code}}
          }
\usepreamble\fdpreamble

\ifToplevel{
\Msg{***********************************************************}
\Msg{*}
\Msg{* To finish the installation you have to move the following}
\Msg{* files into a directory searched by TeX:}
\Msg{*}
\Msg{* \space\space All *.def, *.fd, *.ldf, *.sty}
\Msg{*}
\Msg{* To produce the documentation run the files ending with}
\Msg{* '.dtx' and `.fdd' through LaTeX.}
\Msg{*}
\Msg{* Happy TeXing}
\Msg{***********************************************************}
}
 
\endinput
}
\DeclareOption{italian}{%%
%% This file will generate fast loadable files and documentation
%% driver files from the doc files in this package when run through
%% LaTeX or TeX.
%%
%% Copyright 1989-2008 Johannes L. Braams and any individual authors
%% listed elsewhere in this file.  All rights reserved.
%% 
%% This file is part of the Babel system.
%% --------------------------------------
%% 
%% It may be distributed and/or modified under the
%% conditions of the LaTeX Project Public License, either version 1.3
%% of this license or (at your option) any later version.
%% The latest version of this license is in
%%   http://www.latex-project.org/lppl.txt
%% and version 1.3 or later is part of all distributions of LaTeX
%% version 2003/12/01 or later.
%% 
%% This work has the LPPL maintenance status "maintained".
%% 
%% The Current Maintainer of this work is Johannes Braams.
%% 
%% The list of all files belonging to the LaTeX base distribution is
%% given in the file `manifest.bbl. See also `legal.bbl' for additional
%% information.
%% 
%% The list of derived (unpacked) files belonging to the distribution
%% and covered by LPPL is defined by the unpacking scripts (with
%% extension .ins) which are part of the distribution.
%%
%% --------------- start of docstrip commands ------------------
%%
%%
%% Questo file serve per generare dal file italian.dtx il file
%% italian.ldf mediante il programma LaTeX/docstrip che toglie
%% praticamente tutti i commenti e ne rende il caricamento molto
%% veloce. La documentazione puo' essere ottenuta passando a
%% LaTeX2e il file sorgente italian.dtx
%%
\def\filedate{2001/01/19}
\def\batchfile{italian.ins}
\input docstrip.tex

{\ifx\generate\undefined
\Msg{**********************************************}
\Msg{*}
\Msg{* This installation requires docstrip}
\Msg{* version 2.3c or later.}
\Msg{*}
\Msg{* An older version of docstrip has been input}
\Msg{*}
\Msg{**********************************************}
\errhelp{Move or rename old docstrip.tex.}
\errmessage{Old docstrip in input path}
\batchmode
\csname @@end\endcsname
\fi}

\declarepreamble\mainpreamble
This is a generated file.

Copyright 1989-2008 Johannes L. Braams and any individual authors
listed elsewhere in this file.  All rights reserved.

This file was generated from file(s) of the Babel system.
---------------------------------------------------------

It may be distributed and/or modified under the
conditions of the LaTeX Project Public License, either version 1.3
of this license or (at your option) any later version.
The latest version of this license is in
  http://www.latex-project.org/lppl.txt
and version 1.3 or later is part of all distributions of LaTeX
version 2003/12/01 or later.

This work has the LPPL maintenance status "maintained".

The Current Maintainer of this work is Johannes Braams.

This file may only be distributed together with a copy of the Babel
system. You may however distribute the Babel system without
such generated files.

The list of all files belonging to the Babel distribution is
given in the file `manifest.bbl'. See also `legal.bbl for additional
information.

The list of derived (unpacked) files belonging to the distribution
and covered by LPPL is defined by the unpacking scripts (with
extension .ins) which are part of the distribution.
\endpreamble

\declarepreamble\fdpreamble
This is a generated file.

Copyright 1989-2008 Johannes L. Braams and any individual authors
listed elsewhere in this file.  All rights reserved.

This file was generated from file(s) of the Babel system.
---------------------------------------------------------

It may be distributed and/or modified under the
conditions of the LaTeX Project Public License, either version 1.3
of this license or (at your option) any later version.
The latest version of this license is in
  http://www.latex-project.org/lppl.txt
and version 1.3 or later is part of all distributions of LaTeX
version 2003/12/01 or later.

This work has the LPPL maintenance status "maintained".

The Current Maintainer of this work is Johannes Braams.

This file may only be distributed together with a copy of the Babel
system. You may however distribute the Babel system without
such generated files.

The list of all files belonging to the Babel distribution is
given in the file `manifest.bbl'. See also `legal.bbl for additional
information.

In particular, permission is granted to customize the declarations in
this file to serve the needs of your installation.

However, NO PERMISSION is granted to distribute a modified version
of this file under its original name.

\endpreamble

\keepsilent

\usedir{tex/generic/babel} 

\usepreamble\mainpreamble
\generate{\file{italian.ldf}{\from{italian.dtx}{code}}
          }
\usepreamble\fdpreamble

\ifToplevel{
\Msg{***********************************************************}
\Msg{*}
\Msg{* To finish the installation you have to move the following}
\Msg{* files into a directory searched by TeX:}
\Msg{*}
\Msg{* \space\space All *.def, *.fd, *.ldf, *.sty}
\Msg{*}
\Msg{* To produce the documentation run the files ending with}
\Msg{* '.dtx' and `.fdd' through LaTeX.}
\Msg{*}
\Msg{* Happy TeXing}
\Msg{***********************************************************}
}
 
\endinput
}
\DeclareOption{latin}{%%
%% This file will generate fast loadable files and documentation
%% driver files from the doc files in this package when run through
%% LaTeX or TeX.
%%
%% Copyright 1989-2008 Johannes L. Braams and any individual authors
%% listed elsewhere in this file.  All rights reserved.
%% 
%% This file is part of the Babel system.
%% --------------------------------------
%% 
%% It may be distributed and/or modified under the
%% conditions of the LaTeX Project Public License, either version 1.3
%% of this license or (at your option) any later version.
%% The latest version of this license is in
%%   http://www.latex-project.org/lppl.txt
%% and version 1.3 or later is part of all distributions of LaTeX
%% version 2003/12/01 or later.
%% 
%% This work has the LPPL maintenance status "maintained".
%% 
%% The Current Maintainer of this work is Johannes Braams.
%% 
%% The list of all files belonging to the LaTeX base distribution is
%% given in the file `manifest.bbl. See also `legal.bbl' for additional
%% information.
%% 
%% The list of derived (unpacked) files belonging to the distribution
%% and covered by LPPL is defined by the unpacking scripts (with
%% extension .ins) which are part of the distribution.
%%
%% --------------- start of docstrip commands ------------------
%%
\def\filedate{1999/04/11}
\def\batchfile{latin.ins}
\input docstrip.tex

{\ifx\generate\undefined
\Msg{**********************************************}
\Msg{*}
\Msg{* This installation requires docstrip}
\Msg{* version 2.3c or later.}
\Msg{*}
\Msg{* An older version of docstrip has been input}
\Msg{*}
\Msg{**********************************************}
\errhelp{Move or rename old docstrip.tex.}
\errmessage{Old docstrip in input path}
\batchmode
\csname @@end\endcsname
\fi}

\declarepreamble\mainpreamble
This is a generated file.

Copyright 1989-2008 Johannes L. Braams and any individual authors
listed elsewhere in this file.  All rights reserved.

This file was generated from file(s) of the Babel system.
---------------------------------------------------------

It may be distributed and/or modified under the
conditions of the LaTeX Project Public License, either version 1.3
of this license or (at your option) any later version.
The latest version of this license is in
  http://www.latex-project.org/lppl.txt
and version 1.3 or later is part of all distributions of LaTeX
version 2003/12/01 or later.

This work has the LPPL maintenance status "maintained".

The Current Maintainer of this work is Johannes Braams.

This file may only be distributed together with a copy of the Babel
system. You may however distribute the Babel system without
such generated files.

The list of all files belonging to the Babel distribution is
given in the file `manifest.bbl'. See also `legal.bbl for additional
information.

The list of derived (unpacked) files belonging to the distribution
and covered by LPPL is defined by the unpacking scripts (with
extension .ins) which are part of the distribution.

Copyright 1999-2007 Claudio Beccari All rights reserved.
\endpreamble

\declarepreamble\fdpreamble
This is a generated file.

Copyright 1989-2008 Johannes L. Braams and any individual authors
listed elsewhere in this file.  All rights reserved.

This file was generated from file(s) of the Babel system.
---------------------------------------------------------

It may be distributed and/or modified under the
conditions of the LaTeX Project Public License, either version 1.3
of this license or (at your option) any later version.
The latest version of this license is in
  http://www.latex-project.org/lppl.txt
and version 1.3 or later is part of all distributions of LaTeX
version 2003/12/01 or later.

This work has the LPPL maintenance status "maintained".

The Current Maintainer of this work is Johannes Braams.

This file may only be distributed together with a copy of the Babel
system. You may however distribute the Babel system without
such generated files.

The list of all files belonging to the Babel distribution is
given in the file `manifest.bbl'. See also `legal.bbl for additional
information.

In particular, permission is granted to customize the declarations in
this file to serve the needs of your installation.

However, NO PERMISSION is granted to distribute a modified version
of this file under its original name.

\endpreamble

\keepsilent

\usedir{tex/generic/babel} 

\usepreamble\mainpreamble
\generate{\file{latin.ldf}{\from{latin.dtx}{code}}
          }
\usepreamble\fdpreamble

\ifToplevel{
\Msg{***********************************************************}
\Msg{*}
\Msg{* To finish the installation you have to move the following}
\Msg{* files into a directory searched by TeX:}
\Msg{*}
\Msg{* \space\space All *.def, *.fd, *.ldf, *.sty}
\Msg{*}
\Msg{* To produce the documentation run the files ending with}
\Msg{* '.dtx' and `.fdd' through LaTeX.}
\Msg{*}
\Msg{* Happy TeXing}
\Msg{***********************************************************}
}
 
\endinput
}
\DeclareOption{lowersorbian}{% \iffalse meta-comment
%
% Copyright 1989-2008 Johannes L. Braams and any individual authors
% listed elsewhere in this file.  All rights reserved.
% 
% This file is part of the Babel system.
% --------------------------------------
% 
% It may be distributed and/or modified under the
% conditions of the LaTeX Project Public License, either version 1.3
% of this license or (at your option) any later version.
% The latest version of this license is in
%   http://www.latex-project.org/lppl.txt
% and version 1.3 or later is part of all distributions of LaTeX
% version 2003/12/01 or later.
% 
% This work has the LPPL maintenance status "maintained".
% 
% The Current Maintainer of this work is Johannes Braams.
% 
% The list of all files belonging to the Babel system is
% given in the file `manifest.bbl. See also `legal.bbl' for additional
% information.
% 
% The list of derived (unpacked) files belonging to the distribution
% and covered by LPPL is defined by the unpacking scripts (with
% extension .ins) which are part of the distribution.
% \fi
% \CheckSum{152}
% \iffalse
%
%    Tell the \LaTeX\ system who we are and write an entry on the
%    transcript.
%<*dtx>
\ProvidesFile{lsorbian.dtx}
%</dtx>
%<code>\ProvidesLanguage{lsorbian}
%\fi
%\ProvidesFile{lsorbian.dtx}
        [2008/03/17 v1.0g Lower Sorbian support from the babel system]
%\iffalse
%% File `lsorbian.dtx'
%% Babel package for LaTeX version 2e
%% Copyright (C) 1989 - 2008
%%           by Johannes Braams, TeXniek
%
%% Lower Sorbian Language Definition File
%% Copyright (C) 1994 - 2008
%%           by Eduard Werner
%           Werner, Eduard",
%           Serbski institut z. t.,
%           Dw\'orni\v{s}\'cowa 6
%           02625 Budy\v{s}in/Bautzen
%           Germany",
%           (??)3591 497223",
%           edi at kaihh.hanse.de",
%
%% Please report errors to: Eduard Werner edi at kaihh.hanse.de
%%
%    This file is part of the babel system, it provides the source
%    code for the Lower Sorbian definition file.
%<*filedriver>
\documentclass{ltxdoc}
\newcommand*\TeXhax{\TeX hax}
\newcommand*\babel{\textsf{babel}}
\newcommand*\langvar{$\langle \it lang \rangle$}
\newcommand*\note[1]{}
\newcommand*\Lopt[1]{\textsf{#1}}
\newcommand*\file[1]{\texttt{#1}}
\begin{document}
 \DocInput{lsorbian.dtx}
\end{document}
%</filedriver>
%\fi
%
% \GetFileInfo{lsorbian.dtx}
%
% \changes{lsorbian-0.1}{1994/10/10}{First version}
% \changes{lsorbian-1.0d}{1996/10/10}{Replaced \cs{undefined} with
%    \cs{@undefined} and \cs{empty} with \cs{@empty} for consistency
%    with \LaTeX, moved the definition of \cs{atcatcode} right to the
%    beginning.}
%
%  \section{The Lower Sorbian language}
%
%    The file \file{\filename}\footnote{The file described in this
%    section has version number \fileversion\ and was last revised on
%    \filedate.  It was written by Eduard Werner
%    (\texttt{edi@kaihh.hanse.de}).}  It defines all the
%    language-specific macros for Lower Sorbian.
%
% \StopEventually{}
%
%    The macro |\LdfInit| takes care of preventing that this file is
%    loaded more than once, checking the category code of the
%    \texttt{@} sign, etc.
% \changes{lsorbian-1.0d}{1996/11/03}{Now use \cs{LdfInit} to perform
%    initial checks}
% \changes{lsorbian-1.0g}{2007/10/19}{This file can be loaded under
%    more than one name.}
%    \begin{macrocode}
%<*code>
\LdfInit\CurrentOption{date\CurrentOption}
%    \end{macrocode}
%
%    When this file is read as an option, i.e. by the |\usepackage|
%    command, \texttt{lsorbian} will be an `unknown' language, in which
%    case we have to make it known. So we check for the existence of
%    |\l@lsorbian| to see whether we have to do something here.
% \changes{lsorbian-1.0g}{2007/10/19}{This file can be loaded under
%    more than one name.}
%    As
%    \babel\ also knwos the option \Lopt{lowersorbian} we have to
%    check that as well.
%
%    \begin{macrocode}
\ifx\l@lowersorbian\@undefined
  \ifx\l@lsorbian\@undefined
    \@nopatterns{Lsorbian}
    \adddialect\l@lsorbian\z@
    \let\l@lowersorbian\l@lsorbian
  \else
    \let\l@lowersorbian\l@lsorbian
  \fi
\else
  \let\l@lsorbian\l@lowersorbian
\fi
%    \end{macrocode}
%
%    The next step consists of defining commands to switch to (and
%    from) the Lower Sorbian language.
%
%  \begin{macro}{\captionslsorbian}
%    The macro |\captionslsorbian| defines all strings used in the four
%    standard documentclasses provided with \LaTeX.
% \changes{lsorbian-1.0b}{1995/07/04}{Added \cs{proofname} for
%    AMS-\LaTeX}
% \changes{lsorbian-1.0f}{2000/09/22}{Added \cs{glossaryname}}
% \changes{lsorbian-1.0g}{2007/10/19}{Make this work for more than one
%    option name.}
%    \begin{macrocode}
\@namedef{captions\CurrentOption}{%
  \def\prefacename{Zawod}%
  \def\refname{Referency}%
  \def\abstractname{Abstrakt}%
  \def\bibname{Literatura}%
  \def\chaptername{Kapitl}%
  \def\appendixname{Dodawki}%
  \def\contentsname{Wop\'simje\'se}%
  \def\listfigurename{Zapis wobrazow}%
  \def\listtablename{Zapis tabulkow}%
  \def\indexname{Indeks}%
  \def\figurename{Wobraz}%
  \def\tablename{Tabulka}%
  \def\partname{\'Z\v el}%
  \def\enclname{P\'si\l oga}%
  \def\ccname{CC}%
  \def\headtoname{Komu}%
  \def\pagename{Strona}%
  \def\seename{gl.}%
  \def\alsoname{gl.~teke}%
  \def\proofname{Proof}%  <-- needs translation
  \def\glossaryname{Glossary}% <-- Needs translation
  }%
%    \end{macrocode}
%  \end{macro}
%
%  \begin{macro}{\newdatelsorbian}
%    The macro |\newdatelsorbian| redefines the command |\today| to
%    produce Lower Sorbian dates.
% \changes{lsorbian-1.0e}{1997/10/01}{Use \cs{edef} to define
%    \cs{today} to save memory}
% \changes{lsorbian-1.0e}{1998/03/28}{use \cs{def} instead of
%    \cs{edef}} 
% \changes{lsorbian-1.0g}{2007/10/19}{Make this work for more than one
%    option name.}
%    \begin{macrocode}
\@namedef{newdate\CurrentOption}{%
  \def\today{\number\day.~\ifcase\month\or
    januara\or februara\or m\v erca\or apryla\or maja\or
    junija\or julija\or awgusta\or septembra\or oktobra\or
    nowembra\or decembra\fi
    \space \number\year}}
%    \end{macrocode}
%  \end{macro}
%
%  \begin{macro}{\olddatelsorbian}
%    The macro |\olddatelsorbian| redefines the command |\today| to
%    produce old-style Lower Sorbian dates.
% \changes{lsorbian-1.0g}{2007/10/19}{Make this work for more than one
%    option name.}
%    \begin{macrocode}
\@namedef{olddate\CurrentOption}{%
  \def\today{\number\day.~\ifcase\month\or
    wjelikego ro\v zka\or
    ma\l ego ro\v zka\or
    nal\v etnika\or
    jat\v sownika\or
    ro\v zownika\or
    sma\v znika\or
    pra\v znika\or
    \v znje\'nca\or
    po\v znje\'nca\or
    winowca\or
    nazymnika\or 
    godownika\fi \space \number\year}}
%    \end{macrocode}
%  \end{macro}
%
%    The default will be the new-style dates.
% \changes{lsorbian-1.0g}{2007/10/19}{Make this work for more than one
%    option name.}
%    \begin{macrocode}
\expandafter\let\csname date\CurrentOption\expandafter\endcsname
                \csname newdate\CurrentOption\endcsname
%    \end{macrocode}
%
% \begin{macro}{\extraslsorbian}
% \begin{macro}{\noextraslsorbian}
%    The macro |\extraslsorbian| will perform all the extra
%    definitions needed for the lsorbian language. The macro
%    |\noextraslsorbian| is used to cancel the actions of
%    |\extraslsorbian|.  For the moment these macros are empty but
%    they are defined for compatibility with the other language
%    definition files.
%
%    \begin{macrocode}
\@namedef{extras\CurrentOption}{}
\@namedef{noextras\CurrentOption}{}
%    \end{macrocode}
% \end{macro}
% \end{macro}
%
%    The macro |\ldf@finish| takes care of looking for a
%    configuration file, setting the main language to be switched on
%    at |\begin{document}| and resetting the category code of
%    \texttt{@} to its original value.
% \changes{lsorbian-1.0d}{1996/11/03}{Now use \cs{ldf@finish} to wrap
%    up} 
% \changes{lsorbian-1.0g}{2007/10/19}{Make this work for more than one
%    option name}
%    \begin{macrocode}
\ldf@finish\CurrentOption
%</code>
%    \end{macrocode}
%
% \Finale
%%
%% \CharacterTable
%%  {Upper-case    \A\B\C\D\E\F\G\H\I\J\K\L\M\N\O\P\Q\R\S\T\U\V\W\X\Y\Z
%%   Lower-case    \a\b\c\d\e\f\g\h\i\j\k\l\m\n\o\p\q\r\s\t\u\v\w\x\y\z
%%   Digits        \0\1\2\3\4\5\6\7\8\9
%%   Exclamation   \!     Double quote  \"     Hash (number) \#
%%   Dollar        \$     Percent       \%     Ampersand     \&
%%   Acute accent  \'     Left paren    \(     Right paren   \)
%%   Asterisk      \*     Plus          \+     Comma         \,
%%   Minus         \-     Point         \.     Solidus       \/
%%   Colon         \:     Semicolon     \;     Less than     \<
%%   Equals        \=     Greater than  \>     Question mark \?
%%   Commercial at \@     Left bracket  \[     Backslash     \\
%%   Right bracket \]     Circumflex    \^     Underscore    \_
%%   Grave accent  \`     Left brace    \{     Vertical bar  \|
%%   Right brace   \}     Tilde         \~}
%%
\endinput
}
%^^A\DeclareOption{kannada}{\input{kannada.ldf}}
\DeclareOption{magyar}{% \iffalse meta-comment
%
% Copyright 1989-2005 Johannes L. Braams and any individual authors
% listed elsewhere in this file.  All rights reserved.
% 
% This file is part of the Babel system.
% --------------------------------------
% 
% It may be distributed and/or modified under the
% conditions of the LaTeX Project Public License, either version 1.3
% of this license or (at your option) any later version.
% The latest version of this license is in
%   http://www.latex-project.org/lppl.txt
% and version 1.3 or later is part of all distributions of LaTeX
% version 2003/12/01 or later.
% 
% This work has the LPPL maintenance status "maintained".
% 
% The Current Maintainer of this work is Johannes Braams.
% 
% The list of all files belonging to the Babel system is
% given in the file `manifest.bbl. See also `legal.bbl' for additional
% information.
% 
% The list of derived (unpacked) files belonging to the distribution
% and covered by LPPL is defined by the unpacking scripts (with
% extension .ins) which are part of the distribution.
% \fi
% \CheckSum{1538}
% \iffalse
%    Tell the \LaTeX\ system who we are and write an entry on the
%    transcript.
%<*dtx>
\ProvidesFile{magyar.dtx}
%</dtx>
%<code>\ProvidesLanguage{magyar}
%\fi
%\ProvidesFile{magyar.dtx}
        [2005/03/30 v1.4j Magyar support from the babel system]
%\iffalse
%% File `magyar.dtx'
%% Babel package for LaTeX version 2e
%% Copyright (C) 1989 - 2005
%%           by Johannes Braams, TeXniek
%
%% Magyar Language Definition File
%% Copyright (C) 1989 - 2005
%%           by Johannes Braams, TeXniek
%%              \'Arp\'ad B\'IR\'O
%%              J\'ozsef B\'ERCES
%
% Please report errors to: J.L. Braams babel braams.cistron.nl
%
%    This file is part of the babel system, it provides the source
%    code for the Hungarian language definition file.  A contribution
%    was made by Attila Koppanyi (attila@cernvm.cern.ch).
%<*filedriver>
\documentclass{ltxdoc}
\newcommand*\TeXhax{\TeX hax}
\newcommand*\babel{\textsf{babel}}
\newcommand*\langvar{$\langle \it lang \rangle$}
\newcommand*\note[1]{}
\newcommand*\Lopt[1]{\textsf{#1}}
\newcommand*\file[1]{\texttt{#1}}
\begin{document}
 \DocInput{magyar.dtx}
\end{document}
%</filedriver>
%\fi
% \GetFileInfo{magyar.dtx}
%
% \changes{magyar-1.0a}{1991/07/15}{Renamed \file{babel.sty} in
%    \file{babel.com}}
% \changes{magyar-1.1}{1992/02/16}{Brought up-to-date with babel 3.2a}
% \changes{magyar-1.1d}{1994/02/08}{Further spelling corrections}
% \changes{magyar-1.1e}{1994/02/09}{Still more spelling corrections}
% \changes{magyar-1.2}{1994/02/27}{Update for \LaTeXe}
% \changes{magyar-1.3c}{1994/06/26}{Removed the use of \cs{filedate}
%    and moved identification after the loading of \file{babel.def}}
% \changes{magyar-1.3g}{1996/07/12}{Replaced \cs{undefined} with
%    \cs{@undefined} and \cs{empty} with \cs{@empty} for consistency
%    with \LaTeX}
% \changes{magyar-1.4a}{1998/06/04}{order inverting in
%    headings/titles/captions; definite article handling; active char
%    for special hyphenation}
% \changes{magyar-1.4d}{2001/11/15}{Corrected checksum}
% \changes{magyar-1.4j}{2004/11/17}{Added missing comment characters
%    in the redefinitions of \cs{ps@headings} to prevent spurious
%    spaces} 
%
%  \section{The Hungarian language}
%
%    The file option \file{\filename} defines all the
%    language definition macros for the Hungarian language.
%
%    The \babel{} support for the Hungarian language until file
%    version 1.3i was essentially changing the English document
%    elements to Hungarian ones, but because of the differences
%    between these too languages this was actually unusable (`Part I'
%    was transferred to `R\'esz I' which is not usable instead of `I.\
%    r\'esz'). To enhance the typesetting facilities for Hungarian
%    the following should be considered:
%    \begin{itemize}
%      \item In Hungarian documents there is a period after
%            the part, section, subsection \mbox{etc.} numbers.
%
%      \item In the part, chapter, appendix name the number
%            (or letter) goes before the name, so `Part I' translates to
%            `I.\ r\'esz'.
%
%       \item The same is true with captions
%             (`Table 2.1' goes to `2.1.\ t\'abl\'azat').
%
%       \item There is a period after the caption name instead of a colon.
%             (`Table 2.1:' goes to `2.1.\ t\'abl\'azat.')
%
%       \item There is a period at the end of the title in a run-in head
%             (when |afterskip<0| in |\@startsection|).
%
%       \item Special hyphenation rules must be applied
%             for the so-called long double consonants (ccs, ssz,\dots).
%
%       \item The opening quotation mark is like the German one
%             (the closing is the same as in English).
%
%       \item In Hungarian figure, table, \mbox{etc.}
%             referencing a definite article is also incorporated.
%             The Hungarian definite articles behave like the English
%             indefinite ones (`a/an'). `a' is used for words beginning
%             with a consonant and `az' goes for a vowel.
%             Since some numbers begin with a vowel some others
%             with a consonant some commands should be provided for
%             automatic definite article generation.
%    \end{itemize}
%
%    Until file version 1.3i\footnote{That file was
%    last revised on 1996/12/23 with a contribution by the next
%    authors: Attila Kopp\'anyi (\texttt{attila@cernvm.cern.ch}),
%    \'Arp\'ad B\'{\i}r\'o (\texttt{JZP1104@HUSZEG11.bitnet}),
%    Istv\'an Hamecz (\texttt{hami@ursus.bke.hu)} and
%    Dezs\H{o} Horv\'ath (\texttt{horvath@pisa.infn.it}).} 
%    the special typesetting rules of the Hungarian language
%    mentioned above were not taken into consideration.
%    This version (\fileversion)\footnote{It was written by J\'ozsef
%    B\'erces (\texttt{jozsi@docs4.mht.bme.hu}) with some help from
%    Ferenc Wettl (\texttt{wettl@math.bme.hu}) and an idea from David
%    Carlisle (\texttt{david@dcarlisle.demon.co.uk}).} 
%    enables \babel{} to typeset `good-looking' Hungarian texts.
%
% \DescribeMacro\ontoday
%    The |\ontoday| command works like |\today| but
%    produces a slightly different date format used in expressions such
%    as `on February 10th'.
%
% \DescribeMacro\Az
%    The commands |\Az#1| and |\az#1| write the correct
%    definite article for the argument and the argument itself
%    (separated with a |~|). The star-forms (|\Az*| and |\az*|)
%    produce the article only.
%
% \DescribeMacro\Azr
%    |\Azr#1| and |\azr#1| treat the argument as a label so expand
%    it then write the definite article for |\r@#1|, a non-breakable
%    space then the label expansion. The star-forms do not print
%    the label expansion. |\Azr(#1| and |\azr(#1| are used for
%    equation referencing with the syntax |\azr(|\textit{label}|)|.
%
% \DescribeMacro\Aref
%    There are two aliases |\Aref| and |\aref| for |\Azr| and |\azr|,
%    respectively. During the preparation of a document it is not
%    known in general, if the code `|a~\ref{|\textit{label}|}|' or the code
%    `|az~\ref{|\textit{label}|}|' is the grammatically correct one. Writing
%    `|\aref{|\textit{label}|}|' instead of the previous ones solves the 
%    problem.
%
% \DescribeMacro\Azp
%    |\Azp#1| and |\azp#1| also treat the argument as a label but
%    use the label's page for definite article determination.
%    There are star-forms giving only the definite article without
%    the page number. 
%
% \DescribeMacro\Apageref
%    There are aliases |\Apageref| and |\apageref| for |\Azp| and
%    |\azp|, respectively. The code |\apageref{|\textit{label}|}| 
%    is equivalent either to |a~\pageref{|\textit{label}|}| or 
%    to |az~\pageref{|\textit{label}|}|.
%
% \DescribeMacro\Azc
%    |\Azc| and |\azc| work like the |\cite| command but
%    (of course) they insert the definite article. There can be
%    several comma separated cite labels and in that case the
%    definite article is given for the first one.
%    They accept |\cite|'s optional argument.
%    There are star-forms giving the definite article only.
%
% \DescribeMacro\Acite
%    There are aliases |\Acite| and |\acite| for
%    |\Azc| and |\azc|, respectively. 
%
%    For this language the character |`| is made active.
%    Table~\ref{tab:hun-actives} shows the shortcuts.
%    The main reason for the activation of the |`| character
%    is to handle the special hyphenation of the long double
%    consonants.
%    \begin{table}
%      \begin{center}
%        \begin{tabular}{lp{45mm}p{47mm}}
%          shortcut & explanation & example \\
%          \hline
%          |``|        &  same as |\glqq| in \babel{}, or
%                         |\quotedblbase| in T1 
%                         (opening quotation mark, like ,,) 
%                        & |``id\'ezet''|$\longrightarrow$,,id\'ezet'{}' \\
%          |`c|, |`C|  &  ccs is hyphenated as cs-cs
%                        & |lo`ccsan|$\longrightarrow$locs-csan \\
%          |`d|, |`D|  &  ddz is hyphenated as dz-dz
%                        & |e`ddz\"unk|$\longrightarrow$edz-dz\"unk \\
%          |`g|, |`G|  &  ggy is hyphenated as gy-gy
%                        & |po`ggy\'asz|$\longrightarrow$pogy-gy\'asz \\
%          |`l|, |`L|  &  lly is hyphenated as ly-ly
%                        & |Kod\'a`llyal|$\longrightarrow$Kod\'aly-lyal \\
%          |`n|, |`N|  &  nny is hyphenated as ny-ny
%                        & |me`nnyei|$\longrightarrow$meny-nyei \\
%          |`s|, |`S|  &  ssz is hyphenated as sz-sz
%                        & |vi`ssza|$\longrightarrow$visz-sza \\
%          |`t|, |`T|  &  tty is hyphenated as ty-ty
%                        & |po`ttyan|$\longrightarrow$poty-tyan \\
%          |`z|, |`Z|  &  zzs is hyphenated as zs-zs
%                        & |ri`zzsel|$\longrightarrow$rizs-zsel \\
%        \end{tabular}
%        \caption{The shortcuts defined in \file{magyar.ldf}}
%        \label{tab:hun-actives}
%      \end{center}
%    \end{table}
%
%
% \StopEventually{}
%
%    The macro |\LdfInit| takes care of preventing that this file is
%    loaded more than once, checking the category code of the
%    \texttt{@} sign, etc.
% \changes{magyar-1.3h}{1996/11/03}{Now use \cs{LdfInit} to perform
%    initial checks}
%    \begin{macrocode}
%<*code>
\LdfInit{magyar}{caption\CurrentOption}
%    \end{macrocode}
%
%    When this file is read as an option, i.e. by the |\usepackage|
%    command, \texttt{magyar} will be an `unknown' language in which
%    case we have to make it known.  So we check for the existence of
%    |\l@magyar| or |\l@hungarian| to see whether we have to do
%    something here.
%
% \changes{magyar-1.0b}{1991/10/29}{Removed use of \cs{@ifundefined}}
% \changes{magyar-1.1}{1992/02/16}{Added a warning when no hyphenation
%    patterns were loaded.}
% \changes{magyar-1.3c}{1994/06/26}{Now use \cs{@nopatterns} to
%    produce the warning}
% \changes{magyar-1.4d}{2003/09/18}{The \cs{else} clause got outside
%    of the \cs{if} statement, breaking the Hungarian support}
% \changes{magyar-1.4g}{2003/10/07}{Further change to make it work
%    when neither \cs{l@magyar} nor \cs{l@hugarian} are defined}
%    \begin{macrocode}
\ifx\l@magyar\@undefined
  \ifx\l@hungarian\@undefined
    \@nopatterns{Magyar}
    \adddialect\l@magyar0
  \else
    \let\l@magyar\l@hungarian
  \fi
\fi
%    \end{macrocode}
%    The statement above makes sure that |\l@magyar| is always
%    defined; if |\l@hungarian| is still undefined we make it equal to
%    |\l@magyar|. 
%    \begin{macrocode}
\ifx\l@hungarian\@undefined
  \let\l@hungarian\l@magyar
\fi
%    \end{macrocode}
%
%    The next step consists of defining commands to switch to (and
%    from) the Hungarian language.
%
% \begin{macro}{\captionsmagyar}
%    The macro |\captionsmagyar| defines all strings used in the four
%    standard document classes provided with \LaTeX.
% \changes{magyar-1.1}{1992/02/16}{Added \cs{seename}, \cs{alsoname}
%    and \cs{prefacename}}
% \changes{magyar-1.1}{1993/07/15}{\cs{headpagename} should be
%    \cs{pagename}}
% \changes{magyar-1.1c}{1994/01/05}{Added translations, fixed typos}
% \changes{magyar-1.3e}{1995/07/04}{Added \cs{proofname} for
%    AMS-\LaTeX}
% \changes{magyar-1.3f}{1996/04/18}{translated Proof and replaced some
%    translations}
% \changes{magyar-1.4a}{1998/06/04}{the initial letter of fejezet,
%    t\'abl\'azat, r\'esz, l\'asd changed to lowercase}
%    \begin{macrocode}
\@namedef{captions\CurrentOption}{%
  \def\prefacename{El\H osz\'o}%
%    \end{macrocode}
%    For the list of references at the end of an article we have a
%    choice between two words, `Referenci\'ak' (a Hungarian version of
%    the English word) and `Hivatkoz\'asok'. The latter seems to
%    be in more widespread use.
%    \begin{macrocode}
  \def\refname{Hivatkoz\'asok}%
%    \end{macrocode}
%    If you have a document with a summary instead of an abstract you
%    might want to replace the word `Kivonat' with
%    `\"Osszefoglal\'o'.
%    \begin{macrocode}
  \def\abstractname{Kivonat}%
%    \end{macrocode}
%    The Hungarian version of `Bibliography' is `Bibliogr\'afia', but
%    a more natural word to use is `Irodalomjegyz\'ek'.
%    \begin{macrocode}
  \def\bibname{Irodalomjegyz\'ek}%
  \def\chaptername{fejezet}%
  \def\appendixname{F\"uggel\'ek}%
  \def\contentsname{Tartalomjegyz\'ek}%
  \def\listfigurename{\'Abr\'ak jegyz\'eke}%
  \def\listtablename{T\'abl\'azatok jegyz\'eke}%
  \def\indexname{T\'argymutat\'o}%
  \def\figurename{\'abra}%
  \def\tablename{t\'abl\'azat}%
  \def\partname{r\'esz}%
  \def\enclname{Mell\'eklet}%
  \def\ccname{K\"orlev\'el--c\'\i mzettek}%
  \def\headtoname{C\'\i mzett}%
  \def\pagename{oldal}%
  \def\seename{l\'asd}%
  \def\alsoname{l\'asd m\'eg}%
%    \end{macrocode}
%    Besides the Hungarian word for Proof, `Bizony\'\i t\'as' we can
%    also name Corollary (K\"ovetkezm\'eny), Theorem (T\'etel) and
%    Lemma (Lemma).
% \changes{magyar-1.4b}{2000/09/20}{Added \cs{glossaryname}}
% \changes{magyar-1.4h}{2003/11/20}{Inserted translation for Glossary} 
%    \begin{macrocode}
  \def\proofname{Bizony\'\i t\'as}%
  \def\glossaryname{Sz\'ojegyz\'ek}%
  }%
%    \end{macrocode}
% \end{macro}
%
% \begin{macro}{\datemagyar}
%    The macro |\datemagyar| redefines the command |\today| to produce
%    Hungarian dates.
% \changes{magyar-1.1d}{1994/02/08}{Rewritten to produce the correct
%    date format}
% \changes{magyar-1.4a}{1998/06/10}{Use \cs{number}\cs{day} instead of
%    \cs{ifcase} construct}
%    \begin{macrocode}
\@namedef{date\CurrentOption}{%
  \def\today{%
    \number\year.\nobreakspace\ifcase\month\or
    janu\'ar\or febru\'ar\or m\'arcius\or
    \'aprilis\or m\'ajus\or j\'unius\or
    j\'ulius\or augusztus\or szeptember\or
    okt\'ober\or november\or december\fi
    \space\number\day.}}
%    \end{macrocode}
% \end{macro}
%
% \begin{macro}{\ondatemagyar}
%    The macro |\ondatemagyar| produces Hungarian dates which have the
%    meaning `\emph{on this day}'.  It does not redefine the command
%    |\today|.
% \changes{magyar-1.1c}{1994/01/05}{The date number should not be
%    followed by a dot.}
% \changes{magyar-1.1d}{1994/02/08}{Renamed from \cs{datemagyar};
%    nolonger redefines \cs{today}.}
%    \begin{macrocode}
\@namedef{ondate\CurrentOption}{%
  \number\year.\nobreakspace\ifcase\month\or
  janu\'ar\or febru\'ar\or m\'arcius\or
  \'aprilis\or m\'ajus\or j\'unius\or
  j\'ulius\or augusztus\or szeptember\or
  okt\'ober\or november\or december\fi
    \space\ifcase\day\or
    1-j\'en\or  2-\'an\or  3-\'an\or  4-\'en\or  5-\'en\or
    6-\'an\or  7-\'en\or  8-\'an\or  9-\'en\or 10-\'en\or
   11-\'en\or 12-\'en\or 13-\'an\or 14-\'en\or 15-\'en\or
   16-\'an\or 17-\'en\or 18-\'an\or 19-\'en\or 20-\'an\or
   21-\'en\or 22-\'en\or 23-\'an\or 24-\'en\or 25-\'en\or
   26-\'an\or 27-\'en\or 28-\'an\or 29-\'en\or 30-\'an\or
   31-\'en\fi}
%    \end{macrocode}
% \end{macro}
%
% \begin{macro}{\extrasmagyar}
% \begin{macro}{\noextrasmagyar}
%    The macro |\extrasmagyar| will perform all the extra definitions
%    needed for the Hungarian language. The macro |\noextrasmagyar| is
%    used to cancel the actions of |\extrasmagyar|.
%
%    \begin{macrocode}
 \@namedef{extras\CurrentOption}{%
   \expandafter\let\expandafter\ontoday
     \csname ondate\CurrentOption\endcsname}
\@namedef{noextras\CurrentOption}{\let\ontoday\@undefined}
%    \end{macrocode}
% \end{macro}
% \end{macro}
%
%    Now we redefine some commands included into \file{latex.ltx}.
%    The original form of a command is always saved with
%    |\babel@save| and the changes are added to |\extrasmagyar|.
%    This ensures that the Hungarian version of a macro is alive
%    \emph{only} if the Hungarian language is active.
%
% \begin{macro}{\fnum@figure}
% \begin{macro}{\fnum@table}
%    In figure and table captions the order of the figure/table
%    number and |\figurename|\hspace{0pt}/|\tablename| must be changed.
%    To achieve this |\fnum@figure| and |\fnum@table| are
%    redefined and added to |\extrasmagyar|.
% \changes{magyar-1.4i}{2004/02/20}{Use \cs{nobreakspace} instead of
%    tilde}
%    \begin{macrocode}
\expandafter\addto\csname extras\CurrentOption\endcsname{%
  \babel@save\fnum@figure
  \def\fnum@figure{\thefigure.\nobreakspace\figurename}}
\expandafter\addto\csname extras\CurrentOption\endcsname{%
  \babel@save\fnum@table
  \def\fnum@table{\thetable.\nobreakspace\tablename}}
%    \end{macrocode}
% \end{macro}
% \end{macro}
%
% \begin{macro}{\@makecaption}
%    The colon in a figure/table caption must be replaced by a dot
%    by redefining |\@makecaption|.
%    \begin{macrocode}
\expandafter\addto\csname extras\CurrentOption\endcsname{%
  \babel@save\@makecaption
  \def\@makecaption#1#2{%
    \vskip\abovecaptionskip
    \sbox\@tempboxa{#1. #2}%
    \ifdim \wd\@tempboxa >\hsize
      {#1. #2\csname par\endcsname}
    \else
      \global \@minipagefalse
      \hb@xt@\hsize{\hfil\box\@tempboxa\hfil}%
    \fi
    \vskip\belowcaptionskip}}
%    \end{macrocode}
% \end{macro}
%
% \begin{macro}{\@caption}
%    There should be a dot after the figure/table number in lof/lot,
%    so |\@caption| is redefined.
%    \begin{macrocode}
\expandafter\addto\csname extras\CurrentOption\endcsname{%
  \babel@save\@caption
  \long\def\@caption#1[#2]#3{%
    \csname par\endcsname
    \addcontentsline{\csname ext@#1\endcsname}{#1}%
      {\protect\numberline{\csname the#1\endcsname.}{\ignorespaces #2}}%
    \begingroup
      \@parboxrestore
      \if@minipage
        \@setminipage
      \fi
      \normalsize
      \@makecaption{\csname fnum@#1\endcsname}%
          {\ignorespaces #3}\csname par\endcsname
    \endgroup}}
%    \end{macrocode}
% \end{macro}
%
% \begin{macro}{\@seccntformat}
%    In order to have a dot after the section number
%    |\@seccntformat| is redefined.
%    \begin{macrocode}
\expandafter\addto\csname extras\CurrentOption\endcsname{%
  \babel@save\@seccntformat
  \def\@seccntformat#1{\csname the#1\endcsname.\quad}}
%    \end{macrocode}
% \end{macro}
%
% \begin{macro}{\@sect}
%    Alas, |\@sect| must also be redefined to have that dot in toc too.
%    On the other hand, we include a dot after a run-in head.
%    \begin{macrocode}
\expandafter\addto\csname extras\CurrentOption\endcsname{%
  \babel@save\@sect
  \def\@sect#1#2#3#4#5#6[#7]#8{%
    \ifnum #2>\c@secnumdepth
      \let\@svsec\@empty
    \else
      \refstepcounter{#1}%
      \protected@edef\@svsec{\@seccntformat{#1}\relax}%
    \fi
    \@tempskipa #5\relax
    \ifdim \@tempskipa>\z@
      \begingroup
        #6{%
          \@hangfrom{\hskip #3\relax\@svsec}%
            \interlinepenalty \@M #8\@@par}%
      \endgroup
      \csname #1mark\endcsname{#7}%
      \addcontentsline{toc}{#1}{%
        \ifnum #2>\c@secnumdepth \else
          \protect\numberline{\csname the#1\endcsname.}%
        \fi
        #7}%
    \else
      \def\@svsechd{%
        #6{\hskip #3\relax
        \@svsec #8.}%
        \csname #1mark\endcsname{#7}%
        \addcontentsline{toc}{#1}{%
          \ifnum #2>\c@secnumdepth \else
            \protect\numberline{\csname the#1\endcsname.}%
          \fi
          #7}}%
    \fi
    \@xsect{#5}}}
%    \end{macrocode}
% \end{macro}
%
% \begin{macro}{\@ssect}
%    In order to have that dot after a run-in head when the star form of the
%    sectioning commands is used, we have to redefine |\@ssect|.
%    \begin{macrocode}
\expandafter\addto\csname extras\CurrentOption\endcsname{%
  \babel@save\@ssect
  \def\@ssect#1#2#3#4#5{%
    \@tempskipa #3\relax
    \ifdim \@tempskipa>\z@
      \begingroup
        #4{%
          \@hangfrom{\hskip #1}%
            \interlinepenalty \@M #5\@@par}%
      \endgroup
    \else
      \def\@svsechd{#4{\hskip #1\relax #5.}}%
    \fi
    \@xsect{#3}}}
%    \end{macrocode}
% \end{macro}
%
% \begin{macro}{\@begintheorem}
% \begin{macro}{\@opargbegintheorem}
%    Order changing and dot insertion in theorem by redefining
%    |\@begintheorem| and |\@opargbegintheorem|.
%    \begin{macrocode}
\expandafter\addto\csname extras\CurrentOption\endcsname{%
  \babel@save\@begintheorem
  \def\@begintheorem#1#2{\trivlist
    \item[\hskip \labelsep{\bfseries #2.\ #1.}]\itshape}%
  \babel@save\@opargbegintheorem
  \def\@opargbegintheorem#1#2#3{\trivlist
    \item[\hskip \labelsep{\bfseries #2.\ #1\ (#3).}]\itshape}}
%    \end{macrocode}
% \end{macro}
% \end{macro}
%
%    The next step is to redefine some macros included into the
%    class files. It is determined which class file is loaded
%    then the original form of the macro is saved and the changes
%    are added to |\extrasmagyar|.
%
%    First we check if the \file{book.cls} is loaded.
%    \begin{macrocode}
\@ifclassloaded{book}{%
%    \end{macrocode}
% \begin{macro}{\ps@headings}
%    The look of the headings is changed: we have to insert some dots
%    and change the order of chapter number and |\chaptername|.
%    \begin{macrocode}
  \expandafter\addto\csname extras\CurrentOption\endcsname{%
    \babel@save\ps@headings}
  \expandafter\addto\csname extras\CurrentOption\endcsname{%
    \if@twoside
      \def\ps@headings{%
          \let\@oddfoot\@empty\let\@evenfoot\@empty
          \def\@evenhead{\thepage\hfil\slshape\leftmark}%
          \def\@oddhead{{\slshape\rightmark}\hfil\thepage}%
          \let\@mkboth\markboth
        \def\chaptermark##1{%
          \markboth {\MakeUppercase{%
            \ifnum \c@secnumdepth >\m@ne
              \if@mainmatter
                \thechapter. \@chapapp. \ %
              \fi
            \fi
            ##1}}{}}%
        \def\sectionmark##1{%
          \markright {\MakeUppercase{%
            \ifnum \c@secnumdepth >\z@
              \thesection. \ %
            \fi
            ##1}}}}%
    \else
      \def\ps@headings{%
        \let\@oddfoot\@empty
        \def\@oddhead{{\slshape\rightmark}\hfil\thepage}%
        \let\@mkboth\markboth
        \def\chaptermark##1{%
          \markright {\MakeUppercase{%
            \ifnum \c@secnumdepth >\m@ne
              \if@mainmatter
                \thechapter. \@chapapp. \ %
              \fi
            \fi
            ##1}}}}%
    \fi}
%    \end{macrocode}
% \end{macro}
% \begin{macro}{\@part}
%    At the beginning of a part we need \mbox{eg.} `I.\ r\'esz'
%    instead of `Part I' (in toc too).
%    To achieve this |\@part| is redefined.
%    \begin{macrocode}
  \expandafter\addto\csname extras\CurrentOption\endcsname{%
    \babel@save\@part
    \def\@part[#1]#2{%
        \ifnum \c@secnumdepth >-2\relax
          \refstepcounter{part}%
          \addcontentsline{toc}{part}{\thepart.\hspace{1em}#1}%
        \else
          \addcontentsline{toc}{part}{#1}%
        \fi
        \markboth{}{}%
        {\centering
         \interlinepenalty \@M
         \normalfont
         \ifnum \c@secnumdepth >-2\relax
           \huge\bfseries \thepart.\nobreakspace\partname
           \csname par\endcsname
           \vskip 20\p@
         \fi
         \Huge \bfseries #2\csname par\endcsname}%
        \@endpart}}
%    \end{macrocode}
% \end{macro}
% \begin{macro}{\@chapter}
%    The same changes are made to chapter.
%    First the screen typeout and the toc are changed by
%    redefining |\@chapter|.
%    \begin{macrocode}
  \expandafter\addto\csname extras\CurrentOption\endcsname{%
    \babel@save\@chapter
    \def\@chapter[#1]#2{\ifnum \c@secnumdepth >\m@ne
                           \if@mainmatter
                             \refstepcounter{chapter}%
                             \typeout{\thechapter.\space\@chapapp.}%
                             \addcontentsline{toc}{chapter}%
                                       {\protect\numberline{\thechapter.}#1}%
                           \else
                             \addcontentsline{toc}{chapter}{#1}%
                           \fi
                        \else
                          \addcontentsline{toc}{chapter}{#1}%
                        \fi
                        \chaptermark{#1}%
                        \addtocontents{lof}{\protect\addvspace{10\p@}}%
                        \addtocontents{lot}{\protect\addvspace{10\p@}}%
                        \if@twocolumn
                          \@topnewpage[\@makechapterhead{#2}]%
                        \else
                          \@makechapterhead{#2}%
                          \@afterheading
                        \fi}}
%    \end{macrocode}
% \end{macro}
% \begin{macro}{\@makechapterhead}
%    Then the look of the chapter-start is modified by redefining
%    |\@makechapterhead|.
%    \begin{macrocode}
  \expandafter\addto\csname extras\CurrentOption\endcsname{%
    \babel@save\@makechapterhead
    \def\@makechapterhead#1{%
      \vspace*{50\p@}%
      {\parindent \z@ \raggedright \normalfont
        \ifnum \c@secnumdepth >\m@ne
          \if@mainmatter
            \huge\bfseries \thechapter.\nobreakspace\@chapapp{}
            \csname par\endcsname\nobreak
            \vskip 20\p@
          \fi
        \fi
        \interlinepenalty\@M
        \Huge \bfseries #1\csname par\endcsname\nobreak
        \vskip 40\p@
      }}}%
%    \end{macrocode}
% \end{macro}
%    This the end of the book class modification.
%    \begin{macrocode}
}{}
%    \end{macrocode}
%
%    Now we check if \file{report.cls} is loaded.
%    \begin{macrocode}
\@ifclassloaded{report}{%
%    \end{macrocode}
% \begin{macro}{\ps@headings}
%    First the headings are modified just in case of the book class.
%    \begin{macrocode}
  \expandafter\addto\csname extras\CurrentOption\endcsname{%
    \babel@save\ps@headings}
  \expandafter\addto\csname extras\CurrentOption\endcsname{%
    \if@twoside
      \def\ps@headings{%
          \let\@oddfoot\@empty\let\@evenfoot\@empty
          \def\@evenhead{\thepage\hfil\slshape\leftmark}%
          \def\@oddhead{{\slshape\rightmark}\hfil\thepage}%
          \let\@mkboth\markboth
        \def\chaptermark##1{%
          \markboth {\MakeUppercase{%
            \ifnum \c@secnumdepth >\m@ne
                \thechapter. \@chapapp. \ %
            \fi
            ##1}}{}}%
        \def\sectionmark##1{%
          \markright {\MakeUppercase{%
            \ifnum \c@secnumdepth >\z@
              \thesection. \ %
            \fi
            ##1}}}}%
    \else
      \def\ps@headings{%
        \let\@oddfoot\@empty
        \def\@oddhead{{\slshape\rightmark}\hfil\thepage}%
        \let\@mkboth\markboth
        \def\chaptermark##1{%
          \markright {\MakeUppercase{%
            \ifnum \c@secnumdepth >\m@ne
                \thechapter. \@chapapp. \ %
            \fi
            ##1}}}}%
    \fi}
%    \end{macrocode}
% \end{macro}
% \begin{macro}{\@chapter}
%    Chapter-start modification with |\@chapter|
%    \begin{macrocode}
  \expandafter\addto\csname extras\CurrentOption\endcsname{%
    \babel@save\@chapter
    \def\@chapter[#1]#2{\ifnum \c@secnumdepth >\m@ne
                             \refstepcounter{chapter}%
                             \typeout{\thechapter.\space\@chapapp.}%
                             \addcontentsline{toc}{chapter}%
                                       {\protect\numberline{\thechapter.}#1}%
                        \else
                          \addcontentsline{toc}{chapter}{#1}%
                        \fi
                        \chaptermark{#1}%
                        \addtocontents{lof}{\protect\addvspace{10\p@}}%
                        \addtocontents{lot}{\protect\addvspace{10\p@}}%
                        \if@twocolumn
                          \@topnewpage[\@makechapterhead{#2}]%
                        \else
                          \@makechapterhead{#2}%
                          \@afterheading
                        \fi}}
%    \end{macrocode}
% \end{macro}
% \begin{macro}{\@makechapterhead}
%    and |\@makechapterhead|.
%    \begin{macrocode}
  \expandafter\addto\csname extras\CurrentOption\endcsname{%
    \babel@save\@makechapterhead
    \def\@makechapterhead#1{%
      \vspace*{50\p@}%
      {\parindent \z@ \raggedright \normalfont
        \ifnum \c@secnumdepth >\m@ne
            \huge\bfseries \thechapter.\nobreakspace\@chapapp{}
            \csname par\endcsname\nobreak
            \vskip 20\p@
        \fi
        \interlinepenalty\@M
        \Huge \bfseries #1\csname par\endcsname\nobreak
        \vskip 40\p@
      }}}%
%    \end{macrocode}
% \end{macro}
%    End of report class modification.
%    \begin{macrocode}
}{}
%    \end{macrocode}
%
%    Checking if \file{article.cls} is loaded.
%    \begin{macrocode}
\@ifclassloaded{article}{%
%    \end{macrocode}
% \begin{macro}{\ps@headings}
%    Changing headings by redefining |\ps@headings|.
%    \begin{macrocode}
  \expandafter\addto\csname extras\CurrentOption\endcsname{%
    \babel@save\ps@headings}
  \expandafter\addto\csname extras\CurrentOption\endcsname{%
    \if@twoside
      \def\ps@headings{%
          \let\@oddfoot\@empty\let\@evenfoot\@empty
          \def\@evenhead{\thepage\hfil\slshape\leftmark}%
          \def\@oddhead{{\slshape\rightmark}\hfil\thepage}%
          \let\@mkboth\markboth
        \def\sectionmark##1{%
          \markboth {\MakeUppercase{%
            \ifnum \c@secnumdepth >\z@
              \thesection.\quad
            \fi
            ##1}}{}}%
        \def\subsectionmark##1{%
          \markright {%
            \ifnum \c@secnumdepth >\@ne
              \thesubsection.\quad
            \fi
            ##1}}}%
    \else
      \def\ps@headings{%
        \let\@oddfoot\@empty
        \def\@oddhead{{\slshape\rightmark}\hfil\thepage}%
        \let\@mkboth\markboth
        \def\sectionmark##1{%
          \markright {\MakeUppercase{%
            \ifnum \c@secnumdepth >\m@ne
              \thesection.\quad
            \fi
            ##1}}}}%
    \fi}%
%    \end{macrocode}
% \end{macro}
%    No more necessary changes specific to the article class.
%    \begin{macrocode}
}{}
%    \end{macrocode}
%
%    And now this is the turn of \file{letter.cls}.
%    \begin{macrocode}
\@ifclassloaded{letter}{%
%    \end{macrocode}
% \begin{macro}{\ps@headings}
%    In the headings the page number must be followed by a dot
%    and then |\pagename|.
%    \begin{macrocode}
  \expandafter\addto\csname extras\CurrentOption\endcsname{%
    \babel@save\ps@headings}
  \expandafter\addto\csname extras\CurrentOption\endcsname{%
    \if@twoside
      \def\ps@headings{%
          \let\@oddfoot\@empty\let\@evenfoot\@empty
          \def\@oddhead{\slshape\headtoname{:} \ignorespaces\toname
                        \hfil \@date
                        \hfil \thepage.\nobreakspace\pagename}%
          \let\@evenhead\@oddhead}
    \else
      \def\ps@headings{%
          \let\@oddfoot\@empty
          \def\@oddhead{\slshape\headtoname{:} \ignorespaces\toname
                        \hfil \@date
                        \hfil \thepage.\nobreakspace\pagename}}
    \fi}%
%    \end{macrocode}
% \end{macro}
%    End of letter class.
%    \begin{macrocode}
}{}
%    \end{macrocode}
%
%    After making the changes to the \LaTeX{}
%    macros we define some new ones to handle the problem
%    with definite articles.
% \begin{macro}{\az}
%    |\az| is a user-level command which decides if the next
%    character is a star. |\@az| is called for |\az*| and
%    |\az@| for |\az|.
%    \begin{macrocode}
\def\az{a\@ifstar{\@az}{\az@}}
%    \end{macrocode}
% \end{macro}
% \begin{macro}{\Az}
%    |\Az| is used at the beginning of a sentence. Otherwise it behaves
%    the same as |\az|.
%    \begin{macrocode}
\def\Az{A\@ifstar{\@az}{\az@}}
%    \end{macrocode}
% \end{macro}
% \begin{macro}{\az@}
%    |\az@| is called if there is no star after |\az| or |\Az|.
%    It calls |\@az| and writes |#1| separating with a non-breakable
%    space.
%    \begin{macrocode}
\def\az@#1{\@az{#1}\nobreakspace#1}
%    \end{macrocode}
% \end{macro}
% \begin{macro}{\@az}
%    This macro calls |\hun@tempadef| to remove the accents from
%    the argument then calls |\@@az| that determines if a `z'
%    should be written after a/A (written by |\az|/|\Az|).
%    \begin{macrocode}
\def\@az#1{%
  \hun@tempadef{relax}{relax}{#1}%
  \edef\@tempb{\noexpand\@@az\@tempa\hbox!}%
  \@tempb}
%    \end{macrocode}
% \end{macro}
%
% \begin{macro}{\hun@tempadef}
%    The macro |\hun@tempadef| has three tasks:
%    \begin{itemize}
%      \item Accent removal. Accented letters confuse
%            |\@@az|, the main definite article determinator
%            macro, so they must be changed to their non-accented
%            counterparts. Special letters must also be changed,
%            \mbox{eg.} \oe$\,\rightarrow\,$o.
%      \item Labels must be expanded.
%      \item To handle Roman numerals correctly, commands starting
%            with |\hun@| are defined for labels containing Roman
%            numbers with the Roman numerals
%            replaced by their Arabic representation.
%            This macro can check if there is a |\hun@| command.
%    \end{itemize}
%    There are three arguments:
%    \begin{enumerate}
%      \item The primary command that should be expanded if it exists.
%            This is usually the |\hun@| command for a label.
%      \item The secondary command which is used if the first one
%            is |\relax|. This is usually the original \LaTeX{}
%            command for a label.
%      \item This is used if the first two is |\relax|. For this one
%            no expansion is carried out but the accents are still
%            removed and special letters are changed.
%    \end{enumerate}
% \changes{magyar-1.4f}{2003/09/29}{Added
%    \cs{def}\cs{safe@activesfalse{}} as a fix for PR3426} 
%    \begin{macrocode}
\def\hun@tempadef#1#2#3{%
  \begingroup
    \def\@safe@activesfalse{}%
    \def\setbox ##1{}% to get rid of accents and special letters
    \def\hbox ##1{}%
    \def\accent ##1 ##2{##2}%
    \def\add@accent ##1##2{##2}%
    \def\@text@composite@x ##1##2{##2}%
    \def\i{i}\def\j{j}%
    \def\ae{a}\def\AE{A}\def\oe{o}\def\OE{O}%
    \def\ss{s}\def\L{L}%
    \def\d{}\def\b{}\def\c{}\def\t{}%
    \expandafter\ifx\csname #1\endcsname\relax
      \expandafter\ifx\csname #2\endcsname\relax
        \xdef\@tempa{#3}%
      \else
        \xdef\@tempa{\csname #2\endcsname}%
      \fi
    \else
      \xdef\@tempa{\csname #1\endcsname}%
    \fi
  \endgroup}
%    \end{macrocode}
% \end{macro}
%
%    The following macros are used to determine the definite
%    article for a label's expansion.
% \begin{macro}{\aref}
%    |\aref| is an alias for |\azr|.
%    \begin{macrocode}
\def\aref{\azr}
%    \end{macrocode}
% \end{macro}
% \begin{macro}{\Aref}
%    |\Aref| is an alias for |\Azr|.
%    \begin{macrocode}
\def\Aref{\Azr}
%    \end{macrocode}
% \end{macro}
% \begin{macro}{\azr}
%    |\azr| calls |\@azr| if the next character is a star,
%    otherwise it calls |\azr@|.
%    \begin{macrocode}
\def\azr{a\@ifstar{\@azr}{\azr@}}
%    \end{macrocode}
% \end{macro}
% \begin{macro}{\Azr}
%    |\Azr| is the same as |\azr| except that it writes `A'
%    instead of `a'.
%    \begin{macrocode}
\def\Azr{A\@ifstar{\@azr}{\azr@}}
%    \end{macrocode}
% \end{macro}
% \begin{macro}{\azr@}
%    |\azr@| decides if the next character is |(| and
%    in that case it calls |\azr@@@| which writes an extra |(|
%    for equation referencing. Otherwise |\azr@@| is called.
%    \begin{macrocode}
\def\azr@{\@ifnextchar ({\azr@@@}{\azr@@}}
%    \end{macrocode}
% \end{macro}
% \begin{macro}{\azr@@}
%    Calls |\@azr| then writes the label's expansion preceded by
%    a non-breakable space.
%    \begin{macrocode}
\def\azr@@#1{\@azr{#1}\nobreakspace\ref{#1}}
%    \end{macrocode}
% \end{macro}
% \begin{macro}{\azr@@@}
%    Same as |\azr@@| but inserts a |(| between the
%    non-breakable space and the label expansion.
%    \begin{macrocode}
\def\azr@@@(#1{\@azr{#1}\nobreakspace(\ref{#1}}
%    \end{macrocode}
% \end{macro}
% \begin{macro}{\@azr}
%    Calls |\hun@tempadef| to choose between the label's
%    |\hun@| or original \LaTeX{} command and to expand
%    it with accent removal and special letter substitution.
%    Then calls |\@@az|, the core macro of definite article handling.
%    \begin{macrocode}
\def\@azr#1{%
  \hun@tempadef{hun@r@#1}{r@#1}{}%
  \ifx\@tempa\empty
  \else
    \edef\@tempb{\noexpand\@@az\expandafter\@firstoftwo\@tempa\hbox!}%
    \@tempb
  \fi
}
%    \end{macrocode}
% \end{macro}
%
%    The following commands are used to generate the definite article
%    for the page number of a label.
% \begin{macro}{\apageref}
%    |\apageref| is an alias for |\azp|.
%    \begin{macrocode}
\def\apageref{\azp}
%    \end{macrocode}
% \end{macro}
% \begin{macro}{\Apageref}
%    |\Apageref| is an alias for |\Azp|.
%    \begin{macrocode}
\def\Apageref{\Azp}
%    \end{macrocode}
% \end{macro}
% \begin{macro}{\azp}
%    Checks if the next character is |*| and calls
%    |\@azp| or |\azp@|.
%    \begin{macrocode}
\def\azp{a\@ifstar{\@azp}{\azp@}}
%    \end{macrocode}
% \end{macro}
% \begin{macro}{\Azp}
%    Same as |\azp| except that it writes `A' instead of `a'.
%    \begin{macrocode}
\def\Azp{A\@ifstar{\@azp}{\azp@}}
%    \end{macrocode}
% \end{macro}
% \begin{macro}{\azp@}
%    Calls |\@azp| then writes the label's page preceded by
%    a non-breakable space.
%    \begin{macrocode}
\def\azp@#1{\@azp{#1}\nobreakspace\pageref{#1}}
%    \end{macrocode}
% \end{macro}
% \begin{macro}{\@azp}
%    Calls |\hun@tempadef| then takes the label's page and passes
%    it to |\@@az|.
%    \begin{macrocode}
\def\@azp#1{%
  \hun@tempadef{hun@r@#1}{r@#1}{}%
  \ifx\@tempa\empty
  \else
    \edef\@tempb{\noexpand\@@az\expandafter\@secondoftwo\@tempa\hbox!}%
    \@tempb
  \fi
}
%    \end{macrocode}
% \end{macro}
%
%    The following macros are used to give the definite article to
%    citations.
% \begin{macro}{\acite}
%    This is an alias for |\azc|.
%    \begin{macrocode}
\def\acite{\azc}
%    \end{macrocode}
% \end{macro}
% \begin{macro}{\Acite}
%    This is an alias for |\Azc|.
%    \begin{macrocode}
\def\Acite{\Azc}
%    \end{macrocode}
% \end{macro}
% \begin{macro}{\azc}
%    Checks if the next character is a star and
%    calls |\@azc| or |\azc@|.
%    \begin{macrocode}
\def\azc{a\@ifstar{\@azc}{\azc@}}
%    \end{macrocode}
% \end{macro}
% \begin{macro}{\Azc}
%    Same as |\azc| but used at the beginning of sentences.
%    \begin{macrocode}
\def\Azc{A\@ifstar{\@azc}{\azc@}}
%    \end{macrocode}
% \end{macro}
% \begin{macro}{\azc@}
%    If there is no star we accept an optional argument,
%    just like the |\cite| command.
%    \begin{macrocode}
\def\azc@{\@ifnextchar [{\azc@@}{\azc@@[]}}
%    \end{macrocode}
% \end{macro}
% \begin{macro}{\azc@@}
%    First calls |\@azc| then |\cite|.
%    \begin{macrocode}
\def\azc@@[#1]#2{%
  \@azc{#2}\nobreakspace\def\@tempa{#1}%
    \ifx\@tempa\@empty\cite{#2}\else\cite[#1]{#2}\fi}
%    \end{macrocode}
% \end{macro}
% \begin{macro}{\@azc}
%    This is an auxiliary macro to get the first cite label
%    from a comma-separated list.
%    \begin{macrocode}
\def\@azc#1{\@@azc#1,\hbox!}
%    \end{macrocode}
% \end{macro}
% \begin{macro}{\@@azc}
%    This one uses only the first argument, that is the first
%    element of the comma-separated list of cite labels.
%    Calls |\hun@tempadef| to expand the cite label with accent
%    removal and special letter replacement.
%    Then |\@@az|, the core macro, is called.
%    \begin{macrocode}
\def\@@azc#1,#2\hbox#3!{%
  \hun@tempadef{hun@b@#1}{b@#1}{}%
  \ifx\@tempa\empty
  \else
    \edef\@tempb{\noexpand\@@az\@tempa\hbox!}%
    \@tempb
  \fi}
%    \end{macrocode}
% \end{macro}
%
% \begin{macro}{\hun@number@lehgth}
%    This macro is used to count the number of digits in its
%    argument until a non-digit character is found or
%    the end of the argument is reached.
%    It must be called as
%    |\hun@number@lehgth|\textit{arg}|\hbox\hbox!| and
%    |\count@| must be zeroed.
%    It is called by |\@@az|.
%    \begin{macrocode}
\def\hun@number@lehgth#1#2\hbox#3!{%
  \ifcat\noexpand#11%
    \ifnum\expandafter`\csname#1\endcsname>47
      \ifnum\expandafter`\csname#1\endcsname<58
        \advance\count@ by \@ne
        \hun@number@lehgth#2\hbox\hbox!\fi\fi\fi}
%    \end{macrocode}
% \end{macro}
%
% \begin{macro}{\hun@alph@lehgth}
%    This is used to count the number of letters until a
%    non-letter is found or the end of the argument is reached.
%    It must be called as
%    |\hun@alph@lehgth|\textit{arg}|\hbox\hbox!| and
%    |\count@| must be set to zero.
%    It is called by |\@@az@string|.
%    \begin{macrocode}
\def\hun@alph@lehgth#1#2\hbox#3!{%
  \ifcat\noexpand#1A%
    \advance\count@ by \@ne
    \hun@alph@lehgth#2\hbox\hbox!\fi}
%    \end{macrocode}
% \end{macro}
%
% \begin{macro}{\@@az@string}
%    This macro is called by |\@@az| if the argument begins
%    with a letter.
%    The task of |\@@az@string| is to determine if the argument
%    starts with a vowel and in that case |\let\@tempa\@tempb|.
%    After checking if the first letter is A, E, I, O, or U,
%    |\hun@alph@lehgth| is called
%    to determine the length of the argument. If it gives 1
%    (that is the argument is a single-letter one or the second
%    character is not letter) then the letters L, M, N, R, S, X, and Y
%    are also considered as a vowel since their Hungarian pronounced
%    name starts with a vowel.
%    \begin{macrocode}
\def\@@az@string#1#2{%
  \ifx#1A%
    \let\@tempa\@tempb
  \else\ifx#1E%
    \let\@tempa\@tempb
  \else\ifx#1I%
    \let\@tempa\@tempb
  \else\ifx#1O%
    \let\@tempa\@tempb
  \else\ifx#1U%
    \let\@tempa\@tempb
  \fi\fi\fi\fi\fi
  \ifx\@tempa\@tempb
  \else
    \count@\z@
    \hun@alph@lehgth#1#2\hbox\hbox!%
    \ifnum\count@=\@ne
      \ifx#1F%
        \let\@tempa\@tempb
      \else\ifx#1L%
        \let\@tempa\@tempb
      \else\ifx#1M%
        \let\@tempa\@tempb
      \else\ifx#1N%
        \let\@tempa\@tempb
      \else\ifx#1R%
        \let\@tempa\@tempb
      \else\ifx#1S%
        \let\@tempa\@tempb
      \else\ifx#1X%
        \let\@tempa\@tempb
      \else\ifx#1Y%
        \let\@tempa\@tempb
      \fi\fi\fi\fi\fi\fi\fi\fi
    \fi
  \fi}
%    \end{macrocode}
% \end{macro}
%
% \begin{macro}{\@@az}
%    This macro is the core of definite article handling.
%    It determines if the argument needs `az' or `a' definite
%    article by setting |\@tempa| to `z' or |\@empty|.
%    It sets |\@tempa| to `z' if
%    \begin{itemize}
%      \item the first character of the argument is 5; or
%      \item the first character of the argument is 1 and
%            the $\mathit{length\ of\ the\ number} \pmod 3 = 1$
%            (one--egy, thousand--ezer, million--egymilli\'o,\dots); or
%      \item the first character of the argument is
%            a, A, e, E, i, I, o, O, u, or U; or
%      \item the first character of the argument is
%            l, L, m, M, n, N, r, R, s, S, x, X, y, or Y
%            and the length of the argument is 1 or the second
%            character is a non-letter.
%    \end{itemize}
%    At the end it calls |\@tempa|, that is, it either typesets a `z'
%    or nothing.
%    \begin{macrocode}
\def\@@az#1#2\hbox#3!{%
  \let\@tempa\@empty
  \def\@tempb{z}%
  \uppercase{%
    \ifx5#1%
      \let\@tempa\@tempb
    \else\ifx1#1%
      \count@\@ne
      \hun@number@lehgth#2\hbox\hbox!%
      \loop
      \ifnum\count@>\thr@@
        \advance\count@-\thr@@
      \repeat
      \ifnum\count@=\@ne
        \let\@tempa\@tempb
      \fi
    \else
      \@@az@string{#1}{#2}%
    \fi\fi
  }%
  \@tempa}
%    \end{macrocode}
% \end{macro}
%
% \begin{macro}{\refstepcounter}
%    |\refstepcounter| must be redefined in order to keep
%    |\@currentlabel| unexpanded. This is necessary to enable
%    the |\label| command to write a |\hunnewlabel| command
%    to the aux file with the Roman numerals substituted by
%    their Arabic representations.
%    Of course, the original definition of |\refstepcounter| is
%    saved and restored if the Hungarian language is switched off.
%    \begin{macrocode}
\expandafter\addto\csname extras\CurrentOption\endcsname{%
  \babel@save\refstepcounter
  \def\refstepcounter#1{\stepcounter{#1}%
    \def\@currentlabel{\csname p@#1\endcsname\csname the#1\endcsname}}%
}
%    \end{macrocode}
% \end{macro}
%
% \begin{macro}{\label}
%    |\label| is redefined to write another line into the aux
%    file: |\hunnewlabel{ }{ }| where the Roman numerals
%    are replaced their Arabic representations.
%    The original definition of |\label| is saved into
%    |\old@label| and it is also called by |\label|.
%    On leaving the Hungarian typesetting mode |\label|'s
%    original is restored since it is added to |\noextrasmagyar|.
%    \begin{macrocode}
\expandafter\addto\csname extras\CurrentOption\endcsname{%
  \let\old@label\label
  \def\label#1{\@bsphack
    \old@label{#1}%
    \begingroup
      \let\romannumeral\number
      \def\@roman##1{\number ##1}%
      \def\@Roman##1{\number ##1}%
      {\toks0={\noexpand\noexpand\noexpand\number}%
        \def\number##1{\the\toks0 ##1}\xdef\tempb@{\thepage}}%
      \edef\@tempa##1{\noexpand\protected@write\@auxout{}%
           {\noexpand\string\noexpand\hunnewlabel
           {##1}{{\@currentlabel}{\tempb@}}}}%
      \@tempa{#1}%
    \endgroup
  \@esphack}%
}
\expandafter\addto\csname noextras\CurrentOption\endcsname{%
  \let\label\old@label
}
%    \end{macrocode}
% \end{macro}
%
% \begin{macro}{\hunnewlabel}
%    Finally, |\hunnewlabel| is defined.
%    It checks if the label's expansion (|#2|) differs from that
%    one given in the |\newlabel| command. If yes
%    (that is, the label contains some Roman numerals),
%    it defines the macro |\hun@r@|\textit{label},
%    otherwise it does nothing.
%    \begin{macrocode}
\def\hunnewlabel#1#2{%
  \def\@tempa{#2}%
  \expandafter\ifx\csname r@#1\endcsname\@tempa
    \relax% \message{No need for def: #1}%
  \else
    \global\expandafter\let\csname hun@r@#1\endcsname\@tempa%
  \fi
}
%    \end{macrocode}
% \end{macro}
%
%    For Hungarian the |`| character is made active.
% \changes{magyar-1.4c}{2001/03/05}{Make sure that the grave accent
%    has catcode 12 \emph{before} it is made \cs{active}}
%    \begin{macrocode}
\AtBeginDocument{%
  \if@filesw\immediate\write\@auxout{\catcode096=12}\fi}
\initiate@active@char{`}
\expandafter\addto\csname extras\CurrentOption\endcsname{%
  \languageshorthands{magyar}%
  \bbl@activate{`}}
\expandafter\addto\csname noextras\CurrentOption\endcsname{%
  \bbl@deactivate{`}}
%    \end{macrocode}
%
%    The character sequence |``| is declared as a shorthand
%    in order to produce
%    the opening quotation sign appropriate for Hungarian.
%    \begin{macrocode}
\declare@shorthand{magyar}{``}{\glqq}
%    \end{macrocode}
%
% In Hungarian there are some long double consonants which
% must be hyphenated specially.
% For all these long double consonants (except dzzs, that is
% extremely very-very rare) a shortcut is defined.
%    \begin{macrocode}
\declare@shorthand{magyar}{`c}{\textormath{\bbl@disc{c}{cs}}{c}}
\declare@shorthand{magyar}{`C}{\textormath{\bbl@disc{C}{CS}}{C}}
\declare@shorthand{magyar}{`d}{\textormath{\bbl@disc{d}{dz}}{d}}
\declare@shorthand{magyar}{`D}{\textormath{\bbl@disc{D}{DZ}}{D}}
\declare@shorthand{magyar}{`g}{\textormath{\bbl@disc{g}{gy}}{g}}
\declare@shorthand{magyar}{`G}{\textormath{\bbl@disc{G}{GY}}{G}}
\declare@shorthand{magyar}{`l}{\textormath{\bbl@disc{l}{ly}}{l}}
\declare@shorthand{magyar}{`L}{\textormath{\bbl@disc{L}{LY}}{L}}
\declare@shorthand{magyar}{`n}{\textormath{\bbl@disc{n}{ny}}{n}}
\declare@shorthand{magyar}{`N}{\textormath{\bbl@disc{N}{NY}}{N}}
\declare@shorthand{magyar}{`s}{\textormath{\bbl@disc{s}{sz}}{s}}
\declare@shorthand{magyar}{`S}{\textormath{\bbl@disc{S}{SZ}}{S}}
\declare@shorthand{magyar}{`t}{\textormath{\bbl@disc{t}{ty}}{t}}
\declare@shorthand{magyar}{`T}{\textormath{\bbl@disc{T}{TY}}{T}}
\declare@shorthand{magyar}{`z}{\textormath{\bbl@disc{z}{zs}}{z}}
\declare@shorthand{magyar}{`Z}{\textormath{\bbl@disc{Z}{ZS}}{Z}}
%    \end{macrocode}
%
%    The macro |\ldf@finish| takes care of looking for a
%    configuration file, setting the main language to be switched on
%    at |\begin{document}| and resetting the category code of
%    \texttt{@} to its original value.
% \changes{magyar-1.3h}{1996/10/30}{Now use \cs{ldf@finish} to wrap up}
%    \begin{macrocode}
\ldf@finish\CurrentOption
%</code>
%    \end{macrocode}
%
% \Finale
%%
%% \CharacterTable
%%  {Upper-case    \A\B\C\D\E\F\G\H\I\J\K\L\M\N\O\P\Q\R\S\T\U\V\W\X\Y\Z
%%   Lower-case    \a\b\c\d\e\f\g\h\i\j\k\l\m\n\o\p\q\r\s\t\u\v\w\x\y\z
%%   Digits        \0\1\2\3\4\5\6\7\8\9
%%   Exclamation   \!     Double quote  \"     Hash (number) \#
%%   Dollar        \$     Percent       \%     Ampersand     \&
%%   Acute accent  \'     Left paren    \(     Right paren   \)
%%   Asterisk      \*     Plus          \+     Comma         \,
%%   Minus         \-     Point         \.     Solidus       \/
%%   Colon         \:     Semicolon     \;     Less than     \<
%%   Equals        \=     Greater than  \>     Question mark \?
%%   Commercial at \@     Left bracket  \[     Backslash     \\
%%   Right bracket \]     Circumflex    \^     Underscore    \_
%%   Grave accent  \`     Left brace    \{     Vertical bar  \|
%%   Right brace   \}     Tilde         \~}
%%
\endinput
}
%^^A\DeclareOption{nagari}{\input{nagari.ldf}}
%    \end{macrocode}
%    `New' German orthography, including Austrian variant:
% \changes{babel~3.6p}{1999/04/10}{Added the \Lopt{ngerman} and
%    \Lopt{naustrian} options}
% \changes{babel~3.7f}{2000/09/26}{Added the \Lopt{samin} option}
% \changes{babel~3.8c}{2004/06/12}{Added the \Lopt{newzealand} option}
%    \begin{macrocode}
\DeclareOption{naustrian}{% \iffalse meta-comment
%
% Copyright 1989-2008 Johannes L. Braams and any individual authors
% listed elsewhere in this file.  All rights reserved.
% 
% This file is part of the Babel system.
% --------------------------------------
% 
% It may be distributed and/or modified under the
% conditions of the LaTeX Project Public License, either version 1.3
% of this license or (at your option) any later version.
% The latest version of this license is in
%   http://www.latex-project.org/lppl.txt
% and version 1.3 or later is part of all distributions of LaTeX
% version 2003/12/01 or later.
% 
% This work has the LPPL maintenance status "maintained".
% 
% The Current Maintainer of this work is Johannes Braams.
% 
% The list of all files belonging to the Babel system is
% given in the file `manifest.bbl. See also `legal.bbl' for additional
% information.
% 
% The list of derived (unpacked) files belonging to the distribution
% and covered by LPPL is defined by the unpacking scripts (with
% extension .ins) which are part of the distribution.
% \fi
% \CheckSum{266}
%
% \iffalse
%    Tell the \LaTeX\ system who we are and write an entry on the
%    transcript.
%<*dtx>
\ProvidesFile{ngermanb.dtx}
%</dtx>
%<code>\ProvidesLanguage{ngermanb}
%\fi
%\ProvidesFile{ngermanb.dtx}
        [2008/03/17 v2.6m new German support from the babel system]
%\iffalse
%% File `ngermanb.dtx'
%% Babel package for LaTeX version 2e
%% Copyright (C) 1989 - 2008
%%           by Johannes Braams, TeXniek
%
%% new Germanb Language Definition File
%% Copyright (C) 1989 - 2008
%%           by Bernd Raichle raichle at azu.Informatik.Uni-Stuttgart.de
%%              Johannes Braams, TeXniek,
%%              Walter Schmidt.
% This file is based on german.tex version 2.5e
%                       by Bernd Raichle, Hubert Partl et.al.
%
%% Please report errors to: J.L. Braams
%%                          babel at braams.xs4all.nl
%
%<*filedriver>
\documentclass{ltxdoc}
\font\manual=logo10 % font used for the METAFONT logo, etc.
\newcommand*\MF{{\manual META}\-{\manual FONT}}
\newcommand*\TeXhax{\TeX hax}
\newcommand*\babel{\textsf{babel}}
\newcommand*\langvar{$\langle \it lang \rangle$}
\newcommand*\note[1]{}
\newcommand*\Lopt[1]{\textsf{#1}}
\newcommand*\file[1]{\texttt{#1}}
\begin{document}
 \DocInput{ngermanb.dtx}
\end{document}
%</filedriver>
%\fi
% \GetFileInfo{ngermanb.dtx}
%
% \changes{ngermanb-2.6f}{1999/03/24}{Renamed from \file{germanb.ldf};
%          language names changed from \texttt{german} and \texttt{austrian}
%          to \texttt{ngerman} and \texttt{naustrian}.}
%
%  \section{The German language -- new orthography}
%
%    The file \file{\filename}\footnote{The file described in this
%    section has version number \fileversion\ and was last revised on
%    \filedate.}  defines all the language definition macros for the
%    German language with the `new orthography' introduced in
%    August 1998.  This includes also the Austrian dialect of this
%    language.
%  
%    As with the `traditional'  German orthography, 
%    the character |"| is made active, and 
%    the commands in  table~\ref{tab:german-quote} can be used, except
%    for |"ck| and |"ff| etc., which are no longer required.
%
%    The internal language names are |ngerman| and |naustrian|.
%
% \StopEventually{}
%
%    When this file was read through the option \Lopt{ngermanb} we make
%    it behave as if \Lopt{ngerman} was specified.
%    \begin{macrocode}
\def\bbl@tempa{ngermanb}
\ifx\CurrentOption\bbl@tempa
  \def\CurrentOption{ngerman}
\fi
%    \end{macrocode}
%
%    The macro |\LdfInit| takes care of preventing that this file is
%    loaded more than once, checking the category code of the
%    \texttt{@} sign, etc.
%    \begin{macrocode}
%<*code>
\LdfInit\CurrentOption{captions\CurrentOption}
%    \end{macrocode}
%
%    When this file is read as an option, i.e., by the |\usepackage|
%    command, \texttt{ngerman} will be an `unknown' language, so we
%    have to make it known.  So we check for the existence of
%    |\l@ngerman| to see whether we have to do something here.
%
%    \begin{macrocode}
\ifx\l@ngerman\@undefined
  \@nopatterns{ngerman}
  \adddialect\l@ngerman0
\fi
%    \end{macrocode}
%
%    For the Austrian version of these definitions we just add another
%    language. 
%    \begin{macrocode}
\adddialect\l@naustrian\l@ngerman
%    \end{macrocode}
%
%    The next step consists of defining commands to switch to (and
%    from) the German language.
%
%  \begin{macro}{\captionsngerman}
%  \begin{macro}{\captionsnaustrian}
%    Either the macro |\captionnsgerman| or the macro
%    |\captionsnaustrian| will define all strings used in the four
%    standard document classes provided with \LaTeX.
%
% \changes{ngermanb-2.6k}{2000/09/20}{Added \cs{glossaryname}}
%    \begin{macrocode}
\@namedef{captions\CurrentOption}{%
  \def\prefacename{Vorwort}%
  \def\refname{Literatur}%
  \def\abstractname{Zusammenfassung}%
  \def\bibname{Literaturverzeichnis}%
  \def\chaptername{Kapitel}%
  \def\appendixname{Anhang}%
  \def\contentsname{Inhaltsverzeichnis}%    % oder nur: Inhalt
  \def\listfigurename{Abbildungsverzeichnis}%
  \def\listtablename{Tabellenverzeichnis}%
  \def\indexname{Index}%
  \def\figurename{Abbildung}%
  \def\tablename{Tabelle}%                  % oder: Tafel
  \def\partname{Teil}%
  \def\enclname{Anlage(n)}%                 % oder: Beilage(n)
  \def\ccname{Verteiler}%                   % oder: Kopien an
  \def\headtoname{An}%
  \def\pagename{Seite}%
  \def\seename{siehe}%
  \def\alsoname{siehe auch}%
  \def\proofname{Beweis}%
  \def\glossaryname{Glossar}%
  }
%    \end{macrocode}
%  \end{macro}
%  \end{macro}
%
%  \begin{macro}{\datengerman}
%    The macro |\datengerman| redefines the command
%    |\today| to produce German dates.
%    \begin{macrocode}
\def\month@ngerman{\ifcase\month\or
  Januar\or Februar\or M\"arz\or April\or Mai\or Juni\or
  Juli\or August\or September\or Oktober\or November\or Dezember\fi}
\def\datengerman{\def\today{\number\day.~\month@ngerman
    \space\number\year}}
%    \end{macrocode}
%  \end{macro}
%
%  \begin{macro}{\dateanustrian}
%    The macro |\datenaustrian| redefines the command
%    |\today| to produce Austrian version of the German dates.
%    \begin{macrocode}
\def\datenaustrian{\def\today{\number\day.~\ifnum1=\month
  J\"anner\else \month@ngerman\fi \space\number\year}}
%    \end{macrocode}
%  \end{macro}
%
%  \begin{macro}{\extrasngerman}
%  \begin{macro}{\extrasnaustrian}
%  \begin{macro}{\noextrasngerman}
%  \begin{macro}{\noextrasnaustrian}
%    Either the macro |\extrasngerman| or the macros |\extrasnaustrian|
%    will perform all the extra definitions needed for the German
%    language. The macro |\noextrasngerman| is used to cancel the
%    actions of |\extrasngerman|. 
%
%    For German (as well as for Dutch) the \texttt{"} character is
%    made active. This is done once, later on its definition may vary.
%    \begin{macrocode}
\initiate@active@char{"}
\@namedef{extras\CurrentOption}{%
  \languageshorthands{ngerman}}
\expandafter\addto\csname extras\CurrentOption\endcsname{%
  \bbl@activate{"}}
%    \end{macrocode}
%    Don't forget to turn the shorthands off again.
% \changes{ngermanb-2.6j}{1999/12/16}{Deactivate shorthands ouside of
%    German}
%    \begin{macrocode}
\addto\noextrasngerman{\bbl@deactivate{"}}
%    \end{macrocode}
%
%
%    In order for \TeX\ to be able to hyphenate German words which
%    contain `\ss' (in the \texttt{OT1} position |^^Y|) we have to
%    give the character a nonzero |\lccode| (see Appendix H, the \TeX
%    book).
%    \begin{macrocode}
\expandafter\addto\csname extras\CurrentOption\endcsname{%
  \babel@savevariable{\lccode25}%
  \lccode25=25}
%    \end{macrocode}
%
%    The umlaut accent macro |\"| is changed to lower the umlaut dots.
%    The redefinition is done with the help of |\umlautlow|.
%    \begin{macrocode}
\expandafter\addto\csname extras\CurrentOption\endcsname{%
  \babel@save\"\umlautlow}
\@namedef{noextras\CurrentOption}{\umlauthigh}
%    \end{macrocode}
%    The current 
%    version of the `new' German hyphenation patterns (\file{dehyphn.tex}
%    is to be used with |\lefthyphenmin| and |\righthyphenmin| set to~2. 
% \changes{ngermanb-2.6k}{2000/09/22}{Now use \cs{providehyphenmins} to
%    provide a default value}
%    \begin{macrocode}
\providehyphenmins{\CurrentOption}{\tw@\tw@}
%    \end{macrocode}
%    For German texts we need to make sure that |\frenchspacing| is
%    turned on.
% \changes{ngermanb-2.6m}{2001/01/26}{Turn frenchspacing on, as in
%    \texttt{german.sty}}
%    \begin{macrocode}
\expandafter\addto\csname extras\CurrentOption\endcsname{%
  \bbl@frenchspacing}
\expandafter\addto\csname noextras\CurrentOption\endcsname{%
  \bbl@nonfrenchspacing}
%    \end{macrocode}
%  \end{macro}
%  \end{macro}
%  \end{macro}
%  \end{macro}
%
%    The code above is necessary because we need an extra active
%    character. This character is then used as indicated in
%    table~\ref{tab:german-quote}.
%
%    To be able to define the function of |"|, we first define a
%    couple of `support' macros.
%
%
%  \begin{macro}{\dq}
%    We save the original double quote character in |\dq| to keep
%    it available, the math accent |\"| can now be typed as |"|.
%    \begin{macrocode}
\begingroup \catcode`\"12
\def\x{\endgroup
  \def\@SS{\mathchar"7019 }
  \def\dq{"}}
\x
%    \end{macrocode}
%  \end{macro}
%
%    Now we can define the doublequote macros: the umlauts,
%    \begin{macrocode}
\declare@shorthand{ngerman}{"a}{\textormath{\"{a}\allowhyphens}{\ddot a}}
\declare@shorthand{ngerman}{"o}{\textormath{\"{o}\allowhyphens}{\ddot o}}
\declare@shorthand{ngerman}{"u}{\textormath{\"{u}\allowhyphens}{\ddot u}}
\declare@shorthand{ngerman}{"A}{\textormath{\"{A}\allowhyphens}{\ddot A}}
\declare@shorthand{ngerman}{"O}{\textormath{\"{O}\allowhyphens}{\ddot O}}
\declare@shorthand{ngerman}{"U}{\textormath{\"{U}\allowhyphens}{\ddot U}}
%    \end{macrocode}
%    tremas,
%    \begin{macrocode}
\declare@shorthand{ngerman}{"e}{\textormath{\"{e}}{\ddot e}}
\declare@shorthand{ngerman}{"E}{\textormath{\"{E}}{\ddot E}}
\declare@shorthand{ngerman}{"i}{\textormath{\"{\i}}%
                              {\ddot\imath}}
\declare@shorthand{ngerman}{"I}{\textormath{\"{I}}{\ddot I}}
%    \end{macrocode}
%    german es-zet (sharp s),
%    \begin{macrocode}
\declare@shorthand{ngerman}{"s}{\textormath{\ss}{\@SS{}}}
\declare@shorthand{ngerman}{"S}{\SS}
\declare@shorthand{ngerman}{"z}{\textormath{\ss}{\@SS{}}}
\declare@shorthand{ngerman}{"Z}{SZ}
%    \end{macrocode}
%    german and french quotes,
%    \begin{macrocode}
\declare@shorthand{ngerman}{"`}{\glqq}
\declare@shorthand{ngerman}{"'}{\grqq}
\declare@shorthand{ngerman}{"<}{\flqq}
\declare@shorthand{ngerman}{">}{\frqq}
%    \end{macrocode}
%    and some additional commands:
%    \begin{macrocode}
\declare@shorthand{ngerman}{"-}{\nobreak\-\bbl@allowhyphens}
\declare@shorthand{ngerman}{"|}{%
  \textormath{\penalty\@M\discretionary{-}{}{\kern.03em}%
              \allowhyphens}{}}
\declare@shorthand{ngerman}{""}{\hskip\z@skip}
\declare@shorthand{ngerman}{"~}{\textormath{\leavevmode\hbox{-}}{-}}
\declare@shorthand{ngerman}{"=}{\penalty\@M-\hskip\z@skip}
%    \end{macrocode}
%
%  \begin{macro}{\mdqon}
%  \begin{macro}{\mdqoff}
%    All that's left to do now is to  define a couple of commands
%    for reasons of compatibility with \file{german.sty}.
%    \begin{macrocode}
\def\mdqon{\shorthandon{"}}
\def\mdqoff{\shorthandoff{"}}
%    \end{macrocode}
%  \end{macro}
%  \end{macro}
%
%    The macro |\ldf@finish| takes care of looking for a
%    configuration file, setting the main language to be switched on
%    at |\begin{document}| and resetting the category code of
%    \texttt{@} to its original value.
%    \begin{macrocode}
\ldf@finish\CurrentOption
%</code>
%    \end{macrocode}
%
% \Finale
%%
%% \CharacterTable
%%  {Upper-case    \A\B\C\D\E\F\G\H\I\J\K\L\M\N\O\P\Q\R\S\T\U\V\W\X\Y\Z
%%   Lower-case    \a\b\c\d\e\f\g\h\i\j\k\l\m\n\o\p\q\r\s\t\u\v\w\x\y\z
%%   Digits        \0\1\2\3\4\5\6\7\8\9
%%   Exclamation   \!     Double quote  \"     Hash (number) \#
%%   Dollar        \$     Percent       \%     Ampersand     \&
%%   Acute accent  \'     Left paren    \(     Right paren   \)
%%   Asterisk      \*     Plus          \+     Comma         \,
%%   Minus         \-     Point         \.     Solidus       \/
%%   Colon         \:     Semicolon     \;     Less than     \<
%%   Equals        \=     Greater than  \>     Question mark \?
%%   Commercial at \@     Left bracket  \[     Backslash     \\
%%   Right bracket \]     Circumflex    \^     Underscore    \_
%%   Grave accent  \`     Left brace    \{     Vertical bar  \|
%%   Right brace   \}     Tilde         \~}
%%
\endinput
}
\DeclareOption{newzealand}{%%
%% This file will generate fast loadable files and documentation
%% driver files from the doc files in this package when run through
%% LaTeX or TeX.
%%
%% Copyright 1989-2005 Johannes L. Braams and any individual authors
%% listed elsewhere in this file.  All rights reserved.
%% 
%% This file is part of the Babel system.
%% --------------------------------------
%% 
%% It may be distributed and/or modified under the
%% conditions of the LaTeX Project Public License, either version 1.3
%% of this license or (at your option) any later version.
%% The latest version of this license is in
%%   http://www.latex-project.org/lppl.txt
%% and version 1.3 or later is part of all distributions of LaTeX
%% version 2003/12/01 or later.
%% 
%% This work has the LPPL maintenance status "maintained".
%% 
%% The Current Maintainer of this work is Johannes Braams.
%% 
%% The list of all files belonging to the LaTeX base distribution is
%% given in the file `manifest.bbl. See also `legal.bbl' for additional
%% information.
%% 
%% The list of derived (unpacked) files belonging to the distribution
%% and covered by LPPL is defined by the unpacking scripts (with
%% extension .ins) which are part of the distribution.
%%
%% --------------- start of docstrip commands ------------------
%%
\def\filedate{1999/04/11}
\def\batchfile{english.ins}
\input docstrip.tex

{\ifx\generate\undefined
\Msg{**********************************************}
\Msg{*}
\Msg{* This installation requires docstrip}
\Msg{* version 2.3c or later.}
\Msg{*}
\Msg{* An older version of docstrip has been input}
\Msg{*}
\Msg{**********************************************}
\errhelp{Move or rename old docstrip.tex.}
\errmessage{Old docstrip in input path}
\batchmode
\csname @@end\endcsname
\fi}

\declarepreamble\mainpreamble
This is a generated file.

Copyright 1989-2005 Johannes L. Braams and any individual authors
listed elsewhere in this file.  All rights reserved.

This file was generated from file(s) of the Babel system.
---------------------------------------------------------

It may be distributed and/or modified under the
conditions of the LaTeX Project Public License, either version 1.3
of this license or (at your option) any later version.
The latest version of this license is in
  http://www.latex-project.org/lppl.txt
and version 1.3 or later is part of all distributions of LaTeX
version 2003/12/01 or later.

This work has the LPPL maintenance status "maintained".

The Current Maintainer of this work is Johannes Braams.

This file may only be distributed together with a copy of the Babel
system. You may however distribute the Babel system without
such generated files.

The list of all files belonging to the Babel distribution is
given in the file `manifest.bbl'. See also `legal.bbl for additional
information.

The list of derived (unpacked) files belonging to the distribution
and covered by LPPL is defined by the unpacking scripts (with
extension .ins) which are part of the distribution.
\endpreamble

\declarepreamble\fdpreamble
This is a generated file.

Copyright 1989-2005 Johannes L. Braams and any individual authors
listed elsewhere in this file.  All rights reserved.

This file was generated from file(s) of the Babel system.
---------------------------------------------------------

It may be distributed and/or modified under the
conditions of the LaTeX Project Public License, either version 1.3
of this license or (at your option) any later version.
The latest version of this license is in
  http://www.latex-project.org/lppl.txt
and version 1.3 or later is part of all distributions of LaTeX
version 2003/12/01 or later.

This work has the LPPL maintenance status "maintained".

The Current Maintainer of this work is Johannes Braams.

This file may only be distributed together with a copy of the Babel
system. You may however distribute the Babel system without
such generated files.

The list of all files belonging to the Babel distribution is
given in the file `manifest.bbl'. See also `legal.bbl for additional
information.

In particular, permission is granted to customize the declarations in
this file to serve the needs of your installation.

However, NO PERMISSION is granted to distribute a modified version
of this file under its original name.

\endpreamble

\keepsilent

\usedir{tex/generic/babel} 

\usepreamble\mainpreamble
\generate{\file{english.ldf}{\from{english.dtx}{code}}
          }
\usepreamble\fdpreamble

\ifToplevel{
\Msg{***********************************************************}
\Msg{*}
\Msg{* To finish the installation you have to move the following}
\Msg{* files into a directory searched by TeX:}
\Msg{*}
\Msg{* \space\space All *.def, *.fd, *.ldf, *.sty}
\Msg{*}
\Msg{* To produce the documentation run the files ending with}
\Msg{* '.dtx' and `.fdd' through LaTeX.}
\Msg{*}
\Msg{* Happy TeXing}
\Msg{***********************************************************}
}
 
\endinput
}
\DeclareOption{ngerman}{% \iffalse meta-comment
%
% Copyright 1989-2008 Johannes L. Braams and any individual authors
% listed elsewhere in this file.  All rights reserved.
% 
% This file is part of the Babel system.
% --------------------------------------
% 
% It may be distributed and/or modified under the
% conditions of the LaTeX Project Public License, either version 1.3
% of this license or (at your option) any later version.
% The latest version of this license is in
%   http://www.latex-project.org/lppl.txt
% and version 1.3 or later is part of all distributions of LaTeX
% version 2003/12/01 or later.
% 
% This work has the LPPL maintenance status "maintained".
% 
% The Current Maintainer of this work is Johannes Braams.
% 
% The list of all files belonging to the Babel system is
% given in the file `manifest.bbl. See also `legal.bbl' for additional
% information.
% 
% The list of derived (unpacked) files belonging to the distribution
% and covered by LPPL is defined by the unpacking scripts (with
% extension .ins) which are part of the distribution.
% \fi
% \CheckSum{266}
%
% \iffalse
%    Tell the \LaTeX\ system who we are and write an entry on the
%    transcript.
%<*dtx>
\ProvidesFile{ngermanb.dtx}
%</dtx>
%<code>\ProvidesLanguage{ngermanb}
%\fi
%\ProvidesFile{ngermanb.dtx}
        [2008/03/17 v2.6m new German support from the babel system]
%\iffalse
%% File `ngermanb.dtx'
%% Babel package for LaTeX version 2e
%% Copyright (C) 1989 - 2008
%%           by Johannes Braams, TeXniek
%
%% new Germanb Language Definition File
%% Copyright (C) 1989 - 2008
%%           by Bernd Raichle raichle at azu.Informatik.Uni-Stuttgart.de
%%              Johannes Braams, TeXniek,
%%              Walter Schmidt.
% This file is based on german.tex version 2.5e
%                       by Bernd Raichle, Hubert Partl et.al.
%
%% Please report errors to: J.L. Braams
%%                          babel at braams.xs4all.nl
%
%<*filedriver>
\documentclass{ltxdoc}
\font\manual=logo10 % font used for the METAFONT logo, etc.
\newcommand*\MF{{\manual META}\-{\manual FONT}}
\newcommand*\TeXhax{\TeX hax}
\newcommand*\babel{\textsf{babel}}
\newcommand*\langvar{$\langle \it lang \rangle$}
\newcommand*\note[1]{}
\newcommand*\Lopt[1]{\textsf{#1}}
\newcommand*\file[1]{\texttt{#1}}
\begin{document}
 \DocInput{ngermanb.dtx}
\end{document}
%</filedriver>
%\fi
% \GetFileInfo{ngermanb.dtx}
%
% \changes{ngermanb-2.6f}{1999/03/24}{Renamed from \file{germanb.ldf};
%          language names changed from \texttt{german} and \texttt{austrian}
%          to \texttt{ngerman} and \texttt{naustrian}.}
%
%  \section{The German language -- new orthography}
%
%    The file \file{\filename}\footnote{The file described in this
%    section has version number \fileversion\ and was last revised on
%    \filedate.}  defines all the language definition macros for the
%    German language with the `new orthography' introduced in
%    August 1998.  This includes also the Austrian dialect of this
%    language.
%  
%    As with the `traditional'  German orthography, 
%    the character |"| is made active, and 
%    the commands in  table~\ref{tab:german-quote} can be used, except
%    for |"ck| and |"ff| etc., which are no longer required.
%
%    The internal language names are |ngerman| and |naustrian|.
%
% \StopEventually{}
%
%    When this file was read through the option \Lopt{ngermanb} we make
%    it behave as if \Lopt{ngerman} was specified.
%    \begin{macrocode}
\def\bbl@tempa{ngermanb}
\ifx\CurrentOption\bbl@tempa
  \def\CurrentOption{ngerman}
\fi
%    \end{macrocode}
%
%    The macro |\LdfInit| takes care of preventing that this file is
%    loaded more than once, checking the category code of the
%    \texttt{@} sign, etc.
%    \begin{macrocode}
%<*code>
\LdfInit\CurrentOption{captions\CurrentOption}
%    \end{macrocode}
%
%    When this file is read as an option, i.e., by the |\usepackage|
%    command, \texttt{ngerman} will be an `unknown' language, so we
%    have to make it known.  So we check for the existence of
%    |\l@ngerman| to see whether we have to do something here.
%
%    \begin{macrocode}
\ifx\l@ngerman\@undefined
  \@nopatterns{ngerman}
  \adddialect\l@ngerman0
\fi
%    \end{macrocode}
%
%    For the Austrian version of these definitions we just add another
%    language. 
%    \begin{macrocode}
\adddialect\l@naustrian\l@ngerman
%    \end{macrocode}
%
%    The next step consists of defining commands to switch to (and
%    from) the German language.
%
%  \begin{macro}{\captionsngerman}
%  \begin{macro}{\captionsnaustrian}
%    Either the macro |\captionnsgerman| or the macro
%    |\captionsnaustrian| will define all strings used in the four
%    standard document classes provided with \LaTeX.
%
% \changes{ngermanb-2.6k}{2000/09/20}{Added \cs{glossaryname}}
%    \begin{macrocode}
\@namedef{captions\CurrentOption}{%
  \def\prefacename{Vorwort}%
  \def\refname{Literatur}%
  \def\abstractname{Zusammenfassung}%
  \def\bibname{Literaturverzeichnis}%
  \def\chaptername{Kapitel}%
  \def\appendixname{Anhang}%
  \def\contentsname{Inhaltsverzeichnis}%    % oder nur: Inhalt
  \def\listfigurename{Abbildungsverzeichnis}%
  \def\listtablename{Tabellenverzeichnis}%
  \def\indexname{Index}%
  \def\figurename{Abbildung}%
  \def\tablename{Tabelle}%                  % oder: Tafel
  \def\partname{Teil}%
  \def\enclname{Anlage(n)}%                 % oder: Beilage(n)
  \def\ccname{Verteiler}%                   % oder: Kopien an
  \def\headtoname{An}%
  \def\pagename{Seite}%
  \def\seename{siehe}%
  \def\alsoname{siehe auch}%
  \def\proofname{Beweis}%
  \def\glossaryname{Glossar}%
  }
%    \end{macrocode}
%  \end{macro}
%  \end{macro}
%
%  \begin{macro}{\datengerman}
%    The macro |\datengerman| redefines the command
%    |\today| to produce German dates.
%    \begin{macrocode}
\def\month@ngerman{\ifcase\month\or
  Januar\or Februar\or M\"arz\or April\or Mai\or Juni\or
  Juli\or August\or September\or Oktober\or November\or Dezember\fi}
\def\datengerman{\def\today{\number\day.~\month@ngerman
    \space\number\year}}
%    \end{macrocode}
%  \end{macro}
%
%  \begin{macro}{\dateanustrian}
%    The macro |\datenaustrian| redefines the command
%    |\today| to produce Austrian version of the German dates.
%    \begin{macrocode}
\def\datenaustrian{\def\today{\number\day.~\ifnum1=\month
  J\"anner\else \month@ngerman\fi \space\number\year}}
%    \end{macrocode}
%  \end{macro}
%
%  \begin{macro}{\extrasngerman}
%  \begin{macro}{\extrasnaustrian}
%  \begin{macro}{\noextrasngerman}
%  \begin{macro}{\noextrasnaustrian}
%    Either the macro |\extrasngerman| or the macros |\extrasnaustrian|
%    will perform all the extra definitions needed for the German
%    language. The macro |\noextrasngerman| is used to cancel the
%    actions of |\extrasngerman|. 
%
%    For German (as well as for Dutch) the \texttt{"} character is
%    made active. This is done once, later on its definition may vary.
%    \begin{macrocode}
\initiate@active@char{"}
\@namedef{extras\CurrentOption}{%
  \languageshorthands{ngerman}}
\expandafter\addto\csname extras\CurrentOption\endcsname{%
  \bbl@activate{"}}
%    \end{macrocode}
%    Don't forget to turn the shorthands off again.
% \changes{ngermanb-2.6j}{1999/12/16}{Deactivate shorthands ouside of
%    German}
%    \begin{macrocode}
\addto\noextrasngerman{\bbl@deactivate{"}}
%    \end{macrocode}
%
%
%    In order for \TeX\ to be able to hyphenate German words which
%    contain `\ss' (in the \texttt{OT1} position |^^Y|) we have to
%    give the character a nonzero |\lccode| (see Appendix H, the \TeX
%    book).
%    \begin{macrocode}
\expandafter\addto\csname extras\CurrentOption\endcsname{%
  \babel@savevariable{\lccode25}%
  \lccode25=25}
%    \end{macrocode}
%
%    The umlaut accent macro |\"| is changed to lower the umlaut dots.
%    The redefinition is done with the help of |\umlautlow|.
%    \begin{macrocode}
\expandafter\addto\csname extras\CurrentOption\endcsname{%
  \babel@save\"\umlautlow}
\@namedef{noextras\CurrentOption}{\umlauthigh}
%    \end{macrocode}
%    The current 
%    version of the `new' German hyphenation patterns (\file{dehyphn.tex}
%    is to be used with |\lefthyphenmin| and |\righthyphenmin| set to~2. 
% \changes{ngermanb-2.6k}{2000/09/22}{Now use \cs{providehyphenmins} to
%    provide a default value}
%    \begin{macrocode}
\providehyphenmins{\CurrentOption}{\tw@\tw@}
%    \end{macrocode}
%    For German texts we need to make sure that |\frenchspacing| is
%    turned on.
% \changes{ngermanb-2.6m}{2001/01/26}{Turn frenchspacing on, as in
%    \texttt{german.sty}}
%    \begin{macrocode}
\expandafter\addto\csname extras\CurrentOption\endcsname{%
  \bbl@frenchspacing}
\expandafter\addto\csname noextras\CurrentOption\endcsname{%
  \bbl@nonfrenchspacing}
%    \end{macrocode}
%  \end{macro}
%  \end{macro}
%  \end{macro}
%  \end{macro}
%
%    The code above is necessary because we need an extra active
%    character. This character is then used as indicated in
%    table~\ref{tab:german-quote}.
%
%    To be able to define the function of |"|, we first define a
%    couple of `support' macros.
%
%
%  \begin{macro}{\dq}
%    We save the original double quote character in |\dq| to keep
%    it available, the math accent |\"| can now be typed as |"|.
%    \begin{macrocode}
\begingroup \catcode`\"12
\def\x{\endgroup
  \def\@SS{\mathchar"7019 }
  \def\dq{"}}
\x
%    \end{macrocode}
%  \end{macro}
%
%    Now we can define the doublequote macros: the umlauts,
%    \begin{macrocode}
\declare@shorthand{ngerman}{"a}{\textormath{\"{a}\allowhyphens}{\ddot a}}
\declare@shorthand{ngerman}{"o}{\textormath{\"{o}\allowhyphens}{\ddot o}}
\declare@shorthand{ngerman}{"u}{\textormath{\"{u}\allowhyphens}{\ddot u}}
\declare@shorthand{ngerman}{"A}{\textormath{\"{A}\allowhyphens}{\ddot A}}
\declare@shorthand{ngerman}{"O}{\textormath{\"{O}\allowhyphens}{\ddot O}}
\declare@shorthand{ngerman}{"U}{\textormath{\"{U}\allowhyphens}{\ddot U}}
%    \end{macrocode}
%    tremas,
%    \begin{macrocode}
\declare@shorthand{ngerman}{"e}{\textormath{\"{e}}{\ddot e}}
\declare@shorthand{ngerman}{"E}{\textormath{\"{E}}{\ddot E}}
\declare@shorthand{ngerman}{"i}{\textormath{\"{\i}}%
                              {\ddot\imath}}
\declare@shorthand{ngerman}{"I}{\textormath{\"{I}}{\ddot I}}
%    \end{macrocode}
%    german es-zet (sharp s),
%    \begin{macrocode}
\declare@shorthand{ngerman}{"s}{\textormath{\ss}{\@SS{}}}
\declare@shorthand{ngerman}{"S}{\SS}
\declare@shorthand{ngerman}{"z}{\textormath{\ss}{\@SS{}}}
\declare@shorthand{ngerman}{"Z}{SZ}
%    \end{macrocode}
%    german and french quotes,
%    \begin{macrocode}
\declare@shorthand{ngerman}{"`}{\glqq}
\declare@shorthand{ngerman}{"'}{\grqq}
\declare@shorthand{ngerman}{"<}{\flqq}
\declare@shorthand{ngerman}{">}{\frqq}
%    \end{macrocode}
%    and some additional commands:
%    \begin{macrocode}
\declare@shorthand{ngerman}{"-}{\nobreak\-\bbl@allowhyphens}
\declare@shorthand{ngerman}{"|}{%
  \textormath{\penalty\@M\discretionary{-}{}{\kern.03em}%
              \allowhyphens}{}}
\declare@shorthand{ngerman}{""}{\hskip\z@skip}
\declare@shorthand{ngerman}{"~}{\textormath{\leavevmode\hbox{-}}{-}}
\declare@shorthand{ngerman}{"=}{\penalty\@M-\hskip\z@skip}
%    \end{macrocode}
%
%  \begin{macro}{\mdqon}
%  \begin{macro}{\mdqoff}
%    All that's left to do now is to  define a couple of commands
%    for reasons of compatibility with \file{german.sty}.
%    \begin{macrocode}
\def\mdqon{\shorthandon{"}}
\def\mdqoff{\shorthandoff{"}}
%    \end{macrocode}
%  \end{macro}
%  \end{macro}
%
%    The macro |\ldf@finish| takes care of looking for a
%    configuration file, setting the main language to be switched on
%    at |\begin{document}| and resetting the category code of
%    \texttt{@} to its original value.
%    \begin{macrocode}
\ldf@finish\CurrentOption
%</code>
%    \end{macrocode}
%
% \Finale
%%
%% \CharacterTable
%%  {Upper-case    \A\B\C\D\E\F\G\H\I\J\K\L\M\N\O\P\Q\R\S\T\U\V\W\X\Y\Z
%%   Lower-case    \a\b\c\d\e\f\g\h\i\j\k\l\m\n\o\p\q\r\s\t\u\v\w\x\y\z
%%   Digits        \0\1\2\3\4\5\6\7\8\9
%%   Exclamation   \!     Double quote  \"     Hash (number) \#
%%   Dollar        \$     Percent       \%     Ampersand     \&
%%   Acute accent  \'     Left paren    \(     Right paren   \)
%%   Asterisk      \*     Plus          \+     Comma         \,
%%   Minus         \-     Point         \.     Solidus       \/
%%   Colon         \:     Semicolon     \;     Less than     \<
%%   Equals        \=     Greater than  \>     Question mark \?
%%   Commercial at \@     Left bracket  \[     Backslash     \\
%%   Right bracket \]     Circumflex    \^     Underscore    \_
%%   Grave accent  \`     Left brace    \{     Vertical bar  \|
%%   Right brace   \}     Tilde         \~}
%%
\endinput
}
\DeclareOption{norsk}{% \iffalse meta-comment
%
% Copyright 1989-2005 Johannes L. Braams and any individual authors
% listed elsewhere in this file.  All rights reserved.
% 
% This file is part of the Babel system.
% --------------------------------------
% 
% It may be distributed and/or modified under the
% conditions of the LaTeX Project Public License, either version 1.3
% of this license or (at your option) any later version.
% The latest version of this license is in
%   http://www.latex-project.org/lppl.txt
% and version 1.3 or later is part of all distributions of LaTeX
% version 2003/12/01 or later.
% 
% This work has the LPPL maintenance status "maintained".
% 
% The Current Maintainer of this work is Johannes Braams.
% 
% The list of all files belonging to the Babel system is
% given in the file `manifest.bbl. See also `legal.bbl' for additional
% information.
% 
% The list of derived (unpacked) files belonging to the distribution
% and covered by LPPL is defined by the unpacking scripts (with
% extension .ins) which are part of the distribution.
% \fi
%\CheckSum{305}
% \iffalse
%    Tell the \LaTeX\ system who we are and write an entry on the
%    transcript.
%<*dtx>
\ProvidesFile{norsk.dtx}
%</dtx>
%<code>\ProvidesLanguage{norsk}
%\fi
%\ProvidesFile{norsk.dtx}
        [2012/08/06 v2.0i Norsk support from the babel system]
%\iffalse
%%File `norsk.dtx'
%% Babel package for LaTeX version 2e
%% Copyright (C) 1989 - 2005
%%           by Johannes Braams, TeXniek
%
%% Please report errors to: J.L. Braams
%%                          babel at braams.cistron.nl
%
%    This file is part of the babel system, it provides the source
%    code for the Norwegian language definition file.  Contributions
%    were made by Haavard Helstrup (HAAVARD@CERNVM) and Alv Kjetil
%    Holme (HOLMEA@CERNVM); the `nynorsk' variant has been supplied by
%    Per Steinar Iversen (iversen@vxcern.cern.ch) and Terje Engeset
%    Petterst (TERJEEP@VSFYS1.FI.UIB.NO)
%
%    Rune Kleveland (runekl at math.uio.no) added the shorthand
%    definitions 
%<*filedriver>
\documentclass{ltxdoc}
\newcommand*\TeXhax{\TeX hax}
\newcommand*\babel{\textsf{babel}}
\newcommand*\langvar{$\langle \it lang \rangle$}
\newcommand*\note[1]{}
\newcommand*\Lopt[1]{\textsf{#1}}
\newcommand*\file[1]{\texttt{#1}}
\begin{document}
 \DocInput{norsk.dtx}
\end{document}
%</filedriver>
%\fi
% \GetFileInfo{norsk.dtx}
%
% \changes{norsk-1.0a}{1991/07/15}{Renamed \file{babel.sty} in
%    \file{babel.com}}
% \changes{norsk-1.1a}{1992/02/16}{Brought up-to-date with babel 3.2a}
% \changes{norsk-1.1c}{1993/11/11}{Added a couple of translations
%    (from Per Norman Oma, TeX@itk.unit.no)}
% \changes{norsk-1.2a}{1994/02/27}{Update for \LaTeXe}
% \changes{norsk-1.2d}{1994/06/26}{Removed the use of \cs{filedate}
%    and moved identification after the loading of \file{babel.def}}
% \changes{norsk-1.2h}{1996/07/12}{Replaced \cs{undefined} with
%    \cs{@undefined} and \cs{empty} with \cs{@empty} for consistency
%    with \LaTeX} 
% \changes{norsk-1.2h}{1996/10/10}{Moved the definition of
%    \cs{atcatcode} right to the beginning.}
%
%
%  \section{The Norwegian language}
%
%    The file \file{\filename}\footnote{The file described in this
%    section has version number \fileversion\ and was last revised on
%    \filedate.  Contributions were made by Haavard Helstrup
%    (\texttt{HAAVARD@CERNVM)} and Alv Kjetil Holme
%    (\texttt{HOLMEA@CERNVM}); the `nynorsk' variant has been supplied
%    by Per Steinar Iversen \texttt{iversen@vxcern.cern.ch}) and Terje
%    Engeset Petterst (\texttt{TERJEEP@VSFYS1.FI.UIB.NO)}; the
%    shorthand definitions were provided by Rune Kleveland
%    (\texttt{runekl@math.uio.no}).} defines all the language definition
%    macros for the Norwegian language as well as for an alternative
%    variant `nynorsk' of this language. 
%
%    For this language the character |"| is made active. In
%    table~\ref{tab:norsk-quote} an overview is given of its purpose.
%    \begin{table}[htb]
%     \begin{center}
%     \begin{tabular}{lp{.7\textwidth}}
%      |"ff|& for |ff| to be hyphenated as |ff-f|,
%             this is also implemented for b, d, f, g, l, m, n,
%             p, r, s, and t. (|o"ppussing|)                        \\
%      |"ee|& Hyphenate |"ee| as |\'e-e|. (|komit"een|)             \\
%      |"-| & an explicit hyphen sign, allowing hyphenation in the
%             composing words. Use this for compound words when the
%             hyphenation patterns fail to hyphenate
%             properly. (|alpin"-anlegg|)                           \\
%      \verb="|= & Like |"-|, but inserts 0.03em space.  Use it if
%             the compound point is spanned by a ligature.
%             (\verb=hoff"|intriger=)                               \\
%      |""| & Like |"-|, but producing no hyphen sign.
%             (|i""g\aa{}r|)                                        \\
%      |"~| & Like |-|, but allows no hyphenation at all. (|E"~cup|)\\
%      |"=| & Like |-|, but allowing hyphenation in the composing
%             words. (|marksistisk"=leninistisk|)                   \\
%      |"<| & for French left double quotes (similar to $<<$).      \\
%      |">| & for French right double quotes (similar to $>>$).     \\
%     \end{tabular}
%     \caption{The extra definitions made
%              by \file{norsk.sty}}\label{tab:norsk-quote}
%     \end{center}
%    \end{table}
% \changes{norsk-2.0a}{1998/06/24}{Describe the use of double quote as
%    active character}
%
%    Rune Kleveland distributes a Norwegian dictionary for ispell
%    (570000 words). It can be found at
%    |http://www.uio.no/~runekl/dictionary.html|. 
%
%    This dictionary supports the spellings |spi"sslede| for
%    `spisslede' (hyphenated spiss-slede) and other such words, and
%    also suggest the spelling |spi"sslede| for `spisslede' and
%    `spissslede'.
%
% \StopEventually{}
%
%    The macro |\LdfInit| takes care of preventing that this file is
%    loaded more than once, checking the category code of the
%    \texttt{@} sign, etc.
% \changes{norsk-1.2h}{1996/11/03}{Now use \cs{LdfInit} to perform
%    initial checks} 
%    \begin{macrocode}
%<*code>
\LdfInit\CurrentOption{captions\CurrentOption}
%    \end{macrocode}
%
%    When this file is read as an option, i.e. by the |\usepackage|
%    command, \texttt{norsk} will be an `unknown' language in which
%    case we have to make it known.  So we check for the existence of
%    |\l@norsk| to see whether we have to do something here.
%
% \changes{norsk-1.0c}{1991/10/29}{Removed use of \cs{@ifundefined}}
% \changes{norsk-1.1a}{1992/02/16}{Added a warning when no hyphenation
%    patterns were loaded.}
% \changes{norsk-1.2d}{1994/06/26}{Now use \cs{@nopatterns} to produce
%    the warning}
%    \begin{macrocode}
\ifx\l@norsk\@undefined
    \@nopatterns{Norsk}
    \adddialect\l@norsk0\fi
%    \end{macrocode}
%
%  \begin{macro}{\norskhyphenmins}
%     Some sets of Norwegian hyphenation patterns can be used with
%     |\lefthyphenmin| set to~1 and |\righthyphenmin| set to~2, but
%     the most common set |nohyph.tex| can't.  So we use
%     |\lefthyphenmin=2| by default.
% \changes{norsk-1.2f}{1995/07/02}{Added setting of hyphenmin
%    parameters}
% \changes{norsk-2.0a}{1998/06/24}{Changed setting of hyphenmin
%    parameters to 2~2} 
% \changes{norsk-2.0e}{2000/09/22}{Now use \cs{providehyphenmins} to
%    provide a default value}
%    \begin{macrocode}
\providehyphenmins{\CurrentOption}{\tw@\tw@}
%    \end{macrocode}
%  \end{macro}
%
%    Now we have to decide which version of the captions should be
%    made available. This can be done by checking the contents of
%    |\CurrentOption|. 
%    \begin{macrocode}
\def\bbl@tempa{norsk}
\ifx\CurrentOption\bbl@tempa
%    \end{macrocode}
%
%    The next step consists of defining commands to switch to (and
%    from) the Norwegian language.
%
% \begin{macro}{\captionsnorsk}
%    The macro |\captionsnorsk| defines all strings used
%    in the four standard documentclasses provided with \LaTeX.
% \changes{norsk-1.1a}{1992/02/16}{Added \cs{seename}, \cs{alsoname} and
%    \cs{prefacename}}
% \changes{norsk-1.1b}{1993/07/15}{\cs{headpagename} should be
%    \cs{pagename}}
% \changes{norsk-1.2f}{1995/07/02}{Added \cs{proofname} for
%    AMS-\LaTeX}
% \changes{norsk-1.2g}{1996/04/01}{Replaced `Proof' by its
%    translation} 
% \changes{norsk-2.0e}{2000/09/20}{Added \cs{glossaryname}}
% \changes{norsk-2.0g}{1996/04/01}{Replaced `Glossary' by its
%    translation} 
%    \begin{macrocode}
  \def\captionsnorsk{%
    \def\prefacename{Forord}%
    \def\refname{Referanser}%
    \def\abstractname{Sammendrag}%
    \def\bibname{Bibliografi}%     or Litteraturoversikt
    %                              or Litteratur or Referanser
    \def\chaptername{Kapittel}%
    \def\appendixname{Tillegg}%    or Appendiks
    \def\contentsname{Innhold}%
    \def\listfigurename{Figurer}%  or Figurliste
    \def\listtablename{Tabeller}%  or Tabelliste
    \def\indexname{Register}%
    \def\figurename{Figur}%
    \def\tablename{Tabell}%
    \def\partname{Del}%
    \def\enclname{Vedlegg}%
    \def\ccname{Kopi sendt}%
    \def\headtoname{Til}% in letter
    \def\pagename{Side}%
    \def\seename{Se}%
    \def\alsoname{Se ogs\aa{}}%
    \def\proofname{Bevis}%
    \def\glossaryname{Ordliste}%
    }
\else
%    \end{macrocode}
% \end{macro}
%
%    For the `nynorsk' version of these definitions we just add a
%    ``dialect''.
%    \begin{macrocode}
  \adddialect\l@nynorsk\l@norsk
%    \end{macrocode}
%
% \begin{macro}{\captionsnynorsk}
%    The macro |\captionsnynorsk| defines all strings used in the four
%    standard documentclasses provided with \LaTeX, but using a
%    different spelling than in the command |\captionsnorsk|.
% \changes{norsk-1.1a}{1992/02/16}{Added \cs{seename}, \cs{alsoname} and
%    \cs{prefacename}}
% \changes{norsk-1.1b}{1993/07/15}{\cs{headpagename} should be
%    \cs{pagename}}
% \changes{norsk-1.2g}{1996/04/01}{Replaced `Proof' by its
%    translation} 
% \changes{norsk-2.0e}{2000/09/20}{Added \cs{glossaryname}}
% \changes{norsk-2.0g}{1996/04/01}{Replaced `Glossary' by its
%    translation} 
% \changes{norks-2.0h}{2001/01/12}{Changed \cs{ccname} and \cs{alsoname}}
%    \begin{macrocode}
  \def\captionsnynorsk{%
    \def\prefacename{Forord}%
    \def\refname{Referansar}%
    \def\abstractname{Samandrag}%
    \def\bibname{Litteratur}%     or Litteraturoversyn
     %                             or Referansar
    \def\chaptername{Kapittel}%
    \def\appendixname{Tillegg}%   or Appendiks
    \def\contentsname{Innhald}%
    \def\listfigurename{Figurar}% or Figurliste
    \def\listtablename{Tabellar}% or Tabelliste
    \def\indexname{Register}%
    \def\figurename{Figur}%
    \def\tablename{Tabell}%
    \def\partname{Del}%
    \def\enclname{Vedlegg}%
    \def\ccname{Kopi til}%
    \def\headtoname{Til}% in letter
    \def\pagename{Side}%
    \def\seename{Sj\aa{}}%
    \def\alsoname{Sj\aa{} \`{o}g}%
    \def\proofname{Bevis}%
    \def\glossaryname{Ordliste}%
    }
\fi
%    \end{macrocode}
% \end{macro}
%
% \begin{macro}{\datenorsk}
%    The macro |\datenorsk| redefines the command |\today| to produce
%    Norwegian dates.
% \changes{norsk-1.2i}{1997/10/01}{Use \cs{edef} to define
%    \cs{today} to save memory}
% \changes{norsk-1.2i}{1998/03/28}{use \cs{def} instead of \cs{edef}}
% \changes{norsk-2.0i}{2012/08/06}{Removed extra space after `desember'}
%    \begin{macrocode}
\@namedef{date\CurrentOption}{%
  \def\today{\number\day.~\ifcase\month\or
    januar\or februar\or mars\or april\or mai\or juni\or
    juli\or august\or september\or oktober\or november\or
    desember\fi
    \space\number\year}}
%    \end{macrocode}
% \end{macro}
%
% \begin{macro}{\extrasnorsk}
% \begin{macro}{\extrasnynorsk}
%    The macro |\extrasnorsk| will perform all the extra definitions
%    needed for the Norwegian language. The macro |\noextrasnorsk| is
%    used to cancel the actions of |\extrasnorsk|.  
%
%    Norwegian typesetting requires |\frencspacing| to be in effect.
%    \begin{macrocode}
\@namedef{extras\CurrentOption}{\bbl@frenchspacing}
\@namedef{noextras\CurrentOption}{\bbl@nonfrenchspacing}
%    \end{macrocode}
% \end{macro}
% \end{macro}
%
%    For Norsk the \texttt{"} character is made active. This is done
%    once, later on its definition may vary.
% \changes{norsk-2.0a}{1998/06/24}{Made double quote character active}
%    \begin{macrocode}
\initiate@active@char{"}
\expandafter\addto\csname extras\CurrentOption\endcsname{%
  \languageshorthands{norsk}}
\expandafter\addto\csname extras\CurrentOption\endcsname{%
  \bbl@activate{"}}
%    \end{macrocode}
%    Don't forget to turn the shorthands off again.
% \changes{norsk-2.0c}{1999/12/17}{Deactivate shorthands ouside of
%    Norsk}
%    \begin{macrocode}
\expandafter\addto\csname noextras\CurrentOption\endcsname{%
  \bbl@deactivate{"}}
%    \end{macrocode}
%
%    The code above is necessary because we need to define a number of
%    shorthand commands. These sharthand commands are then used as
%    indicated in table~\ref{tab:norsk-quote}.
%
%    To be able to define the function of |"|, we first define a
%    couple of `support' macros.
%
%  \begin{macro}{\dq}
%    We save the original double quote character in |\dq| to keep
%    it available, the math accent |\"| can now be typed as |"|.
%    \begin{macrocode}
\begingroup \catcode`\"12
\def\x{\endgroup
  \def\@SS{\mathchar"7019 }
  \def\dq{"}}
\x
%    \end{macrocode}
%  \end{macro}
%
%    Now we can define the discretionary shorthand commands.
%    The number of words where such hyphenation is required is for
%    each character
%    \begin{center}
%      \begin{tabular}{*{11}c}
%        b&d&f &g&k &l &n&p &r&s &t \\
%        4&4&15&3&43&30&8&12&1&33&35
%       \end{tabular}
%    \end{center}
%    taken from a list of 83000 ispell-roots.
%
% \changes{norsk-2.0d}{2000/02/29}{Shorthands are the same for both
%    spelling variants, no need to use \cs{CurrentOption}}
%    \begin{macrocode}
\declare@shorthand{norsk}{"b}{\textormath{\bbl@disc b{bb}}{b}}
\declare@shorthand{norsk}{"B}{\textormath{\bbl@disc B{BB}}{B}}
\declare@shorthand{norsk}{"d}{\textormath{\bbl@disc d{dd}}{d}}
\declare@shorthand{norsk}{"D}{\textormath{\bbl@disc D{DD}}{D}}
\declare@shorthand{norsk}{"e}{\textormath{\bbl@disc e{\'e}}{}}
\declare@shorthand{norsk}{"E}{\textormath{\bbl@disc E{\'E}}{}}
\declare@shorthand{norsk}{"F}{\textormath{\bbl@disc F{FF}}{F}}
\declare@shorthand{norsk}{"g}{\textormath{\bbl@disc g{gg}}{g}}
\declare@shorthand{norsk}{"G}{\textormath{\bbl@disc G{GG}}{G}}
\declare@shorthand{norsk}{"k}{\textormath{\bbl@disc k{kk}}{k}}
\declare@shorthand{norsk}{"K}{\textormath{\bbl@disc K{KK}}{K}}
\declare@shorthand{norsk}{"l}{\textormath{\bbl@disc l{ll}}{l}}
\declare@shorthand{norsk}{"L}{\textormath{\bbl@disc L{LL}}{L}}
\declare@shorthand{norsk}{"n}{\textormath{\bbl@disc n{nn}}{n}}
\declare@shorthand{norsk}{"N}{\textormath{\bbl@disc N{NN}}{N}}
\declare@shorthand{norsk}{"p}{\textormath{\bbl@disc p{pp}}{p}}
\declare@shorthand{norsk}{"P}{\textormath{\bbl@disc P{PP}}{P}}
\declare@shorthand{norsk}{"r}{\textormath{\bbl@disc r{rr}}{r}}
\declare@shorthand{norsk}{"R}{\textormath{\bbl@disc R{RR}}{R}}
\declare@shorthand{norsk}{"s}{\textormath{\bbl@disc s{ss}}{s}}
\declare@shorthand{norsk}{"S}{\textormath{\bbl@disc S{SS}}{S}}
\declare@shorthand{norsk}{"t}{\textormath{\bbl@disc t{tt}}{t}}
\declare@shorthand{norsk}{"T}{\textormath{\bbl@disc T{TT}}{T}}
%    \end{macrocode}
%    We need to treat |"f| a bit differently in order to preserve the
%    ff-ligature. 
% \changes{norsk-2.0b}{1999/11/19}{Copied the coding for \texttt{"f}
%    from germanb.dtx version 2.6g} 
%    \begin{macrocode}
\declare@shorthand{norsk}{"f}{\textormath{\bbl@discff}{f}}
\def\bbl@discff{\penalty\@M
  \afterassignment\bbl@insertff \let\bbl@nextff= }
\def\bbl@insertff{%
  \if f\bbl@nextff
    \expandafter\@firstoftwo\else\expandafter\@secondoftwo\fi
  {\relax\discretionary{ff-}{f}{ff}\allowhyphens}{f\bbl@nextff}}
\let\bbl@nextff=f
%    \end{macrocode}
%    We now  define the French double quotes and some commands 
%    concerning hyphenation:
% \changes{norsk-2.0b}{1999/11/22}{added the french double quotes}
% \changes{norsk-2.0d}{2000/01/28}{Use \cs{bbl@allowhyphens} in
%    \texttt{"-}}
%    \begin{macrocode}
\declare@shorthand{norsk}{"<}{\flqq}
\declare@shorthand{norsk}{">}{\frqq}
\declare@shorthand{norsk}{"-}{\penalty\@M\-\bbl@allowhyphens}
\declare@shorthand{norsk}{"|}{%
  \textormath{\penalty\@M\discretionary{-}{}{\kern.03em}%
              \allowhyphens}{}}
\declare@shorthand{norsk}{""}{\hskip\z@skip}
\declare@shorthand{norsk}{"~}{\textormath{\leavevmode\hbox{-}}{-}}
\declare@shorthand{norsk}{"=}{\penalty\@M-\hskip\z@skip}
%    \end{macrocode}
%
%    The macro |\ldf@finish| takes care of looking for a
%    configuration file, setting the main language to be switched on
%    at |\begin{document}| and resetting the category code of
%    \texttt{@} to its original value.
% \changes{norsk-1.2h}{1996/11/03}{Now use \cs{ldf@finish} to wrap up}
%    \begin{macrocode}
\ldf@finish\CurrentOption
%</code>
%    \end{macrocode}
%
% \Finale
%%
%% \CharacterTable
%%  {Upper-case    \A\B\C\D\E\F\G\H\I\J\K\L\M\N\O\P\Q\R\S\T\U\V\W\X\Y\Z
%%   Lower-case    \a\b\c\d\e\f\g\h\i\j\k\l\m\n\o\p\q\r\s\t\u\v\w\x\y\z
%%   Digits        \0\1\2\3\4\5\6\7\8\9
%%   Exclamation   \!     Double quote  \"     Hash (number) \#
%%   Dollar        \$     Percent       \%     Ampersand     \&
%%   Acute accent  \'     Left paren    \(     Right paren   \)
%%   Asterisk      \*     Plus          \+     Comma         \,
%%   Minus         \-     Point         \.     Solidus       \/
%%   Colon         \:     Semicolon     \;     Less than     \<
%%   Equals        \=     Greater than  \>     Question mark \?
%%   Commercial at \@     Left bracket  \[     Backslash     \\
%%   Right bracket \]     Circumflex    \^     Underscore    \_
%%   Grave accent  \`     Left brace    \{     Vertical bar  \|
%%   Right brace   \}     Tilde         \~}
%%
\endinput
}
\DeclareOption{samin}{%%
%% This file will generate fast loadable files and documentation
%% driver files from the doc files in this package when run through
%% LaTeX or TeX.
%%
%% Copyright 1989-2005 Johannes L. Braams and any individual authors
%% listed elsewhere in this file.  All rights reserved.
%% 
%% This file is part of the Babel system.
%% --------------------------------------
%% 
%% It may be distributed and/or modified under the
%% conditions of the LaTeX Project Public License, either version 1.3
%% of this license or (at your option) any later version.
%% The latest version of this license is in
%%   http://www.latex-project.org/lppl.txt
%% and version 1.3 or later is part of all distributions of LaTeX
%% version 2003/12/01 or later.
%% 
%% This work has the LPPL maintenance status "maintained".
%% 
%% The Current Maintainer of this work is Johannes Braams.
%% 
%% The list of all files belonging to the LaTeX base distribution is
%% given in the file `manifest.bbl. See also `legal.bbl' for additional
%% information.
%% 
%% The list of derived (unpacked) files belonging to the distribution
%% and covered by LPPL is defined by the unpacking scripts (with
%% extension .ins) which are part of the distribution.
%%
%% --------------- start of docstrip commands ------------------
%%
\def\filedate{2000/09/26}
\def\batchfile{samin.ins}
\input docstrip.tex

{\ifx\generate\undefined
\Msg{**********************************************}
\Msg{*}
\Msg{* This installation requires docstrip}
\Msg{* version 2.3c or later.}
\Msg{*}
\Msg{* An older version of docstrip has been input}
\Msg{*}
\Msg{**********************************************}
\errhelp{Move or rename old docstrip.tex.}
\errmessage{Old docstrip in input path}
\batchmode
\csname @@end\endcsname
\fi}

\declarepreamble\mainpreamble
This is a generated file.

Copyright 1989-2005 Johannes L. Braams and any individual authors
listed elsewhere in this file.  All rights reserved.

This file was generated from file(s) of the Babel system.
---------------------------------------------------------

It may be distributed and/or modified under the
conditions of the LaTeX Project Public License, either version 1.3
of this license or (at your option) any later version.
The latest version of this license is in
  http://www.latex-project.org/lppl.txt
and version 1.3 or later is part of all distributions of LaTeX
version 2003/12/01 or later.

This work has the LPPL maintenance status "maintained".

The Current Maintainer of this work is Johannes Braams.

This file may only be distributed together with a copy of the Babel
system. You may however distribute the Babel system without
such generated files.

The list of all files belonging to the Babel distribution is
given in the file `manifest.bbl'. See also `legal.bbl for additional
information.

The list of derived (unpacked) files belonging to the distribution
and covered by LPPL is defined by the unpacking scripts (with
extension .ins) which are part of the distribution.
\endpreamble

\declarepreamble\fdpreamble
This is a generated file.

Copyright 1989-2005 Johannes L. Braams and any individual authors
listed elsewhere in this file.  All rights reserved.

This file was generated from file(s) of the Babel system.
---------------------------------------------------------

It may be distributed and/or modified under the
conditions of the LaTeX Project Public License, either version 1.3
of this license or (at your option) any later version.
The latest version of this license is in
  http://www.latex-project.org/lppl.txt
and version 1.3 or later is part of all distributions of LaTeX
version 2003/12/01 or later.

This work has the LPPL maintenance status "maintained".

The Current Maintainer of this work is Johannes Braams.

This file may only be distributed together with a copy of the Babel
system. You may however distribute the Babel system without
such generated files.

The list of all files belonging to the Babel distribution is
given in the file `manifest.bbl'. See also `legal.bbl for additional
information.

In particular, permission is granted to customize the declarations in
this file to serve the needs of your installation.

However, NO PERMISSION is granted to distribute a modified version
of this file under its original name.

\endpreamble

\keepsilent

\usedir{tex/generic/babel} 

\usepreamble\mainpreamble
\generate{\file{samin.ldf}{\from{samin.dtx}{code}}
          }
\usepreamble\fdpreamble

\ifToplevel{
\Msg{***********************************************************}
\Msg{*}
\Msg{* To finish the installation you have to move the following}
\Msg{* files into a directory searched by TeX:}
\Msg{*}
\Msg{* \space\space All *.def, *.fd, *.ldf, *.sty}
\Msg{*}
\Msg{* To produce the documentation run the files ending with}
\Msg{* '.dtx' and `.fdd' through LaTeX.}
\Msg{*}
\Msg{* Happy TeXing}
\Msg{***********************************************************}
}
 
\endinput}
%    \end{macrocode}
%    For Norwegian two spelling variants are provided.
%    \begin{macrocode}
\DeclareOption{nynorsk}{% \iffalse meta-comment
%
% Copyright 1989-2005 Johannes L. Braams and any individual authors
% listed elsewhere in this file.  All rights reserved.
% 
% This file is part of the Babel system.
% --------------------------------------
% 
% It may be distributed and/or modified under the
% conditions of the LaTeX Project Public License, either version 1.3
% of this license or (at your option) any later version.
% The latest version of this license is in
%   http://www.latex-project.org/lppl.txt
% and version 1.3 or later is part of all distributions of LaTeX
% version 2003/12/01 or later.
% 
% This work has the LPPL maintenance status "maintained".
% 
% The Current Maintainer of this work is Johannes Braams.
% 
% The list of all files belonging to the Babel system is
% given in the file `manifest.bbl. See also `legal.bbl' for additional
% information.
% 
% The list of derived (unpacked) files belonging to the distribution
% and covered by LPPL is defined by the unpacking scripts (with
% extension .ins) which are part of the distribution.
% \fi
%\CheckSum{305}
% \iffalse
%    Tell the \LaTeX\ system who we are and write an entry on the
%    transcript.
%<*dtx>
\ProvidesFile{norsk.dtx}
%</dtx>
%<code>\ProvidesLanguage{norsk}
%\fi
%\ProvidesFile{norsk.dtx}
        [2012/08/06 v2.0i Norsk support from the babel system]
%\iffalse
%%File `norsk.dtx'
%% Babel package for LaTeX version 2e
%% Copyright (C) 1989 - 2005
%%           by Johannes Braams, TeXniek
%
%% Please report errors to: J.L. Braams
%%                          babel at braams.cistron.nl
%
%    This file is part of the babel system, it provides the source
%    code for the Norwegian language definition file.  Contributions
%    were made by Haavard Helstrup (HAAVARD@CERNVM) and Alv Kjetil
%    Holme (HOLMEA@CERNVM); the `nynorsk' variant has been supplied by
%    Per Steinar Iversen (iversen@vxcern.cern.ch) and Terje Engeset
%    Petterst (TERJEEP@VSFYS1.FI.UIB.NO)
%
%    Rune Kleveland (runekl at math.uio.no) added the shorthand
%    definitions 
%<*filedriver>
\documentclass{ltxdoc}
\newcommand*\TeXhax{\TeX hax}
\newcommand*\babel{\textsf{babel}}
\newcommand*\langvar{$\langle \it lang \rangle$}
\newcommand*\note[1]{}
\newcommand*\Lopt[1]{\textsf{#1}}
\newcommand*\file[1]{\texttt{#1}}
\begin{document}
 \DocInput{norsk.dtx}
\end{document}
%</filedriver>
%\fi
% \GetFileInfo{norsk.dtx}
%
% \changes{norsk-1.0a}{1991/07/15}{Renamed \file{babel.sty} in
%    \file{babel.com}}
% \changes{norsk-1.1a}{1992/02/16}{Brought up-to-date with babel 3.2a}
% \changes{norsk-1.1c}{1993/11/11}{Added a couple of translations
%    (from Per Norman Oma, TeX@itk.unit.no)}
% \changes{norsk-1.2a}{1994/02/27}{Update for \LaTeXe}
% \changes{norsk-1.2d}{1994/06/26}{Removed the use of \cs{filedate}
%    and moved identification after the loading of \file{babel.def}}
% \changes{norsk-1.2h}{1996/07/12}{Replaced \cs{undefined} with
%    \cs{@undefined} and \cs{empty} with \cs{@empty} for consistency
%    with \LaTeX} 
% \changes{norsk-1.2h}{1996/10/10}{Moved the definition of
%    \cs{atcatcode} right to the beginning.}
%
%
%  \section{The Norwegian language}
%
%    The file \file{\filename}\footnote{The file described in this
%    section has version number \fileversion\ and was last revised on
%    \filedate.  Contributions were made by Haavard Helstrup
%    (\texttt{HAAVARD@CERNVM)} and Alv Kjetil Holme
%    (\texttt{HOLMEA@CERNVM}); the `nynorsk' variant has been supplied
%    by Per Steinar Iversen \texttt{iversen@vxcern.cern.ch}) and Terje
%    Engeset Petterst (\texttt{TERJEEP@VSFYS1.FI.UIB.NO)}; the
%    shorthand definitions were provided by Rune Kleveland
%    (\texttt{runekl@math.uio.no}).} defines all the language definition
%    macros for the Norwegian language as well as for an alternative
%    variant `nynorsk' of this language. 
%
%    For this language the character |"| is made active. In
%    table~\ref{tab:norsk-quote} an overview is given of its purpose.
%    \begin{table}[htb]
%     \begin{center}
%     \begin{tabular}{lp{.7\textwidth}}
%      |"ff|& for |ff| to be hyphenated as |ff-f|,
%             this is also implemented for b, d, f, g, l, m, n,
%             p, r, s, and t. (|o"ppussing|)                        \\
%      |"ee|& Hyphenate |"ee| as |\'e-e|. (|komit"een|)             \\
%      |"-| & an explicit hyphen sign, allowing hyphenation in the
%             composing words. Use this for compound words when the
%             hyphenation patterns fail to hyphenate
%             properly. (|alpin"-anlegg|)                           \\
%      \verb="|= & Like |"-|, but inserts 0.03em space.  Use it if
%             the compound point is spanned by a ligature.
%             (\verb=hoff"|intriger=)                               \\
%      |""| & Like |"-|, but producing no hyphen sign.
%             (|i""g\aa{}r|)                                        \\
%      |"~| & Like |-|, but allows no hyphenation at all. (|E"~cup|)\\
%      |"=| & Like |-|, but allowing hyphenation in the composing
%             words. (|marksistisk"=leninistisk|)                   \\
%      |"<| & for French left double quotes (similar to $<<$).      \\
%      |">| & for French right double quotes (similar to $>>$).     \\
%     \end{tabular}
%     \caption{The extra definitions made
%              by \file{norsk.sty}}\label{tab:norsk-quote}
%     \end{center}
%    \end{table}
% \changes{norsk-2.0a}{1998/06/24}{Describe the use of double quote as
%    active character}
%
%    Rune Kleveland distributes a Norwegian dictionary for ispell
%    (570000 words). It can be found at
%    |http://www.uio.no/~runekl/dictionary.html|. 
%
%    This dictionary supports the spellings |spi"sslede| for
%    `spisslede' (hyphenated spiss-slede) and other such words, and
%    also suggest the spelling |spi"sslede| for `spisslede' and
%    `spissslede'.
%
% \StopEventually{}
%
%    The macro |\LdfInit| takes care of preventing that this file is
%    loaded more than once, checking the category code of the
%    \texttt{@} sign, etc.
% \changes{norsk-1.2h}{1996/11/03}{Now use \cs{LdfInit} to perform
%    initial checks} 
%    \begin{macrocode}
%<*code>
\LdfInit\CurrentOption{captions\CurrentOption}
%    \end{macrocode}
%
%    When this file is read as an option, i.e. by the |\usepackage|
%    command, \texttt{norsk} will be an `unknown' language in which
%    case we have to make it known.  So we check for the existence of
%    |\l@norsk| to see whether we have to do something here.
%
% \changes{norsk-1.0c}{1991/10/29}{Removed use of \cs{@ifundefined}}
% \changes{norsk-1.1a}{1992/02/16}{Added a warning when no hyphenation
%    patterns were loaded.}
% \changes{norsk-1.2d}{1994/06/26}{Now use \cs{@nopatterns} to produce
%    the warning}
%    \begin{macrocode}
\ifx\l@norsk\@undefined
    \@nopatterns{Norsk}
    \adddialect\l@norsk0\fi
%    \end{macrocode}
%
%  \begin{macro}{\norskhyphenmins}
%     Some sets of Norwegian hyphenation patterns can be used with
%     |\lefthyphenmin| set to~1 and |\righthyphenmin| set to~2, but
%     the most common set |nohyph.tex| can't.  So we use
%     |\lefthyphenmin=2| by default.
% \changes{norsk-1.2f}{1995/07/02}{Added setting of hyphenmin
%    parameters}
% \changes{norsk-2.0a}{1998/06/24}{Changed setting of hyphenmin
%    parameters to 2~2} 
% \changes{norsk-2.0e}{2000/09/22}{Now use \cs{providehyphenmins} to
%    provide a default value}
%    \begin{macrocode}
\providehyphenmins{\CurrentOption}{\tw@\tw@}
%    \end{macrocode}
%  \end{macro}
%
%    Now we have to decide which version of the captions should be
%    made available. This can be done by checking the contents of
%    |\CurrentOption|. 
%    \begin{macrocode}
\def\bbl@tempa{norsk}
\ifx\CurrentOption\bbl@tempa
%    \end{macrocode}
%
%    The next step consists of defining commands to switch to (and
%    from) the Norwegian language.
%
% \begin{macro}{\captionsnorsk}
%    The macro |\captionsnorsk| defines all strings used
%    in the four standard documentclasses provided with \LaTeX.
% \changes{norsk-1.1a}{1992/02/16}{Added \cs{seename}, \cs{alsoname} and
%    \cs{prefacename}}
% \changes{norsk-1.1b}{1993/07/15}{\cs{headpagename} should be
%    \cs{pagename}}
% \changes{norsk-1.2f}{1995/07/02}{Added \cs{proofname} for
%    AMS-\LaTeX}
% \changes{norsk-1.2g}{1996/04/01}{Replaced `Proof' by its
%    translation} 
% \changes{norsk-2.0e}{2000/09/20}{Added \cs{glossaryname}}
% \changes{norsk-2.0g}{1996/04/01}{Replaced `Glossary' by its
%    translation} 
%    \begin{macrocode}
  \def\captionsnorsk{%
    \def\prefacename{Forord}%
    \def\refname{Referanser}%
    \def\abstractname{Sammendrag}%
    \def\bibname{Bibliografi}%     or Litteraturoversikt
    %                              or Litteratur or Referanser
    \def\chaptername{Kapittel}%
    \def\appendixname{Tillegg}%    or Appendiks
    \def\contentsname{Innhold}%
    \def\listfigurename{Figurer}%  or Figurliste
    \def\listtablename{Tabeller}%  or Tabelliste
    \def\indexname{Register}%
    \def\figurename{Figur}%
    \def\tablename{Tabell}%
    \def\partname{Del}%
    \def\enclname{Vedlegg}%
    \def\ccname{Kopi sendt}%
    \def\headtoname{Til}% in letter
    \def\pagename{Side}%
    \def\seename{Se}%
    \def\alsoname{Se ogs\aa{}}%
    \def\proofname{Bevis}%
    \def\glossaryname{Ordliste}%
    }
\else
%    \end{macrocode}
% \end{macro}
%
%    For the `nynorsk' version of these definitions we just add a
%    ``dialect''.
%    \begin{macrocode}
  \adddialect\l@nynorsk\l@norsk
%    \end{macrocode}
%
% \begin{macro}{\captionsnynorsk}
%    The macro |\captionsnynorsk| defines all strings used in the four
%    standard documentclasses provided with \LaTeX, but using a
%    different spelling than in the command |\captionsnorsk|.
% \changes{norsk-1.1a}{1992/02/16}{Added \cs{seename}, \cs{alsoname} and
%    \cs{prefacename}}
% \changes{norsk-1.1b}{1993/07/15}{\cs{headpagename} should be
%    \cs{pagename}}
% \changes{norsk-1.2g}{1996/04/01}{Replaced `Proof' by its
%    translation} 
% \changes{norsk-2.0e}{2000/09/20}{Added \cs{glossaryname}}
% \changes{norsk-2.0g}{1996/04/01}{Replaced `Glossary' by its
%    translation} 
% \changes{norks-2.0h}{2001/01/12}{Changed \cs{ccname} and \cs{alsoname}}
%    \begin{macrocode}
  \def\captionsnynorsk{%
    \def\prefacename{Forord}%
    \def\refname{Referansar}%
    \def\abstractname{Samandrag}%
    \def\bibname{Litteratur}%     or Litteraturoversyn
     %                             or Referansar
    \def\chaptername{Kapittel}%
    \def\appendixname{Tillegg}%   or Appendiks
    \def\contentsname{Innhald}%
    \def\listfigurename{Figurar}% or Figurliste
    \def\listtablename{Tabellar}% or Tabelliste
    \def\indexname{Register}%
    \def\figurename{Figur}%
    \def\tablename{Tabell}%
    \def\partname{Del}%
    \def\enclname{Vedlegg}%
    \def\ccname{Kopi til}%
    \def\headtoname{Til}% in letter
    \def\pagename{Side}%
    \def\seename{Sj\aa{}}%
    \def\alsoname{Sj\aa{} \`{o}g}%
    \def\proofname{Bevis}%
    \def\glossaryname{Ordliste}%
    }
\fi
%    \end{macrocode}
% \end{macro}
%
% \begin{macro}{\datenorsk}
%    The macro |\datenorsk| redefines the command |\today| to produce
%    Norwegian dates.
% \changes{norsk-1.2i}{1997/10/01}{Use \cs{edef} to define
%    \cs{today} to save memory}
% \changes{norsk-1.2i}{1998/03/28}{use \cs{def} instead of \cs{edef}}
% \changes{norsk-2.0i}{2012/08/06}{Removed extra space after `desember'}
%    \begin{macrocode}
\@namedef{date\CurrentOption}{%
  \def\today{\number\day.~\ifcase\month\or
    januar\or februar\or mars\or april\or mai\or juni\or
    juli\or august\or september\or oktober\or november\or
    desember\fi
    \space\number\year}}
%    \end{macrocode}
% \end{macro}
%
% \begin{macro}{\extrasnorsk}
% \begin{macro}{\extrasnynorsk}
%    The macro |\extrasnorsk| will perform all the extra definitions
%    needed for the Norwegian language. The macro |\noextrasnorsk| is
%    used to cancel the actions of |\extrasnorsk|.  
%
%    Norwegian typesetting requires |\frencspacing| to be in effect.
%    \begin{macrocode}
\@namedef{extras\CurrentOption}{\bbl@frenchspacing}
\@namedef{noextras\CurrentOption}{\bbl@nonfrenchspacing}
%    \end{macrocode}
% \end{macro}
% \end{macro}
%
%    For Norsk the \texttt{"} character is made active. This is done
%    once, later on its definition may vary.
% \changes{norsk-2.0a}{1998/06/24}{Made double quote character active}
%    \begin{macrocode}
\initiate@active@char{"}
\expandafter\addto\csname extras\CurrentOption\endcsname{%
  \languageshorthands{norsk}}
\expandafter\addto\csname extras\CurrentOption\endcsname{%
  \bbl@activate{"}}
%    \end{macrocode}
%    Don't forget to turn the shorthands off again.
% \changes{norsk-2.0c}{1999/12/17}{Deactivate shorthands ouside of
%    Norsk}
%    \begin{macrocode}
\expandafter\addto\csname noextras\CurrentOption\endcsname{%
  \bbl@deactivate{"}}
%    \end{macrocode}
%
%    The code above is necessary because we need to define a number of
%    shorthand commands. These sharthand commands are then used as
%    indicated in table~\ref{tab:norsk-quote}.
%
%    To be able to define the function of |"|, we first define a
%    couple of `support' macros.
%
%  \begin{macro}{\dq}
%    We save the original double quote character in |\dq| to keep
%    it available, the math accent |\"| can now be typed as |"|.
%    \begin{macrocode}
\begingroup \catcode`\"12
\def\x{\endgroup
  \def\@SS{\mathchar"7019 }
  \def\dq{"}}
\x
%    \end{macrocode}
%  \end{macro}
%
%    Now we can define the discretionary shorthand commands.
%    The number of words where such hyphenation is required is for
%    each character
%    \begin{center}
%      \begin{tabular}{*{11}c}
%        b&d&f &g&k &l &n&p &r&s &t \\
%        4&4&15&3&43&30&8&12&1&33&35
%       \end{tabular}
%    \end{center}
%    taken from a list of 83000 ispell-roots.
%
% \changes{norsk-2.0d}{2000/02/29}{Shorthands are the same for both
%    spelling variants, no need to use \cs{CurrentOption}}
%    \begin{macrocode}
\declare@shorthand{norsk}{"b}{\textormath{\bbl@disc b{bb}}{b}}
\declare@shorthand{norsk}{"B}{\textormath{\bbl@disc B{BB}}{B}}
\declare@shorthand{norsk}{"d}{\textormath{\bbl@disc d{dd}}{d}}
\declare@shorthand{norsk}{"D}{\textormath{\bbl@disc D{DD}}{D}}
\declare@shorthand{norsk}{"e}{\textormath{\bbl@disc e{\'e}}{}}
\declare@shorthand{norsk}{"E}{\textormath{\bbl@disc E{\'E}}{}}
\declare@shorthand{norsk}{"F}{\textormath{\bbl@disc F{FF}}{F}}
\declare@shorthand{norsk}{"g}{\textormath{\bbl@disc g{gg}}{g}}
\declare@shorthand{norsk}{"G}{\textormath{\bbl@disc G{GG}}{G}}
\declare@shorthand{norsk}{"k}{\textormath{\bbl@disc k{kk}}{k}}
\declare@shorthand{norsk}{"K}{\textormath{\bbl@disc K{KK}}{K}}
\declare@shorthand{norsk}{"l}{\textormath{\bbl@disc l{ll}}{l}}
\declare@shorthand{norsk}{"L}{\textormath{\bbl@disc L{LL}}{L}}
\declare@shorthand{norsk}{"n}{\textormath{\bbl@disc n{nn}}{n}}
\declare@shorthand{norsk}{"N}{\textormath{\bbl@disc N{NN}}{N}}
\declare@shorthand{norsk}{"p}{\textormath{\bbl@disc p{pp}}{p}}
\declare@shorthand{norsk}{"P}{\textormath{\bbl@disc P{PP}}{P}}
\declare@shorthand{norsk}{"r}{\textormath{\bbl@disc r{rr}}{r}}
\declare@shorthand{norsk}{"R}{\textormath{\bbl@disc R{RR}}{R}}
\declare@shorthand{norsk}{"s}{\textormath{\bbl@disc s{ss}}{s}}
\declare@shorthand{norsk}{"S}{\textormath{\bbl@disc S{SS}}{S}}
\declare@shorthand{norsk}{"t}{\textormath{\bbl@disc t{tt}}{t}}
\declare@shorthand{norsk}{"T}{\textormath{\bbl@disc T{TT}}{T}}
%    \end{macrocode}
%    We need to treat |"f| a bit differently in order to preserve the
%    ff-ligature. 
% \changes{norsk-2.0b}{1999/11/19}{Copied the coding for \texttt{"f}
%    from germanb.dtx version 2.6g} 
%    \begin{macrocode}
\declare@shorthand{norsk}{"f}{\textormath{\bbl@discff}{f}}
\def\bbl@discff{\penalty\@M
  \afterassignment\bbl@insertff \let\bbl@nextff= }
\def\bbl@insertff{%
  \if f\bbl@nextff
    \expandafter\@firstoftwo\else\expandafter\@secondoftwo\fi
  {\relax\discretionary{ff-}{f}{ff}\allowhyphens}{f\bbl@nextff}}
\let\bbl@nextff=f
%    \end{macrocode}
%    We now  define the French double quotes and some commands 
%    concerning hyphenation:
% \changes{norsk-2.0b}{1999/11/22}{added the french double quotes}
% \changes{norsk-2.0d}{2000/01/28}{Use \cs{bbl@allowhyphens} in
%    \texttt{"-}}
%    \begin{macrocode}
\declare@shorthand{norsk}{"<}{\flqq}
\declare@shorthand{norsk}{">}{\frqq}
\declare@shorthand{norsk}{"-}{\penalty\@M\-\bbl@allowhyphens}
\declare@shorthand{norsk}{"|}{%
  \textormath{\penalty\@M\discretionary{-}{}{\kern.03em}%
              \allowhyphens}{}}
\declare@shorthand{norsk}{""}{\hskip\z@skip}
\declare@shorthand{norsk}{"~}{\textormath{\leavevmode\hbox{-}}{-}}
\declare@shorthand{norsk}{"=}{\penalty\@M-\hskip\z@skip}
%    \end{macrocode}
%
%    The macro |\ldf@finish| takes care of looking for a
%    configuration file, setting the main language to be switched on
%    at |\begin{document}| and resetting the category code of
%    \texttt{@} to its original value.
% \changes{norsk-1.2h}{1996/11/03}{Now use \cs{ldf@finish} to wrap up}
%    \begin{macrocode}
\ldf@finish\CurrentOption
%</code>
%    \end{macrocode}
%
% \Finale
%%
%% \CharacterTable
%%  {Upper-case    \A\B\C\D\E\F\G\H\I\J\K\L\M\N\O\P\Q\R\S\T\U\V\W\X\Y\Z
%%   Lower-case    \a\b\c\d\e\f\g\h\i\j\k\l\m\n\o\p\q\r\s\t\u\v\w\x\y\z
%%   Digits        \0\1\2\3\4\5\6\7\8\9
%%   Exclamation   \!     Double quote  \"     Hash (number) \#
%%   Dollar        \$     Percent       \%     Ampersand     \&
%%   Acute accent  \'     Left paren    \(     Right paren   \)
%%   Asterisk      \*     Plus          \+     Comma         \,
%%   Minus         \-     Point         \.     Solidus       \/
%%   Colon         \:     Semicolon     \;     Less than     \<
%%   Equals        \=     Greater than  \>     Question mark \?
%%   Commercial at \@     Left bracket  \[     Backslash     \\
%%   Right bracket \]     Circumflex    \^     Underscore    \_
%%   Grave accent  \`     Left brace    \{     Vertical bar  \|
%%   Right brace   \}     Tilde         \~}
%%
\endinput
}
\DeclareOption{polish}{% \iffalse meta-comment
%
% Copyright 1989-2005 Johannes L. Braams and any individual authors
% listed elsewhere in this file.  All rights reserved.
% 
% This file is part of the Babel system.
% --------------------------------------
% 
% It may be distributed and/or modified under the
% conditions of the LaTeX Project Public License, either version 1.3
% of this license or (at your option) any later version.
% The latest version of this license is in
%   http://www.latex-project.org/lppl.txt
% and version 1.3 or later is part of all distributions of LaTeX
% version 2003/12/01 or later.
% 
% This work has the LPPL maintenance status "maintained".
% 
% The Current Maintainer of this work is Johannes Braams.
% 
% The list of all files belonging to the Babel system is
% given in the file `manifest.bbl. See also `legal.bbl' for additional
% information.
% 
% The list of derived (unpacked) files belonging to the distribution
% and covered by LPPL is defined by the unpacking scripts (with
% extension .ins) which are part of the distribution.
% \fi
% \CheckSum{543}
%
% \iffalse
%    Tell the \LaTeX\ system who we are and write an entry on the
%    transcript.
%<*dtx>
\ProvidesFile{polish.dtx}
%</dtx>
%<code>\ProvidesLanguage{polish}
%\fi
%\ProvidesFile{polish.dtx}
        [2005/03/31 v1.2l Polish support from the babel system]
%\iffalse
%% File `polish.dtx'
%% Babel package for LaTeX version 2e
%% Copyright (C) 1989 -- 2005
%%           by Johannes Braams, TeXniek
%
%% Polish Language Definition File
%% Copyright (C) 1989 - 2005
%%           by Elmar Schalueck, Michael Janich
%              Universitaet-Gesamthochschule Paderborn
%              Warburger Strasse 100
%              4790 Paderborn
%              Germany
%              elmar at uni-paderborn.de
%              massa at uni-paderborn.de
%
%% Please report errors to: J.L. Braams
%%                          babel at braams.cistron.nl
%
%    This file is part of the babel system, it provides the source
%    code for the Polish language definition file. It was developped
%    out of Polish.tex, which was written by Elmar Schalueck and
%    Michael Janich. Polish.tex was based on code by Leszek
%    Holenderski, Jerzy Ryll and J. S. Bie\'n from Faculty of
%    Mathematics,Informatics and Mechanics of Warsaw University, exept
%    of Jerzy Ryll (Instytut Informatyki Uniwersytetu Warszawskiego).
%<*filedriver>
\documentclass{ltxdoc}
\newcommand*\TeXhax{\TeX hax}
\newcommand*\babel{\textsf{babel}}
\newcommand*\langvar{$\langle \it lang \rangle$}
\newcommand*\note[1]{}
\newcommand*\Lopt[1]{\textsf{#1}}
\newcommand*\file[1]{\texttt{#1}}
\begin{document}
 \DocInput{polish.dtx}
\end{document}
%</filedriver>
%\fi
% \GetFileInfo{polish.dtx}
%
% \changes{polish-1.1c}{1994/06/26}{Removed the use of \cs{filedate}
%    and moved identification after the loading of babel.def}
% \changes{polish-1.2d}{1996/10/10}{Replaced \cs{undefined} with
%    \cs{@undefined} and \cs{empty} with \cs{@empty} for consistency
%    with \LaTeX, moved the definition of \cs{atcatcode} right to the
%    beginning.}
%
%  \section{The Polish language}
%
%    The file \file{\filename}\footnote{The file described in this
%    section has version number \fileversion\ and was last revised on
%    \filedate.}  defines all the language-specific macros for the
%    Polish language.
%
%    For this language the character |"| is made active. In
%    table~\ref{tab:polish-quote} an overview is given of its purpose.
%    \begin{table}[htb]
%     \begin{center}
%     \begin{tabular}{lp{8cm}}
%      |"a| & or |\aob|, for tailed-a (like \c{a})\\
%      |"A| & or |\Aob|, for tailed-A (like \c{A})\\
%      |"e| & or |\eob|, for tailed-e (like \c{e})\\
%      |"E| & or |\Eob|, for tailed-E (like \c{E})\\
%      |"c| & or |\'c|,  for accented c (like \'c),
%                      same with uppercase letters and n,o,s\\
%      |"l| & or |\lpb{}|, for l with stroke (like \l)\\
%      |"L| & or |\Lpb{}|, for L with stroke (like \L)\\
%      |"r| & or |\zkb{}|, for pointed z (like \.z), cf.
%      pronounciation\\
%      |"R| & or |\Zkb{}|, for pointed Z (like \.Z)\\
%      |"z| & or |\'z|,  for accented z\\
%      |"Z| & or |\'Z|,  for accented Z\\
%      \verb="|= & disable ligature at this position.\\
%      |"-| & an explicit hyphen sign, allowing hyphenation
%                  in the rest of the word.\\
%      |""| & like |"-|, but producing no hyphen sign
%                  (for compund words with hyphen, e.g.\ |x-""y|). \\
%      |"`| & for German left double quotes (looks like ,,).   \\
%      |"'| & for German right double quotes.                  \\
%      |"<| & for French left double quotes (similar to $<<$). \\
%      |">| & for French right double quotes (similar to $>>$).\\
%     \end{tabular}
%     \caption{The extra definitions made by \file{polish.sty}}
%     \label{tab:polish-quote}
%     \end{center}
%    \end{table}
%
% \StopEventually{}
%
%    The macro |\LdfInit| takes care of preventing that this file is
%    loaded more than once, checking the category code of the
%    \texttt{@} sign, etc.
% \changes{polish-1.2d}{1996/11/03}{Now use \cs{LdfInit} to perform
%    initial checks} 
%    \begin{macrocode}
%<*code>
\LdfInit{polish}\captionspolish
%    \end{macrocode}
%
%    When this file is read as an option, i.e. by the |\usepackage|
%    command, \texttt{polish} could be an `unknown' language in which
%    case we have to make it known. So we check for the existence of
%    |\l@polish| to see whether we have to do something here.
%
% \changes{polish-1.1c}{1994/06/26}{Now use \cs{@nopatterns} to
%    produce the warning}
%    \begin{macrocode}
\ifx\l@polish\@undefined
  \@nopatterns{Polish}
  \adddialect\l@polish0\fi
%    \end{macrocode}
%
%    The next step consists of defining commands to switch to (and
%    from) the Polish language.
%
% \begin{macro}{\captionspolish}
%    The macro |\captionspolish| defines all strings used in the four
%    standard documentclasses provided with \LaTeX.
% \changes{polish-1.2b}{1995/07/04}{Added \cs{proofname} for
%    AMS-\LaTeX}
% \changes{polish-1.2f}{1997/09/11}{Added translation for Proof and
%    changed translation of Contents}
% \changes{polish-1.2i}{2000/09/19}{\cs{bibname} and \cs{refname} were
%    swapped}
% \changes{polish-1.2i}{2000/09/19}{Added \cs{glossaryname}}
%    \begin{macrocode}
\addto\captionspolish{%
  \def\prefacename{Przedmowa}%
  \def\refname{Literatura}%
  \def\abstractname{Streszczenie}%
  \def\bibname{Bibliografia}%
  \def\chaptername{Rozdzia\l}%
  \def\appendixname{Dodatek}%
  \def\contentsname{Spis tre\'sci}%
  \def\listfigurename{Spis rysunk\'ow}%
  \def\listtablename{Spis tablic}%
  \def\indexname{Indeks}%
  \def\figurename{Rysunek}%
  \def\tablename{Tablica}%
  \def\partname{Cz\eob{}\'s\'c}%
  \def\enclname{Za\l\aob{}cznik}%
  \def\ccname{Kopie:}%
  \def\headtoname{Do}%
  \def\pagename{Strona}%
  \def\seename{Por\'ownaj}%
  \def\alsoname{Por\'ownaj tak\.ze}%
  \def\proofname{Dow\'od}%
  \def\glossaryname{Glossary}% <-- Needs translation
}
%    \end{macrocode}
% \end{macro}
%
% \begin{macro}{\datepolish}
%    The macro |\datepolish| redefines the command |\today| to produce
%    Polish dates.
% \changes{polish-1.2f}{1997/10/01}{Use \cs{edef} to define
%    \cs{today} to save memory}
% \changes{polish-1.2f}{1998/03/28}{use \cs{def} instead of \cs{edef}}
% \changes{polish-1.2i}{2000/01/08}{A missing comment character caused
%    an unwanted space character in the output}
%    \begin{macrocode}
\def\datepolish{%
  \def\today{\number\day~\ifcase\month\or
  stycznia\or lutego\or marca\or kwietnia\or maja\or czerwca\or lipca\or
  sierpnia\or wrze\'snia\or pa\'zdziernika\or listopada\or grudnia\fi
  \space\number\year}%
}
%    \end{macrocode}
% \end{macro}
%
% \begin{macro}{\extraspolish}
% \begin{macro}{\noextraspolish}
%    The macro |\extraspolish| will perform all the extra definitions
%    needed for the Polish language. The macro |\noextraspolish| is
%    used to cancel the actions of |\extraspolish|.
%
%    For Polish the \texttt{"} character is made active. This is
%    done once, later on its definition may vary. Other languages in
%    the same document may also use the \texttt{"} character for
%    shorthands; we specify that the polish group of shorthands
%    should be used.
%
%    \begin{macrocode}
\initiate@active@char{"}
\addto\extraspolish{\languageshorthands{polish}}
\addto\extraspolish{\bbl@activate{"}}
%    \end{macrocode}
%    Don't forget to turn the shorthands off again.
% \changes{polish-1.2h}{1999/12/17}{Deactivate shorthands ouside of
%    Polish}
%    \begin{macrocode}
\addto\noextraspolish{\bbl@deactivate{"}}
%    \end{macrocode}
% \end{macro}
% \end{macro}
%
%    The code above is necessary because we need an extra
%    active character. This character is then used as indicated in
%    table~\ref{tab:polish-quote}.
%
%    If you have problems at the end of a word with a linebreak, use
%    the other version without hyphenation tricks. Some TeX wizard may
%    produce a better solution with forcasting another token to decide
%    whether the character after the double quote is the last in a
%    word. Do it and let us know.
%
%    In Polish texts some letters get special diacritical marks.
%    Leszek Holenderski designed the following code to position the
%    diacritics correctly for every font in every size. These macros
%    need a few extra dimension variables.
%
%    \begin{macrocode}
\newdimen\pl@left
\newdimen\pl@down
\newdimen\pl@right
\newdimen\pl@temp
%    \end{macrocode}
%
%  \begin{macro}{\sob}
%    The macro |\sob| is used to put the `ogonek' in the right
%    place.
%
% \changes{polish-1.2d}{1996/08/18}{This macro is meant to be used in
%    horizontal mode; so leave vertical mode if necessary} 
%    \begin{macrocode}
\def\sob#1#2#3#4#5{%parameters: letter and fractions hl,ho,vl,vo
  \setbox0\hbox{#1}\setbox1\hbox{$_\mathchar'454$}\setbox2\hbox{p}%
  \pl@right=#2\wd0 \advance\pl@right by-#3\wd1
  \pl@down=#5\ht1 \advance\pl@down by-#4\ht0
  \pl@left=\pl@right \advance\pl@left by\wd1
  \pl@temp=-\pl@down \advance\pl@temp by\dp2 \dp1=\pl@temp
  \leavevmode
  \kern\pl@right\lower\pl@down\box1\kern-\pl@left #1}
%    \end{macrocode}
%  \end{macro}
%
%  \begin{macro}{\aob}
%  \begin{macro}{\Aob}
%  \begin{macro}{\eob}
%  \begin{macro}{\Eob}
%    The ogonek is placed with the letters `a', `A', `e', and `E'.
% \changes{polish-1.2d}{1996/08/18}{Use the constructed version of the
%    characters only in OT1; use proper characters in T1.} 
%    \begin{macrocode}
\DeclareTextCommand{\aob}{OT1}{\sob a{.66}{.20}{0}{.90}}
\DeclareTextCommand{\Aob}{OT1}{\sob A{.80}{.50}{0}{.90}}
\DeclareTextCommand{\eob}{OT1}{\sob e{.50}{.35}{0}{.93}}
\DeclareTextCommand{\Eob}{OT1}{\sob E{.60}{.35}{0}{.90}}
%    \end{macrocode}
%    For the 'new' \texttt{T1} encoding we can provide simpler
%    definitions. 
%    \begin{macrocode}
\DeclareTextCommand{\aob}{T1}{\k a}
\DeclareTextCommand{\Aob}{T1}{\k A}
\DeclareTextCommand{\eob}{T1}{\k e}
\DeclareTextCommand{\Eob}{T1}{\k E}
%    \end{macrocode}
%    Construct the characters by default from the OT1 encoding.
%    \begin{macrocode}
\ProvideTextCommandDefault{\aob}{\UseTextSymbol{OT1}{\aob}}
\ProvideTextCommandDefault{\Aob}{\UseTextSymbol{OT1}{\Aob}}
\ProvideTextCommandDefault{\eob}{\UseTextSymbol{OT1}{\eob}}
\ProvideTextCommandDefault{\Eob}{\UseTextSymbol{OT1}{\Eob}}
%    \end{macrocode}
%  \end{macro}
%  \end{macro}
%  \end{macro}
%  \end{macro}
%
%  \begin{macro}{\spb}
%    The macro |\spb| is used to put the `poprzeczka' in the
%    right place.
%
% \changes{polish-1.2d}{1996/08/18}{\cs{spb} is meant to be used in
%    horizontal mode; so leave vertical mode if necessary} 
%    \begin{macrocode}
\def\spb#1#2#3#4#5{%
  \setbox0\hbox{#1}\setbox1\hbox{\char'023}%
  \pl@right=#2\wd0 \advance\pl@right by-#3\wd1
  \pl@down=#5\ht1 \advance\pl@down by-#4\ht0
  \pl@left=\pl@right \advance\pl@left by\wd1
  \ht1=\pl@down \dp1=-\pl@down
  \leavevmode
  \kern\pl@right\lower\pl@down\box1\kern-\pl@left #1}
%    \end{macrocode}
%  \end{macro}
%
%  \begin{macro}{\skb}
%    The macro |\skb| is used to put the `kropka' in the
%    right place.
%
% \changes{polish-1.2d}{1996/08/18}{\cs{skb} is meant to be used in
%    horizontal mode; so leave vertical mode if necessary} 
%    \begin{macrocode}
\def\skb#1#2#3#4#5{%
  \setbox0\hbox{#1}\setbox1\hbox{\char'056}%
  \pl@right=#2\wd0 \advance\pl@right by-#3\wd1
  \pl@down=#5\ht1 \advance\pl@down by-#4\ht0
  \pl@left=\pl@right \advance\pl@left by\wd1
  \leavevmode
  \kern\pl@right\lower\pl@down\box1\kern-\pl@left #1}
%    \end{macrocode}
%  \end{macro}
%
%  \begin{macro}{\textpl}
%    For the `poprzeczka' and the `kropka' in text fonts we don't need
%    any special coding, but we can (almost) use what is already
%    available.
%
%    \begin{macrocode}
\def\textpl{%
  \def\lpb{\plll}%
  \def\Lpb{\pLLL}%
  \def\zkb{\.z}%
  \def\Zkb{\.Z}}
%    \end{macrocode}
%    Initially we assume that typesetting is done with text fonts.
% \changes{polish-1.0a}{1993/11/05}{Initially execute `textpl}
%    \begin{macrocode}
\textpl
%    \end{macrocode}
%
%    \begin{macrocode}
\let\lll=\l \let\LLL=\L
\def\plll{\lll}
\def\pLLL{\LLL}
%    \end{macrocode}
%  \end{macro}
%
%  \begin{macro}{\telepl}
%    But for the `teletype' font in `OT1' encoding we have to take some
%    special actions, involving the macros defined above.
%
%    \begin{macrocode}
\def\telepl{%
  \def\lpb{\spb l{.45}{.5}{.4}{.8}}%
  \def\Lpb{\spb L{.23}{.5}{.4}{.8}}%
  \def\zkb{\skb z{.5}{.5}{1.2}{0}}%
  \def\Zkb{\skb Z{.5}{.5}{1.1}{0}}}
%    \end{macrocode}
%  \end{macro}
%
%    To activate these codes the font changing commands as they are
%    defined in \LaTeX\ are modified. The same is done for plain
%    \TeX's font changing commands.
%
%    When |\selectfont| is undefined the current format is spposed to be
%    either plain (based) or \LaTeX$\:$2.09.
% \changes{polish-1.2a}{1995/06/06}{Don't modify \cs{rm} and friends for
%    \LaTeXe, take \cs{selectfont } instead}
%    \begin{macrocode}
\ifx\selectfont\@undefined
  \ifx\prm\@undefined \addto\rm{\textpl}\else \addto\prm{\textpl}\fi
  \ifx\pit\@undefined \addto\it{\textpl}\else \addto\pit{\textpl}\fi
  \ifx\pbf\@undefined \addto\bf{\textpl}\else \addto\pbf{\textpl}\fi
  \ifx\psl\@undefined \addto\sl{\textpl}\else \addto\psl{\textpl}\fi
  \ifx\psf\@undefined                   \else \addto\psf{\textpl}\fi
  \ifx\psc\@undefined                   \else \addto\psc{\textpl}\fi
  \ifx\ptt\@undefined \addto\tt{\telepl}\else \addto\ptt{\telepl}\fi
\else
%    \end{macrocode}
%    When |\selectfont| exists we assume \LaTeXe.
%    \begin{macrocode}
  \expandafter\addto\csname selectfont \endcsname{%
    \csname\f@encoding @pl\endcsname}
\fi
%    \end{macrocode}
%    Currently we support the OT1 and T1 encodings. For T1 we don't
%    have to make a difference between typewriter fonts and other
%    fonts, they all have the same glyphs.
%    \begin{macrocode}
\expandafter\let\csname T1@pl\endcsname\textpl
%    \end{macrocode}
%    For OT1 we need to check the current font family, stored in
%    |\f@family|. Unfortunately we need a hack as |\ttdefault| is
%    defined as a |\long| macro, while |\f@family| is not.
%    \begin{macrocode}
\expandafter\def\csname OT1@pl\endcsname{%
  \long\edef\curr@family{\f@family}%
  \ifx\curr@family\ttdefault
    \telepl
  \else
    \textpl
  \fi}
%    \end{macrocode}
%
%  \begin{macro}{\dq}
%    We save the original double quote character in |\dq| to keep
%    it available, the math accent |\"| can now be typed as |"|.
%    \begin{macrocode}
\begingroup \catcode`\"12
\def\x{\endgroup
  \def\dq{"}}
\x
%    \end{macrocode}
%  \end{macro}
%
%    Now we can define the doublequote macros for diacritics,
% \changes{polish-1.1d}{1995/01/31}{The dqmacro for C used \cs{'c}}
%    \begin{macrocode}
\declare@shorthand{polish}{"a}{\textormath{\aob}{\ddot a}}
\declare@shorthand{polish}{"A}{\textormath{\Aob}{\ddot A}}
\declare@shorthand{polish}{"c}{\textormath{\'c}{\acute c}}
\declare@shorthand{polish}{"C}{\textormath{\'C}{\acute C}}
\declare@shorthand{polish}{"e}{\textormath{\eob}{\ddot e}}
\declare@shorthand{polish}{"E}{\textormath{\Eob}{\ddot E}}
\declare@shorthand{polish}{"l}{\textormath{\lpb}{\ddot l}}
\declare@shorthand{polish}{"L}{\textormath{\Lpb}{\ddot L}}
\declare@shorthand{polish}{"n}{\textormath{\'n}{\acute n}}
\declare@shorthand{polish}{"N}{\textormath{\'N}{\acute N}}
\declare@shorthand{polish}{"o}{\textormath{\'o}{\acute o}}
\declare@shorthand{polish}{"O}{\textormath{\'O}{\acute O}}
\declare@shorthand{polish}{"s}{\textormath{\'s}{\acute s}}
\declare@shorthand{polish}{"S}{\textormath{\'S}{\acute S}}
%    \end{macrocode}
%  \begin{macro}{\polishrz}
%  \begin{macro}{\polishzx}
% \changes{polish-1.2j}{2000/11/11}{Added support for two
%    notationstyles for kropka and accented z}
%    The command |\polishrz| defines the shorthands |"r|, |"z| and
%    |"x| to produce pointed z, accented z and |"x|. This is the
%    default as these shorthands were defined by this language
%    definition file for quite some time.
%    \begin{macrocode}
\newcommand*{\polishrz}{%
  \declare@shorthand{polish}{"r}{\textormath{\zkb}{\ddot r}}%
  \declare@shorthand{polish}{"R}{\textormath{\Zkb}{\ddot R}}%
  \declare@shorthand{polish}{"z}{\textormath{\'z}{\acute z}}%
  \declare@shorthand{polish}{"Z}{\textormath{\'Z}{\acute Z}}%
  \declare@shorthand{polish}{"x}{\dq x}%
  \declare@shorthand{polish}{"X}{\dq X}%
  }
\polishrz
%    \end{macrocode}
%    The command |\polishzx| switches to a different set of
%    shorthands, |"z|, |"x| and |"r| to produce pointed z, accented z
%    and |"r|; a different shorthand notation also in use.
% \changes{polish-1.2k}{2003/10/12}{Fixed a typo}
%    \begin{macrocode}
\newcommand*{\polishzx}{%
  \declare@shorthand{polish}{"z}{\textormath{\zkb}{\ddot z}}%
  \declare@shorthand{polish}{"Z}{\textormath{\Zkb}{\ddot Z}}%
  \declare@shorthand{polish}{"x}{\textormath{\'z}{\acute x}}%
  \declare@shorthand{polish}{"X}{\textormath{\'Z}{\acute X}}%
  \declare@shorthand{polish}{"r}{\dq r}%
  \declare@shorthand{polish}{"R}{\dq R}%
  }
%    \end{macrocode}
%  \end{macro}
%  \end{macro}
%
%    Then we define access to two forms of quotation marks, similar
%    to the german and french quotation marks.
% \changes{polish-1.2e}{1997/04/03}{Removed empty groups after
%    double quote and guillemot characters}
% \changes{polish-1.2l}{2004/02/18}{Changed closing quote}
%    \begin{macrocode}
\declare@shorthand{polish}{"`}{%
  \textormath{\quotedblbase}{\mbox{\quotedblbase}}}
\declare@shorthand{polish}{"'}{%
  \textormath{\textquotedblright}{\mbox{\textquotedblright}}}
\declare@shorthand{polish}{"<}{%
  \textormath{\guillemotleft}{\mbox{\guillemotleft}}}
\declare@shorthand{polish}{">}{%
  \textormath{\guillemotright}{\mbox{\guillemotright}}}
%    \end{macrocode}
%    then we define two shorthands to be able to specify hyphenation
%    breakpoints that behavew a little different from |\-|.
%    \begin{macrocode}
\declare@shorthand{polish}{"-}{\nobreak-\bbl@allowhyphens}
\declare@shorthand{polish}{""}{\hskip\z@skip}
%    \end{macrocode}
%    And we want to have a shorthand for disabling a ligature.
%    \begin{macrocode}
\declare@shorthand{polish}{"|}{%
  \textormath{\discretionary{-}{}{\kern.03em}}{}}
%    \end{macrocode}
%
%
%  \begin{macro}{\mdqon}
%  \begin{macro}{\mdqoff}
%    All that's left to do now is to  define a couple of commands
%    for reasons of compatibility with \file{polish.tex}.
% \changes{polish-1.2f}{1998/06/07}{Now use \cs{shorthandon} and
%    \cs{shorthandoff}} 
%    \begin{macrocode}
\def\mdqon{\shorthandon{"}}
\def\mdqoff{\shorthandoff{"}}
%    \end{macrocode}
%  \end{macro}
%  \end{macro}
%
%    The macro |\ldf@finish| takes care of looking for a
%    configuration file, setting the main language to be switched on
%    at |\begin{document}| and resetting the category code of
%    \texttt{@} to its original value.
% \changes{polish-1.2d}{1996/11/03}{Now use \cs{ldf@finish} to wrap
%    up} 
%    \begin{macrocode}
\ldf@finish{polish}
%</code>
%    \end{macrocode}
%
% \Finale
%
%% \CharacterTable
%%  {Upper-case    \A\B\C\D\E\F\G\H\I\J\K\L\M\N\O\P\Q\R\S\T\U\V\W\X\Y\Z
%%   Lower-case    \a\b\c\d\e\f\g\h\i\j\k\l\m\n\o\p\q\r\s\t\u\v\w\x\y\z
%%   Digits        \0\1\2\3\4\5\6\7\8\9
%%   Exclamation   \!     Double quote  \"     Hash (number) \#
%%   Dollar        \$     Percent       \%     Ampersand     \&
%%   Acute accent  \'     Left paren    \(     Right paren   \)
%%   Asterisk      \*     Plus          \+     Comma         \,
%%   Minus         \-     Point         \.     Solidus       \/
%%   Colon         \:     Semicolon     \;     Less than     \<
%%   Equals        \=     Greater than  \>     Question mark \?
%%   Commercial at \@     Left bracket  \[     Backslash     \\
%%   Right bracket \]     Circumflex    \^     Underscore    \_
%%   Grave accent  \`     Left brace    \{     Vertical bar  \|
%%   Right brace   \}     Tilde         \~}
%%
\endinput
}
\DeclareOption{portuges}{%%
%% This file will generate fast loadable files and documentation
%% driver files from the doc files in this package when run through
%% LaTeX or TeX.
%%
%% Copyright 1989-2008 Johannes L. Braams and any individual authors
%% listed elsewhere in this file.  All rights reserved.
%% 
%% This file is part of the Babel system.
%% --------------------------------------
%% 
%% It may be distributed and/or modified under the
%% conditions of the LaTeX Project Public License, either version 1.3
%% of this license or (at your option) any later version.
%% The latest version of this license is in
%%   http://www.latex-project.org/lppl.txt
%% and version 1.3 or later is part of all distributions of LaTeX
%% version 2003/12/01 or later.
%% 
%% This work has the LPPL maintenance status "maintained".
%% 
%% The Current Maintainer of this work is Johannes Braams.
%% 
%% The list of all files belonging to the LaTeX base distribution is
%% given in the file `manifest.bbl. See also `legal.bbl' for additional
%% information.
%% 
%% The list of derived (unpacked) files belonging to the distribution
%% and covered by LPPL is defined by the unpacking scripts (with
%% extension .ins) which are part of the distribution.
%%
%% --------------- start of docstrip commands ------------------
%%
\def\filedate{1999/04/11}
\def\batchfile{portuges.ins}
\input docstrip.tex

{\ifx\generate\undefined
\Msg{**********************************************}
\Msg{*}
\Msg{* This installation requires docstrip}
\Msg{* version 2.3c or later.}
\Msg{*}
\Msg{* An older version of docstrip has been input}
\Msg{*}
\Msg{**********************************************}
\errhelp{Move or rename old docstrip.tex.}
\errmessage{Old docstrip in input path}
\batchmode
\csname @@end\endcsname
\fi}

\declarepreamble\mainpreamble
This is a generated file.

Copyright 1989-2008 Johannes L. Braams and any individual authors
listed elsewhere in this file.  All rights reserved.

This file was generated from file(s) of the Babel system.
---------------------------------------------------------

It may be distributed and/or modified under the
conditions of the LaTeX Project Public License, either version 1.3
of this license or (at your option) any later version.
The latest version of this license is in
  http://www.latex-project.org/lppl.txt
and version 1.3 or later is part of all distributions of LaTeX
version 2003/12/01 or later.

This work has the LPPL maintenance status "maintained".

The Current Maintainer of this work is Johannes Braams.

This file may only be distributed together with a copy of the Babel
system. You may however distribute the Babel system without
such generated files.

The list of all files belonging to the Babel distribution is
given in the file `manifest.bbl'. See also `legal.bbl for additional
information.

The list of derived (unpacked) files belonging to the distribution
and covered by LPPL is defined by the unpacking scripts (with
extension .ins) which are part of the distribution.
\endpreamble

\declarepreamble\fdpreamble
This is a generated file.

Copyright 1989-2008 Johannes L. Braams and any individual authors
listed elsewhere in this file.  All rights reserved.

This file was generated from file(s) of the Babel system.
---------------------------------------------------------

It may be distributed and/or modified under the
conditions of the LaTeX Project Public License, either version 1.3
of this license or (at your option) any later version.
The latest version of this license is in
  http://www.latex-project.org/lppl.txt
and version 1.3 or later is part of all distributions of LaTeX
version 2003/12/01 or later.

This work has the LPPL maintenance status "maintained".

The Current Maintainer of this work is Johannes Braams.

This file may only be distributed together with a copy of the Babel
system. You may however distribute the Babel system without
such generated files.

The list of all files belonging to the Babel distribution is
given in the file `manifest.bbl'. See also `legal.bbl for additional
information.

In particular, permission is granted to customize the declarations in
this file to serve the needs of your installation.

However, NO PERMISSION is granted to distribute a modified version
of this file under its original name.

\endpreamble

\keepsilent

\usedir{tex/generic/babel} 

\usepreamble\mainpreamble
\generate{\file{portuges.ldf}{\from{portuges.dtx}{code}}
          }
\usepreamble\fdpreamble

\ifToplevel{
\Msg{***********************************************************}
\Msg{*}
\Msg{* To finish the installation you have to move the following}
\Msg{* files into a directory searched by TeX:}
\Msg{*}
\Msg{* \space\space All *.def, *.fd, *.ldf, *.sty}
\Msg{*}
\Msg{* To produce the documentation run the files ending with}
\Msg{* '.dtx' and `.fdd' through LaTeX.}
\Msg{*}
\Msg{* Happy TeXing}
\Msg{***********************************************************}
}
 
\endinput
}
\DeclareOption{portuguese}{%%
%% This file will generate fast loadable files and documentation
%% driver files from the doc files in this package when run through
%% LaTeX or TeX.
%%
%% Copyright 1989-2008 Johannes L. Braams and any individual authors
%% listed elsewhere in this file.  All rights reserved.
%% 
%% This file is part of the Babel system.
%% --------------------------------------
%% 
%% It may be distributed and/or modified under the
%% conditions of the LaTeX Project Public License, either version 1.3
%% of this license or (at your option) any later version.
%% The latest version of this license is in
%%   http://www.latex-project.org/lppl.txt
%% and version 1.3 or later is part of all distributions of LaTeX
%% version 2003/12/01 or later.
%% 
%% This work has the LPPL maintenance status "maintained".
%% 
%% The Current Maintainer of this work is Johannes Braams.
%% 
%% The list of all files belonging to the LaTeX base distribution is
%% given in the file `manifest.bbl. See also `legal.bbl' for additional
%% information.
%% 
%% The list of derived (unpacked) files belonging to the distribution
%% and covered by LPPL is defined by the unpacking scripts (with
%% extension .ins) which are part of the distribution.
%%
%% --------------- start of docstrip commands ------------------
%%
\def\filedate{1999/04/11}
\def\batchfile{portuges.ins}
\input docstrip.tex

{\ifx\generate\undefined
\Msg{**********************************************}
\Msg{*}
\Msg{* This installation requires docstrip}
\Msg{* version 2.3c or later.}
\Msg{*}
\Msg{* An older version of docstrip has been input}
\Msg{*}
\Msg{**********************************************}
\errhelp{Move or rename old docstrip.tex.}
\errmessage{Old docstrip in input path}
\batchmode
\csname @@end\endcsname
\fi}

\declarepreamble\mainpreamble
This is a generated file.

Copyright 1989-2008 Johannes L. Braams and any individual authors
listed elsewhere in this file.  All rights reserved.

This file was generated from file(s) of the Babel system.
---------------------------------------------------------

It may be distributed and/or modified under the
conditions of the LaTeX Project Public License, either version 1.3
of this license or (at your option) any later version.
The latest version of this license is in
  http://www.latex-project.org/lppl.txt
and version 1.3 or later is part of all distributions of LaTeX
version 2003/12/01 or later.

This work has the LPPL maintenance status "maintained".

The Current Maintainer of this work is Johannes Braams.

This file may only be distributed together with a copy of the Babel
system. You may however distribute the Babel system without
such generated files.

The list of all files belonging to the Babel distribution is
given in the file `manifest.bbl'. See also `legal.bbl for additional
information.

The list of derived (unpacked) files belonging to the distribution
and covered by LPPL is defined by the unpacking scripts (with
extension .ins) which are part of the distribution.
\endpreamble

\declarepreamble\fdpreamble
This is a generated file.

Copyright 1989-2008 Johannes L. Braams and any individual authors
listed elsewhere in this file.  All rights reserved.

This file was generated from file(s) of the Babel system.
---------------------------------------------------------

It may be distributed and/or modified under the
conditions of the LaTeX Project Public License, either version 1.3
of this license or (at your option) any later version.
The latest version of this license is in
  http://www.latex-project.org/lppl.txt
and version 1.3 or later is part of all distributions of LaTeX
version 2003/12/01 or later.

This work has the LPPL maintenance status "maintained".

The Current Maintainer of this work is Johannes Braams.

This file may only be distributed together with a copy of the Babel
system. You may however distribute the Babel system without
such generated files.

The list of all files belonging to the Babel distribution is
given in the file `manifest.bbl'. See also `legal.bbl for additional
information.

In particular, permission is granted to customize the declarations in
this file to serve the needs of your installation.

However, NO PERMISSION is granted to distribute a modified version
of this file under its original name.

\endpreamble

\keepsilent

\usedir{tex/generic/babel} 

\usepreamble\mainpreamble
\generate{\file{portuges.ldf}{\from{portuges.dtx}{code}}
          }
\usepreamble\fdpreamble

\ifToplevel{
\Msg{***********************************************************}
\Msg{*}
\Msg{* To finish the installation you have to move the following}
\Msg{* files into a directory searched by TeX:}
\Msg{*}
\Msg{* \space\space All *.def, *.fd, *.ldf, *.sty}
\Msg{*}
\Msg{* To produce the documentation run the files ending with}
\Msg{* '.dtx' and `.fdd' through LaTeX.}
\Msg{*}
\Msg{* Happy TeXing}
\Msg{***********************************************************}
}
 
\endinput
}
\DeclareOption{romanian}{% \iffalse meta-comment
%
% Copyright 1989-2005 Johannes L. Braams and any individual authors
% listed elsewhere in this file.  All rights reserved.
% 
% This file is part of the Babel system.
% --------------------------------------
% 
% It may be distributed and/or modified under the
% conditions of the LaTeX Project Public License, either version 1.3
% of this license or (at your option) any later version.
% The latest version of this license is in
%   http://www.latex-project.org/lppl.txt
% and version 1.3 or later is part of all distributions of LaTeX
% version 2003/12/01 or later.
% 
% This work has the LPPL maintenance status "maintained".
% 
% The Current Maintainer of this work is Johannes Braams.
% 
% The list of all files belonging to the Babel system is
% given in the file `manifest.bbl. See also `legal.bbl' for additional
% information.
% 
% The list of derived (unpacked) files belonging to the distribution
% and covered by LPPL is defined by the unpacking scripts (with
% extension .ins) which are part of the distribution.
% \fi
% \CheckSum{89}
% \iffalse
%    Tell the \LaTeX\ system who we are and write an entry on the
%    transcript.
%<*dtx>
\ProvidesFile{romanian.dtx}
%</dtx>
%<code>\ProvidesLanguage{romanian}
%\fi
%\ProvidesFile{romanian.dtx}
        [2005/03/31 v1.2l Romanian support from the babel system]
%\iffalse
%% File `romanian.dtx'
%% Babel package for LaTeX version 2e
%% Copyright (C) 1989 - 2005
%%           by Johannes Braams, TeXniek
%
%% Romanian Language Definition File
%% Copyright (C) 1989 - 2005
%%           by Johannes Braams, TeXniek
%
%% Please report errors to: J.L. Braams
%%                          babel at braams.cistron.nl
%
%    This file is part of the babel system, it provides the source
%    code for the Romanian language definition file. A contribution
%    was made by Umstatter Horst (hhu at cernvm.cern.ch) and Robert
%    Juhasz (robertj at uni-paderborn.de)
%<*filedriver>
\documentclass{ltxdoc}
\newcommand*\TeXhax{\TeX hax}
\newcommand*\babel{\textsf{babel}}
\newcommand*\langvar{$\langle \it lang \rangle$}
\newcommand*\note[1]{}
\newcommand*\Lopt[1]{\textsf{#1}}
\newcommand*\file[1]{\texttt{#1}}
\begin{document}
 \DocInput{romanian.dtx}
\end{document}
%</filedriver>
%\fi
%
% \GetFileInfo{romanian.dtx}
%
% \changes{romanian-1.0a}{1991/07/15}{Renamed babel.sty in babel.com}
% \changes{romanian-1.1}{1992/02/16}{Brought up-to-date with babel 3.2a}
% \changes{romanian-1.2}{1994/02/27}{Update for LaTeX2e}
% \changes{romanian-1.2d}{1994/06/26}{Removed the use of \cs{filedate}
%    and moved identification after the loading of babel.def}
% \changes{romanian-1.2e}{1995/05/25}{Updated for babel release 3.5}
% \changes{romanian-1.2h}{1996/10/10}{Replaced \cs{undefined} with
%    \cs{@undefined} and \cs{empty} with \cs{@empty} for consistency
%    with \LaTeX, moved the definition of \cs{atcatcode} right to the
%    beginning.}
%
%  \section{The Romanian language}
%
%    The file \file{\filename}\footnote{The file described in this
%    section has version number \fileversion\ and was last revised on
%    \filedate.  A contribution was made by Umstatter Horst
%    (\texttt{hhu@cernvm.cern.ch}).}  defines all the
%    language-specific macros for the Romanian language.
%
%    For this language currently no special definitions are needed or
%    available.
%
% \StopEventually{}
%
%    The macro |\LdfInit| takes care of preventing that this file is
%    loaded more than once, checking the category code of the
%    \texttt{@} sign, etc.
% \changes{romanian-1.2h}{1996/11/03}{Now use \cs{LdfInit} to perform
%    initial checks} 
%    \begin{macrocode}
%<*code>
\LdfInit{romanian}\captionsromanian
%    \end{macrocode}
%
%    When this file is read as an option, i.e. by the |\usepackage|
%    command, \texttt{romanian} will be an `unknown' language in which
%    case we have to make it known. So we check for the existence of
%    |\l@romanian| to see whether we have to do something here.
%
% \changes{romanian-1.0b}{1991/10/29}{Removed use of
%    \cs{@ifundefined}}
% \changes{romanian-1.1}{1992/02/16}{Added a warning when no
%    hyphenation patterns were loaded.}
% \changes{romanian-1.2d}{1994/06/26}{Now use \cs{@nopatterns} to
%    produce the warning}
%    \begin{macrocode}
\ifx\l@romanian\@undefined
    \@nopatterns{Romanian}
    \adddialect\l@romanian0\fi
%    \end{macrocode}
%
%    The next step consists of defining commands to switch to (and
%    from) the Romanian language.
%
% \begin{macro}{\captionsromanian}
%    The macro |\captionsromanian| defines all strings used in the
%    four standard documentclasses provided with \LaTeX.
% \changes{romanian-1.1}{1992/02/16}{Added \cs{seename}, \cs{alsoname}
%    and \cs{prefacename}}
% \changes{romanian-1.1}{1992/02/17}{Translation errors found by
%    Robert Juhasz fixed}
% \changes{romanian-1.1}{1993/07/15}{\cs{headpagename} should be
%    \cs{pagename}}
% \changes{romanian-1.2f}{1995/07/04}{Added \cs{proofname} for
%    AMS-\LaTeX}
% \changes{romanian-1.2g}{1995/11/10}{Added translation of `Proof'}
% \changes{romanian-1.2k}{2000/09/20}{Added \cs{glossaryname}}
% \changes{romanian-1.2l}{2003/06/10}{Added translation for Glossary}
%    \begin{macrocode}ls
\addto\captionsromanian{%
  \def\prefacename{Prefa\c{t}\u{a}}%
  \def\refname{Bibliografie}%
  \def\abstractname{Rezumat}%
  \def\bibname{Bibliografie}%
  \def\chaptername{Capitolul}%
  \def\appendixname{Anexa}%
  \def\contentsname{Cuprins}%
  \def\listfigurename{List\u{a} de figuri}%
  \def\listtablename{List\u{a} de tabele}%
  \def\indexname{Glosar}%
  \def\figurename{Figura}%    % sau Plan\c{s}a
  \def\tablename{Tabela}%
  \def\partname{Partea}%
  \def\enclname{Anex\u{a}}%   % sau Anexe
  \def\ccname{Copie}%
  \def\headtoname{Pentru}%
  \def\pagename{Pagina}%
  \def\seename{Vezi}%
  \def\alsoname{Vezi de asemenea}%
  \def\proofname{Demonstra\c{t}ie} %
  \def\glossaryname{Glosar}%
  }%
%    \end{macrocode}
% \end{macro}
%
% \begin{macro}{\dateromanian}
%    The macro |\dateromanian| redefines the command |\today| to
%    produce Romanian dates.
% \changes{romanian-1.1}{1992/02/17}{Translation errors found by Robert
%    Juhasz fixed}
% \changes{romanian-1.2i}{1997/10/01}{Use \cs{edef} to define
%    \cs{today} to save memory}
% \changes{romanian-1.2i}{1998/03/28}{use \cs{def} instead of
%    \cs{edef}} 
%    \begin{macrocode}
\def\dateromanian{%
  \def\today{\number\day~\ifcase\month\or
    ianuarie\or februarie\or martie\or aprilie\or mai\or
    iunie\or iulie\or august\or septembrie\or octombrie\or
    noiembrie\or decembrie\fi
    \space \number\year}}
%    \end{macrocode}
% \end{macro}
%
% \begin{macro}{\extrasromanian}
% \begin{macro}{\noextrasromanian}
%    The macro |\extrasromanian| will perform all the extra
%    definitions needed for the Romanian language. The macro
%    |\noextrasromanian| is used to cancel the actions of
%    |\extrasromanian| For the moment these macros are empty but they
%    are defined for compatibility with the other language definition
%    files.
%
%    \begin{macrocode}
\addto\extrasromanian{}
\addto\noextrasromanian{}
%    \end{macrocode}
% \end{macro}
% \end{macro}
%
%    The macro |\ldf@finish| takes care of looking for a
%    configuration file, setting the main language to be switched on
%    at |\begin{document}| and resetting the category code of
%    \texttt{@} to its original value.
% \changes{romanian-1.2h}{1996/11/03}{Now use \cs{ldf@finish} to wrap
%    up} 
%    \begin{macrocode}
\ldf@finish{romanian}
%</code>
%    \end{macrocode}
%
% \Finale
%%
%% \CharacterTable
%%  {Upper-case    \A\B\C\D\E\F\G\H\I\J\K\L\M\N\O\P\Q\R\S\T\U\V\W\X\Y\Z
%%   Lower-case    \a\b\c\d\e\f\g\h\i\j\k\l\m\n\o\p\q\r\s\t\u\v\w\x\y\z
%%   Digits        \0\1\2\3\4\5\6\7\8\9
%%   Exclamation   \!     Double quote  \"     Hash (number) \#
%%   Dollar        \$     Percent       \%     Ampersand     \&
%%   Acute accent  \'     Left paren    \(     Right paren   \)
%%   Asterisk      \*     Plus          \+     Comma         \,
%%   Minus         \-     Point         \.     Solidus       \/
%%   Colon         \:     Semicolon     \;     Less than     \<
%%   Equals        \=     Greater than  \>     Question mark \?
%%   Commercial at \@     Left bracket  \[     Backslash     \\
%%   Right bracket \]     Circumflex    \^     Underscore    \_
%%   Grave accent  \`     Left brace    \{     Vertical bar  \|
%%   Right brace   \}     Tilde         \~}
%%
\endinput
}
\DeclareOption{russian}{%%
%% This file will generate fast loadable files and documentation
%% driver files from the doc files in this package when run through
%% LaTeX or TeX.
%%
%% Copyright 1989-2008 Johannes L. Braams and any individual authors
%% listed elsewhere in this file.  All rights reserved.
%% 
%% This file is part of the Babel system.
%% --------------------------------------
%% 
%% It may be distributed and/or modified under the
%% conditions of the LaTeX Project Public License, either version 1.3
%% of this license or (at your option) any later version.
%% The latest version of this license is in
%%   http://www.latex-project.org/lppl.txt
%% and version 1.3 or later is part of all distributions of LaTeX
%% version 2003/12/01 or later.
%% 
%% This work has the LPPL maintenance status "maintained".
%% 
%% The Current Maintainer of this work is Johannes Braams.
%% 
%% The list of all files belonging to the LaTeX base distribution is
%% given in the file `manifest.bbl. See also `legal.bbl' for additional
%% information.
%% 
%% The list of derived (unpacked) files belonging to the distribution
%% and covered by LPPL is defined by the unpacking scripts (with
%% extension .ins) which are part of the distribution.
%%
%% --------------- start of docstrip commands ------------------
%%
\def\filedate{1999/03/13}
\def\batchfile{russianb.ins}
\input docstrip.tex

{\ifx\generate\undefined
\Msg{**********************************************}
\Msg{*}
\Msg{* This installation requires docstrip}
\Msg{* version 2.3c or later.}
\Msg{*}
\Msg{* An older version of docstrip has been input}
\Msg{*}
\Msg{**********************************************}
\errhelp{Move or rename old docstrip.tex.}
\errmessage{Old docstrip in input path}
\batchmode
\csname @@end\endcsname
\fi}

\declarepreamble\mainpreamble
This is a generated file.

Copyright 1989-2008 Johannes L. Braams and any individual authors
listed elsewhere in this file.  All rights reserved.

This file was generated from file(s) of the Babel system.
---------------------------------------------------------

It may be distributed and/or modified under the
conditions of the LaTeX Project Public License, either version 1.3
of this license or (at your option) any later version.
The latest version of this license is in
  http://www.latex-project.org/lppl.txt
and version 1.3 or later is part of all distributions of LaTeX
version 2003/12/01 or later.

This work has the LPPL maintenance status "maintained".

The Current Maintainer of this work is Johannes Braams.

This file may only be distributed together with a copy of the Babel
system. You may however distribute the Babel system without
such generated files.

The list of all files belonging to the Babel distribution is
given in the file `manifest.bbl'. See also `legal.bbl for additional
information.

The list of derived (unpacked) files belonging to the distribution
and covered by LPPL is defined by the unpacking scripts (with
extension .ins) which are part of the distribution.
\endpreamble

\declarepreamble\fdpreamble

This is a generated file.

Copyright 1989-2008 Johannes L. Braams and any individual authors
listed elsewhere in this file.  All rights reserved.

This file was generated from file(s) of the Babel system.
---------------------------------------------------------

It may be distributed and/or modified under the
conditions of the LaTeX Project Public License, either version 1.3
of this license or (at your option) any later version.
The latest version of this license is in
  http://www.latex-project.org/lppl.txt
and version 1.3 or later is part of all distributions of LaTeX
version 2003/12/01 or later.

This work has the LPPL maintenance status "maintained".

The Current Maintainer of this work is Johannes Braams.

This file may only be distributed together with a copy of the Babel
system. You may however distribute the Babel system without
such generated files.

The list of all files belonging to the Babel distribution is
given in the file `manifest.bbl'. See also `legal.bbl for additional
information.

In particular, permission is granted to customize the declarations in
this file to serve the needs of your installation.

However, NO PERMISSION is granted to distribute a modified version
of this file under its original name.

\endpreamble

\usedir{tex/generic/babel}
\keepsilent
 
\usepreamble\mainpreamble

\generate{\file{russianb.ldf}{\from{russianb.dtx}{code}}}
 
\ifToplevel{
\Msg{***********************************************************}
\Msg{*}
\Msg{* To finish the installation you have to move the following}
\Msg{* files into a directory searched by TeX:}
\Msg{*}
\Msg{* \space\space All *.fd}
\Msg{*}
\Msg{* To produce the documentation run the files ending with}
\Msg{* `.fdd' through LaTeX.}
\Msg{*}
\Msg{* Happy TeXing}
\Msg{***********************************************************}
}
 
\endinput
}
%    \end{macrocode}
%^^A \changes{babel~3.7a}{1997/02/07}{Added the \Lopt{sanskrit} option}
% \changes{babel~3.6i}{1997/02/21}{Added the \Lopt{ukrainian} option}
%^^A \changes{babel~3.7a}{1998/03/30}{Added the \Lopt{tamil} option}
%    \begin{macrocode}
%^^A\DeclareOption{sanskrit}{\input{sanskrit.ldf}}
\DeclareOption{scottish}{% \iffalse meta-comment
%
% Copyright 1989-2005 Johannes L. Braams and any individual authors
% listed elsewhere in this file.  All rights reserved.
% 
% This file is part of the Babel system.
% --------------------------------------
% 
% It may be distributed and/or modified under the
% conditions of the LaTeX Project Public License, either version 1.3
% of this license or (at your option) any later version.
% The latest version of this license is in
%   http://www.latex-project.org/lppl.txt
% and version 1.3 or later is part of all distributions of LaTeX
% version 2003/12/01 or later.
% 
% This work has the LPPL maintenance status "maintained".
% 
% The Current Maintainer of this work is Johannes Braams.
% 
% The list of all files belonging to the Babel system is
% given in the file `manifest.bbl. See also `legal.bbl' for additional
% information.
% 
% The list of derived (unpacked) files belonging to the distribution
% and covered by LPPL is defined by the unpacking scripts (with
% extension .ins) which are part of the distribution.
% \fi
% \CheckSum{95}
%
% \iffalse
%    Tell the \LaTeX\ system who we are and write an entry on the
%    transcript.
%<*dtx>
\ProvidesFile{scottish.dtx}
%</dtx>
%<code>\ProvidesLanguage{scottish}
%\fi
%\ProvidesFile{scottish.dtx}
        [2005/03/31 v1.0g Scottish support from the babel system]
%\iffalse
%% File `scottish.dtx'
%% Babel package for LaTeX version 2e
%% Copyright (C) 1989 -- 2005
%%           by Johannes Braams, TeXniek
%
%% Please report errors to: J.L. Braams
%%                          babel at braams.cistron.nl
%
%    This file is part of the babel system, it provides the source
%    code for the Scottish language definition file.
%
%    The Gaidhlig or Scottish Gaelic terms were provided by Fraser
%    Grant \texttt{FRASER@CERNVM}.
%<*filedriver>
\documentclass{ltxdoc}
\newcommand*{\TeXhax}{\TeX hax}
\newcommand*{\babel}{\textsf{babel}}
\newcommand*{\langvar}{$\langle \mathit lang \rangle$}
\newcommand*{\note}[1]{}
\newcommand*{\Lopt}[1]{\textsf{#1}}
\newcommand*{\file}[1]{\texttt{#1}}
\begin{document}
 \DocInput{scottish.dtx}
\end{document}
%</filedriver>
%\fi
% \GetFileInfo{scottish.dtx}
%
% \changes{scottish-1.0b}{1995/06/14}{Corrected typos (PR1652)}
% \changes{scottish-1.0d}{1996/10/10}{Replaced \cs{undefined} with
%    \cs{@undefined} and \cs{empty} with \cs{@empty} for consistency
%    with \LaTeX, moved the definition of \cs{atcatcode} right to the
%    beginning.}
%
%  \section{The Scottish language}
%
%    The file \file{\filename}\footnote{The file described in this
%    section has version number \fileversion\ and was last revised on
%    \filedate. A contribution was made by Fraser Grant
%    (\texttt{FRASER@CERNVM}).}  defines all the language definition
%    macros for the Scottish language.
%
%    For this language currently no special definitions are needed or
%    available.
%
% \StopEventually{}
%
%    The macro |\LdfInit| takes care of preventing that this file is
%    loaded more than once, checking the category code of the
%    \texttt{@} sign, etc.
% \changes{scottish-1.0d}{1996/11/03}{Now use \cs{LdfInit} to perform
%    initial checks} 
%    \begin{macrocode}
%<*code>
\LdfInit{scottish}\captionsscottish
%    \end{macrocode}
%
%    When this file is read as an option, i.e. by the |\usepackage|
%    command, \texttt{scottish} could be an `unknown' language in
%    which case we have to make it known.  So we check for the
%    existence of |\l@scottish| to see whether we have to do something
%    here.
%
%    \begin{macrocode}
\ifx\l@scottish\@undefined
  \@nopatterns{scottish}
  \adddialect\l@scottish0\fi
%    \end{macrocode}
%    The next step consists of defining commands to switch to (and
%    from) the Scottish language.
%
% \begin{macro}{\captionsscottish}
%    The macro |\captionsscottish| defines all strings used in the
%    four standard documentclasses provided with \LaTeX.
% \changes{scottish-1.0c}{1995/07/04}{Added \cs{proofname} for
%    AMS-\LaTeX}
% \changes{scottish-1.0g}{2000/09/20}{Added \cs{glossaryname}}
%    \begin{macrocode}
\addto\captionsscottish{%
  \def\prefacename{Preface}%    <-- needs translation
  \def\refname{Iomraidh}%
  \def\abstractname{Br\`{\i}gh}%
  \def\bibname{Leabhraichean}%
  \def\chaptername{Caibideil}%
  \def\appendixname{Ath-sgr`{\i}obhadh}%
  \def\contentsname{Cl\`ar-obrach}%
  \def\listfigurename{Liosta Dhealbh }%
  \def\listtablename{Liosta Chl\`ar}%
  \def\indexname{Cl\`ar-innse}%
  \def\figurename{Dealbh}%
  \def\tablename{Cl\`ar}%
  \def\partname{Cuid}%
  \def\enclname{a-staigh}%
  \def\ccname{lethbhreac gu}%
  \def\headtoname{gu}%
  \def\pagename{t.d.}%             abrv. `taobh duilleag'
  \def\seename{see}%    <-- needs translation
  \def\alsoname{see also}%    <-- needs translation
  \def\proofname{Proof}%    <-- needs translation
  \def\glossaryname{Glossary}% <-- Needs translation
}
%    \end{macrocode}
% \end{macro}
%
% \begin{macro}{\datescottish}
%    The macro |\datescottish| redefines the command |\today| to
%    produce Scottish dates.
% \changes{scottish-1.0e}{1997/10/01}{Use \cs{edef} to define
%    \cs{today} to save memory}
% \changes{scottish-1.0e}{1998/03/28}{use \cs{def} instead of
%    \cs{edef}} 
%    \begin{macrocode}
\def\datescottish{%
  \def\today{%
    \number\day\space \ifcase\month\or
    am Faoilteach\or an Gearran\or am M\`art\or an Giblean\or
    an C\`eitean\or an t-\`Og mhios\or an t-Iuchar\or
    L\`unasdal\or an Sultuine\or an D\`amhar\or
    an t-Samhainn\or an Dubhlachd\fi
    \space \number\year}}
%    \end{macrocode}
% \end{macro}
%
% \begin{macro}{\extrasscottish}
% \begin{macro}{\noextrasscottish}
%    The macro |\extrasscottish| will perform all the extra
%    definitions needed for the Scottish language. The macro
%    |\noextrasscottish| is used to cancel the actions of
%    |\extrasscottish|.  For the moment these macros are empty but
%    they are defined for compatibility with the other language
%    definition files.
%
%    \begin{macrocode}
\addto\extrasscottish{}
\addto\noextrasscottish{}
%    \end{macrocode}
% \end{macro}
% \end{macro}
%
%    The macro |\ldf@finish| takes care of looking for a
%    configuration file, setting the main language to be switched on
%    at |\begin{document}| and resetting the category code of
%    \texttt{@} to its original value.
% \changes{scottish-1.0d}{1996/11/03}{Now use \cs{ldf@finish} to wrap
%    up} 
%    \begin{macrocode}
\ldf@finish{scottish}
%</code>
%    \end{macrocode}
%
% \Finale
%\endinput
%% \CharacterTable
%%  {Upper-case    \A\B\C\D\E\F\G\H\I\J\K\L\M\N\O\P\Q\R\S\T\U\V\W\X\Y\Z
%%   Lower-case    \a\b\c\d\e\f\g\h\i\j\k\l\m\n\o\p\q\r\s\t\u\v\w\x\y\z
%%   Digits        \0\1\2\3\4\5\6\7\8\9
%%   Exclamation   \!     Double quote  \"     Hash (number) \#
%%   Dollar        \$     Percent       \%     Ampersand     \&
%%   Acute accent  \'     Left paren    \(     Right paren   \)
%%   Asterisk      \*     Plus          \+     Comma         \,
%%   Minus         \-     Point         \.     Solidus       \/
%%   Colon         \:     Semicolon     \;     Less than     \<
%%   Equals        \=     Greater than  \>     Question mark \?
%%   Commercial at \@     Left bracket  \[     Backslash     \\
%%   Right bracket \]     Circumflex    \^     Underscore    \_
%%   Grave accent  \`     Left brace    \{     Vertical bar  \|
%%   Right brace   \}     Tilde         \~}
%%
}
\DeclareOption{serbian}{% \iffalse meta-comme

% Copyright 1989-2005 Johannes L. Braams and any individual autho
% listed elsewhere in this file.  All rights reserve

% This file is part of the Babel syste
% ------------------------------------

% It may be distributed and/or modified under t
% conditions of the LaTeX Project Public License, either version 1
% of this license or (at your option) any later versio
% The latest version of this license is
%   http://www.latex-project.org/lppl.t
% and version 1.3 or later is part of all distributions of LaT
% version 2003/12/01 or late

% This work has the LPPL maintenance status "maintained

% The Current Maintainer of this work is Johannes Braam

% The list of all files belonging to the Babel system
% given in the file `manifest.bbl. See also `legal.bbl' for addition
% informatio

% The list of derived (unpacked) files belonging to the distributi
% and covered by LPPL is defined by the unpacking scripts (wi
% extension .ins) which are part of the distributio
% \
% \CheckSum{26
% \iffal
%    Tell the \LaTeX\ system who we are and write an entry on t
%    transcrip
%<*dt
\ProvidesFile{serbian.dt
%</dt
%<code>\ProvidesLanguage{serbia
%\
%\ProvidesFile{serbian.dt
       [2005/03/31 v1.0d Serbocroatian support from the babel syste
%\iffal
% Babel package for LaTeX version
% Copyright (C) 1989 - 20
%           by Johannes Braams, TeXni

% Please report errors to: J.L. Braa
%                          babel at braams.cistron.

%    This file is part of the babel system, it provides the sour
%    code for the Serbocroatian language definition file.  A contributi
%    was made by Dejan Muhamedagi\'{c} (dejan@yunix.com) and Jankov
%    Slobodan <slobodan@archimed.filfak.ac.ni.y

%<*filedrive
\documentclass{ltxdo
\newcommand*\TeXhax{\TeX ha
\newcommand*\babel{\textsf{babel
\newcommand*\langvar{$\langle \it lang \rangle
\newcommand*\note[1]
\newcommand*\Lopt[1]{\textsf{#1
\newcommand*\file[1]{\texttt{#1
\begin{documen
 \DocInput{serbian.dt
\end{documen
%</filedrive
%\
% \GetFileInfo{serbian.dt
% \changes{serbian-1.0b}{1998/06/16}{Added suggestions for shorthan
%    and so on from Jankovic Sloboda

%  \section{The Serbocroatian languag

%    The file \file{\filename}\footnote{The file described in th
%    section has version number \fileversion\ and was last revised
%    \filedate.  A contribution was made by Dejan Muhamedagi\'{
%    (\texttt{dejan@yunix.com}).}  defines all the language definiti
%    macros for the Serbian language, typeset in a latin script. In
%    future version support for typesetting in a cyrillic script m
%    be adde

%    For this language the character |"| is made active.
%    table~\ref{tab:serbian-quote} an overview is given of i
%    purpose. One of the reasons for this is that in the Serbi
%    language some special characters are use

%    \begin{table}[ht
%     \begin{cente
%     \begin{tabular}{lp{8cm
%      |"c| & |\"c|, also implemented for th
%                  lowercase and uppercase s and z.
%      |"d| & |\dj|, also implemented for |"D|
%      |"-| & an explicit hyphen sign, allowing hyphenati
%                  in the rest of the word.
%      \verb="|= & disable ligature at this position
%      |""| & like |"-|, but producing no hyphen si
%                  (for compund words with hyphen, e.g.\ |x-""y|).
%      |"`| & for Serbian left double quotes (looks like ,,).
%      |"'| & for Serbian right double quotes.
%      |"<| & for French left double quotes (similar to $<<$).
%      |">| & for French right double quotes (similar to $>>$).
%     \end{tabula
%     \caption{The extra definitions ma
%              by \file{serbian.ldf}}\label{tab:serbian-quot
%     \end{cente
%    \end{tabl

%    Apart from defining shorthands we need to make sure taht t
%    first paragraph of each section is intended. Furthermore t
%    following new math operators are defined (|\tg|, |\ctg
%    |\arctg|, |\arcctg|, |\sh|, |\ch|, |\th|, |\cth|, |\arsh
%    |\arch|, |\arth|, |\arcth|, |\Prob|, |\Expect|, |\Variance|

% \StopEventually

%    The macro |\LdfInit| takes care of preventing that this file
%    loaded more than once, checking the category code of t
%    \texttt{@} sign, et
%    \begin{macrocod
%<*cod
\LdfInit{serbian}\captionsserbi
%    \end{macrocod

%    When this file is read as an option, i.e. by the |\usepackag
%    command, \texttt{serbian} will be an `unknown' language in whi
%    case we have to make it known. So we check for the existence
%    |\l@serbian| to see whether we have to do something her

%    \begin{macrocod
\ifx\l@serbian\@undefin
    \@nopatterns{Serbia
    \adddialect\l@serbian0\
%    \end{macrocod

%    The next step consists of defining commands to switch to (a
%    from) the Serbocroatian languag

%  \begin{macro}{\captionsserbia
%    The macro |\captionsserbian| defines all strings us
%    in the four standard documentclasses provided with \LaTe
% \changes{serbian-1.0d}{2000/09/20}{Added \cs{glossaryname
%    \begin{macrocod
\addto\captionsserbian
  \def\prefacename{Predgovor
  \def\refname{Literatura
  \def\abstractname{Sa\v{z}etak
  \def\bibname{Bibliografija
  \def\chaptername{Glava
  \def\appendixname{Dodatak
  \def\contentsname{Sadr\v{z}aj
  \def\listfigurename{Slike
  \def\listtablename{Tabele
  \def\indexname{Indeks
  \def\figurename{Slika
  \def\tablename{Tabela
  \def\partname{Deo
  \def\enclname{Prilozi
  \def\ccname{Kopije
  \def\headtoname{Prima
  \def\pagename{Strana
  \def\seename{Vidi
  \def\alsoname{Vidi tako\dj e
  \def\proofname{Dokaz
  \def\glossaryname{Glossary}% <-- Needs translati

%    \end{macrocod
%  \end{macr

%  \begin{macro}{\dateserbia
%    The macro |\dateserbian| redefines the command |\today|
%    produce Serbocroatian date
%    \begin{macrocod
\def\dateserbian
  \def\today{\number\day .~\ifcase\month\
    januar\or februar\or mart\or april\or maj\
    juni\or juli\or avgust\or septembar\or oktobar\or novembar\
    decembar\fi \space \number\year
%    \end{macrocod
%  \end{macr

%  \begin{macro}{\extrasserbia
%  \begin{macro}{\noextrasserbia
%    The macro |\extrasserbian| will perform all the ext
%    definitions needed for the Serbocroatian language. The mac
%    |\noextrasserbian| is used to cancel the actions
%    |\extrasserbian|.

%    For Serbian the \texttt{"} character is made active. This is do
%    once, later on its definition may vary. Other languages in t
%    same document may also use the \texttt{"} character f
%    shorthands; we specify that the serbian group of shorthan
%    should be use

% \changes{serbian-1.0b}{1998/06/16}{Introduced the active \texttt{"
%    \begin{macrocod
\initiate@active@char{
\addto\extrasserbian{\languageshorthands{serbian
\addto\extrasserbian{\bbl@activate{"
%    \end{macrocod
%    Don't forget to turn the shorthands off agai
% \changes{serbian-1.0c}{1999/12/17}{Deactivate shorthands ouside
%    Serbia
%    \begin{macrocod
\addto\noextrasserbian{\bbl@deactivate{"
%    \end{macrocod
%    First we define shorthands to facilitate the occurence of lette
%    such as \v{c
%    \begin{macrocod
\declare@shorthand{serbian}{"c}{\textormath{\v c}{\check c
\declare@shorthand{serbian}{"d}{\textormath{\dj}{\dj}}
\declare@shorthand{serbian}{"s}{\textormath{\v s}{\check s
\declare@shorthand{serbian}{"z}{\textormath{\v z}{\check z
\declare@shorthand{serbian}{"C}{\textormath{\v C}{\check C
\declare@shorthand{serbian}{"D}{\textormath{\DJ}{\DJ}}
\declare@shorthand{serbian}{"S}{\textormath{\v S}{\check S
\declare@shorthand{serbian}{"Z}{\textormath{\v Z}{\check Z
%    \end{macrocod

%    Then we define access to two forms of quotation marks, simil
%    to the german and french quotation mark
%    \begin{macrocod
\declare@shorthand{serbian}{"`}
  \textormath{\quotedblbase{}}{\mbox{\quotedblbase}
\declare@shorthand{serbian}{"'}
  \textormath{\textquotedblleft{}}{\mbox{\textquotedblleft}
\declare@shorthand{serbian}{"<}
  \textormath{\guillemotleft{}}{\mbox{\guillemotleft}
\declare@shorthand{serbian}{">}
  \textormath{\guillemotright{}}{\mbox{\guillemotright}
%    \end{macrocod
%    then we define two shorthands to be able to specify hyphenati
%    breakpoints that behave a little different from |\-
% \changes{serbian-1.0d}{2000/09/20}{Changed definition of \texttt{"
%    to be the same as for other language
%    \begin{macrocod
\declare@shorthand{serbian}{"-}{\nobreak-\bbl@allowhyphen
\declare@shorthand{serbian}{""}{\hskip\z@ski
%    \end{macrocod
%    And we want to have a shorthand for disabling a ligatur
%    \begin{macrocod
\declare@shorthand{serbian}{"|}
  \textormath{\discretionary{-}{}{\kern.03em}}{
%    \end{macrocod
%  \end{macr
%  \end{macr

%  \begin{macro}{\bbl@frenchinden
%  \begin{macro}{\bbl@nonfrenchinden
%    In Serbian the first paragraph of each section should be indente
%    Add this code only in \LaTe
%    \begin{macrocod
\ifx\fmtname plain \el
  \let\@aifORI\@afterindentfal
  \def\bbl@frenchindent{\let\@afterindentfalse\@afterindenttr
                        \@afterindenttru
  \def\bbl@nonfrenchindent{\let\@afterindentfalse\@aifO
                          \@afterindentfals
  \addto\extrasserbian{\bbl@frenchinden
  \addto\noextrasserbian{\bbl@nonfrenchinden
\
%    \end{macrocod
%  \end{macr
%  \end{macr

%  \begin{macro}{\mathserbia
%    Some math functions in Serbian math books have other name
%    e.g. |sinh| in Serbian is written as |sh| etc. So we define
%    number of new math operator
%    \begin{macrocod
\def\sh{\mathop{\operator@font sh}\nolimits} % same as \si
\def\ch{\mathop{\operator@font ch}\nolimits} % same as \co
\def\th{\mathop{\operator@font th}\nolimits} % same as \ta
\def\cth{\mathop{\operator@font cth}\nolimits} % same as \co
\def\arsh{\mathop{\operator@font arsh}\nolimit
\def\arch{\mathop{\operator@font arch}\nolimit
\def\arth{\mathop{\operator@font arth}\nolimit
\def\arcth{\mathop{\operator@font arcth}\nolimit
\def\tg{\mathop{\operator@font tg}\nolimits} % same as \t
\def\ctg{\mathop{\operator@font ctg}\nolimits} % same as \c
\def\arctg{\mathop{\operator@font arctg}\nolimits} % same as \arct
\def\arcctg{\mathop{\operator@font arcctg}\nolimit
\def\Prob{\mathop{\mathsf P\hskip0pt}\nolimit
\def\Expect{\mathop{\mathsf E\hskip0pt}\nolimit
\def\Variance{\mathop{\mathsf D\hskip0pt}\nolimit
%    \end{macrocod
%  \end{macr

%    The macro |\ldf@finish| takes care of looking for
%    configuration file, setting the main language to be switched
%    at |\begin{document}| and resetting the category code
%    \texttt{@} to its original valu
%    \begin{macrocod
\ldf@finish{serbia
%</cod
%    \end{macrocod

% \Fina
%% \CharacterTab
%%  {Upper-case    \A\B\C\D\E\F\G\H\I\J\K\L\M\N\O\P\Q\R\S\T\U\V\W\X\Y
%%   Lower-case    \a\b\c\d\e\f\g\h\i\j\k\l\m\n\o\p\q\r\s\t\u\v\w\x\y
%%   Digits        \0\1\2\3\4\5\6\7\8
%%   Exclamation   \!     Double quote  \"     Hash (number)
%%   Dollar        \$     Percent       \%     Ampersand
%%   Acute accent  \'     Left paren    \(     Right paren
%%   Asterisk      \*     Plus          \+     Comma
%%   Minus         \-     Point         \.     Solidus
%%   Colon         \:     Semicolon     \;     Less than
%%   Equals        \=     Greater than  \>     Question mark
%%   Commercial at \@     Left bracket  \[     Backslash
%%   Right bracket \]     Circumflex    \^     Underscore
%%   Grave accent  \`     Left brace    \{     Vertical bar
%%   Right brace   \}     Tilde         \

\endinp
}
\DeclareOption{slovak}{% \iffalse meta-comment
%
% Copyright 1989-2008 Johannes L. Braams and any individual authors
% listed elsewhere in this file.  All rights reserved.
% 
% This file is part of the Babel system.
% --------------------------------------
% 
% It may be distributed and/or modified under the
% conditions of the LaTeX Project Public License, either version 1.3
% of this license or (at your option) any later version.
% The latest version of this license is in
%   http://www.latex-project.org/lppl.txt
% and version 1.3 or later is part of all distributions of LaTeX
% version 2003/12/01 or later.
% 
% This work has the LPPL maintenance status "maintained".
% 
% The Current Maintainer of this work is Johannes Braams.
% 
% The list of all files belonging to the Babel system is
% given in the file `manifest.bbl. See also `legal.bbl' for additional
% information.
% 
% The list of derived (unpacked) files belonging to the distribution
% and covered by LPPL is defined by the unpacking scripts (with
% extension .ins) which are part of the distribution.
% \fi
% \CheckSum{1382}
% \iffalse
%    Tell the \LaTeX\ system who we are and write an entry on the
%    transcript.
%<*dtx>
\ProvidesFile{slovak.dtx}
%</dtx>
%<+code>\ProvidesLanguage{slovak}
%\fi
%\ProvidesFile{slovak.dtx}
        [2008/07/06 v3.1a Slovak support from the babel system]
%\iffalse
%% File `slovak.dtx' 
%% Babel package for LaTeX version 2e
%% Copyright (C) 1989 - 2008
%%           by Johannes Braams, TeXniek
%
%% Slovak Language Definition File
%% Copyright (C) 1989 - 2001
%%           by Jana Chlebikova
%            Department of Artificial Intelligence
%            Faculty of Mathematics and Physics
%            Mlynska dolina
%            84215 Bratislava
%            Slovakia
%            (42)(7) 720003 l. 835
%            (42)(7) 725882
%            chlebikj at mff.uniba.cs (Internet)
%            and Johannes Braams, TeXniek
%
%% Copyright (C) 2002-2005
%%           by Tobias Schlemmer
%            Braunsdorfer Stra\ss e 101
%            01159 Dresden
%            Deutschland
%            Tobias.Schlemmer at web.de
%
%% Copyright (C) 2005-2008
%%           by Petr Tesa\v r\'ik (babel at tesarici.cz)
%
% This file is also based on CSLaTeX
%                       by Ji\v r\'i Zlatu\v ska, Zden\v ek Wagner,
%                          Jaroslav \v Snajdr and Petr Ol\v s\'ak.
%
%% Please report errors to: Petr Tesa\v r\'ik
%%                          babel at tesarici.cz
%
%    This file is part of the babel system, it provides the source
%    code for the Slovak language definition file.
%<*filedriver>
\documentclass{ltxdoc}
\newcommand*\TeXhax{\TeX hax}
\newcommand*\babel{\textsf{babel}}
\newcommand*\langvar{$\langle \it lang \rangle$}
\newcommand*\note[1]{}
\newcommand*\Lopt[1]{\textsf{#1}}
\newcommand*\file[1]{\texttt{#1}}
\begin{document}
 \DocInput{slovak.dtx}
\end{document}
%</filedriver>
%\fi
%
% \def\CS{$\cal C\kern-.1667em
%   \lower.5ex\hbox{$\cal S$}\kern-.075em$}
%
% \GetFileInfo{slovak.dtx}
%
% \changes{slovak-1.0}{1992/07/15}{First version}
% \changes{slovak-1.2}{1994/02/27}{Update for \LaTeXe}
% \changes{slovak-1.2d}{1994/06/26}{Removed the use of \cs{filedate}
%    and moved identification after the loading of \file{babel.def}}
% \changes{slovak-1.2i}{1996/10/10}{Replaced \cs{undefined} with
%    \cs{@undefined} and \cs{empty} with \cs{@empty} for consistency
%    with \LaTeX, moved the definition of \cs{atcatcode} right to the
%    beginning.}
% \changes{slovak-1.3a}{2004/02/20}{Added contributed shorthand
%    definitions} 
% \changes{slovak-3.0}{2005/09/10}{Implemented the functionality of
%    \CS\LaTeX's slovak.sty.  The version number was bumped to 3.0
%    to minimize confusion by being higher than the last version
%    of \CS\LaTeX.}
%
%  \section{The Slovak language}
%
%    The file \file{\filename}\footnote{The file described in this
%    section has version number \fileversion\ and was last revised on
%    \filedate.  It was originally written by Jana Chlebikova
%    (\texttt{chlebik@euromath.dk}) and modified by Tobias Schlemmer
%    (\texttt{Tobias.Schlemmer@web.de}). It was then rewritten by
%    Petr Tesa\v r\'ik (\texttt{babel@tesarici.cz}).} defines all the
%    language-specific macros for the Slovak language.
%
%    For this language the macro |\q| is defined. It was used with the
%    letters (\texttt{t}, \texttt{d}, \texttt{l}, and \texttt{L}) and
%    adds a \texttt{'} to them to simulate a `hook' that should be
%    there.  The result looks like t\kern-2pt\char'47. Since the the T1
%    font encoding has the corresponding characters it is mapped to |\v|.
%    Therefore we recommend using T1 font encoding. If you don't want to
%    use this encoding, please, feel free to redefine |\q| in your file.
%    I think babel will honour this |;-)|.
%
%    For this language the characters |"|, |'| and |^| are made
%    active. In table~\ref{tab:slovak-quote} an overview is given of
%    its purpose. Also the vertical placement of the
%    umlaut can be controlled this way.
%
%    \begin{table}[htb]
%     \begin{center}
%     \begin{tabular}{lp{8cm}}
%      |"a| & |\"a|, also implemented for the other
%                  lowercase and uppercase vowels.                 \\
%      |^d| & |\q d|, also implemented for l, t and L.             \\
%      |^c| & |\v c|, also implemented for C, D, N, n, T, Z and z. \\
%      |^o| & |\^o|, also implemented for O.                       \\
%      |'a| & |\'a|, also implemented for the other lowercase and
%                    uppercase l, r, y and vowels.                 \\
%      \verb="|= & disable ligature at this position.              \\
%      |"-| & an explicit hyphen sign, allowing hyphenation
%             in the rest of the word.                             \\
%      |""| & like |"-|, but producing no hyphen sign
%             (for compund words with hyphen, e.g.\ |x-""y|).      \\
%      |"~| & for a compound word mark without a breakpoint.       \\
%      |"=| & for a compound word mark with a breakpoint, allowing
%             hyphenation in the composing words.                  \\
%      |"`| & for German left double quotes (looks like ,,).       \\
%      |"'| & for German right double quotes.                      \\
%      |"<| & for French left double quotes (similar to $<<$).     \\
%      |">| & for French right double quotes (similar to $>>$).    \\
%     \end{tabular}
%     \caption{The extra definitions made
%              by \file{slovak.ldf}}\label{tab:slovak-quote}
%     \end{center}
%    \end{table}
%
%    The quotes in table~\ref{tab:slovak-quote} can also be typeset by
%    using the commands in table~\ref{tab:smore-quote}.
%    \begin{table}[htb]
%     \begin{center}
%     \begin{tabular}{lp{8cm}}
%      |\glqq| & for German left double quotes (looks like ,,).   \\
%      |\grqq| & for German right double quotes (looks like ``).  \\
%      |\glq|  & for German left single quotes (looks like ,).    \\
%      |\grq|  & for German right single quotes (looks like `).   \\
%      |\flqq| & for French left double quotes (similar to $<<$). \\
%      |\frqq| & for French right double quotes (similar to $>>$).\\
%      |\flq|  & for (French) left single quotes (similar to $<$).  \\
%      |\frq|  & for (French) right single quotes (similar to $>$). \\
%      |\dq|   & the original quotes character (|"|).        \\
%      |\sq|   & the original single quote (|'|).            \\
%     \end{tabular}
%     \caption{More commands which produce quotes, defined
%              by \file{slovak.ldf}}\label{tab:smore-quote}
%     \end{center}
%    \end{table}
%
%  \subsection{Compatibility}
%
%    Great care has been taken to ensure backward compatibility with
%    \CS\LaTeX.  In particular, documents which load this file with
%    |\usepackage{slovak}| should produce identical output with no
%    modifications to the source.  Additionally, all the \CS\LaTeX{}
%    options are recognized:
%
%    \label{tab:slovak-options}
%    \begin{list}{}
%     {\def\makelabel#1{\sbox0{\Lopt{#1}}%
%        \ifdim\wd0>\labelwidth
%          \setbox0\vbox{\box0\hbox{}} \wd0=0pt \fi
%        \box0\hfil}
%      \setlength{\labelwidth}{2\parindent}
%      \setlength{\leftmargin}{2\parindent}
%      \setlength{\rightmargin}{\parindent}}
%     \item[IL2, T1, OT1]
%       These options set the default font encoding.  Please note
%       that their use is deprecated. You should use the |fontenc|
%       package to select font encoding.
%
%     \item[split, nosplit]
%       These options control whether hyphenated words are
%       automatically split according to Slovak typesetting rules.
%       With the \Lopt{split} option ``je-li'' is hyphenated as
%       ``je-/-li''. The \Lopt{nosplit} option disables this behavior.
%
%       The use of this option is strongly discouraged, as it breaks
%       too many common things---hyphens cannot be used in labels,
%       negative arguments to \TeX{} primitives will not work in
%       horizontal mode (use \cs{minus} as a workaround), and there are
%       a few other peculiarities with using this mode.
%
%     \item[nocaptions]
%
%       This option was used in \CS\LaTeX{} to set up Czech/Slovak
%       typesetting rules, but leave the original captions and dates.
%       The recommended way to achieve this is to use English as the main
%       language of the document and use the environment |otherlanguage*|
%       for Czech text.
%
%     \item[olduv]
%       There are two version of \cs{uv}.  The older one allows the use
%       of \cs{verb} inside the quotes but breaks any respective kerning
%       with the quotes (like that in \CS{} fonts).  The newer one honors
%       the kerning in the font but does not allow \cs{verb} inside the
%       quotes.
%
%       The new version is used by default in \LaTeXe{} and the old version
%       is used with plain \TeX.  You may use \Lopt{olduv} to override the
%       default in \LaTeXe.
%       
%     \item[cstex]
%       This option was used to include the commands \cs{csprimeson} and
%       \cs{csprimesoff}.  Since these commands are always included now,
%       it has been removed and the empty definition lasts for compatibility.
%    \end{list}
%
% \StopEventually{}
%
%  \subsection{Implementation}
%
%    The macro |\LdfInit| takes care of preventing that this file is
%    loaded more than once, checking the category code of the
%    \texttt{@} sign, etc.
% \changes{slovak-1.2i}{1996/11/03}{Now use \cs{LdfInit} to perform
%    initial checks} 
%    \begin{macrocode}
%<*code>
\LdfInit\CurrentOption{date\CurrentOption}
%    \end{macrocode}
%
%    When this file is read as an option, i.e. by the |\usepackage|
%    command, \texttt{slovak} will be an `unknown' language in which
%    case we have to make it known. So we check for the existence of
%    |\l@slovak| to see whether we have to do something here.
%
% \changes{slovak-1.2d}{1994/06/26}{Now use \cs{@nopatterns} to
%    produce the warning}
%    \begin{macrocode}
\ifx\l@slovak\@undefined
    \@nopatterns{Slovak}
    \adddialect\l@slovak0\fi
%    \end{macrocode}
%
%    We need to define these macros early in the process.
%
%    \begin{macrocode}
\def\cs@iltw@{IL2}
\newif\ifcs@splithyphens
\cs@splithyphensfalse
%    \end{macrocode}
%
%    If Babel is not loaded, we provide compatibility with \CS\LaTeX.
%    However, if macro \cs{@ifpackageloaded} is not defined, we assume
%    to be loaded from plain and provide compatibility with csplain.
%    Of course, this does not work well with \LaTeX$\:$2.09, but I
%    doubt anyone will ever want to use this file with \LaTeX$\:$2.09.
%
%    \begin{macrocode}
\ifx\@ifpackageloaded\@undefined
  \let\cs@compat@plain\relax
  \message{csplain compatibility mode}
\else
  \@ifpackageloaded{babel}{}{%
    \let\cs@compat@latex\relax
    \message{cslatex compatibility mode}}
\fi
\ifx\cs@compat@latex\relax
  \ProvidesPackage{slovak}[2008/07/06 v3.1a CSTeX Slovak style]
%    \end{macrocode}
%
%    Declare \CS\LaTeX{} options (see also the descriptions on page
%    \pageref{tab:slovak-options}).
%
%    \begin{macrocode}
  \DeclareOption{IL2}{\def\encodingdefault{IL2}}
  \DeclareOption {T1}{\def\encodingdefault {T1}}
  \DeclareOption{OT1}{\def\encodingdefault{OT1}}
  \DeclareOption{nosplit}{\cs@splithyphensfalse}
  \DeclareOption{split}{\cs@splithyphenstrue}
  \DeclareOption{nocaptions}{\let\cs@nocaptions=\relax}
  \DeclareOption{olduv}{\let\cs@olduv=\relax}
  \DeclareOption{cstex}{\relax}
%    \end{macrocode}
%
%    Make |IL2| encoding the default.  This can be overriden with
%    the other font encoding options.
%    \begin{macrocode}
  \ExecuteOptions{\cs@iltw@}
%    \end{macrocode}
%
%    Now, process the user-supplied options.
%    \begin{macrocode}
  \ProcessOptions
%    \end{macrocode}
%
%    Standard \LaTeXe{} does not include the IL2 encoding in the format.
%    The encoding can be loaded later using the fontenc package, but
%    \CS\LaTeX{} included IL2 by default.  This means existing documents
%    for \CS\LaTeX{} do not load that package, so load the encoding
%    ourselves in compatibility mode.
%
%    \begin{macrocode}
  \ifx\encodingdefault\cs@iltw@
    \input il2enc.def
  \fi
%    \end{macrocode}
%
%    Restore the definition of \cs{CurrentOption}, clobbered by processing
%    the options.
%
%    \begin{macrocode}
  \def\CurrentOption{slovak}
\fi
%    \end{macrocode}
%
%    The next step consists of defining commands to switch to (and
%    from) the Slovak language.
%
%  \begin{macro}{\captionsslovak}
%    The macro \cs{captionsslovak} defines all strings used in the four
%    standard documentclasses provided with \LaTeX.
% \changes{slovak-1.2g}{1995/07/04}{Added \cs{proofname} for
%    AMS-\LaTeX}
% \changes{slovak-1.2k}{1999/02/28}{Repaired a few spelling mistakes}
% \changes{slovak-1.2l}{2000/09/20}{Added \cs{glossaryname}}
% \changes{slovak-3.0}{2005/12/22}{Updated some translations.  Former
%    translations were: `\'Uvod' for \cs{prefacename}, `Referencie' for
%    \cs{refname}, `Index' for \cs{indexname}, `Obr\'azok' for
%    \cs{figurename}, `Pr\'ilohy' for \cs{enclname}, `CC' for \cs{ccname},
%    `Komu' for \cs{headtoname}, `Strana' for \cs{pagename}}
%    \begin{macrocode}
\@namedef{captions\CurrentOption}{%
  \def\prefacename{Predhovor}%
  \def\refname{Literat\'ura}%
  \def\abstractname{Abstrakt}%
  \def\bibname{Literat\'ura}%
  \def\chaptername{Kapitola}%
  \def\appendixname{Dodatok}%
  \def\contentsname{Obsah}%
  \def\listfigurename{Zoznam obr\'azkov}%
  \def\listtablename{Zoznam tabuliek}%
  \def\indexname{Register}%
  \def\figurename{Obr.}%
  \def\tablename{Tabu\v{l}ka}%
  \def\partname{\v{C}as\v{t}}%
  \def\enclname{Pr\'{\i}loha}%
  \def\ccname{cc.}%
  \def\headtoname{Pre}%
  \def\pagename{Str.}%
  \def\seename{vi\v{d}}%
  \def\alsoname{vi\v{d} tie\v{z}}%
  \def\proofname{D\^okaz}%
  \def\glossaryname{Slovn\'{\i}k}%
  }%
%    \end{macrocode}
%  \end{macro}
%
%  \begin{macro}{\dateslovak}
%    The macro \cs{dateslovak} redefines the command \cs{today}
%    to produce Slovak dates.
% \changes{slovak-1.2j}{1997/10/01}{Use \cs{edef} to define
%    \cs{today} to save memory}
% \changes{slovak-1.2j}{1998/03/28}{use \cs{def} instead of \cs{edef}}
% \changes{slovak-1.2k}{1999/02/28}{Repaired a spelling mistake}
%    \begin{macrocode}
\@namedef{date\CurrentOption}{%
  \def\today{\number\day.~\ifcase\month\or
    janu\'ara\or febru\'ara\or marca\or apr\'{\i}la\or m\'aja\or
    j\'una\or j\'ula\or augusta\or septembra\or okt\'obra\or
    novembra\or decembra\fi
    \space \number\year}}
%    \end{macrocode}
%  \end{macro}
%
%  \begin{macro}{\extrasslovak}
%  \begin{macro}{\noextrasslovak}
%    The macro \cs{extrasslovak} will perform all the extra definitions
%    needed for the Slovak language. The macro \cs{noextrasslovak} is
%    used to cancel the actions of \cs{extrasslovak}.
%
%    For Slovak texts \cs{frenchspacing} should be in effect.  Language
%    group for shorthands is also set here.
% \changes{slovak-3.0}{1995/04/21}{now use \cs{bbl@frenchspacing} and
%    \cs{bbl@nonfrenchspacing}}
% \changes{slovak-3.1}{2006/10/07}{move \cs{languageshorthands} here,
%    so that the language group is always initialized correctly}
%    \begin{macrocode}
\expandafter\addto\csname extras\CurrentOption\endcsname{%
  \bbl@frenchspacing
  \languageshorthands{slovak}}
\expandafter\addto\csname noextras\CurrentOption\endcsname{%
  \bbl@nonfrenchspacing}
%    \end{macrocode}
%
% \changes{slovak-1.2e}{1995/05/28}{Use \LaTeX's \cs{v} accent
%    command}
%    \begin{macrocode}
\expandafter\addto\csname extras\CurrentOption\endcsname{%
  \babel@save\q\let\q\v}
%    \end{macrocode}
%
%    For Slovak three characters are used to define shorthands, they
%    need to be made active.
% \changes{slovak-1.3a}{2004/02/20}{Make three characters available
%    for shorthands}
%    \begin{macrocode}
\ifx\cs@compat@latex\relax\else
  \initiate@active@char{^}
  \addto\extrasslovak{\bbl@activate{^}}
  \addto\noextrasslovak{\bbl@deactivate{^}}
  \initiate@active@char{"}
  \addto\extrasslovak{\bbl@activate{"}\umlautlow}
  \addto\noextrasslovak{\bbl@deactivate{"}\umlauthigh}
  \initiate@active@char{'}
  \@ifpackagewith{babel}{activeacute}{%
    \addto\extrasslovak{\bbl@activate{'}}
    \addto\noextrasslovak{\bbl@deactivate{'}}%
    }{}
\fi
%    \end{macrocode}
%  \end{macro}
%  \end{macro}
%
%  \begin{macro}{\sq}
%  \begin{macro}{\dq}
%    We save the original single and double quote characters in
%    \cs{sq} and \cs{dq} to make them available later.  The math
%    accent |\"| can now be typed as |"|.
%    \begin{macrocode}
\begingroup\catcode`\"=12\catcode`\'=12
\def\x{\endgroup
  \def\sq{'}
  \def\dq{"}}
\x
%    \end{macrocode}
%  \end{macro}
%  \end{macro}
%
% \changes{slovak-1.2b}{1994/06/04}{Added setting of left- and
%    righthyphenmin}
%
%    The slovak hyphenation patterns should be used with
%    |\lefthyphenmin| set to~2 and |\righthyphenmin| set to~3.
% \changes{slovak-1.2e}{1995/05/28}{Now use \cs{slovakhyphenmins}}
% \changes{slovak-1.2l}{2000/09/22}{Now use \cs{providehyphenmins} to
%    provide a default value}
% \changes{slovak-3.0}{2005/12/22}{Changed default \cs{righthyphenmin}
%    to 3 characters.}
%    \begin{macrocode}
\providehyphenmins{\CurrentOption}{\tw@\thr@@}
%    \end{macrocode}
%
%    In order to prevent problems with the active |^| we add a
%    shorthand on system level which expands to a `normal |^|.
%    \begin{macrocode}
\ifx\cs@compat@latex\relax\else
  \declare@shorthand{system}{^}{\csname normal@char\string^\endcsname}
%    \end{macrocode}
%
%    Now we can define the doublequote macros: the umlauts,
%    \begin{macrocode}
  \declare@shorthand{slovak}{"a}{\textormath{\"{a}\allowhyphens}{\ddot a}}
  \declare@shorthand{slovak}{"o}{\textormath{\"{o}\allowhyphens}{\ddot o}}
  \declare@shorthand{slovak}{"u}{\textormath{\"{u}\allowhyphens}{\ddot u}}
  \declare@shorthand{slovak}{"A}{\textormath{\"{A}\allowhyphens}{\ddot A}}
  \declare@shorthand{slovak}{"O}{\textormath{\"{O}\allowhyphens}{\ddot O}}
  \declare@shorthand{slovak}{"U}{\textormath{\"{U}\allowhyphens}{\ddot U}}
%    \end{macrocode}
%    tremas,
%    \begin{macrocode}
  \declare@shorthand{slovak}{"e}{\textormath{\"{e}\allowhyphens}{\ddot e}}
  \declare@shorthand{slovak}{"E}{\textormath{\"{E}\allowhyphens}{\ddot E}}
  \declare@shorthand{slovak}{"i}{\textormath{\"{\i}\allowhyphens}%
                                {\ddot\imath}}
  \declare@shorthand{slovak}{"I}{\textormath{\"{I}\allowhyphens}{\ddot I}}
%    \end{macrocode}
%    other slovak characters
%    \begin{macrocode}
  \declare@shorthand{slovak}{^c}{\textormath{\v{c}\allowhyphens}{\check{c}}}
  \declare@shorthand{slovak}{^d}{\textormath{\q{d}\allowhyphens}{\check{d}}}
  \declare@shorthand{slovak}{^l}{\textormath{\q{l}\allowhyphens}{\check{l}}}
  \declare@shorthand{slovak}{^n}{\textormath{\v{n}\allowhyphens}{\check{n}}}
  \declare@shorthand{slovak}{^o}{\textormath{\^{o}\allowhyphens}{\hat{o}}}
  \declare@shorthand{slovak}{^s}{\textormath{\v{s}\allowhyphens}{\check{s}}}
  \declare@shorthand{slovak}{^t}{\textormath{\q{t}\allowhyphens}{\check{t}}}
  \declare@shorthand{slovak}{^z}{\textormath{\v{z}\allowhyphens}{\check{z}}}
  \declare@shorthand{slovak}{^C}{\textormath{\v{C}\allowhyphens}{\check{C}}}
  \declare@shorthand{slovak}{^D}{\textormath{\v{D}\allowhyphens}{\check{D}}}
  \declare@shorthand{slovak}{^L}{\textormath{\q{L}\allowhyphens}{\check{L}}}
  \declare@shorthand{slovak}{^N}{\textormath{\v{N}\allowhyphens}{\check{N}}}
  \declare@shorthand{slovak}{^O}{\textormath{\^{O}\allowhyphens}{\hat{O}}}
  \declare@shorthand{slovak}{^S}{\textormath{\v{S}\allowhyphens}{\check{S}}}
  \declare@shorthand{slovak}{^T}{\textormath{\v{T}\allowhyphens}{\check{T}}}
  \declare@shorthand{slovak}{^Z}{\textormath{\v{Z}\allowhyphens}{\check{Z}}}
%    \end{macrocode}
%     acute accents,
%    \begin{macrocode}
  \@ifpackagewith{babel}{activeacute}{%
    \declare@shorthand{slovak}{'a}{\textormath{\'a\allowhyphens}{^{\prime}a}}
    \declare@shorthand{slovak}{'e}{\textormath{\'e\allowhyphens}{^{\prime}e}}
    \declare@shorthand{slovak}{'i}{\textormath{\'\i{}\allowhyphens}{^{\prime}i}}
    \declare@shorthand{slovak}{'l}{\textormath{\'l\allowhyphens}{^{\prime}l}}
    \declare@shorthand{slovak}{'o}{\textormath{\'o\allowhyphens}{^{\prime}o}}
    \declare@shorthand{slovak}{'r}{\textormath{\'r\allowhyphens}{^{\prime}r}}
    \declare@shorthand{slovak}{'u}{\textormath{\'u\allowhyphens}{^{\prime}u}}
    \declare@shorthand{slovak}{'y}{\textormath{\'y\allowhyphens}{^{\prime}y}}
    \declare@shorthand{slovak}{'A}{\textormath{\'A\allowhyphens}{^{\prime}A}}
    \declare@shorthand{slovak}{'E}{\textormath{\'E\allowhyphens}{^{\prime}E}}
    \declare@shorthand{slovak}{'I}{\textormath{\'I\allowhyphens}{^{\prime}I}}
    \declare@shorthand{slovak}{'L}{\textormath{\'L\allowhyphens}{^{\prime}l}}
    \declare@shorthand{slovak}{'O}{\textormath{\'O\allowhyphens}{^{\prime}O}}
    \declare@shorthand{slovak}{'R}{\textormath{\'R\allowhyphens}{^{\prime}R}}
    \declare@shorthand{slovak}{'U}{\textormath{\'U\allowhyphens}{^{\prime}U}}
    \declare@shorthand{slovak}{'Y}{\textormath{\'Y\allowhyphens}{^{\prime}Y}}
    \declare@shorthand{slovak}{''}{%
      \textormath{\textquotedblright}{\sp\bgroup\prim@s'}}
    }{}
  %    \end{macrocode}
%    and some additional commands:
%    \begin{macrocode}
  \declare@shorthand{slovak}{"-}{\nobreak\-\bbl@allowhyphens}
  \declare@shorthand{slovak}{"|}{%
    \textormath{\penalty\@M\discretionary{-}{}{\kern.03em}%
                \bbl@allowhyphens}{}}
  \declare@shorthand{slovak}{""}{\hskip\z@skip}
  \declare@shorthand{slovak}{"~}{\textormath{\leavevmode\hbox{-}}{-}}
  \declare@shorthand{slovak}{"=}{\cs@splithyphen}
\fi
%    \end{macrocode}
%
%  \begin{macro}{\v}
%    \LaTeX's normal |\v| accent places a caron over the letter that
%    follows it (\v{o}). This is not what we want for the letters d,
%    t, l and L; for those the accent should change shape. This is
%    acheived by the following.
%    \begin{macrocode}
\AtBeginDocument{%
  \DeclareTextCompositeCommand{\v}{OT1}{t}{%
    t\kern-.23em\raise.24ex\hbox{'}}
  \DeclareTextCompositeCommand{\v}{OT1}{d}{%
    d\kern-.13em\raise.24ex\hbox{'}}
  \DeclareTextCompositeCommand{\v}{OT1}{l}{\lcaron{}}
  \DeclareTextCompositeCommand{\v}{OT1}{L}{\Lcaron{}}}
%    \end{macrocode}
%
%  \begin{macro}{\lcaron}
%  \begin{macro}{\Lcaron}
%    For the letters \texttt{l} and \texttt{L} we want to disinguish
%    between normal fonts and monospaced fonts.
%    \begin{macrocode}
\def\lcaron{%
  \setbox0\hbox{M}\setbox\tw@\hbox{i}%
  \ifdim\wd0>\wd\tw@\relax
    l\kern-.13em\raise.24ex\hbox{'}\kern-.11em%
  \else
    l\raise.45ex\hbox to\z@{\kern-.35em '\hss}%
  \fi}
\def\Lcaron{%
  \setbox0\hbox{M}\setbox\tw@\hbox{i}%
  \ifdim\wd0>\wd\tw@\relax
    L\raise.24ex\hbox to\z@{\kern-.28em'\hss}%
  \else
    L\raise.45ex\hbox to\z@{\kern-.40em '\hss}%
  \fi}
%    \end{macrocode}
%  \end{macro}
%  \end{macro}
%  \end{macro}
%
%    Initialize active quotes.  \CS\LaTeX{} provides a way of
%    converting English-style quotes into Slovak-style ones.  Both
%    single and double quotes are affected, i.e. |``text''| is
%    converted to something like |,,text``| and |`text'| is converted
%    to |,text`|.  This conversion can be switched on and off with
%    \cs{csprimeson} and \cs{csprimesoff}.\footnote{By the way, the
%    names of these macros are misleading, because the handling of
%    primes in math mode is rather marginal, the most important thing
%    being the handling of quotes\ldots}
%
%    These quotes present various troubles, e.g. the kerning is broken,
%    apostrophes are converted to closing single quote, some primitives
%    are broken (most notably the |\catcode`\|\meta{char} syntax will
%    not work any more), and writing them to \file{.aux} files cannot
%    be handled correctly.  For these reasons, these commands are only
%    available in \CS\LaTeX{} compatibility mode.
%
%    \begin{macrocode}
\ifx\cs@compat@latex\relax
  \let\cs@ltxprim@s\prim@s
  \def\csprimeson{%
    \catcode``\active \catcode`'\active \let\prim@s\bbl@prim@s}
  \def\csprimesoff{%
    \catcode``12 \catcode`'12 \let\prim@s\cs@ltxprim@s}
  \begingroup\catcode``\active
  \def\x{\endgroup
    \def`{\futurelet\cs@next\cs@openquote}
    \def\cs@openquote{%
      \ifx`\cs@next \expandafter\cs@opendq
      \else \expandafter\clq
      \fi}%
  }\x
  \begingroup\catcode`'\active
  \def\x{\endgroup
    \def'{\textormath{\futurelet\cs@next\cs@closequote}
                     {^\bgroup\prim@s}}
    \def\cs@closequote{%
      \ifx'\cs@next \expandafter\cs@closedq
      \else \expandafter\crq
      \fi}%
  }\x
  \def\cs@opendq{\clqq\let\cs@next= }
  \def\cs@closedq{\crqq\let\cs@next= }
%    \end{macrocode}
%
%    The way I recommend for typesetting quotes in Slovak documents
%    is to use shorthands similar to those used in German.
%    
% \changes{3.0}{2006/04/21}{|"`| and |"'| changed from German quotes
%   to Slovak quotes}
%    \begin{macrocode}
\else
  \declare@shorthand{slovak}{"`}{\clqq}
  \declare@shorthand{slovak}{"'}{\crqq}
  \declare@shorthand{slovak}{"<}{\flqq}
  \declare@shorthand{slovak}{">}{\frqq}
\fi
%    \end{macrocode}
%
%  \begin{macro}{\clqq}
%    This is the CS opening quote, which is similar to the German
%    quote (\cs{glqq}) but the kerning is different.
%
%    For the OT1 encoding, the quote is constructed from the right
%    double quote (i.e. the ``Opening quotes'' character) by moving
%    it down to the baseline and shifting it to the right, or to the
%    left if italic correction is positive.
%
%    For T1, the ``German Opening quotes'' is used.  It is moved to
%    the right and the total width is enlarged.  This is done in an
%    attempt to minimize the difference between the OT1 and T1
%    versions.
%
% \changes{3.0}{2006/04/20}{Added \cs{leavevmode} to allow an opening
%   quote at the beginning of a paragraph}
%    \begin{macrocode}
\ProvideTextCommand{\clqq}{OT1}{%
  \set@low@box{\textquotedblright}%
  \setbox\@ne=\hbox{l\/}\dimen\@ne=\wd\@ne
  \setbox\@ne=\hbox{l}\advance\dimen\@ne-\wd\@ne
  \leavevmode
  \ifdim\dimen\@ne>\z@\kern-.1em\box\z@\kern.1em
    \else\kern.1em\box\z@\kern-.1em\fi\allowhyphens}
\ProvideTextCommand{\clqq}{T1}
  {\kern.1em\quotedblbase\kern-.0158em\relax}
\ProvideTextCommandDefault{\clqq}{\UseTextSymbol{OT1}\clqq}
%    \end{macrocode}
%  \end{macro}
%
%  \begin{macro}{\crqq}
%    For OT1, the CS closing quote is basically the same as
%    \cs{grqq}, only the \cs{textormath} macro is not used, because
%    as far as I know, \cs{grqq} does not work in math mode anyway.
%
%    For T1, the character is slightly wider and shifted to the
%    right to match its OT1 counterpart.
%
%    \begin{macrocode}
\ProvideTextCommand{\crqq}{OT1}
  {\save@sf@q{\nobreak\kern-.07em\textquotedblleft\kern.07em}}
\ProvideTextCommand{\crqq}{T1}
  {\save@sf@q{\nobreak\kern.06em\textquotedblleft\kern.024em}}
\ProvideTextCommandDefault{\crqq}{\UseTextSymbol{OT1}\crqq}
%    \end{macrocode}
%  \end{macro}
%
%  \begin{macro}{\clq}
%  \begin{macro}{\crq}
%
%    Single CS quotes are similar to double quotes (see the
%    discussion above).
%
%    \begin{macrocode}
\ProvideTextCommand{\clq}{OT1}
  {\set@low@box{\textquoteright}\box\z@\kern.04em\allowhyphens}
\ProvideTextCommand{\clq}{T1}
  {\quotesinglbase\kern-.0428em\relax}
\ProvideTextCommandDefault{\clq}{\UseTextSymbol{OT1}\clq}
\ProvideTextCommand{\crq}{OT1}
  {\save@sf@q{\nobreak\textquoteleft\kern.17em}}
\ProvideTextCommand{\crq}{T1}
  {\save@sf@q{\nobreak\textquoteleft\kern.17em}}
\ProvideTextCommandDefault{\crq}{\UseTextSymbol{OT1}\crq}
%    \end{macrocode}
%  \end{macro}
%  \end{macro}
%
%  \begin{macro}{\uv}
%    There are two versions of \cs{uv}.  The older one opens a group
%    and uses \cs{aftergroup} to typeset the closing quotes.  This
%    version allows using \cs{verb} inside the quotes, because the
%    enclosed text is not passed as an argument, but unfortunately
%    it breaks any kerning with the quotes.  Although the kerning
%    with the opening quote could be fixed, the kerning with the
%    closing quote cannot.
%
%    The newer version is defined as a command with one parameter.
%    It preserves kerning but since the quoted text is passed as an
%    argument, it cannot contain \cs{verb}.
%
%    Decide which version of \cs{uv} should be used.  For sake
%    of compatibility, we use the older version with plain \TeX{}
%    and the newer version with \LaTeXe.
%    \begin{macrocode}
\ifx\cs@compat@plain\@undefined\else\let\cs@olduv=\relax\fi
\ifx\cs@olduv\@undefined
  \DeclareRobustCommand\uv[1]{{\leavevmode\clqq#1\crqq}}
\else
  \DeclareRobustCommand\uv{\bgroup\aftergroup\closequotes
    \leavevmode\clqq\let\cs@next=}
  \def\closequotes{\unskip\crqq\relax}
\fi
%    \end{macrocode}
%  \end{macro}
%
%  \begin{macro}{\cs@wordlen}
%    Declare a counter to hold the length of the word after the
%    hyphen.
%
%    \begin{macrocode}
\newcount\cs@wordlen
%    \end{macrocode}
%  \end{macro}
%
%  \begin{macro}{\cs@hyphen}
%  \begin{macro}{\cs@endash}
%  \begin{macro}{\cs@emdash}
%    Store the original hyphen in a macro. Ditto for the ligatures.
%
% \changes{slovak-3.1}{2006/10/07}{ensure correct catcode for the
%    saved hyphen}
%    \begin{macrocode}
\begingroup\catcode`\-12
\def\x{\endgroup
  \def\cs@hyphen{-}
  \def\cs@endash{--}
  \def\cs@emdash{---}
%    \end{macrocode}
%  \end{macro}
%  \end{macro}
%  \end{macro}
%
%  \begin{macro}{\cs@boxhyphen}
%    Provide a non-breakable hyphen to be used when a compound word
%    is too short to be split, i.e. the second part is shorter than
%    \cs{righthyphenmin}.
%
%    \begin{macrocode}
  \def\cs@boxhyphen{\hbox{-}}
%    \end{macrocode}
%  \end{macro}
%
%  \begin{macro}{\cs@splithyphen}
%    The macro \cs{cs@splithyphen} inserts a split hyphen, while
%    allowing both parts of the compound word to be hyphenated at
%    other places too.
%
%    \begin{macrocode}
  \def\cs@splithyphen{\kern\z@
    \discretionary{-}{\char\hyphenchar\the\font}{-}\nobreak\hskip\z@}
}\x
%    \end{macrocode}
%  \end{macro}
%
%  \begin{macro}{-}
%    To minimize the effects of activating the hyphen character,
%    the active definition expands to the non-active character
%    in all cases where hyphenation cannot occur, i.e. if not
%    typesetting (check \cs{protect}), not in horizontal mode,
%    or in inner horizontal mode.
%
%    \begin{macrocode}
\initiate@active@char{-}
\declare@shorthand{slovak}{-}{%
  \ifx\protect\@typeset@protect
    \ifhmode
      \ifinner
        \bbl@afterelse\bbl@afterelse\bbl@afterelse\cs@hyphen
      \else
        \bbl@afterfi\bbl@afterelse\bbl@afterelse\cs@firsthyphen
      \fi
    \else
      \bbl@afterfi\bbl@afterelse\cs@hyphen
    \fi
  \else
    \bbl@afterfi\cs@hyphen
  \fi}
%    \end{macrocode}
%  \end{macro}
%
%  \begin{macro}{\cs@firsthyphen}
%  \begin{macro}{\cs@firsthyph@n}
%  \begin{macro}{\cs@secondhyphen}
%  \begin{macro}{\cs@secondhyph@n}
%    If we encounter a hyphen, check whether it is followed
%    by a second or a third hyphen and if so, insert the
%    corresponding ligature.
%
%    If we don't find a hyphen, the token found will be placed
%    in \cs{cs@token} for further analysis, and it will also stay
%    in the input.
%
%    \begin{macrocode}
\begingroup\catcode`\-\active
\def\x{\endgroup
  \def\cs@firsthyphen{\futurelet\cs@token\cs@firsthyph@n}
  \def\cs@firsthyph@n{%
    \ifx -\cs@token
      \bbl@afterelse\cs@secondhyphen
    \else
      \bbl@afterfi\cs@checkhyphen
    \fi}
  \def\cs@secondhyphen ##1{%
    \futurelet\cs@token\cs@secondhyph@n}
  \def\cs@secondhyph@n{%
    \ifx -\cs@token
      \bbl@afterelse\cs@emdash\@gobble
    \else
      \bbl@afterfi\cs@endash
    \fi}
}\x
%    \end{macrocode}
%  \end{macro}
%  \end{macro}
%  \end{macro}
%  \end{macro}
%
%  \begin{macro}{\cs@checkhyphen}
%    Check that hyphenation is enabled, and if so, start analyzing
%    the rest of the word, i.e. initialize \cs{cs@word} and \cs{cs@wordlen}
%    and start processing input with \cs{cs@scanword}.
%
%    \begin{macrocode}
\def\cs@checkhyphen{%
  \ifnum\expandafter\hyphenchar\the\font=`\-
    \def\cs@word{}\cs@wordlen\z@
    \bbl@afterelse\cs@scanword
  \else
    \cs@hyphen
  \fi}
%    \end{macrocode}
%  \end{macro}
%
%  \begin{macro}{\cs@scanword}
%  \begin{macro}{\cs@continuescan}
%  \begin{macro}{\cs@gettoken}
%  \begin{macro}{\cs@gett@ken}
%    Each token is first analyzed with \cs{cs@scanword}, which expands
%    the token and passes the first token of the result to
%    \cs{cs@gett@ken}. If the expanded token is not identical to the
%    unexpanded one, presume that it might be expanded further and
%    pass it back to \cs{cs@scanword} until you get an unexpandable
%    token. Then analyze it in \cs{cs@examinetoken}.
%
%    The \cs{cs@continuescan} macro does the same thing as
%    \cs{cs@scanword}, but it does not require the first token to be
%    in \cs{cs@token} already.
%
%    \begin{macrocode}
\def\cs@scanword{\let\cs@lasttoken= \cs@token\expandafter\cs@gettoken}
\def\cs@continuescan{\let\cs@lasttoken\@undefined\expandafter\cs@gettoken}
\def\cs@gettoken{\futurelet\cs@token\cs@gett@ken}
\def\cs@gett@ken{%
  \ifx\cs@token\cs@lasttoken \def\cs@next{\cs@examinetoken}%
  \else \def\cs@next{\cs@scanword}%
  \fi \cs@next}
%    \end{macrocode}
%  \end{macro}
%  \end{macro}
%  \end{macro}
%  \end{macro}
%
%  \begin{macro}{cs@examinetoken}
%    Examine the token in \cs{cs@token}:
%
%    \begin{itemize}
%    \item
%      If it is a letter (catcode 11) or other (catcode 12), add it
%      to \cs{cs@word} with \cs{cs@addparam}.
%
%    \item
%      If it is the \cs{char} primitive, add it with \cs{cs@expandchar}.
%
%    \item
%      If the token starts or ends a group, ignore it with
%      \cs{cs@ignoretoken}.
%
%    \item
%      Otherwise analyze the meaning of the token with
%      \cs{cs@checkchardef} to detect primitives defined with
%      \cs{chardef}.
%
%    \end{itemize}
%
%    \begin{macrocode}
\def\cs@examinetoken{%
  \ifcat A\cs@token
    \def\cs@next{\cs@addparam}%
  \else\ifcat 0\cs@token
    \def\cs@next{\cs@addparam}%
  \else\ifx\char\cs@token
    \def\cs@next{\afterassignment\cs@expandchar\let\cs@token= }%
  \else\ifx\bgroup\cs@token
    \def\cs@next{\cs@ignoretoken\bgroup}%
  \else\ifx\egroup\cs@token
    \def\cs@next{\cs@ignoretoken\egroup}%
  \else\ifx\begingroup\cs@token
    \def\cs@next{\cs@ignoretoken\begingroup}%
  \else\ifx\endgroup\cs@token
    \def\cs@next{\cs@ignoretoken\endgroup}%
  \else
    \def\cs@next{\expandafter\expandafter\expandafter\cs@checkchardef
      \expandafter\meaning\expandafter\cs@token\string\char\end}%
  \fi\fi\fi\fi\fi\fi\fi\cs@next}
%    \end{macrocode}
%  \end{macro}
%
%  \begin{macro}{\cs@checkchardef}
%    Check the meaning of a token and if it is a primitive defined
%    with \cs{chardef}, pass it to \cs{\cs@examinechar} as if it were
%    a \cs{char} sequence. Otherwise, there are no more word characters,
%    so do the final actions in \cs{cs@nosplit}.
%
%    \begin{macrocode}
\expandafter\def\expandafter\cs@checkchardef
  \expandafter#\expandafter1\string\char#2\end{%
    \def\cs@token{#1}%
    \ifx\cs@token\@empty
      \def\cs@next{\afterassignment\cs@examinechar\let\cs@token= }%
    \else
      \def\cs@next{\cs@nosplit}%
    \fi \cs@next}
%    \end{macrocode}
%  \end{macro}
%
%  \begin{macro}{\cs@ignoretoken}
%    Add a token to \cs{cs@word} but do not update the \cs{cs@wordlen}
%    counter. This is mainly useful for group starting and ending
%    primitives, which need to be preserved, but do not affect the word
%    boundary.
%
%    \begin{macrocode}
\def\cs@ignoretoken#1{%
  \edef\cs@word{\cs@word#1}%
  \afterassignment\cs@continuescan\let\cs@token= }
%    \end{macrocode}
%  \end{macro}
%
%  \begin{macro}{cs@addparam}
%    Add a token to \cs{cs@word} and check its lccode. Note that
%    this macro can only be used for tokens which can be passed as
%    a parameter.
%
%    \begin{macrocode}
\def\cs@addparam#1{%
  \edef\cs@word{\cs@word#1}%
  \cs@checkcode{\lccode`#1}}
%    \end{macrocode}
%  \end{macro}
%
%  \begin{macro}{\cs@expandchar}
%  \begin{macro}{\cs@examinechar}
%    Add a \cs{char} sequence to \cs{cs@word} and check its lccode.
%    The charcode is first parsed in \cs{cs@expandchar} and then the
%    resulting \cs{chardef}-defined sequence is analyzed in
%    \cs{cs@examinechar}.
%
%    \begin{macrocode}
\def\cs@expandchar{\afterassignment\cs@examinechar\chardef\cs@token=}
\def\cs@examinechar{%
  \edef\cs@word{\cs@word\char\the\cs@token\space}%
  \cs@checkcode{\lccode\cs@token}}
%    \end{macrocode}
%  \end{macro}
%  \end{macro}
%
%  \begin{macro}{\cs@checkcode}
%    Check the lccode of a character. If it is zero, it does not count
%    to the current word, so finish it with \cs{cs@nosplit}. Otherwise
%    update the \cs{cs@wordlen} counter and go on scanning the word
%    with \cs{cs@continuescan}. When enough characters are gathered in
%    \cs{cs@word} to allow word break, insert the split hyphen and
%    finish.
%
%    \begin{macrocode}
\def\cs@checkcode#1{%
  \ifnum0=#1
    \def\cs@next{\cs@nosplit}%
  \else
    \advance\cs@wordlen\@ne
    \ifnum\righthyphenmin>\the\cs@wordlen
      \def\cs@next{\cs@continuescan}%
    \else
      \cs@splithyphen
      \def\cs@next{\cs@word}%
    \fi
  \fi \cs@next}
%    \end{macrocode}
%  \end{macro}
%
%  \begin{macro}{\cs@nosplit}
%    Insert a non-breakable hyphen followed by the saved word.
%
%    \begin{macrocode}
\def\cs@nosplit{\cs@boxhyphen\cs@word}
%    \end{macrocode}
%  \end{macro}
%
%  \begin{macro}{\cs@hyphen}
%    The \cs{minus} sequence can be used where the active hyphen
%    does not work, e.g. in arguments to \TeX{} primitives in outer
%    horizontal mode.
%
%    \begin{macrocode}
\let\minus\cs@hyphen
%    \end{macrocode}
%  \end{macro}

%  \begin{macro}{\standardhyphens}
%  \begin{macro}{\splithyphens}
%    These macros control whether split hyphens are allowed in Czech
%    and/or Slovak texts. You may use them in any language, but the
%    split hyphen is only activated for Czech and Slovak.
%
% \changes{slovak-3.1}{2006/10/07}{activate with split hyphens and
%    deactivate with standard hyphens, not vice versa}
%    \begin{macrocode}
\def\standardhyphens{\cs@splithyphensfalse\cs@deactivatehyphens}
\def\splithyphens{\cs@splithyphenstrue\cs@activatehyphens}
%    \end{macrocode}
%  \end{macro}
%  \end{macro}
%
%  \begin{macro}{\cs@splitattr}
%    Now we declare the |split| language attribute.  This is
%    similar to the |split| package option of cslatex, but it
%    only affects Slovak, not Czech.
%
% \changes{slovak-3.1}{2006/10/07}{attribute added}
%    \begin{macrocode}
\def\cs@splitattr{\babel@save\ifcs@splithyphens\splithyphens}
\bbl@declare@ttribute{slovak}{split}{%
  \addto\extrasslovak{\cs@splitattr}}
%    \end{macrocode}
%  \end{macro}
%
%  \begin{macro}{\cs@activatehyphens}
%  \begin{macro}{\cs@deactivatehyphens}
%    These macros are defined as \cs{relax} by default to prevent
%    activating/deactivating the hyphen character. They are redefined
%    when the language is switched to Czech/Slovak. At that moment
%    the hyphen is also activated if split hyphens were requested with
%    \cs{splithyphens}.
%
%    When the language is de-activated, de-activate the hyphen and
%    restore the bogus definitions of these macros.
%
%    \begin{macrocode}
\let\cs@activatehyphens\relax
\let\cs@deactivatehyphens\relax
\expandafter\addto\csname extras\CurrentOption\endcsname{%
  \def\cs@activatehyphens{\bbl@activate{-}}%
  \def\cs@deactivatehyphens{\bbl@deactivate{-}}%
  \ifcs@splithyphens\cs@activatehyphens\fi}
\expandafter\addto\csname noextras\CurrentOption\endcsname{%
  \cs@deactivatehyphens
  \let\cs@activatehyphens\relax
  \let\cs@deactivatehyphens\relax}
%    \end{macrocode}
%  \end{macro}
%  \end{macro}
%
%  \begin{macro}{\cs@looseness}
%  \begin{macro}{\looseness}
%    One of the most common situations where an active hyphen will not
%    work properly is the \cs{looseness} primitive. Change its definition
%    so that it deactivates the hyphen if needed.
%
%    \begin{macrocode}
\let\cs@looseness\looseness
\def\looseness{%
  \ifcs@splithyphens
    \cs@deactivatehyphens\afterassignment\cs@activatehyphens \fi
  \cs@looseness}
%    \end{macrocode}
%  \end{macro}
%  \end{macro}
%
%  \begin{macro}{\cs@selectlanguage}
%  \begin{macro}{\cs@main@language}
%    Specifying the |nocaptions| option means that captions and dates
%    are not redefined by default, but they can be switched on later
%    with \cs{captionsslovak} and/or \cs{dateslovak}. 
%
%    We mimic this behavior by redefining \cs{selectlanguage}.  This
%    macro is called once at the beginning of the document to set the
%    main language of the document.  If this is \cs{cs@main@language},
%    it disables the macros for setting captions and date.  In any
%    case, it restores the original definition of \cs{selectlanguage}
%    and expands it.
%
%    The definition of \cs{selectlanguage} can be shared between Czech
%    and Slovak; the actual language is stored in \cs{cs@main@language}.
%
%    \begin{macrocode}
\ifx\cs@nocaptions\@undefined\else
  \edef\cs@main@language{\CurrentOption}
  \ifx\cs@origselect\@undefined
    \let\cs@origselect=\selectlanguage
    \def\selectlanguage{%
      \let\selectlanguage\cs@origselect
      \ifx\bbl@main@language\cs@main@language
        \expandafter\cs@selectlanguage
      \else
        \expandafter\selectlanguage
      \fi}
    \def\cs@selectlanguage{%
      \cs@tempdisable{captions}%
      \cs@tempdisable{date}%
      \selectlanguage}
%    \end{macrocode}
%
%  \begin{macro}{\cs@tempdisable}
%    \cs{cs@tempdisable} disables a language setup macro temporarily,
%    i.e. the macro with the name of \meta{\#1}|\bbl@main@language|
%    just restores the original definition and purges the saved macro
%    from memory.
%
%    \begin{macrocode}
    \def\cs@tempdisable#1{%
      \def\@tempa{cs@#1}%
      \def\@tempb{#1\bbl@main@language}%
      \expandafter\expandafter\expandafter\let
        \expandafter \csname\expandafter \@tempa \expandafter\endcsname
        \csname \@tempb \endcsname
      \expandafter\edef\csname \@tempb \endcsname{%
        \let \expandafter\noexpand \csname \@tempb \endcsname
          \expandafter\noexpand \csname \@tempa \endcsname
        \let \expandafter\noexpand\csname \@tempa \endcsname
          \noexpand\@undefined}}
%    \end{macrocode}
%  \end{macro}
%
%    These macros are not needed, once the initialization is over.
%
%    \begin{macrocode}
    \@onlypreamble\cs@main@language
    \@onlypreamble\cs@origselect
    \@onlypreamble\cs@selectlanguage
    \@onlypreamble\cs@tempdisable
  \fi
\fi
%    \end{macrocode}
%  \end{macro}
%  \end{macro}
%
%    The encoding of mathematical fonts should be changed to IL2.  This
%    allows to use accented letter in some font families.  Besides,
%    documents do not use CM fonts if there are equivalents in CS-fonts,
%    so there is no need to have both bitmaps of CM-font and CS-font.
%
%    \cs{@font@warning} and \cs{@font@info} are temporarily redefined
%    to avoid annoying font warnings.
%
%    \begin{macrocode}
\ifx\cs@compat@plain\@undefined
\ifx\cs@check@enc\@undefined\else
  \def\cs@check@enc{
    \ifx\encodingdefault\cs@iltw@
      \let\cs@warn\@font@warning \let\@font@warning\@gobble
      \let\cs@info\@font@info    \let\@font@info\@gobble
      \SetSymbolFont{operators}{normal}{\cs@iltw@}{cmr}{m}{n}
      \SetSymbolFont{operators}{bold}{\cs@iltw@}{cmr}{bx}{n}
      \SetMathAlphabet\mathbf{normal}{\cs@iltw@}{cmr}{bx}{n}
      \SetMathAlphabet\mathit{normal}{\cs@iltw@}{cmr}{m}{it}
      \SetMathAlphabet\mathrm{normal}{\cs@iltw@}{cmr}{m}{n}
      \SetMathAlphabet\mathsf{normal}{\cs@iltw@}{cmss}{m}{n}
      \SetMathAlphabet\mathtt{normal}{\cs@iltw@}{cmtt}{m}{n}
      \SetMathAlphabet\mathbf{bold}{\cs@iltw@}{cmr}{bx}{n}
      \SetMathAlphabet\mathit{bold}{\cs@iltw@}{cmr}{bx}{it}
      \SetMathAlphabet\mathrm{bold}{\cs@iltw@}{cmr}{bx}{n}
      \SetMathAlphabet\mathsf{bold}{\cs@iltw@}{cmss}{bx}{n}
      \SetMathAlphabet\mathtt{bold}{\cs@iltw@}{cmtt}{m}{n}
      \let\@font@warning\cs@warn \let\cs@warn\@undefined
      \let\@font@info\cs@info    \let\cs@info\@undefined
    \fi
    \let\cs@check@enc\@undefined}
  \AtBeginDocument{\cs@check@enc}
\fi
\fi
%    \end{macrocode}
%
%  \begin{macro}{cs@undoiltw@}
%
%    The thing is that \LaTeXe{} core only supports the T1 encoding
%    and does not bother changing the uc/lc/sfcodes when encoding
%    is switched. :( However, the IL2 encoding \emph{does} change
%    these codes, so if encoding is switched back from IL2, we must
%    also undo the effect of this change to be compatible with
%    \LaTeXe.  OK, this is not the right\textsuperscript{TM} solution
%    but it works.  Cheers to Petr Ol\v s\'ak.
%
%    \begin{macrocode}
\def\cs@undoiltw@{%
  \uccode158=208 \lccode158=158 \sfcode158=1000
  \sfcode159=1000
  \uccode165=133 \lccode165=165 \sfcode165=1000
  \uccode169=137 \lccode169=169 \sfcode169=1000
  \uccode171=139 \lccode171=171 \sfcode171=1000
  \uccode174=142 \lccode174=174 \sfcode174=1000
  \uccode181=149
  \uccode185=153
  \uccode187=155
  \uccode190=0   \lccode190=0
  \uccode254=222 \lccode254=254 \sfcode254=1000
  \uccode255=223 \lccode255=255 \sfcode255=1000}
%    \end{macrocode}
%  \end{macro}
%
%  \begin{macro}{@@enc@update}
%
%    Redefine the \LaTeXe{} internal function \cs{@@enc@update} to
%    change the encodings correctly.
%
%    \begin{macrocode}
\ifx\cs@enc@update\@undefined
\ifx\@@enc@update\@undefined\else
  \let\cs@enc@update\@@enc@update
  \def\@@enc@update{\ifx\cf@encoding\cs@iltw@\cs@undoiltw@\fi
    \cs@enc@update
    \expandafter\ifnum\csname l@\languagename\endcsname=\the\language
      \expandafter\ifx
      \csname l@\languagename:\f@encoding\endcsname\relax
      \else
        \expandafter\expandafter\expandafter\let
          \expandafter\csname
          \expandafter l\expandafter @\expandafter\languagename
          \expandafter\endcsname\csname l@\languagename:\f@encoding\endcsname
      \fi
      \language=\csname l@\languagename\endcsname\relax
    \fi}
\fi\fi
%    \end{macrocode}
%  \end{macro}
%
%    The macro |\ldf@finish| takes care of looking for a
%    configuration file, setting the main language to be switched on
%    at |\begin{document}| and resetting the category code of
%    \texttt{@} to its original value.
% \changes{slovak-1.2i}{1996/11/03}{Now use \cs{ldf@finish} to wrap up}
%    \begin{macrocode}
\ldf@finish\CurrentOption
%</code>
%    \end{macrocode}
%
% \Finale
%%
%% \CharacterTable
%%  {Upper-case    \A\B\C\D\E\F\G\H\I\J\K\L\M\N\O\P\Q\R\S\T\U\V\W\X\Y\Z
%%   Lower-case    \a\b\c\d\e\f\g\h\i\j\k\l\m\n\o\p\q\r\s\t\u\v\w\x\y\z
%%   Digits        \0\1\2\3\4\5\6\7\8\9
%%   Exclamation   \!     Double quote  \"     Hash (number) \#
%%   Dollar        \$     Percent       \%     Ampersand     \&
%%   Acute accent  \'     Left paren    \(     Right paren   \)
%%   Asterisk      \*     Plus          \+     Comma         \,
%%   Minus         \-     Point         \.     Solidus       \/
%%   Colon         \:     Semicolon     \;     Less than     \<
%%   Equals        \=     Greater than  \>     Question mark \?
%%   Commercial at \@     Left bracket  \[     Backslash     \\
%%   Right bracket \]     Circumflex    \^     Underscore    \_
%%   Grave accent  \`     Left brace    \{     Vertical bar  \|
%%   Right brace   \}     Tilde         \~}
%%
\endinput
}
\DeclareOption{slovene}{% \iffalse meta-comme

% Copyright 1989-2005 Johannes L. Braams and any individual autho
% listed elsewhere in this file.  All rights reserve

% This file is part of the Babel syste
% ------------------------------------

% It may be distributed and/or modified under t
% conditions of the LaTeX Project Public License, either version 1
% of this license or (at your option) any later versio
% The latest version of this license is
%   http://www.latex-project.org/lppl.t
% and version 1.3 or later is part of all distributions of LaT
% version 2003/12/01 or late

% This work has the LPPL maintenance status "maintained

% The Current Maintainer of this work is Johannes Braam

% The list of all files belonging to the Babel system
% given in the file `manifest.bbl. See also `legal.bbl' for addition
% informatio

% The list of derived (unpacked) files belonging to the distributi
% and covered by LPPL is defined by the unpacking scripts (wi
% extension .ins) which are part of the distributio
% \
% \CheckSum{14
% \iffal
%    Tell the \LaTeX\ system who we are and write an entry on t
%    transcrip
%<*dt
\ProvidesFile{slovene.dt
%</dt
%<code>\ProvidesLanguage{sloven
%\
%\ProvidesFile{slovene.dt
        [2005/03/31 v1.2m Slovene support from the babel syste
%\iffal
%% File `slovene.dt
%% Babel package for LaTeX version
%% Copyright (C) 1989 - 20
%%           by Johannes Braams, TeXni

%% Please report errors to: J.L. Braa
%%                          babel at braams.cistron.

%    This file is part of the babel system, it provides the sour
%    code for the Slovenian language definition file.  The Sloveni
%    words were contributed by Danilo Zavrtanik, University
%    Ljubljana (Slovenia, former Yugoslavia
%    The usage of the active " was introduced
%        Leon \v{Z}lajp
%        Jo\v{z}ef Stefan Institut
%        Jamova 39, Ljubljan
%        Sloven
%        e-mail: leon.zlajpah at ijs.

%<*filedrive
\documentclass{ltxdo
\newcommand*\TeXhax{\TeX ha
\newcommand*\babel{\textsf{babel
\newcommand*\langvar{$\langle \it lang \rangle
\newcommand*\note[1]
\newcommand*\Lopt[1]{{textsf \1
\newcommand*\file[1]{\texttt{#1
\begin{documen
 \DocInput{slovene.dt
\end{documen
%</filedrive
%\
% \GetFileInfo{slovene.dt

% \changes{slovene-1.0a}{1991/07/15}{Renamed babel.sty in babel.co
% \changes{slovene-1.1}{1992/02/16}{Brought up-to-date with babel 3.2
% \changes{slovene-1.2}{1994/02/27}{Update for \LaTeX
% \changes{slovene-1.2d}{1994/06/26}{Removed the use of \cs{filedat
%    and moved identification after the loading of \file{babel.def
% \changes{slovene-1.2i}{1995/11/03}{Replaced `Slovanian' with corre
%    `Slovenian'
% \changes{slovene-1.2i}{1996/10/10}{Replaced \cs{undefined} wi
%    \cs{@undefined} and \cs{empty} with \cs{@empty} for consisten
%    with \LaTeX, moved the definition of \cs{atcatcode} right to t
%    beginning

%  \section{The Slovenian languag

%    The file \file{\filename}\footnote{The file described in th
%    section has version number \fileversion\ and was last revised
%    \filedate.  Contributions were made by Danilo Zavrtani
%    University of Ljubljana (YU) and Leon \v{Z}lajp
%    (\texttt{leon.zlajpah@ijs.si}).}  defines all t
%    language-specific macros for the Slovenian languag

%    For this language the character |"| is made active.
%    table~\ref{tab:slovene-quote} an overview is given of i
%    purpose. One of the reasons for this is that in the Slove
%    language some special characters are use

%    \begin{table}[ht
%     \begin{cente
%     \begin{tabular}{lp{8cm
%      |"c| & |\"c|, also implemented for th
%                  lowercase and uppercase s and z.
%      |"-| & an explicit hyphen sign, allowing hyphenati
%                  in the rest of the word.
%      |""| & like |"-|, but producing no hyphen si
%                  (for compund words with hyphen, e.g.\ |x-""y|).
%      |"`| & for Slovene left double quotes (looks like ,,).
%      |"'| & for Slovene right double quotes.
%      |"<| & for French left double quotes (similar to $<<$).
%      |">| & for French right double quotes (similar to $>>$).
%     \end{tabula
%     \caption{The extra definitions ma
%              by \file{slovene.ldf}}\label{tab:slovene-quot
%     \end{cente
%    \end{tabl

% \StopEventually

%    The macro |\LdfInit| takes care of preventing that this file
%    loaded more than once, checking the category code of t
%    \texttt{@} sign, et
% \changes{slovene-1.2i}{1996/11/03}{Now use \cs{LdfInit} to perfo
%    initial checks
%    \begin{macrocod
%<*cod
\LdfInit{slovene}\captionsslove
%    \end{macrocod

%    When this file is read as an option, i.e. by the |\usepackag
%    command, \texttt{slovene} will be an `unknown' language in whi
%    case we have to make it known. So we check for the existence
%    |\l@slovene| to see whether we have to do something her

% \changes{slovene-1.0b}{1991/10/29}{Removed use of \cs{@ifundefined
% \changes{slovene-1.1}{1992/02/16}{Added a warning when
%    hyphenation patterns were loaded
% \changes{slovene-1.2d}{1994/06/26}{Now use \cs{@nopatterns}
%    produce the warnin
%    \begin{macrocod
\ifx\l@slovene\@undefin
    \@nopatterns{Sloven
    \adddialect\l@slovene0\
%    \end{macrocod

%    The next step consists of defining commands to switch to t
%    Slovenian language. The reason for this is that a user might wa
%    to switch back and forth between language

% \begin{macro}{\captionssloven
%    The macro |\captionsslovene| defines all strings used in the fo
%    standard documentclasses provided with \LaTe
% \changes{slovene-1.1}{1992/02/16}{Added \cs{seename}, \cs{alsonam
%    and \cs{prefacename
% \changes{slovene-1.1}{1993/07/15}{\cs{headpagename} should
%    \cs{pagename
% \changes{slovene-1.2b}{1994/06/04}{Added extra translations fr
%    Josef Leydold, \texttt{leydold@statrix2.wu-wien.ac.at
% \changes{slovene-1.2g}{1995/07/04}{Added \cs{proofname} f
%    AMS-\LaTe
% \changes{slovene-1.2h}{1995/07/25}{Added translation of `Proof
% \changes{slovene-1.2m}{2000/09/20}{Added \cs{glossaryname
%    \begin{macrocod
\addto\captionsslovene
  \def\prefacename{Predgovor
  \def\refname{Literatura
  \def\abstractname{Povzetek
  \def\bibname{Literatura
  \def\chaptername{Poglavje
  \def\appendixname{Dodatek
  \def\contentsname{Kazalo
  \def\listfigurename{Slike
  \def\listtablename{Tabele
  \def\indexname{Stvarno kazalo}% used to be Inde
  \def\figurename{Slika
  \def\tablename{Tabela
  \def\partname{Del
  \def\enclname{Priloge
  \def\ccname{Kopije
  \def\headtoname{Prejme
  \def\pagename{Stran
  \def\seename{glej
  \def\alsoname{glej tudi
  \def\proofname{Dokaz
  \def\glossaryname{Glossary}% <-- Needs translati

%    \end{macrocod
% \end{macr

% \begin{macro}{\datesloven
%    The macro |\dateslovene| redefines the command |\today|
%    produce Slovenian date
% \changes{slovene-1.2j}{1997/10/01}{Use \cs{edef} to defi
%    \cs{today} to save memor
% \changes{slovene-1.2j}{1998/03/28}{use \cs{def} instead
%    \cs{edef
%    \begin{macrocod
\def\dateslovene
  \def\today{\number\day.~\ifcase\month\
    januar\or februar\or marec\or april\or maj\or junij\
    julij\or avgust\or september\or oktober\or november\or december\
    \space \number\year
%    \end{macrocod
% \end{macr

% \begin{macro}{\extrassloven
% \begin{macro}{\noextrassloven
%    The macro |\extrasslovene| performs all the extra definitio
%    needed for the Slovenian language. The macro |\noextrassloven
%    is used to cancel the actions of |\extrasslovene|

%    For Slovene the \texttt{"} character is made active. This is do
%    once, later on its definition may vary. Other languages in t
%    same document may also use the \texttt{"} character f
%    shorthands; we specify that the slovenian group of shorthan
%    should be use

% \changes{slovene-1.2f}{1995/06/04}{Introduced the active \texttt{"
%    \begin{macrocod
\initiate@active@char{
\addto\extrasslovene{\languageshorthands{slovene
\addto\extrasslovene{\bbl@activate{"
%    \end{macrocod
%    Don't forget to turn the shorthands off agai
% \changes{slovene-1.2l}{1999/12/17}{Deactivate shorthands ouside
%    Sloven
%    \begin{macrocod
\addto\noextrasslovene{\bbl@deactivate{"
%    \end{macrocod
%    First we define shorthands to facilitate the occurence of lette
%    such as \v{c
% \changes{slovene-1.2i}{1996/09/21}{removed shorthand for \texttt{"
%    as it is not needed for slovenian
%    \begin{macrocod
\declare@shorthand{slovene}{"c}{\textormath{\v c}{\check c
\declare@shorthand{slovene}{"s}{\textormath{\v s}{\check s
\declare@shorthand{slovene}{"z}{\textormath{\v z}{\check z
\declare@shorthand{slovene}{"C}{\textormath{\v C}{\check C
\declare@shorthand{slovene}{"S}{\textormath{\v S}{\check S
\declare@shorthand{slovene}{"Z}{\textormath{\v Z}{\check Z
%    \end{macrocod

%    Then we define access to two forms of quotation marks, simil
%    to the german and french quotation mark
% \changes{slovene-1.2j}{1997/04/03}{Removed empty groups aft
%    double quote and guillemot character
%    \begin{macrocod
\declare@shorthand{slovene}{"`}
  \textormath{\quotedblbase}{\mbox{\quotedblbase}
\declare@shorthand{slovene}{"'}
  \textormath{\textquotedblleft}{\mbox{\textquotedblleft}
\declare@shorthand{slovene}{"<}
  \textormath{\guillemotleft}{\mbox{\guillemotleft}
\declare@shorthand{slovene}{">}
  \textormath{\guillemotright}{\mbox{\guillemotright}
%    \end{macrocod
%    then we define two shorthands to be able to specify hyphenati
%    breakpoints that behavew a little different from |\-
%    \begin{macrocod
\declare@shorthand{slovene}{"-}{\nobreak-\bbl@allowhyphen
\declare@shorthand{slovene}{""}{\hskip\z@ski
%    \end{macrocod
%    And we want to have a shorthand for disabling a ligatur
%    \begin{macrocod
\declare@shorthand{slovene}{"|}
  \textormath{\discretionary{-}{}{\kern.03em}}{
%    \end{macrocod
% \end{macr
% \end{macr

%    The macro |\ldf@finish| takes care of looking for
%    configuration file, setting the main language to be switched
%    at |\begin{document}| and resetting the category code
%    \texttt{@} to its original valu
% \changes{slovene-1.2i}{1996/11/03}{Now use \cs{ldf@finish} to wr
%    up
%    \begin{macrocod
\ldf@finish{sloven
%</cod
%    \end{macrocod

% \Fina

%% \CharacterTab
%%  {Upper-case    \A\B\C\D\E\F\G\H\I\J\K\L\M\N\O\P\Q\R\S\T\U\V\W\X\Y
%%   Lower-case    \a\b\c\d\e\f\g\h\i\j\k\l\m\n\o\p\q\r\s\t\u\v\w\x\y
%%   Digits        \0\1\2\3\4\5\6\7\8
%%   Exclamation   \!     Double quote  \"     Hash (number)
%%   Dollar        \$     Percent       \%     Ampersand
%%   Acute accent  \'     Left paren    \(     Right paren
%%   Asterisk      \*     Plus          \+     Comma
%%   Minus         \-     Point         \.     Solidus
%%   Colon         \:     Semicolon     \;     Less than
%%   Equals        \=     Greater than  \>     Question mark
%%   Commercial at \@     Left bracket  \[     Backslash
%%   Right bracket \]     Circumflex    \^     Underscore
%%   Grave accent  \`     Left brace    \{     Vertical bar
%%   Right brace   \}     Tilde         \

\endinp
}
\DeclareOption{spanish}{% \iffalse meta-comment
%
% Copyright 1989-2011 Johannes L. Braams and any individual authors
% listed elsewhere in this file.  All rights reserved.
% 
% This file is part of the Babel system.
% --------------------------------------
% 
% It may be distributed and/or modified under the
% conditions of the LaTeX Project Public License, either version 1.3
% of this license or (at your option) any later version.
% The latest version of this license is in
%   http://www.latex-project.org/lppl.txt
% and version 1.3 or later is part of all distributions of LaTeX
% version 2003/12/01 or later.
% 
% This work has the LPPL maintenance status "maintained".
% 
% The Current Maintainer of this work is Johannes Braams.
% 
% The list of all files belonging to the Babel system is
% given in the file `manifest.bbl. See also `legal.bbl' for additional
% information.
% 
% The list of derived (unpacked) files belonging to the distribution
% and covered by LPPL is defined by the unpacking scripts (with
% extension .ins) which are part of the distribution.
% \fi
% \ProvidesFile{spanish.dtx}
%       [2010/05/23 v5.0j Spanish support from the babel system] 
%\iffalse
%% File `spanish.dtx'
%% Babel package for LaTeX version 2e
%% Copyright (C) 1989 - 2008
%%           by Johannes Braams, TeXniek
%
%% Spanish Language Definition File
%% Copyright (C) 1997 - 2010
%%        Javier Bezos (www.tex-tipografia.com)
%%     and
%%        CervanTeX (www.cervantex.es)
%
%% Please report errors to: Javier Bezos (preferably)
%%                          www.tex-tipografia.com
%%                          J.L. Braams
%%                          www.latex-project.org
%
%    This file is part of the babel system, it provides the source
%    code for the Spanish language definition file.
%    The original version of this file was written by Javier Bezos.
%    The latest release is available on CTAN:/language/spanish/
% \fi
%
% \iffalse
%<*filedriver>
\let\ooverb\verb
\documentclass[spanish,a4paper]{ltxdoc}
\let\verb\ooverb
\usepackage{babel}
\usepackage{hyperref}

\let\meta\emph

\usepackage{pslatex,mathptmx,color}
\usepackage[cp1252]{inputenc}
\usepackage[T1]{fontenc}
\newcommand\act[1]{%
 \\%
 \makebox[1.5pc][l]{\textcolor{green}{$\surd$}}%
 \textsf{#1}\ignorespaces}
\newcommand\deact[1]{%
 \\%
 \makebox[1.5pc][l]{\textcolor{red}{$\times$}}%
 \texttt{#1}\ignorespaces}
\newcommand\txt{\makebox[1.5pc][l]{\textcolor{blue}{$\Rightarrow$}}\ignorespaces}
\newcommand\con{\makebox[1.5pc][l]{\textcolor{magenta}{$\star$}}\ignorespaces}
\newcommand\alw{%
 \\%
 \makebox[1.5pc][l]{\textcolor{green}{$\surd$}}%
 Se define siempre, sin depender de un grupo.}
\newcommand\opp{\qquad Opci�n de paquete}

\newcommand*\babel{\textsf{babel}}
\newcommand*\file[1]{\texttt{#1}}

\setlength{\arrayrulewidth}{2\arrayrulewidth}
\newcommand\toprule[1]{\cline{1-#1}\\[-2ex]}
\newcommand\botrule[1]{\\[.6ex]\cline{1-#1}}
\newcommand\hmk{$\string|$}

\newenvironment{decl}[1][]%
 {\par\small\addvspace{4.5ex plus 1ex}%
  \vskip-\parskip
  \ifx\relax#1\relax
   \def\@decl@date{}%
  \else
   \def\@decl@date{\NEWfeature{#1}}%
  \fi
  \noindent
  \begin{tabular}{|l|}\hline\ignorespaces}%
 {\\\hline\end{tabular}\nobreak\@decl@date\par\nobreak
  \vspace{2.3ex}\vskip-\parskip}

\newcommand\New[1]{%
 \leavevmode\marginpar{\raggedleft\sffamily Nuevo en #1}}

\newcommand\nm[1]{\unskip\,$^{#1}$}
\newcommand\nt[1]{\quad$^{#1}$\,\ignorespaces}

\makeatletter
 \renewcommand\@biblabel{}
\makeatother

\newcommand\DOT[1]{\lsc{DOT},~#1}
\newcommand\DTL[1]{\lsc{DTL},~#1}
\newcommand\MEA[1]{\lsc{MEA},~#1}

\raggedright
\setlength{\parindent}{0em}
\setlength{\parskip}{3pt}

\addtolength{\oddsidemargin}{-4pc}
\addtolength{\textwidth}{7pc}

\OnlyDescription
\begin{document}
  \DocInput{spanish.dtx}
\end{document}
%</filedriver>
%\fi
%
% \begingroup
% \ifx\langdeffile\undefined
%
%^^A ======= Beginning of text as typeset by spanish.dtx =========
%
%
% \title{Estilo \textsf{spanish}\\
%  para el sistema \babel.\footnote{Este
%  archivo est� actualmente en la versi�n
%  5.0j con fecha 23 de mayo del 2010. ^^A@#
%  Esta copia del manual se compuso el~\today.}}
%
% \author{Javier Bezos\footnote{Por favor, env�en comentarios y
% sugerencias en http://www.tex-tipografia.com/spanish.html.  Han
% colaborado de una u otra forma muchas personas, a las cuales
% agradezco sus comentarios y sugerencias; en particular, han sido muy
% activos Juan Luis Varona y Jos� Luis Rivera.  Para m�s informaci�n
% sobre los criterios seguidos, v�ase la referencia: Javier Bezos,
% \textit{Tipograf�a espa�ola con \TeX.} Para informaci�n sobre
% actualizaciones: http://www.cervantex.es/}}
%
% \date{23 de mayo del 2010} ^^A@#
%
% \maketitle
% 
% {\small\tableofcontents}
%
% \section*{S�mbolos empleados}
%
% \begin{itemize}
% \item[\textcolor{blue}{$\Rightarrow$}] Macros para 
% ser usadas en el texto (generan texto o lo estructuran).
% \item[\textcolor{magenta}{$\star$}] Macros de 
% configuraci�n y preferencias.
% \item[\textcolor{green}{$\surd$}] Grupo que 
% activa la orden.
% \item[\textcolor{red}{$\times$}] Opciones de 
% paquete que anulan la orden. En redonda van las destinadas 
% espec�ficamente a anular ese punto, y en cursiva las que adem�s
% anulan otros aspectos del estilo.
% \end{itemize}
%
% \section{Uso de \textsf{spanish} para babel}
%
% El estilo \textsf{spanish} para babel adapta una serie de elementos
% de los documentos de \LaTeX\ al castellano, tanto en las
% traducciones como en la tipograf�a.  Para usarlo, basta con dar
% la opci�n \textsf{spanish} al cargar babel: \begin{verbatim}
% \usepackage[spanish]{babel} \end{verbatim}
%
% Esto es todo lo que hace falta para conseguir que el documento tenga
% un aspecto espa�ol. En caso de estar en M�xico, v�ase, adem�s, el
% apartado \ref{paises} (<<Opciones por pa�ses>>):\footnote{En pr�ximas
% versiones se a�adir�n m�s pa�ses.}
%\begin{verbatim}
%\usepackage[spanish,mexico]{babel}
%\end{verbatim}
%
% El estilo \textsf{spanish} se puede cargar junto con otras lenguas (v�ase el 
% manual de babel). Si \textsf{spanish} es la �ltima de las lenguas cargadas,
% entonces se considera la lengua principal y se hacen una serie de
% ajustes tipogr�ficos adicionales. En particular, se modifican
% �rdenes y entornos como:
%\begin{center}
%\begin{tabular}{lll}
%  |enumerate| &  |\roman|    &  |\section|\\
%  |itemize|   &  |\fnsymbol| &  |\subsection|\\
%  |\%|        &  |\alph|     &  |\subsubsection|\\
%                  &  |\Alph|     & \\
%\end{tabular}
%\end{center}
%
% El estilo est� pensado para que sea muy configurable.  Para ello, se
% proporcionan una serie de opciones de paquete, que en caso de
% emplearse deben ir \textit{despu�s} de \textsf{spanish}.  Por
% ejemplo:
% \begin{verbatim}
% \usepackage[french,spanish,es-noindentfirst]{babel} \end{verbatim}
% carga los estilos para el franc�s y el espa�ol, esta �ltima como
% lengua principal; adem�s, evita que \textsf{spanish} sangre el
% primer p�rrafo tras un t�tulo.  Otras opciones se pueden ajustar por
% medio de macros, en particular aquellas que se puede desear cambiar
% en medio del documento (por ejemplo, el formato de la fecha).
%
% Los cambios est�n organizados en una serie de grupos: 
% \textsf{captions, date, text, math} y \textsf{shorthands}.
% Los tres ultimos corresponden a lo que en babel ser�a normalmente
% \textsf{extras}.
%
% \section{\textsf{spanish} como lengua principal}
% 
% Si la lengua principal es \textsf{spanish}, se introducen una serie de
% cambios en el momento de cargar la lengua para adaptar varios
% elementos a los usos tipogr�ficos espa�oles.  Estos cambios
% funcionan con las clases est�ndar "+--con otras tal vez alguno de
% ellos no tenga efecto--- y persisten durante todo el documento.
% Ninguno de ellos es necesario para componer el documento, aunque
% naturalmente el resultado ser� distinto.
%
% \subsection{Listas}
%
% \begin{decl} \txt |\begin{enumerate} ... \end{enumerate}|%
% \deact{es-nolists, es-noenumerate, \textit{es-nolayout, es-minimal,
% es-sloppy}} \end{decl}
%
% Usa la siguiente secuencia:\\
% \quad 1.\\
% \qquad \emph{a})\\
% \quad\qquad 1)\\
% \qquad\qquad \emph{a$'$})
%
% \begin{decl} \txt |\begin{itemize} ... \end{itemize}|%
% \deact{es-nolists, es-noitemize, \textit{es-nolayout, es-minimal,
% es-sloppy}} \end{decl}
%
% Usa la siguiente secuencia:\\
% \quad\leavevmode\hbox to 1.2ex
%     {\hss\vrule height .95ex width .8ex depth -.15ex\hss}\\
% \qquad\textbullet\\
% \quad\qquad $\circ$\\
% \qquad\qquad $\diamond$
%
% \begin{decl}
% \con  |\spanishdashitems    \spanishsignitems|
% \end{decl}
%
% Dos �rdenes  para cambiar a otros estilos en
% |itemize|: rayas en todos los niveles y \textbullet{} $\circ$
% $\diamond$ $\triangleright$, respectivamente.
%
% \begin{decl}
% \con  |es-nolists|\opp
% \end{decl}
%
% Desactiva los cambios en las listas (aunque  |\es@enumerate| y 
%  |\es@itemize| siguen disponibles).
%
% \subsection{Contadores}
% 
% \begin{decl}
% \txt |\alph    \Alph|\deact{\textit{es-nolayout, es-sloppy}}
% \end{decl}
%
% Incluyen la e�e.
%
% \begin{decl}
% \txt  |\fnsymbol|\deact{\textit{es-nolayout, es-sloppy}}
% \end{decl}
%
% Se emplean uno, dos, tres... asteriscos (*, **, ***, etc.),
% en lugar de la sucesi�n angloamericana de cruces, barras,
% etc.\footnote{\DOT{162}.}
%
% \begin{decl}
% \txt  |\roman|\deact{es-ucroman, es-lcroman, \textit{es-nolayout, es-minimal, es-sloppy}}
% \end{decl}
%
% Como en castellano no se usan n�meros romanos en min�scula,
% |\roman| se redefine para que los d� en
% versalitas.\footnote{\DTL{197}.} La opci�n de paquete
%  |es-minimal| los desactiva con  |es-ucroman|, y |es-sloppy|
% con  |es-lcroman|.

% \begin{decl}
% \con  |es-ucroman|\opp
% \end{decl}
%
% Opci�n de paquete adicional, que pasa todos los romanos a versales,
% en caso de que no se quiera la versalita o por incompatibilidad con
% alg�n paquete que use de forma indebida  |\roman|.\footnote{En
% el momento de escribir esto, como m�nimo son: \textsf{dramatist,
% epiolmec, flashcards, lipsum, ntheorem, ntheorem-hyper,
% texmate.} Otros paquetes como \textsf{hyperref, easy} y \textsf{exam} 
% ya han sido corregidos.}
%
% \begin{decl}
% \con  |es-lcroman|\opp
% \end{decl}
%
% Como �ltimo recurso, de haber problemas con el valor predeterminado
% o con  |es-ucroman|, con esta opci�n de paquete puede dejarse la
% definici�n de \LaTeX, aunque en espa�ol los romanos en min�scula
% sean una falta ortogr�fica.
%
% \begin{decl}
% \con  |es-preindex|\opp
% \end{decl}
%
% \textit{MakeIndex} no puede entender la forma en que |\roman|
% escribe el n�mero de p�gina, por lo que elimina las l�neas
% afectadas.  Por ello el archivo |.idx| ha de ser convertido antes de
% procesarlo con \textit{MakeIndex}.  Con este paquete se proporciona
% la utilidad |romanidx.sty| que se encarga de ello.  Simplemente se
% compone ese archivo con \LaTeX{} y a continuaci�n se responde a las
% preguntas que se formulan; el archivo resultante, es decir, el que
% hay que procesar con \textit{MakeIndex,} tiene la extensi�n
% \texttt{eix}.  Este proceso no es necesario si no se introdujo
% ninguna entrada de �ndice en p�ginas numeradas con |\roman| (lo
% cual ser� lo m�s normal).  Si un s�mbolo propio de
% \emph{MakeIndex} generara problemas, debe encerrarse entre llaves:
% \verb={"|}=.
%
% Con la opci�n de paquete  |es-preindex| se llama desde el
% documento |romanidx.sty|, de forma que no es necesaria su ejecuci�n
% aparte. Tampoco pide ning�n dato, sino que ha de darse en el 
% documento principal con la siguiente orden.
%
% \begin{decl}
% \con  |\spanishindexchars|\marg{encap}\marg{open\_range}\marg{close\_range}
% \end{decl}
%
% De usarse |es-preindex| con un estilo de �ndice que no tiene los
% valores predeterminados de estos tres caracteres especiales, hay que
% darlos con esta orden (es decir, por omisi�n es
% \verb+\spanishindexchars{|}{(}{)}+).
%
% \begin{decl}
% \con  |\spanishscroman    \spanishlcroman    \spanishucroman|
% \end{decl}
%
% Finalmente, tres macros permiten cambios temporales en el
% documento de  |\roman| a versalitas, min�sculas y may�sculas,
% respectivamente.
%
% \subsection{Otros}
%
% \begin{decl}
% \txt  |\guillemotleft    \guillemotright|\deact{\textit{es-nolayout, es-sloppy}}
% \end{decl}
%
% Las comillas latinas para |OT1| son menos angulosas y se generan
% con unas puntas de flecha de |lasy|. En T1 no hay cambios.
%
% \begin{decl}
% \txt  |\section|, |\subsection|, etc., 
% |\tableofcontents|\deact{es-nosectiondot, es-noindentfirst, \textit{es-nolayout, 
% es-mininal, es-sloppy}}
% \end{decl}
%
% Los n�meros en los t�tulos est�n seguidos de un punto 
% tanto en el texto como en el �ndice. Adem�s,
% el primer p�rrafo tras el t�tulo no elimina la sangr�a.
% 
% \begin{decl}
% \con  |es-nolayout|\opp
% \end{decl}
% 
% Si no se desea ninguno de estos cambios, basta con usar esta opci�n 
% de paquete.
%
% \section{Traducciones}
%
% \subsection{Nombres}
% 
% \begin{decl}
% \txt  |\refname|, |\tablename|, |\contentsname|, etc.\\
% \con  |\spanishrefname|, |\spanishtablename|, |\spanishcontentsname|, etc.
% \act{captions}
% \end{decl}
%
%    Establecen las traducciones al castellano de algunos t�rminos,
%    tal y como se describe en el cuadro 1.  Para cambiar el texto
%    de ellas, conviene redefinir la forma que empieza con
%     |\spanish...|, ya que, al contrario que las �rdenes
%    |\refname|, |\abstractname|, etc., se pueden redefinir cuando
%    se desee y entran en acci�n al momento y de forma permanente, sin
%    necesidad de |\addto|.
% 
% \begin{table}
% \center\small
% \newcommand\name[2]{%
% \texttt{\textbackslash#1name}&%
% \texttt{\textbackslash spanish#1name}&}
% \caption{Traducciones}
% \vspace{1.5ex}
% \begin{tabular}{l@{\hspace{3em}}l@{\hspace{3em}}l}
% \toprule3
% \name{ref}        & Referencias\\
% \name{abstract}   & Resumen\\
% \name{bib}        & Bibliograf�a\\
% \name{chapter}    & Cap�tulo\\
% \name{appendix}   & Ap�ndice\\
% \name{contents}   & �ndice general\nm{a}\\
% \name{listfigure} & �ndice de figuras\\
% \name{listtable}  & �ndice de cuadros\\
% \name{index}      & �ndice alfab�tico\\
% \name{figure}     & Figura\\
% \name{table}      & Cuadro\\
% \name{part}       & Parte\\
% \name{encl}       & Adjunto\\
% \name{cc}         & Copia a\\
% \name{headto}     & A\\
% \name{page}       & p�gina\\
% \name{see}        & v�ase\\
% \name{also}       & v�ase tambi�n\\
% \name{proof}     & Demostraci�n
% \botrule3
% \end{tabular}
%
% \vspace{1.5ex}
%
% \begin{minipage}{10cm}\footnotesize
% \nt{a} Pero s�lo <<�ndice>> en \textsf{article}.
% \end{minipage}
% \end{table}
% 
% \begin{decl}
% \con  |es-uppernames|\opp
% \end{decl}
%
% Aunque sea un anglicismo,\footnote{\DOT{197}.} con esta opci�n de
% paquete los sustantivos tienen may�scula inicial.
%
% \begin{decl}
% \con  |es-tabla|\opp
% \end{decl}
%
% En caso de que todos los cuadros sean tablas, esta opci�n permite
% cambiar \textit{cuadro} por \textit{tabla} (en cierto modo,
% \textit{cuadro} es a \textit{tabla} lo que \texttt{table} es a
% \texttt{tabular}).
%
% \subsection{Fechas}
%
% \begin{decl}
% \txt  |\today    \Today|
% \act{date}
% \end{decl}
%
%    Fecha  actual, en la forma \textit{1 de enero de 
%    2004.} Con  |\Today| el mes va en may�scula.
%
% \begin{decl}
% \con  |\spanishdatedel    \spanishdatede|
% \end{decl}
%
%    Con la primera se cambia el formato para que a partir del 2000 se
%    emplee \textit{del} y no \textit{de} (recomendado).  La segunda
%    hace justo lo contrario (predeterminado).
%
% \begin{decl}
% \con  |\spanishreverseddate|
% \end{decl}
%
%   Cambia el formato de  |\today| a la forma
%   \textit{enero 1 del 2004.} Con  |\Today| el mes va en 
%   may�scula.
%
% \section{Abreviaciones (\textit{shorthands})}
% 
%    La lista completa se puede encontrar en el cuadro 2.  En los
%    siguientes apartados se dar�n m�s detalles sobre algunas de
%    ellas.
%
% \begin{table}[!t]
% \center\small
% \caption{Abreviaciones}
% \vspace{1.5ex}
% \begin{tabular}{l@{\hspace{3em}}l@{\hspace{3em}}l}
% \toprule2
% |� � � � �| & � � � � �\\
% |� � � � �| & � � � � �\\
% |� �|          & � �\nm{a}\\
% |"u "U|          & "u "U\\
% |"i "I|          & "i "I\\
% |"a "A "o "O|    & Ordinales: 1"a, 1"A, 1"o, 1"O\\
% |"er "ER|        & Ordinales: 1"er, 1"ER\\
% |"c "C|          & "c "C\\
% |"rr "RR|        & rr, pero -r cuando se divide\\
% |"y|             & El antiguo signo para <<y>>\\
% |"-|             & Como |\-|, pero permite m�s divisiones\\
% |"=|             & Como |-|, pero permite mas divisiones\nm{b}\\
% |"~|             & Gui�n estil�stico\nm{c}\\
% |"+ "+- "+--|    & Como |-|, |--| y |---|, pero sin divisi�n\\
% |~- ~-- ~---|    & Lo mismo que el anterior.\\
% |""|             & Permite mas divisiones antes y despu�s\nm{d}\\
% |"/|             & Una barra algo m�s baja\\
% \verb+"|+        & Divide un logotipo\nm{e}\\
% |"< ">|          & "< ">\\
% |"` "'|          & |\begin{quoting}| |\end{quoting}|\nm{f}\\
% |<< >>|          & Lo mismo que el anterior.\\
% |?` !`|          & ?` !`\nm{g}\\
%|"? "!|           & "? "! alineados con la linea base\nm{h}
% \botrule2
% \end{tabular}
%
% \vspace{1.5ex}
%
% \begin{minipage}{11cm}
% \footnotesize
% \nt{a} La forma |~n| no est� activada por omisi�n a partir de 
% la versi�n 5.
% \nt{b} |"=| viene a ser lo mismo que |""-""|.
% \nt{c} Esta abreviaci�n tiene un uso distinto
% en otras lenguas de babel.
% \nt{d} Como en <<entrada/salida>>.
% \nt{e} Carece de uso en castellano.
% \nt{f} V�ase sec.~2.7. 
% \nt{g} No proporcionadas por este paquete, sino por cada tipo;
% figuran aqu� como simple recordatorio.
% \nt{h} �tiles en r�tulos en may�sculas.
% \end{minipage}
% \end{table}
%
% Los caracteres usados como abreviaciones se comportan
% como otras �rdenes de \TeX{} y por tanto se hace caso
% omiso de los espacios que le puedan seguir: \verb*|' a| es lo mismo
% que |�|. Eso tambi�n implica que tras esos caracteres no
% puede ir una llave de cierre y que deber� escribirse
% |{... '{}}| en lugar de |{... '}|; en modo matem�tico no hay
% ning�n problema y |$x^{a'}$| ($x^{a'}$) es v�lido.
%
% \begin{decl}
% \con  |activeacute|\opp
% \end{decl}
%
% Para poder usar ap�strofos como abreviaciones de acentos es
% necesaria esta opci�n en |\usepackage|.  Puede cambiarse este
% comportamiento con |\es@acuteactive| en el archivo de
% configuraci�n |spanish.cfg|; en ese caso los ap�strofos se activan
% siempre.
%
% \begin{decl}
% \con  |es-tilden|\opp
% \end{decl}
% 
% Esta orden activa las abreviaciones |~n| y |~N| por compatibilidad
% con versiones anteriores de \textsf{spanish} (y siempre que no se
% empleado tambi�n |es-notilde|). En la versi�n 5 no est�n
% activadas de forma predeterminada.
% 
% \begin{decl}
% \con  |\spanishdeactivate|\marg{caracteres}
% \end{decl}
% 
% Permite desactivar las abreviaciones correspondientes a los
% caracteres dados. Para evitar entrar en conflicto con otras lenguas,
% al salir de \textsf{spanish} se reactivan,\footnote{El punto para
% los decimales no es estrictamente una abreviaci�n y no se
% reactiva.} por lo que si se desea que 
% persistan hay que a�adir la orden a |\shorthandsspanish| con 
%|\addto|. La orden |\renewcommand\shorthandsspanish{}| es una 
% variante optimizada de
%\begin{verbatim}
% \addto\shorthandsspanish{\spanishdeactivate{.'"~<>}}
%\end{verbatim}
%
% \begin{decl}
% \con  |es-noshorthands|\opp
% \end{decl}
%
% No activa ninguna abreviaci�n.
% 
% \subsection{Coma decimal}
%
% \begin{decl}
% \txt  |.|\textit{n�mero}\act{shorthands}\deact{es-nodecimaldot, 
% \textit{es-noshorthands, es-minimal, es-sloppy}}
% \end{decl}
%
% En \textsf{spanish}, el punto en matem�ticas sirve como marca decimal
% gen�rica que puede representarse como coma o punto; funciona
% por tanto como marcado l�gico del signo para decimales. Por
% omisi�n, se siguen las normas internacionales ISO y la legislaci�n
% de diversos pa�ses (como de Espa�a y M�xico) de emplear la coma.
% Ya que \TeX\ usa la coma como separador en intervalos o expresiones 
% similares, lo que a�ade un espacio fino, \textsf{spanish}
% interpreta todo punto en modo matem�tico de esta forma siempre
% que est� seguido de una cifra, pero no en otras circunstancias:
% \begin{quote}\small\begin{tabbing}
% |$1\,234.567\,890$|     \quad \=  $1\,234.567\,890$\\
% |$f(1,2)=12.34.$|        \> $f(1,2)=12.34.$\\
% |$1{.}000$|              \> $1{.}000$, pero\\
% |1.000|                  \> 1.000, pues no es modo matem�tico.
% \end{tabbing}\end{quote}
%
%
% \begin{decl}
% \con  |\decimalcomma    \decimalpoint    \spanishdecimal|\marg{math}
% \end{decl}
%
% Las dos primeras establecen si se usa una coma (predeterminado)
% o un punto, mientras que |\spanishdecimal|\marg{math}
% permite darle una definici�n arbitraria.
%
% \begin{decl}
% \con  |es-nodecimaldot|\opp
% \end{decl}
%
% Cancela el mecanismo del punto decimal.
%
% \subsection{Divisi�n de palabras}
%
% \textsf{Spanish} comprueba la codificaci�n en el momento en que se
% usa un acento: si es |OT1|, se toman medidas para facilitar la
% divisi�n, que pese a todo nunca ser� perfecta, y si es |T1|,
% se accede directamente al car�cter correspondiente.
%
% \begin{decl}
% \txt  |"-    "=    "~|\act{shorthands}
% \deact{\textit{es-noshorthands, es-sloppy}}
% \end{decl}
%
% Para matizar la divisi�n de palabras hay cuatro posibilidades, dos 
% de ellas con el m�todo de abreviaciones:
% \begin{itemize}
% \item  |\-| es un gui�n opcional que no permite
% m�s divisiones, 
%
% \item |"-| es similar pero permite m�s divisiones,
%
% \item |-| es un gui�n que no permite m�s divisiones ni
% antes ni despu�s, y
% 
% \item |"=| es el equivalente que s� las permite,\footnote{No
% es una buena idea usar esta orden, pero en 
% medidas muy cortas puede resultar necesario.}
%
% \end{itemize}
% Por ejemplo (con las posibles divisiones marcadas con \hmk):
% \begin{quote}\small\begin{tabbing}
% |Zaragoza-Barcelona|\qquad \= Zaragoza-\hmk Barcelona\\
% |Zaragoza"=Barcelona| \>
%    Za\hmk ra\hmk go\hmk za-\hmk Bar\hmk ce\hmk lo\hmk na\\
% |semi\-abierto| \> semi\hmk abierto\\
% |semi"-abierto| \> se\hmk mi\hmk abier\hmk to.\footnotemark
% \end{tabbing}\footnotetext{Justo antes y despu�s de
% {\ttfamily\string"\string-} y {\ttfamily\string"\string=} se
% aplican los correspondientes
% valores de {\ttfamily\string\...hyphenmin} lo que implica que la
% divis�n semia\hmk bierto no es posible.
% �ste es un comportamiento correcto.}
% \end{quote}
%
% Con la abreviaci�n |"~|, el gui�n
% tambi�n aparece al comienzo de la siguiente l�nea. Por ejemplo:
% \begin{quote}\small\begin{tabbing}
% |infra"~rojo|  \quad \= in\hmk fra-ro\hmk jo, pero infra-\hmk-rojo.
% \end{tabbing}\end{quote}
%
% \begin{decl}
%\txt  |"+  "+-  "+--|\act{shorthands}\deact{\emph{no-shorthands, 
% es-sloppy}}
% \end{decl}
%\vskip-1.5pc\vskip0pt
% \begin{decl}
% \txt  |~-  ~--  ~---|\act{shorthands}\deact{es-notilde, \emph{no-shorthands, 
% es-minimal, es-sloppy}}
% \end{decl}

% Evitan divisiones: |~-|, que resulta �til para expresar una serie
% de n�meros sin que el gui�n los divida (12~-14, |12~-14|), y
% |~---|, que es la forma que debe usarse para abrir incisos con
% rayas, ya que de lo contrario puede haber una divisi�n entre la
% raya de abrir y la palabra que le sigue:
% \begin{quote}\small\begin{tabbing}
%|Los conciertos ~---o % academias--- que organiz�...|
% \end{tabbing}\end{quote}
%
% Tambi�n pueden emplearse para esta misma funci�n las abreviaciones
% |"+|, |"+-| y |"+---|.  Mientras que este gui�n evita toda posible
% divisi�n en los elementos que une, la raya (---) y la semirraya
% (--) las permiten en las palabras que le precedan o le sigan.
% 
% Otra abreviaci�n es |"rr| que sirve para el 
% �nico cambio de escritura del castellano en caso de haber divisi�n.  
% La \lsc{RAE} indica que al a�adir un prefijo que termina en vocal a 
% una palabra que comienza con \emph{r}, �sta �ltima debe 
% doblarse a menos que se unan por un gui�n. Por ejemplo:
% \begin{quote}\small\begin{tabbing}
% |extra"rradio|  \quad \= ex\hmk trarra\hmk dio, pero extra-\hmk 
%    radio.
% \end{tabbing}\end{quote}
% No hay acuerdo sobre si esta regla y otras similares han de 
% aplicarse o no, aunque la opini�n mayoritaria actual est� en
% contra.
%
% \subsection{Otros}
%
% \begin{decl}
% \txt  |"/|\act{shorthands}\deact{\textit{es-noshorthands, es-sloppy}}
% \end{decl}
%
% Es una utilidad tipogr�fica m�s que espec�ficamente espa�ola.
% En ciertos tipos, como Times, el extremo inferior de la barra est�
% en la l�nea de base y expresiones como <<am/pm>> resultan poco
% est�ticas.  |"/| produce una barra que, de ser necesario, se baja
% ligeramente.  Computer Modern tiene una barra bien dise�ada y no es
% posible ilustrar aqu� este punto, pero se escribir�a
% |am"/pm|.
%
% \begin{decl}
% \txt  |"y|\act{shorthand}\deact{\textit{es-noshorthands, es-sloppy}}
% \end{decl}
%
% El signo \textit{et tironiano}, que en espa�ol se emple� muy a
% menudo, se puede <<imitar>> con |"y|, siempre que se haya cargado el
% paquete |graphics|; de no ser as�, se usa la letra $\tau$, aunque
% la variante normal de \TeX{} no es demasiado apropiada.
%
% \section{Funciones de texto y matem�ticas}
%
% \subsection{Abreviaturas}
%
% \begin{decl}
% \txt  |\sptext|\marg{texto}\act{text}\deact{\textit{es-sloppy}}
% \end{decl}
%
% Pone un punto y le sigue el argumento en voladitas.  Para
% abreviaturas como |adm\sptext{�n}| que da adm\sptext{�n}.  Hay seis
% abreviaciones asociadas a ordinales: |"a|, |"A|, |"o|, |"O|, |"er| y
% |"ER| que equivalen a |\sptext{a}|, etc. Muchos tipos
% a�aden un peque�o subrayado que debe evitarse, y por tanto no se
% deben escribir los ordinales con \textsf{inputenc}.
%
% Para ajustar el tama�o lo mejor posible, se usa el de
% �ndices en curso. Esto funciona bien salvo para tama�os muy 
% grandes o muy peque�os, donde los resultados son meramente
% aceptables. 
%
% En Plain \TeX{} se ejecuta |\sptextfont| para la
% letra voladita, de forma que |{\bf\let\sptextfont\bf 1"o}| da el
% resultado correcto (|\mit| si es para cursiva). Para usar un tipo
% nuevo con |\sptext| hay que definir tambi�n las variantes 
% matem�ticas con |\newfam|.
%
% \subsection{Espaciado}
% 
% El espaciado espa�ol difiere relativamente poco del ingl�s, con
% alguna excepci�n; una de ellas es que en \textsf{spanish}
% |\frenchspacing| est� activo.
% 
% \begin{decl}
% \txt  |\...|\act{text}\deact{\textit{es-sloppy}}
% \end{decl}
%
% Puntos suspensivos menos espaciados que  |\dots|. El espacio 
% que sigue se conserva:
% \begin{quote}\small\begin{tabbing}
% |\... y solo estaba\... ella.|\quad\=\... y solo estaba\... ella.
% \end{tabbing}\end{quote}
% Tambi�n podr�an escribirse los tres puntos sin m�s |...|, y en
% la pr�ctica no hay diferencia, a menos que se cambie el
% valor del espacio tras punto; en ese caso, la forma con barra
% da los valores apropiados \emph{dentro} de una sentencia, y
% los tres puntos \emph{al final} de ella. Esta orden no 
% interfiere con el valor original de |\.| (un punto suprascrito).
%
% \begin{decl}
% \txt  |\%|\act{text}\deact{\textit{es-minimal, es-sloppy}}
% \end{decl}
%
% Se a�ade un espacio fino antes del signo (en concreto |\,|), con
% lo cual se puede "<recuperar"> con su opuesto |\!| si |\%| no sigue
% a una cifra; tambi�n se puede emplear |\percentsign|).
%
% \begin{decl}
% \con  |\spanishplainpercent|
% \end{decl}
%
% Orden para que |\%| no a�ada el espacio fino. Puede ser �til 
% en cuadros, si |\%| aparece siempre entre par�ntesis.
%
% \subsection{Fuentes}
%
% \begin{decl}
% \txt |\lsc|\marg{texto}\act{text}\deact{\textit{es-sloppy}}
% \end{decl}
%
% Pasa \textit{texto} a versalitas:
% \begin{quote}\small\begin{tabbing}
% |\lsc{RAE}| \quad \= \lsc{RAE}\\
% |\lsc{ReNFe}| \quad \= \lsc{ReNFe}.\\
% |siglo \lsc{XVII}| \quad \= siglo \lsc{XVII}\\
% |cap�tulo \lsc{II}| \quad \= cap�tulo \lsc{II}.
% \end{tabbing}\end{quote}
% 
% Para evitar que con un tipo que carece de versalitas acabe
% apareciendo (por substituci�n) un texto de min�sculas se intenta
% usar en estos casos las versales \emph{reales} de un tama�o menor
% (\LaTeX\ tiende a sustituir versalitas por versalitas, pero hay
% excepciones, como con las negritas).
% 
% \begin{decl}
% \txt |\�|\alw
% \end{decl}
%
% Lo mismo que |�|.
%
% \subsection{Entrecomillados}
% 
% \begin{decl}
% \txt |\begin{quoting} ... \end{quoting}|\alw
% \end{decl}
%
% El entorno |quoting| entrecomilla un texto, a�adiendo comillas de
% seguir al comienzo de cada p�rrafo en su interior.\footnote{Se puede
% encontrar una detallada exposici�n de las comillas en \DTL{44 ss.}
% De ah� se ha tomado alg�n ejemplo.}
%
% \begin{decl}
% \txt |<<    >>|\act{shorthands}\deact{es-noquoting, \textit{es-noshorthands, es-minimal, 
% es-sloppy}}
% \end{decl}
%\vskip-1.5pc\vskip0pt
% \begin{decl}
% \txt |"`    "'|\act{shorthands}\deact{\textit{es-noshorthands, es-sloppy}}
% \end{decl}
%
% Tambi�n se pueden emplear las abreviaciones |<<| y |>>| (o
% alternativamente |"`| y |"'|) que se limitan a llamar a |quoting|,
% que por ser entorno considera sus cambios internos como locales.
% (Es decir, |<< ... >>| implica |{<< ... >>}|.) Las abreviaciones
% |"<| y |">| contin�an dando sin m�s los caracteres de comillas de
% abrir y cerrar, respectivamente.
% 
% Por ejemplo:
%\begin{verbatim}
% <<Se llaman <<comillas de seguir>> a las que son de cierre,
% pero se colocan al comienzo de cada p�rrafo cuando se transcribe
% un texto entrecomillado con m�s de un p�rrafo.
% 
% En su interior, como de costumbre, se usan inglesas.>>
%\end{verbatim}
% cuyo resultado es:
% \begin{quotation}\small
% <<Se llaman <<comillas de seguir>> a las que son de cierre,
% pero se colocan al comienzo de cada p�rrafo cuando se transcribe
% un texto entrecomillado con m�s de un p�rrafo.
% 
% En su interior, como de costumbre, se usan inglesas.>>
% \end{quotation}
%
% Tambi�n se a�aden comillas de seguir en listas, excepto con la
% opci�n \texttt{es-nolists} o cualquier otra que las desactive.
% 
% Este entorno se puede redefinir, como por ejemplo:
%\begin{verbatim}
% \renewenvironment{quoting}{\itshape}{}
%\end{verbatim}
% pero en principio no implica un nuevo p�rrafo, ya que 
% est� pensado para ser usado tambi�n en el texto.
%
% \begin{decl}
% \con |\lquoti| |\rquoti| |\lquotii| |\rquotii| |\lquotiii| 
% |\rquotiii|
% \end{decl}
% 
% Controlan las comillas en |quoting|, seg�n el
% nivel en que nos encontremos. |\lquoti| son las comillas de abrir
% m�s exteriores, |\lquotii| las de segundo nivel, etc., y lo mismo
% para las de cerrar con |\rquoti|... Para las de seguir siempre se
% usan las de cerrar. Los valores predefinidos est�n en el cuadro 3.
% \begin{table}
% \center\small
% \caption{Entrecomillados}
% \vspace{1.5ex}
% \begin{tabular}{l@{\hspace{5em}}l}
% \toprule2
% |\lquoti|   &|"<|\\
% |\rquoti|   &|">|\\
% |\lquotii|  &|``|\\
% |\rquotii|  &|''|\\
% |\lquotiii| &|`|\\
% |\rquotiii| &|'|
% \botrule2
% \end{tabular}
% \end{table}
% 
% \begin{decl}
% \con |\activatequoting    \deactivatequoting|
% \end{decl}
% 
% Las incompatibilidades potenciales de estas abreviaciones son
% enormes. Por ejemplo, en \textsf{ifthen} se cancelan las
% comparaciones entre n�meros;\,\footnote{Y en \texttt{\textbackslash 
% ifnum},
% \texttt{\textbackslash ifdim}, etc., usado por los desarrolladores en
% los paquetes.} tambi�n
% resultan inoperantes |@>>>| y |@<<<| de
% \textsf{amstex}.\footnote{Aunque en 
% este caso cabe usar los sin�nimos |@)))| y |@(((|.}
% Por ello, se da la posibilidad de cancelarlas y reactivarlas con
% estas �rdenes, aunque si se est� usando 
% \textsf{xmltex} ya se
% desactivan por completo de forma autom�tica. El entorno
% |quoting| siempre permanece disponible.\footnote{Algunos tipos
% disponen de esta ligadura de forma interna para
% generar los caracteres de comillas, por lo que en ellos tambi�n
% podemos usarlos siempre, aunque los ajustes proporcionados por
% \textsf{spanish} se pueden perder; por otra parte, tampoco se
% usan demasiado a menudo.}
% 
% \subsection{Funciones matem�ticas}
%
% \begin{decl}
% \txt |\lim \limsup \liminf \bmod \pmod \sen \tg| 
% etc.\act{math}\deact{\textit{es-minimal, es-sloppy}}
% \end{decl}
%
% Tradicionalmente, las abreviaciones de lo que en \TeX\ se conocen
% como operadores se han formado a partir del nombre castellano, lo
% que implica la presencia del acento en l�m (en sus tres formas
% |\lim|, |\limsup| y |\liminf|), m�x, m�n, �nf y m�d (en sus dos
% formas |\bmod| y |\pmod|).
%
% Con \textsf{spanish} pueden seguirse varias convenciones con ayuda
% de las siguientes �rdenes:
% \begin{decl} \con |\accentedoperators| |\unaccentedoperators|
% \end{decl}
% Activa o desactiva los acentos.
% Por omisi�n se acent�an, como por ejemplo: $\lim_{x\to 0}(1/x)$
% (|$\lim_{x\to 0}(1/x)$|).
% 
% \begin{decl}
% \con |\spacedoperators| |\unspacedoperators|
% \end{decl}
% Activa o desactiva el espacio entre "<arc"> y la funci�n.
% Lo habitual ha sido con espacio; as� pues, por omisi�n
% se espacia.
%
% Tambi�n se a�aden |\sen|, |\arcsen|, |\tg| y |\arctg|,
% que dan las funciones respectivas.
% \begin{decl}
% \con |\spanishoperators|
% \end{decl}
%
% Otras funciones trigonom�tricas se encuentran almacenadas en el
% par�metro |\spanishoperators|, que inicialmente incluye cotg,
% cosec, senh y tgh. La raz�n por la que estas funciones se han
% separado es porque su forma no est� normalizada en el �mbito
% hispanohablante. De esta forma se puede cambiar por otras con, por
% ejemplo:
%\begin{verbatim}
% \renewcommand{\spanishoperators}{ctg arc\,ctg sh ch th}
%\end{verbatim}
% (separadas con espacio). Cuando se selecciona \textsf{spanish} se crean
% �rdenes con esos nombres
% y que dan esas funciones (siempre con |\nolimits|). Adem�s de
% las letras sin acentuar se aceptan las �rdenes |\,| y |\acute|, que
% se pasan por alto para formar el nombre. Por ejemplo, |arc\,ctg|
% se escribe en el documento con
% |\arcctg|, |M\acute{a}x| como |\Max| y |cr\acute{i}t| como |\crit|
% (hay que usar |i| y no |\dotlessi|).
% La orden |\,| responde a |\|(|un|)|spacedoperators|, y |\acute|
% a |\|(|un|)|accentedoperators|.
% 
% Conviene que |\spanishoperators| est� en el pre�mbulo del
% documento en s�, antes de |\selectspanish| o de 
% |\begin{document}|.
%
% \begin{decl}
% \txt |\dotlessi|\act{math}\deact{\textit{es-sloppy}}
% \end{decl}
% 
% La \textit{i} sin punto tambi�n es accesible directamente en modo
% matem�tico con |\dotlessi|, de forma que se puede escribir
% |\acute{\dotlessi}|. Por ejemplo,
% |$V_{\mathbf{cr\acute{\dotlessi}t}}$| da
% $V_{\mathbf{cr\acute{\dotlessi}t}}$.
%
%
% \section{Opciones generales}
%
% Est�n pensadas principalmente para documentos basados en una clase
% o un estilo editorial muy preciso que no debe tocarse. Para conocer 
% los cambios exactos, v�anse las diferentes entradas que describen 
% las funciones de \textsf{spanish}.
%
% \begin{decl}
% \con |es-minimal|\opp
% \end{decl}
%
% Anula la mayor�a de los cambios pero deja unas cuantas utilidades
% que pueden resultar utiles en el momento de escribir el texto. 
% 
% \begin{decl}
% \con |es-sloppy|\opp
% \end{decl}
% 
% Anula, adem�s, todas las ligaduras sin excepci�n, la e�e en listas y los 
% grupos \textsf{text} y \textsf{math}.
%
% \section{Selecci�n}
%
% \begin{decl}
% |\selectspanish|
% \end{decl}
% 
% Por omisi�n, \babel{} deja <<dormidas>> las lenguas hasta que se
% llega a |\begin{document}| con el fin de evitar conflictos por
% las abreviaciones; a cambio,
% se priva de la posibilidad de usar las lenguas en el pre�mbulo 
% en �rdenes como |\savebox|, |\title|, |\newtheorem|, etc.
% 
% La orden |\selectspanish| permite activar \textsf{spanish} con sus
% extensiones y abreviaciones antes de
% |\begin{document}|.\footnote{Algunos detalles, que
% apenas afectan a \textsf{spanish}, siguen sin activarse hasta el
% comienzo del documento.}
% De esta forma, podr�amos decir
%\begin{verbatim}
% \documentclass{book}
% \usepackage[T1]{fontenc}
% \usepackage[cp1252]{inputenc}
% \usepackage[spanish,activeacute,es-notilde]{babel}
% ... % Mas paquetes
% 
% \selectspanish
% 
% \title{T�tulo}
% \author{Autor}
% \newcommand{\pste}{para"-psicol�gicamente}
% ...   % Mas definiciones
%
% \begin{document}
%\end{verbatim}
%
% \section{Adaptaci�n}
%
% \subsection{Opciones por pa�ses}
% \label{paises}
%
% % \begin{decl} \con |mexico| \quad |mexico-com|
% \end{decl}
%
% La primera cambia \textit{cuadro} a \textit{tabla} y desactiva tanto
% |<||<>||>| como el punto decimal. Tambi�n cambia
% |"`| y |"'| a ``\,`\,"<\,">\,'\,''. Es decir, aparte de
% redefinir las comillas, equivale a:
% a:
%\begin{verbatim}
%\usepackage[spanish,es-nodecimaldot,es-tabla,es-noquoting]{spanish}
%\end{verbatim}
% La segunda es similar
% pero s� activa el punto decimal. (Obs�rvese que no van precedidas
% de |es-|.)
%
% Probablemente, esta opci�n tambi�n sea apropiada en algunos
% pa�ses de Am�rica Central y del Sur.
%
% \subsection{Configuraci�n}
%
% En sus �ltimas versiones, \babel{} ofrece la posibilidad
% de cargar autom�ticamente un archivo con el mismo nombre que
% el principal, pero con extensi�n |.cfg|. \textsf{Spanish}
% proporciona unas pocas �rdenes para ser usadas en este archivo:
%
% \begin{decl}
% \con |\es@activeacute|
% \end{decl}
% Activa las abreviaciones con ap�strofos, sin que sea
% necesario incluir |activeacute| como opci�n en |\usepackage|.
%
% \begin{decl} \con |\es@enumerate{<leveli>}|%
%      |{<levelii>}{<leveliii>}{<leveliv>}|\alw
% \end{decl} 
% Cambia los valores preestablecidos por \textsf{spanish} para
% |enumerate|. \textit{leveln} consiste en una letra, que
% indica qu� formato tendr� el n�mero, seguida
% de cualquier texto. La letra tiene que ser: |1| (ar�bigo),
% |a| (min�scula \emph{cursiva}\,\footnote{La letra es cursiva
% pero no los signos que le puedan seguir. M�s bien deber�a
% decirse destacada, ya que se usa |\string\emph|.
% V�ase \DTL{11}.}), |A| (versal),
% |i| (romano \emph{versalita}), |I| (romano versal) o
% finalmente |o| (ordinal\,\footnote{Lo normal es no a�adir ning�n 
% signo tras ordinal.}).
%
% Esta orden no est� pensada para hacer cambios elaborados, sino
% s�lo meros reajustes. Los valores preestablecidos 
% equivalen a
%\begin{verbatim}
% \es@enumerate{1.}{a)}{1)}{a$'$)}
%\end{verbatim}
%
% \begin{decl} \con |\es@itemize{<leveli>}|%
%      |{<levelii>}{<leveliii>}{<leveliv>}|\alw
% \end{decl} 
% Lo mismo para |itemize|, s�lo que los argumentos se
% usan de forma literal. Los valores originales de \LaTeX{} son
% similares a
%\begin{verbatim}
% \es@itemize{\textbullet}{\normalfont\bfseries\textendash}
%    {\textasteriskcentered}{\textperiodcentered}
%\end{verbatim}
%
% \begin{decl}\con |\es@operators|\act{math}
% \end{decl}
% Todo lo relativo a operadores se cancela con
%\begin{verbatim}
% \let\es@operators\relax
%\end{verbatim}
% Es buena idea incluirlo si no se van a usar, ya que ahorra memoria.
%
% Otros ajustes �tiles en este contexto son |\spanishoperators|,
% |\selectspanish| y |\deactivatequoting|.
%
%
% Recordemos que todos los cambios
% operados desde este archivo restan compatibilidad al
% documento, por lo que si se distribuye conviene adjuntarlo
% con el entorno |filecontents|.
%
% \subsection{Pasar opciones desde un paquete o clase}
%
% \begin{decl} \con |\spanishoptions|
% \end{decl}
%
% Como |\PassOptionsToPackage| a�ade opciones al comienzo y
% las opciones espec�ficas de \textsf{spanish} han de ir al final, definiendo
% esta macro se puede controlar el comportamiento de \textsf{spanish} antes
% de su carga.
%
% \subsection{Otros cambios}
%
% Las adaptaciones se encuentran organizadas en varios grupos, a los 
% que corresponden sendas macros: 
% |\textspanish|, |\mathspanish|,
% |\shorthandsspanish|, |\datespanish| y |\captionsspanish|. Pueden
% cancelarse con:
%\begin{verbatim}
%  \renewcommand\textspanish{}
%\end{verbatim}
%
% \section{Plain \TeX}
% 
% Con Plain hay que hacer:
%\begin{verbatim}
% \input spanish.sty
%\end{verbatim}
% 
% Se incluyen: traducciones, casi todas las abreviaciones, coma
% decimal, utilidades para divisi�n de palabras, ordinales en una
% versi�n simplificada (y no muy elegante), funciones matem�ticas,
% |\�| y espaciado. La selecci�n de la lengua es inmediata al
% cargar el archivo.
% 
% En cambio no est�n disponibles: entrecomillados, 
% |\lsc| ni las adaptaciones de lengua principal.
%
% \section{Compatibilidad con versiones anteriores}
%
% En versiones de \textsf{babel} bastante antiguas, las abreviaciones 
% con |'| se activaban por omisi�n, mientras que ahora es necesario
% |activeacute|.
%
% En la versi�n 4, la abreviaci�n |~n| se consider� para extinguir.
% En la versi�n 5 sigue estando, pero \textit{no} se activa por
% omisi�n, sino que hay que emplear |es-tilden|.
%
% En la versi�n 5 el grupo \textsf{layout} no se retrasa a
% |\begin{document}|, como en la 4, sino que se ejecuta
% inmediatamente. Esto permite cambios en el pre�mbulo con otros
% paquetes. Con ello, adem�s, |\selectspanish*| carece de utilidad.
% La opci�n de paquete |es-delayed| restaura el comportamiento
% anterior, por si hubiera alguna incompatibilidad.
%
% La compatibilidad con la versi�n 2.09 de \LaTeX{} se ha suprimido.
%
% \section*{Referencias}
% \addcontentsline{toc}{section}{Referencias}
%
% \begingroup
% \small
% \leftskip1.5cm \parindent-1.5cm
%
% \makebox[1.5cm][l]{\lsc{DRAE}}\textit{Diccionario de la Academia
%    Espa�ola}, Madrid, Espasa-Calpe, 21"a ed., 1992.
%
% \makebox[1.5cm][l]{\lsc{DOT}}Jos� Mart�nez de Sousa,
%   \textit{Diccionario de ortograf�a t�cnica}, 
%   Madrid, Germ�n S�nchez Ruip�rez/Pir�mide, 1987.
%   (Biblioteca del libro.)
%
% \makebox[1.5cm][l]{\lsc{DTL}}Jos� Mart�nez de Sousa,
%   \textit{Diccionario de tipograf�a y del libro}, 
%   Madrid, Paraninfo, 3"a ed., 1992.
%
% \makebox[1.5cm][l]{\lsc{MEA}}Jos� Mart�nez de Sousa,
%   \textit{Manual de edici�n y autoedici�n}, 
%   Madrid, Pir�mide, 1994.
%
% \leftskip0pt \parindent0pt \vspace{6pt}
%
% {\itshape
% Para otras cuestiones tipogr�ficas, las referencias
% usadas son, entre otras:}
%
% \parindent-1.5pc \leftskip1.5pc \vspace{3pt}
%
% Asociaci�n de Academias de la Lengua Espa�ola,
% \textit{Diccionario panhisp�nico de dudas}, Madrid, Santillana, 2005.
%
% Javier Bezos,
% \textit{Tipograf�a espa�ola con \TeX}, documento electr�nico
% disponible en 
% \textsf{http://perso.wanadoo.es/jbezos/tipografia.html}. 
%
% Javier Bezos,
% \textit{Tipograf�a y notaciones cient�ficas}, Gij�n, 
% Trea, 2008.
%
% Bureau International des Poids et mesures,
% \textit{Le Sist\`{e}me international d�nit�s},
% 8"a ed., Par�s, {\footnotesize BIPM}, 2006, 
% \textsf{http://www.bipm.org/""fr/""si/""si\_brochure/}, 2006-11-10.
%
% Jorge de Buen,
% \textit{Manual de dise�o editorial,} M�xico, Santillana, 2000.
%
%
% \textit{The Chicago Manual of Style}, Chicago, University of
% Chicago Press, 14"a~ed., 1993, esp.~p�gs.~333~-335.
%
% Jos� Fern�ndez Castillo,
% \textit{Normas para correctores y compositores tip�grafos},
% Madrid, Espasa-Calpe, 1959.
%
% IRANOR [AENOR], Normas \lsc{UNE} n�meros 5010 (<<Signos 
% matem�ticos>>), 5028 (<<S�mbolos 
% geom�tricos>>) y 5029 (<<Impresi�n de los
% s�mbolos de magnitudes y unidades y de los n�meros>>). 
% [Obsoletas.]
%
% Llerena, Mario,
% \textit{Un manual de estilo,} Miami, Unilit, 1999.
%
% Real Academia Espa�ola,
% \textit{Esbozo de una nueva gram�tica de la
% lengua espa�ola}, Madrid, Espasa-Calpe, 1973.
% 
% V.\ Mart�nez Sicluna,
% \textit{Teor�a y pr�ctica de la tipograf�a},
% Barcelona, Gustavo Gili, 1945.
%
% Jos� Mart�nez de Sousa,
% \textit{Diccionario de ortograf�a de la lengua espa�ola},
% Madrid, Paraninfo, 1996.
%
% Juan Mart�nez Val, \textit{Tipograf�a pr�ctica}, Madrid,
% Laberinto, 2002.
%
% Juan Jos� Morato, \textit{Gu�a pr�ctica del compositor 
% tipogr�fico}, Madrid, Hernando, 2"a ed., 1908 (1"a ed., 1900,
% 3"a ed., 1933).
%
% Marion Neubauer,
% <<Feinheiten bei wissenschaftlichen Publikationen>>,
% \textit{Die \TeX nisches Kom\"odie},  parte I, vol. 8, n"o 4, 1996,
% p�gs. 23-40; parte II, vol. 9, n"o 1, 1997, p�gs.~25~-44.
%
% Notimex, \textit{Manual de operaci�n y estilo editorial}, M�xico, 
% Notimex, 1999.
%
% Jos� Polo,
% \textit{Ortograf�a y ciencia del lenguaje}, Madrid, Paraninfo, 
% 1974.
%
% Siglo 21, \textit{Libro de estilo}, M�xico, Alda, 
% $\mathrm{^s}\!$/$\mathrm{_f}$
% (impr.  1995).
%
% Pedro Valle,
% \textit{C�mo corregir sin ofender}, Buenos Aires, Lumen, 1998.
%
% Hugh C. Wolfe, <<S�mbolos, unidades y nomenclatura>>, 
% \textit{Enciclopedia de F�sica}, dir. Rita G. Lerner y George L. 
% Trigg, Madrid, Alianza, 1987, t.~2, p�gs.~1423~-1451.
%
%\endgroup
%
% \else
%
%^^A  ======= Beginning of text as typeset by user.drv =========
%
% \GetFileInfo{spanish.dtx}
%
% \section{The Spanish language}
%
% The file \file{\filename}\footnote{The file described in this
% section has version number \fileversion\ and was last revised on
% \filedate.  The maintainer from v4.0 on is Javier Bezos
% (http://www.tex-tipografia.com).  Previous
% versions were made by Julio S\'anchez.  The English documentation
% has been improved by José Luis Rivera; thanks to him it is now a lot
% clearer.} defines all the language-specific macros for the Spanish
% language.
%
% Spanish support is implemented following mainly the guidelines given
% by Jos\'e Mart\'\i nez de Sousa.  You may get the the full
% documentation (more comprehensive, but regrettably only in Spanish)
% by typesetting |spanish.dtx| directly.  There are examples and some
% additional features documented in the Spanish version only.
% Cross-references in this section point to that document.
% 
% \paragraph{Features} This style provides:
%
% \begin{itemize}
% \item Translations following the International \LaTeX{} 
% conventions, as well as |\today|.
% 
% \item Shorthands listed in Table~\ref{tab:spanish-quote-def}.
% Examples in subsection~3.4 are illustrative.  Notice that |"~| has a
% special meaning in \textsf{spanish} different to other languages,
% and is used mainly in linguistic contexts.
% 
% \begin{table}[htb]
% \centering
% \begin{tabular}{lp{8cm}}
% |'a|      & Acute accented a. Works for e, i, o, u, too (both
%             lowercase and uppercase).\\
% |'n|      & \~n (uppercase too).\\
% |"i|      & \"i (uppercase too).\\
% |"u|      & \"u (uppercase too).\\
% |"a| |"o| & Ordinal numbers (uppercase |"A|, |"O| too).\\
% |"er "ER| & Ordinal 1.\textsuperscript{er} 1.\textsuperscript{\textsc{er}}\\
% |"c|      & \c{c} (uppercase too).\\
% |"rr|     & rr, but -r when hyphenated.\\
% |"y|      & An old ligature for ``et'' (like the English \&).\\
% |"-|      & Like |\-|, but allowing hyphenation in the rest 
%             the word.\\
% |"=|      & Like |-|, but allowing hyphenation in the rest 
%             the word.\\
% |"~|      & The hyphen is repeated at the very beginning of 
%             the next line if the word is hyphenated at this
%             point.\\
% |""|      & Like |"-| but producing no hyphen sign.\\
% |~-|      & Like |"-| but with no break after the hyphen. Works for
%             en-dashes (|~--|) and em-dashes (|~---|). |"+|, |"+-| 
%             and |"+--| are synonymous.\\
% |"/|      & A slash slightly lowered, if necessary.\\
% \verb+"|+ & Disable ligatures at this point.\\
% |"<|      & Left guillemets.\\
% |">|      & Right guillemets.\\
% |<<| |>>| & |\begin{quoting}| and |\end{quoting}|. (See below.) 
%             |"`| and |"'| are synonymous.\\
% |"? "!|   & Opening question and exlamation marks (?`!`) 
%             aligned on the baseline, useful for all-caps headings, etc.
% \end{tabular}
% \caption{Extra definitions made by file \file{spanish.ldf}}
% \label{tab:spanish-quote-def}
% \end{table}
% 
% \item |\frenchspacing|.
%
% \item \emph{In math mode}, a dot followed by a digit is replaced
% by a decimal comma.
% 
% \item Spanish ordinals and abbreviations with the |\sptext|\marg{text}
% command as, for instance, |1\sptext{er}|. The preceptive dot is included.
% 
% \item Accented (l\'\i m, m\'ax, m\'\i n, m\'od) and spaced
% (arc\,cos, etc.) functions. 
%
% \item |\dotlessi| is provided for use in math mode.
%
% \item A |quoting| environment and a related pair of shorthands |<<|
% and |>>|. Useful for traditional spanish multi-paragraph quoting.
%
% \item There is a small space before the percent |\%| sign.
% 
% \item |\lsc| provides lowercase small caps. (See subsection~3.10.)
%
% \item Ellipsis is best typed as |...| or, within a sentence, as |\...|
% 
% \end{itemize}
% 
% If \textsf{spanish} is the main language, the command 
% |\layoutspanish| is added to the main group, modifying the standard
% classes throughout the whole document in the following way:
%
% \begin{itemize}
%
% \item Paragraphs are set with |\indentfirst|.
% 
% \item Both |enumerate| and |itemize| are adapted to Spanish rules.
% 
% \item Both |\alph| and |\Alph| include \textit{\~n} after \textit{n}.
% 
% \item Symbol footmarks are one, two, three, etc., asterisks.
% 
% \item |OT1| guillemets are generated with two |lasy| symbols instead
% of small |\ll| and |\gg|.
% 
% \item |\roman| is redefined to write small caps Roman numerals, since
% lowercase Roman numerals are discouraged (see below).
% 
% \item There is a dot after section numbers in titles, headings, and toc.
%
% \end{itemize}
%
% A subset of these features is implemented for Plain \TeX{}
% (accesible with the command |\input spanish.sty|).  Most
% significantly, |\lsc|, the |quoting| environment, and features
% provided by |\layoutspanish| are missing.
%
% \paragraph{Customization}
%
% Beginning with version 5.0, customization is made following two paths:
% via |options| or via |commands|; these options and commands override 
% the layout for Spanish documents at different levels: options are meant for 
% use at the preamble only, while commands may be used in the configuration
% file or at document level.
%
% Global options control the overall appearance of the document, and may 
% be set on the |{babel}| call, right after calling |spanish|, or 
% shortly before the call to |{babel}|, to ensure their proper loading
% at runtime. Thus, the following calls are roughly equivalent:
% 
% \begin{verbatim}
% \usepackage[...,spanish,es-nosectiondot,es-nodecimaldot,...]{babel}
% 
% \def\spanishoptions{es-nosectiondot,es-nodecimaldot}
% \usepackage[...,spanish,...]{babel}
% \end{verbatim}
%
% \begin{table}
% \centering
% \begin{tabular}{cccc}\hline
%  Basic Options       & |es-minimal| & |es-sloppy| & |es-noshorthands| \\\hline
%  |es-noindentfirst|  & X  & X  &   \\
%  |es-nosectiondot|   & X  & X  &   \\
%  |es-nolists|        & X  & X  &   \\
%  |es-noquoting|      & X  & X  & X \\
%  |es-notilde|        & X  & X  & X \\
%  |es-nodecimaldot|   & X  & X  & X \\
%  |es-nolayout|       &    & X  &   \\
%  |es-ucroman|        & X  &    &   \\
%  |es-lcroman|        & X  & X  &   \\\hline
% \end{tabular}
%  \caption{Spanish Customization Options}
%  \label{tab:SpanishCustomizationOptions}
% \end{table}
%
% Some global options are built upon lower level options, and may be
% used as shorthand for more global customizations.
% Table~\ref{tab:SpanishCustomizationOptions} gives an overview of the
% global options constructed this way.  Most of these options are
% self-explanatory: they disable the changes made to the basic \LaTeX\
% layout by |spanish|.  |es-lcroman| however, and a few others, need a
% bit of explanation, and they may be described as follows:
%
% \begin{itemize}
%
% \item Traditional Spanish typography discourages the use of
% lowercase Roman numerals; instead, a smallcaps variant is
% implemented.  However, since |Makeindex| seems to choke on the code
% implementing lowercase Roman numerals (via the |\lsc| macro), two
% workarounds are implemented: the |es-ucroman| option converts all
% Roman numerals to uppercase, and the |es-lcroman| option turns all
% Roman numerals to lowercase; the former should be preferred over the
% latter.  Three macros control local changes to Roman numbers:
% |\spanishscroman|, |\spanishucroman|, and |\spanishlcroman|.
% 
% \item The |es-preindex| option calls the |romanidx.sty| package
% automatically to fix index entries in smallcaps roman form.  An
% additional macro,
% |\spanishindexchars|\marg{encap}\marg{openrange}\marg{closerange}
% determines the characters delimiting index entries.  Defaults are
% \verb=\spanishindexchars{|}{(}{)}=.
%
% \item The |es-tilden| option restores the old tilde |~| shorthand
% for \~n.  This shorthand is however \emph{strongly} deprecated.
%
% \item The |es-nolayout| option disables layout changes in the
% document when |spanish| is the main language.  These changes affect
% enumerated and itemized lists, enumerations (alphabetic order
% excludes \~n), and symbolic footnotes.
%
% \item The |es-noshorthands| disables the shorthand mechanism
% completely: neither |"| nor |'| nor |<| nor |>| nor |~| nor |.| work
% at all.
%
% \item The |es-noquoting| option disables the macros |<<| and |>>|
% calling the |quoting| environment; the alternative macros |"`| and
% |"'| are still available.
%
% \item The |es-uppernames| option makes uppercase versions of
% captions for chapter, tablename, etc.
%
% \item The |es-tabla| option changes ``cuadro'' for ``tabla'' in
% captions.
%
% \end{itemize}
%
% Finally, the Spanish 5 series begins the implementation of national
% variations of Spanish typography, beginning with Mexico.  Thus the
% global options |mexico| and |mexico-com| are adapted to practices
% spread in Mexico, and perhaps Central America, the Caribbean, and
% some countries in South America.\footnote{The main difference is
% that |mexico| disables the |decimaldot| mechanism, while
% |mexico-com| keeps it enabled; both change the |quoting|
% environment, disabling the use of guillemets.}
%
% Many of the global options are implemented via macros, which may be
% included in the configuration file |spanish.cfg|, in the preamble,
% after the call to |babel|, and in the body of the document.  These
% macros are the following.
%
% \begin{itemize}
%
% \item The macros |\spanishdashitems| and |\spanishsignitems| change
% the values of itemized lists to a series of dashes or an alternative
% series of symbols, respectively.
%
% \item The command |\deactivatequoting| deactivates the |<<| and |>>|
% shorthands if you want to use |<| and |>| in numerical comparisons
% and some AMS\TeX\ commands.
% 
% \item You may kill the space in spaced operators with
% |\unspacedoperators|. 
%
% \item You may kill the accents on accented operators with 
% |\unaccentedoperators|.  
%
% \item The command |\decimalpoint| resets the decimal separator to
% its default (dot) value, while |\spanishdecimal|\marg{symbol} allows
% for an arbitrary definition.
%
% \item |\spanishplainpercent| prevents the addition of a thinspace before 
% the percent sign in texts. This might be useful for parenthesized percent
% signs in tables, etc.
%
% \item The macros |\spanishdatedel| and |\spanishdatede| control the 
% if the article is given in years (|del| or |de|). 
%
% \item The macro |\spanishreverseddate| sets the date of the format
% ``Month Day del Year''.
%
% \item The macro |\Today| gives months in uppercase.
%
% \item The macros |\spanish|\textit{caption} change the value of the \emph{caption} 
% automatically (no need to add an |\addto|).
%
% \item The command |\spanishdeactivate|\marg{characters} disables the
% shorthand characters listed in the argument.  Elegible characters
% are the set |.'"~<>|.  These shorthand characters may be globally
% deactivated for Spanish adding this command to |\shorthandsspanish|.
%
% \item Extras are divided in groups controlled by the commands
% |\textspanish|, |\mathspanish|, |\shorthandsspanish| y
% |\layoutspanish|; their values may be cancelled typing
% |\renewcommand|\marg{command}|{}|, or changed at will (check the
% Spanish documentation or the code for details).
% 
% \item The command |\spanishoperators|\marg{operators} defines
% command names for operators in Spanish.  There is no standard name
% for some of them, so they may be created or changed at will.  For
% instance, the command
% |\renewcommand{\spanishoperators}{arc\,ctg m\acute{i}n}|
% creates commands for these functions.  The command
% |\,| adds thinspaces at the appropriate places for spaced operators
% (like |\arcctg| in this case), and the command |\acute|\marg{letter}
% adds an accent to the letter included in the definition (thus,
% |m\acute{i}n| defines the accented function |\min| (m\'\i{}n);
% please notice that |\dotlessi| is not necessary).
% 
% \item The commands
% |\lquoti|\marg{string} |\rquoti|\marg{string} 
% |\lquotii|\marg{string} |\rquotii|\marg{string} 
% |\lquotiii|\marg{string} |\rquotiii|\marg{string} 
% set the quoting signs in the |quoting| environment, 
% nested from outside in. They may be |\renew|ed at will. 
% Default values are shown in table~\ref{tab:spanish-quote-ref}.
% \begin{table}
% \center\small
% \vspace{1.5ex}
% \begin{tabular}{l@{\hspace{5em}}l}
% |\lquoti|   &|"<|\\
% |\rquoti|   &|">|\\
% |\lquotii|  &|``|\\
% |\rquotii|  &|''|\\
% |\lquotiii| &|`|\\
% |\rquotiii| &|'|
% \end{tabular}
% \caption{Default quoting signs set for the \texttt{quoting} environment.}
% \label{tab:spanish-quote-ref}
% \end{table}
%
% \item The command |\selectspanish*| is obsolete: if |spanish| is the
% main language, all its features are available right after loading
% |babel|.  The |es-delayed| option is provided to restore the
% previous behavior and macros for backwards compatibility.
% 
% \end{itemize}
%
% \fi
% \endgroup
% 
% \StopEventually{}
%
%^^A ========== End of manual ===============
%
% \ifx\langdeffile\undefined
% 
% \section{The Code}
% 
% \else
%
% \subsection{The Code}
% 
% \fi
%
% \changes{spanish~5.0a}{2007/02/21}{Reimplemented in full, which some 
%   parts rewritten from scratch. Added the es- mechanism and the mexico
%   option. Many bug fixes.}
% \changes{spanish~5.0d}{2008/05/25}{Fixed two bugs: misplaced 
%   subscripts with lim and the like; problem with \cs{roman} and hyperref.} 
% \changes{spanish~5.0h}{2009/01/02}{Removed unnecessary \cs{string}s 
%   with two acutes. Added es-noenumerate, es-noitemize.}
% \changes{spanish~5.0i}{2009/05/11}{romanidx not working. Some \cs{es@roman}
%   replaced with \cs{es@scroman}.}
% \changes{spanish~5.0j}{2010/01/06}{Overdot \cs{.} was not robust.}
% \changes{spanish~5.0j}{2010/04/04}{Colon in saved catcodes, because 
%   babel doesn't restore it after french}
% \changes{spanish~5.0j}{2010/05/08}{Changed order of tests in 
%   shorthands, to fix a bug with ifthen}
%
%    This file is for both \LaTeXe{} and Plain formats.
%
%    \begin{macrocode}
%<*code>
\ProvidesLanguage{spanish.ldf}
    [2010/05/23 v5.0j Spanish support from the babel system]
\LdfInit{spanish}\captionsspanish

\edef\es@savedcatcodes{%
 \catcode`\noexpand\~=\the\catcode`\~
 \catcode`\noexpand\"=\the\catcode`\"
 \catcode`\noexpand\:=\the\catcode`\:}
\catcode`\~=\active
\catcode`\"=12
\catcode`\:=12

\ifx\undefined\l@spanish
 \@nopatterns{Spanish}
 \adddialect\l@spanish0
\fi

\def\es@sdef#1{\babel@save#1\def#1}
\def\es@sDRC#1{\babel@save#1\DeclareRobustCommand*#1}

\@ifundefined{documentclass}
 {\let\ifes@latex\iffalse}
 {\let\ifes@latex\iftrue}
%    \end{macrocode}
%
%    Package options for spanish. To avoid error messages dummy
%    options are created on the fly when neccessary.
%
%    \begin{macrocode}
\ifes@latex

\@ifundefined{spanishoptions}{}
{\PassOptionsToPackage{\spanishoptions}{babel}}

\def\es@genoption#1#2#3{%
 \DeclareOption{#1}{}%
 \@ifpackagewith{babel}{#1}%
  {\def\es@a{#1}%
   \expandafter\let\expandafter\es@b\csname opt@babel.sty\endcsname
   \addto\es@b{,#2}%
   \expandafter\let\csname opt@babel.sty\endcsname\es@b
   \AtEndOfPackage{#3}}%
  {}}

\es@genoption{es-minimal}
 {es-ucroman,es-noindentfirst,es-nosectiondot,es-noenumerate,%
  es-noitemize,es-noquoting,es-notilde,es-nodecimaldot}
 {\spanishplainpercent
  \let\es@operators\relax}
\es@genoption{es-nolists}
 {es-noenumerate,es-noitemize}{}
\es@genoption{es-sloppy}
 {es-nolayout,es-noshorthands}{}
\es@genoption{es-noshorthands}
 {es-noquoting,es-nodecimaldot,es-notilde}{}
\es@genoption{mexico}
 {mexico-com,es-nodecimaldot}{}
\es@genoption{mexico-com}
 {es-tabla,es-noquoting}
 {\def\lquoti{``}\def\rquoti{''}%
  \def\lquotii{`}\def\rquotii{'}%
  \def\lquotiii{\guillemotleft{}}%
  \def\rquotiii{\guillemotright{}}}

\def\es@ifoption#1#2#3{%
 \DeclareOption{es-#1}{}%
 \@ifpackagewith{babel}{es-#1}{#2}{#3}}%

\def\es@optlayout#1#2{\es@ifoption{#1}{}{\addto\layoutspanish{#2}}}

\else

\def\es@ifoption#1#2#3{\@namedef{spanish#1}{#2}}

\fi

\let\es@uclc\@secondoftwo
\es@ifoption{uppernames}{\let\es@uclc\@firstoftwo}{}

\def\es@tablename{Ccuadro}
\es@ifoption{tabla}{\def\es@tablename{Ttabla}}{}
\es@ifoption{cuadro}{\def\es@tablename{Ccuadro}}{}
%    \end{macrocode}
%
%    Captions follow a two step schema, so that, say, |\refname| is 
%    defined as |\spanishrefname| which in turn contains the string
%    to be printed. The final definition of |\captionsspanish|
%    is built below.
%
%    \begin{macrocode}
\def\captionsspanish{%
 \es@a{preface}{Prefacio}%
 \es@a{ref}{Referencias}%
 \es@a{abstract}{Resumen}%
 \es@a{bib}{Bibliograf\'{\i}a}%
 \es@a{chapter}{Cap\'{\i}tulo}%
 \es@a{appendix}{Ap\'{e}ndice}%
 \es@a{listfigure}{\'{I}ndice de \es@uclc Ffiguras}%
 \es@a{listtable}{\'{I}ndice de \expandafter\es@uclc\es@tablename s}%
 \es@a{index}{\'{I}ndice \es@uclc Aalfab\'{e}tico}%
 \es@a{figure}{Figura}%
 \es@a{table}{\expandafter\@firstoftwo\es@tablename}%
 \es@a{part}{Parte}%
 \es@a{encl}{Adjunto}%
 \es@a{cc}{Copia a}%
 \es@a{headto}{A}%
 \es@a{page}{p\'{a}gina}%
 \es@a{see}{v\'{e}ase}%
 \es@a{also}{v\'{e}ase tambi\'{e}n}%
 \es@a{proof}{Demostraci\'{o}n}%
 \es@a{glossary}{Glosario}%
 \@ifundefined{chapter}
  {\es@a{contents}{\'Indice}}%
  {\es@a{contents}{\'Indice \es@uclc Ggeneral}}}

\def\es@a#1{\@namedef{spanish#1name}}
\captionsspanish
\def\es@a#1#2{%
 \def\expandafter\noexpand\csname#1name\endcsname
 {\expandafter\noexpand\csname spanish#1name\endcsname}}
\edef\captionsspanish{\captionsspanish}
%    \end{macrocode}
%
%    Now two macros for dates (upper and lowercase).
%
%    \begin{macrocode}
\def\es@month#1{%
 \expandafter#1\ifcase\month\or Eenero\or Ffebrero\or
 Mmarzo\or Aabril\or Mmayo\or Jjunio\or Jjulio\or Aagosto\or
 Sseptiembre\or Ooctubre\or Nnoviembre\or Ddiciembre\fi}

\def\es@today#1{%
 \ifcase\es@datefmt
  \the\day~de \es@month#1%
 \else
  \es@month#1~\the\day
 \fi
 \ de\ifnum\year>1999\es@yearl\fi~\the\year}

\def\datespanish{%
 \def\today{\es@today\@secondoftwo}%
 \def\Today{\es@today\@firstoftwo}}
\newcount\es@datefmt
\def\spanishreverseddate{\es@datefmt\@ne}
\def\spanishdatedel{\def\es@yearl{l}}
\def\spanishdatede{\let\es@yearl\@empty}
\spanishdatede
%    \end{macrocode}
%
%    The basic macros to select the language in the preamble or the
%    config file. Use of |\selectlanguage| should be avoided at this
%    early stage because the active chars are not yet
%    active. |\selectspanish| makes them active.
%
%    \begin{macrocode}
\def\selectspanish{%
 \def\selectspanish{%
  \def\selectspanish{%
   \PackageWarning{spanish}{Extra \string\selectspanish ignored}}%
  \es@select}}
\@onlypreamble\selectspanish
\def\es@select{%
 \let\es@select\@undefined
 \selectlanguage{spanish}}

\let\es@shlist\@empty
%    \end{macrocode}
%
%    Instead of joining all the extras directly in |\extrasspanish|,
%    we subdivide them in three further groups.
%
%    \begin{macrocode}
\def\extrasspanish{%
 \textspanish
 \mathspanish
 \ifx\shorthandsspanish\@empty
  \expandafter\spanishdeactivate\expandafter{\es@shlist}%
  \languageshorthands{none}%
 \else
  \shorthandsspanish
 \fi}
\def\noextrasspanish{%
 \ifx\textspanish\@empty\else
  \notextspanish
 \fi
 \ifx\mathspanish\@empty\else
  \nomathspanish
 \fi
 \ifx\shorthandsspanish\@empty\else
  \noshorthandsspanish
 \fi
 \csname es@restorelist\endcsname}

\addto\textspanish{\es@sDRC\sptext{\es@sptext}}

\def\es@orddot{.}
%    \end{macrocode}
%
%    The definition of |\sptext| is more elaborated than that of
%     |\textsuperscript|. With uppercase superscript text
%    the scriptscriptsize is used. The mandatory dot is already
%    included. There are two versions, depending on the
%    format.
%
%    \begin{macrocode}
\ifes@latex
 \def\es@sptext#1{%
  {\es@orddot
   \setbox\z@\hbox{8}\dimen@\ht\z@
   \csname S@\f@size\endcsname
   \edef\@tempa{\def\noexpand\@tempc{#1}%
    \lowercase{\def\noexpand\@tempb{#1}}}\@tempa
   \ifx\@tempb\@tempc
    \fontsize\sf@size\z@
    \selectfont
    \advance\dimen@-1.15ex
   \else
    \fontsize\ssf@size\z@
    \selectfont
    \advance\dimen@-1.5ex
   \fi
   \math@fontsfalse\raise\dimen@\hbox{#1}}}
\else
 \let\sptextfont\rm
 \def\es@sptext#1{%
  {\es@orddot
   \setbox\z@\hbox{8}\dimen@\ht\z@
   \edef\@tempa{\def\noexpand\@tempc{#1}%
    \lowercase{\def\noexpand\@tempb{#1}}}\@tempa
   \ifx\@tempb\@tempc
    \advance\dimen@-0.75ex
    \raise\dimen@\hbox{$\scriptstyle\sptextfont#1$}%
   \else
    \advance\dimen@-0.8ex
    \raise\dimen@\hbox{$\scriptscriptstyle\sptextfont#1$}%
   \fi}}
\fi
%    \end{macrocode}
%
%    Lowercase small caps.  We check if the current font has small
%    caps.  If not, we fakes them.  \cs{selectfont} in \cs{es@lsc}
%    seems redundant, but it's not.  An intermediate macro allows
%    using an optimized variant for Roman numerals.
%
%    \begin{macrocode}
\ifes@latex
 \addto\textspanish{\es@sDRC\lsc{\es@lsc}} 
 \def\es@lsc{\es@xlsc\MakeUppercase\MakeLowercase}
 \def\es@xlsc#1#2#3{%
  \leavevmode
  \hbox{%
   \scshape\selectfont
   \expandafter\ifx\csname\f@encoding/\f@family/\f@series
      /n/\f@size\expandafter\endcsname
    \csname\curr@fontshape/\f@size\endcsname
    \csname S@\f@size\endcsname
    \fontsize\sf@size\z@\selectfont
    \PackageWarning{spanish}{Replacing `\curr@fontshape' by
      \MessageBreak faked small caps}%
    #1{#3}%
   \else
    #2{#3}%
   \fi}}
\fi
%    \end{macrocode}
%
%    The |quoting| environment (not available in Plain).  Overriding
%    the default |\everypar| is a bit tricky.
%
%    \begin{macrocode}
\newif\ifes@listquot

\ifes@latex
 \csname newtoks\endcsname\es@quottoks
 \csname newcount\endcsname\es@quotdepth
 \newenvironment{quoting}
  {\leavevmode
  \advance\es@quotdepth\@ne
  \csname lquot\romannumeral\es@quotdepth\endcsname%
  \ifnum\es@quotdepth=\@ne
   \es@listquotfalse
   \let\es@quotpar\everypar
   \let\everypar\es@quottoks
   \everypar\expandafter{\the\es@quotpar}%
   \es@quotpar{\the\everypar
    \ifes@listquot\global\es@listquotfalse\else\es@quotcont\fi}%
  \fi
  \toks@\expandafter{\es@quotcont}%
  \edef\es@quotcont{\the\toks@
   \expandafter\noexpand
   \csname rquot\romannumeral\es@quotdepth\endcsname}}
  {\csname rquot\romannumeral\es@quotdepth\endcsname}
 \def\lquoti{\guillemotleft{}}
 \def\rquoti{\guillemotright{}}
 \def\lquotii{``}
 \def\rquotii{''}
 \def\lquotiii{`}
 \def\rquotiii{'}
 \let\es@quotcont\@empty
%    \end{macrocode}
%
%    If there is a marginpar inside quoting, we don't add the
%    quotes. |\es@listqout| stores the quotes to be used before
%    item labels; otherwise they could appear after the labels.
%
%    \begin{macrocode}  
 \addto\@marginparreset{\let\es@quotcont\@empty}
 \DeclareRobustCommand\es@listquot{%
  \csname rquot\romannumeral\es@quotdepth\endcsname
  \global\es@listquottrue}
\fi
%    \end{macrocode}
%
%    |\frenchspacing|, |\...| and |\%|. 
%
%    \begin{macrocode}
\addto\textspanish{\bbl@frenchspacing}
\addto\notextspanish{\bbl@nonfrenchspacing}
\addto\textspanish{%
 \let\es@save@dot\.%
 \es@sDRC\.{\@ifnextchar.{\es@dots}{\es@save@dot}}}
\def\es@dots..{\leavevmode\hbox{...}\spacefactor\@M}
\def\es@sppercent{\unskip\textormath{$\m@th\,$}{\,}}
\def\spanishplainpercent{\let\es@sppercent\@empty}
\addto\textspanish{%
 \let\percentsign\%%
 \es@sDRC\%{\es@sppercent\percentsign{}}}
%    \end{macrocode}
%    
%    Now, the math group.  It's not easy to add an accent to an
%    operator, because we must avoid using text (that is, |\mbox|)
%    where we have no control on font and size, and at the same time
%    we need |\i|, which is forbidden in math mode.  |\dotlessi| must
%    be converted to uppercase if necessary in \LaTeXe.  There are two
%    versions, depending on the format.
%
%    \begin{macrocode}
\addto\mathspanish{\es@sDRC\dotlessi{\es@dotlessi}}
\let\nomathspanish\relax

\ifes@latex
 \def\es@texti{\i}
 \addto\@uclclist{\dotlessi\es@texti}
\fi

\ifes@latex
 \def\es@dotlessi{%
  \ifmmode
   {\ifnum\mathgroup=\m@ne
     \imath
    \else
     \count@\escapechar \escapechar=\m@ne
     \expandafter\expandafter\expandafter
      \split@name\expandafter\string\the\textfont\mathgroup\@nil
     \escapechar=\count@
     \@ifundefined{\f@encoding\string\i}%
      {\edef\f@encoding{\string?}}{}%
     \expandafter\count@\the\csname\f@encoding\string\i\endcsname
     \advance\count@"7000
     \mathchar\count@
    \fi}%
  \else
   \i
  \fi}
\else
 \def\es@dotlessi{\textormath{\i}{\mathchar"7010}}
\fi

\def\accentedoperators{%
 \def\es@op@ac##1{\acute{\if i##1\dotlessi\else##1\fi}}}
\def\unaccentedoperators{%
 \def\es@op@ac##1{##1}}
\accentedoperators
\def\spacedoperators{\let\es@op@sp\,}
\def\unspacedoperators{\let\es@op@sp\@empty}
\spacedoperators
\addto\mathspanish{\es@operators}

\ifes@latex\else
 \let\operator@font\rm
\fi
%    \end{macrocode}
%
%    Operators are stored in |\es@operators|, which is
%    included in the math group. Since |\operator@font| is
%    defined in \LaTeXe{} only, we define it in the plain variant.
%
%    \begin{macrocode}
\def\es@operators{%
 \es@sdef\bmod{\nonscript\mskip-\medmuskip\mkern5mu
  \mathbin{\operator@font m\es@op@ac od}\penalty900\mkern5mu
  \nonscript\mskip-\medmuskip}%
 \@ifundefined{@amsmath@err}%
  {\es@sdef\pmod##11{\allowbreak\mkern18mu
    ({\operator@font m\es@op@ac od}\,\,##11)}}%
  {\es@sdef\mod##1{\allowbreak\if@display\mkern18mu
    \else\mkern12mu\fi{\operator@font m\es@op@ac od}\,\,##1}%
   \es@sdef\pmod##1{\pod{{\operator@font m\es@op@ac od}%
    \mkern6mu##1}}}%
 \def\es@a##1 {%
  \if^##1^% empty? continue
   \bbl@afterelse
   \es@a
  \else
   \bbl@afterfi
   {\if&##1% &? finish
   \else
    \bbl@afterfi
    \begingroup
    \let\,\@empty % ignore when def'ing name
    \let\acute\@firstofone % id
    \edef\es@b{\expandafter\noexpand\csname##1\endcsname}%
    \def\,{\noexpand\es@op@sp}%
    \def\acute{\noexpand\es@op@ac}%
    \edef\es@a{\endgroup
     \noexpand\es@sdef\expandafter\noexpand\es@b{%
       \mathop{\noexpand\operator@font##1}\es@c}}%
    \es@a % restores itself
   \es@a
  \fi}%
 \fi}%
 \let\es@b\spanishoperators
 \addto\es@b{ }%
 \let\es@c\@empty
 \expandafter\es@a\es@b l\acute{i}m l\acute{i}m\,sup
  l\acute{i}m\,inf m\acute{a}x \acute{i}nf m\acute{i}n & %
 \def\es@c{\nolimits}%
 \expandafter\es@a\es@b sen tg arc\,sen arc\,cos arc\,tg & }
\def\spanishoperators{cotg cosec senh tgh }
%    \end{macrocode}
%
%    Now comes the text shorthands. They are grouped in
%    |\shorthandsspanish| and this style performs some
%    operations before the babel shortands are called.
%    The aims are to allow espression like |$a^{x'}$|
%    and to deactivate shorthands by making them of
%    category `other.' After providing a |\'i| shorthand,
%    the new macros are defined. 
%    
%    \begin{macrocode}
\DeclareTextCompositeCommand{\'}{OT1}{i}{\@tabacckludge'{\i}}

\def\es@set@shorthand#1{%
 \expandafter\edef\csname es@savecat\string#1\endcsname
  {\the\catcode`#1}%
 \initiate@active@char{#1}%
 \catcode`#1=\csname es@savecat\string#1\endcsname\relax
 \if.#1\else
  \addto\es@restorelist{\es@restore{#1}}%
  \addto\es@select{\shorthandon{#1}}%
  \addto\shorthandsspanish{\es@activate{#1}}%
  \addto\es@shlist{#1}%
 \fi}

\def\es@use@shorthand{%
 \if@safe@actives
  \bbl@afterelse
  \string
 \else
  \bbl@afterfi
  {\ifx\thepage\relax
   \bbl@afterelse
   \string
  \else
   \bbl@afterfi
   \es@use@sh
  \fi}%
 \fi}

\def\es@use@sh#1{%
 \ifx\protect\@unexpandable@protect
  \bbl@afterelse
  \noexpand#1%
 \else%
  \bbl@afterfi
  \textormath
   {\csname active@char\string#1\endcsname}%
   {\csname normal@char\string#1\endcsname}%
 \fi}

\gdef\es@activate#1{%
 \begingroup
  \lccode`\~=`#1
  \lowercase{%
 \endgroup
 \def~{\es@use@shorthand~}}}

\def\spanishdeactivate#1{%
 \@tfor\@tempa:=#1\do{\expandafter\es@spdeactivate\@tempa}}

\def\es@spdeactivate#1{%
 \if.#1%
  \mathcode`\.=\es@period@code
 \else
  \begingroup
   \lccode`\~=`#1
   \lowercase{%
  \endgroup
  \expandafter\let\expandafter~%
   \csname normal@char\string#1\endcsname}%
  \catcode`#1=\csname es@savecat\string#1\endcsname\relax
 \fi}
%    \end{macrocode}
%
%    |\es@restore| is used in the list |\es@restorelist|, which in
%    turn restores all shorthands as defined by \babel. The latter
%    macros also has |\es@quoting|.
%
%    \begin{macrocode}
\def\es@restore#1{%
 \shorthandon{#1}%
 \begingroup
  \lccode`\~=`#1
  \lowercase{%
 \endgroup
 \bbl@deactivate{~}}}
%    \end{macrocode}
%
%    To selectively define the shorthands we have a couple of
%    macros, which defines a certain combination if the first
%    character has been activated as a shorthand. The second
%    one is intended for a few shorthands with an alternative
%    form.                                                      
%
%    \begin{macrocode}
\def\es@declare#1{%
 \@ifundefined{es@savecat\expandafter\string\@firstoftwo#1}%
  {\@gobble}%
  {\declare@shorthand{spanish}{#1}}}
\def\es@declarealt#1#2#3{%
 \es@declare{#1}{#3}%
 \es@declare{#2}{#3}}

\ifes@latex\else
 \def\@tabacckludge#1{\csname\string#1\endcsname}
\fi

\@ifundefined{add@accent}{\def\add@accent#1#2{\accent#1 #2}}{}
%    \end{macrocode}
%
%    Instead of redefining |\'|, we redefine the internal
%    macro for the OT1 encoding.
%
%    \begin{macrocode}
\ifes@latex
 \def\es@accent#1#2#3{%
  \expandafter\@text@composite
  \csname OT1\string#1\endcsname#3\@empty\@text@composite
  {\bbl@allowhyphens\add@accent{#2}{#3}\bbl@allowhyphens
   \setbox\@tempboxa\hbox{#3%
    \global\mathchardef\accent@spacefactor\spacefactor}%
   \spacefactor\accent@spacefactor}}
\else
 \def\es@accent#1#2#3{%
  \bbl@allowhyphens\add@accent{#2}{#3}\bbl@allowhyphens
  \spacefactor\sfcode`#3 }
\fi

\addto\shorthandsspanish{\languageshorthands{spanish}}%
\es@ifoption{noshorthands}{}{\es@set@shorthand{"}}
%    \end{macrocode}
%
%    We override the default |"| of babel, intended for german.
%
%    \begin{macrocode}
\def\es@umlaut#1{%
 \bbl@allowhyphens\add@accent{127}#1\bbl@allowhyphens
 \spacefactor\sfcode`#1 }

\addto\shorthandsspanish{%
 \babel@save\bbl@umlauta
 \let\bbl@umlauta\es@umlaut}
\let\noshorthandsspanish\relax

\ifes@latex
\addto\shorthandsspanish{%
 \expandafter\es@sdef\csname OT1\string\~\endcsname{\es@accent\~{126}}%
 \expandafter\es@sdef\csname OT1\string\'\endcsname{\es@accent\'{19}}}
\else
\addto\shorthandsspanish{%
 \es@sdef\~{\es@accent\~{126}}%
 \es@sdef\'#1{\if#1i\es@accent\'{19}\i\else\es@accent\'{19}{#1}\fi}}
\fi

\def\es@sptext@r#1#2{\es@sptext{#1#2}}
\es@declare{"a}{\sptext{a}}
\es@declare{"A}{\sptext{A}}
\es@declare{"o}{\sptext{o}}
\es@declare{"O}{\sptext{O}}
\es@declare{"e}{\protect\es@sptext@r{e}}
\es@declare{"E}{\protect\es@sptext@r{E}}
\es@declare{"u}{\"u}
\es@declare{"U}{\"U}
\es@declare{"i}{\"{\i}}
\es@declare{"I}{\"I}
\es@declare{"c}{\c{c}}
\es@declare{"C}{\c{C}}
\es@declare{"<}{\guillemotleft{}}
\es@declare{">}{\guillemotright{}}
\def\es@chf{\char\hyphenchar\font}
\es@declare{"-}{\bbl@allowhyphens\-\bbl@allowhyphens}
\es@declare{"=}{\bbl@allowhyphens\es@chf\hskip\z@skip}
\es@declare{"~}
 {\bbl@allowhyphens
  \discretionary{\es@chf}{\es@chf}{\es@chf}%
  \bbl@allowhyphens}
\es@declare{"r}
 {\bbl@allowhyphens
  \discretionary{\es@chf}{}{r}%
  \bbl@allowhyphens}
\es@declare{"R}
 {\bbl@allowhyphens
  \discretionary{\es@chf}{}{R}%
  \bbl@allowhyphens}
\es@declare{"y}
 {\@ifundefined{scalebox}%
   {\ensuremath{\tau}}%
   {\raisebox{1ex}{\scalebox{-1}{\resizebox{.45em}{1ex}{2}}}}}
\es@declare{""}{\hskip\z@skip}
\es@declare{"/}
 {\setbox\z@\hbox{/}%
  \dimen@\ht\z@
  \advance\dimen@-1ex
  \advance\dimen@\dp\z@
  \dimen@.31\dimen@
  \advance\dimen@-\dp\z@
  \ifdim\dimen@>0pt
   \kern.01em\lower\dimen@\box\z@\kern.03em
  \else
   \box\z@
  \fi}
\es@declare{"?}
 {\setbox\z@\hbox{?`}%
  \leavevmode\raise\dp\z@\box\z@}
\es@declare{"!}
 {\setbox\z@\hbox{!`}%
  \leavevmode\raise\dp\z@\box\z@}

\def\spanishdecimal#1{\def\es@decimal{{#1}}}
\def\decimalcomma{\spanishdecimal{,}}
\def\decimalpoint{\spanishdecimal{.}}
\decimalcomma
\es@ifoption{nodecimaldot}{}
 {\AtBeginDocument{\bgroup\@fileswfalse}%
  \es@set@shorthand{.}%
  \AtBeginDocument{\egroup}%
  \@namedef{normal@char\string.}{%
   \@ifnextchar\egroup
    {\mathchar\es@period@code\relax}%
    {\csname active@char\string.\endcsname}}%
  \declare@shorthand{system}{.}{\mathchar\es@period@code\relax}%
  \addto\shorthandsspanish{%
   \mathchardef\es@period@code\the\mathcode`\.%
   \babel@savevariable{\mathcode`\.}%
   \mathcode`\.="8000 %
   \es@activate{.}}%
  \def\es@a#1{\es@declare{.#1}{\es@decimal#1}}%
  \es@a1\es@a2\es@a3\es@a4\es@a5\es@a6\es@a7\es@a8\es@a9\es@a0}

\es@ifoption{notilde}{}{\es@set@shorthand{~}}
\def\deactivatetilden{%
 \expandafter\let\csname spanish@sh@\string~@n@\endcsname\relax
 \expandafter\let\csname spanish@sh@\string~@N@\endcsname\relax}
\es@ifoption{tilden}
 {\es@declare{~n}{\~n}%
  \es@declare{~N}{\~N}}
 {\let\deactivatetilden\relax}
\es@declarealt{~-}{"+}{%
 \leavevmode
 \bgroup
 \let\@sptoken\es@dashes % Changes \@ifnextchar behaviour
 \@ifnextchar-%       
  {\es@dashes}%
  {\hbox{\es@chf}\egroup}}
\def\es@dashes-{%
 \@ifnextchar-%
  {\bbl@allowhyphens\hbox{---}\bbl@allowhyphens\egroup\@gobble}%
  {\bbl@allowhyphens\hbox{--}\bbl@allowhyphens\egroup}}

\es@ifoption{noquoting}%
 {\let\es@quoting\relax
 \let\activatequoting\relax
 \let\deactivatequoting\relax}
 {\@ifundefined{XML@catcodes}%
 {\es@set@shorthand{<}%
  \es@set@shorthand{>}%
  \declare@shorthand{system}{<}{\csname normal@char\string<\endcsname}%
  \declare@shorthand{system}{>}{\csname normal@char\string>\endcsname}%
  \addto\es@restorelist{\es@quoting}%
  \addto\es@select{\es@quoting}%
  \ifes@latex
   \AtBeginDocument{%
    \es@quoting
    \if@filesw
     \immediate\write\@mainaux{\string\@nameuse{es@quoting}}%
    \fi}%
  \fi
  \def\activatequoting{%
   \shorthandon{<>}%
   \let\es@quoting\activatequoting}%
  \def\deactivatequoting{%
   \shorthandoff{<>}%
   \let\es@quoting\deactivatequoting}}{}}

\es@declarealt{<<}{"`}{\begin{quoting}}
\es@declarealt{>>}{"'}{\end{quoting}}
%    \end{macrocode}
%
%    Acute accent shorthands are stored in a macro.  If |activeacute|
%    was set as an option it's executed.  If not is not deleted for a
%    possible later use in the |cfg| file.  In non \LaTeXe{} formats
%    it's always executed.
%
% \changes{spanish~5.0e}{2008/07/06}{Two acutes in a row should be
%    turned into a double right quote}
% \changes{spanish~5.0g}{2008/07/20}{Fixed bad kerning before two
%    acutes}
%
%    \begin{macrocode}
\begingroup
\catcode`\'=12
\gdef\es@activeacute{%
 \es@set@shorthand{'}%
 \def\es@a##1{\es@declare{'##1}{\@tabacckludge'##1}}%
 \es@a a\es@a e\es@a i\es@a o\es@a u%
 \es@a A\es@a E\es@a I\es@a O\es@a U%
 \es@declare{'n}{\~n}%
 \es@declare{'N}{\~N}%
 \es@declare{''}{''}%
%    \end{macrocode}
%
%    But \textsf{spanish} allows two category codes for |'|,
%    so both should be taken into account in \cs{bbl@pr@m@s}.
%    
%    \begin{macrocode}
 \let\es@pr@m@s\bbl@pr@m@s
 \def\bbl@pr@m@s{%
  \ifx'\@let@token
   \bbl@afterelse
   \pr@@@s
  \else
   \bbl@afterfi
   \es@pr@m@s
  \fi}%
 \let\es@activeacute\relax}
\endgroup

\ifes@latex
 \@ifpackagewith{babel}{activeacute}{\es@activeacute}{}
\else
 \es@activeacute
\fi
%    \end{macrocode}
%
%    And the customization. By default these macros only
%    store the values and do nothing.
%
%    \begin{macrocode}
\def\es@enumerate#1#2#3#4{\def\es@enum{{#1}{#2}{#3}{#4}}}
\def\es@itemize#1#2#3#4{\def\es@item{{#1}{#2}{#3}{#4}}}

\ifes@latex
\es@enumerate{1.}{a)}{1)}{a$'$}
\def\spanishdashitems{\es@itemize{---}{---}{---}{---}}
\def\spanishsymbitems{%
 \es@itemize
  {\leavevmode\hbox to 1.2ex
   {\hss\vrule height .9ex width .7ex depth -.2ex\hss}}%
  {\textbullet}%
  {$\m@th\circ$}%
  {$\m@th\diamond$}}
\def\spanishsignitems{%
 \es@itemize{\textbullet}%
  {$\m@th\circ$}%
  {$\m@th\diamond$}%
  {$\m@th\triangleright$}}
\spanishsymbitems
\def\es@enumdef#1#2#3\@@{%
 \if#21%
  \@namedef{theenum#1}{\arabic{enum#1}}%
 \else\if#2a%
  \@namedef{theenum#1}{\emph{\alph{enum#1}}}%
 \else\if#2A%
  \@namedef{theenum#1}{\Alph{enum#1}}%
 \else\if#2i%
  \@namedef{theenum#1}{\roman{enum#1}}%
 \else\if#2I%
  \@namedef{theenum#1}{\Roman{enum#1}}%
 \else\if#2o%
  \@namedef{theenum#1}{\arabic{enum#1}\sptext{o}}%
 \fi\fi\fi\fi\fi\fi
 \toks@\expandafter{\csname theenum#1\endcsname}%
 \expandafter\edef\csname labelenum#1\endcsname
   {\noexpand\es@listquot\the\toks@#3}}
\def\es@guillemot#1#2{%
 \ifmmode#1%
 \else
  \save@sf@q{\penalty\@M
  \leavevmode\hbox{\usefont{U}{lasy}{m}{n}%
   \char#2 \kern-0.19em\char#2 }}%
 \fi}
\def\layoutspanish{%
 \let\layoutspanish\@empty
 \DeclareTextCommand{\guillemotleft}{OT1}{\es@guillemot\ll{40}}%
 \DeclareTextCommand{\guillemotright}{OT1}{\es@guillemot\gg{41}}%
 \def\@fnsymbol##1%
  {\ifcase##1\or*\or**\or***\or****\or
   *****\or******\else\@ctrerr\fi}%
 \def\@alph##1%
  {\ifcase##1\or a\or b\or c\or d\or e\or f\or g\or h\or i\or j\or
   k\or l\or m\or n\or \~n\or o\or p\or q\or r\or s\or t\or u\or v\or
   w\or x\or y\or z\else\@ctrerr\fi}%
 \def\@Alph##1%
  {\ifcase##1\or A\or B\or C\or D\or E\or F\or G\or H\or I\or J\or
   K\or L\or M\or N\or \~N\or O\or P\or Q\or R\or S\or T\or U\or V\or
   W\or X\or Y\or Z\else\@ctrerr\fi}}

\es@optlayout{noenumerate}{%
 \def\es@enumerate#1#2#3#4{%
  \es@enumdef{i}#1\@empty\@empty\@@
  \es@enumdef{ii}#2\@empty\@empty\@@
  \es@enumdef{iii}#3\@empty\@empty\@@
  \es@enumdef{iv}#4\@empty\@empty\@@}%
  \def\p@enumii{\theenumi}%
  \def\p@enumiii{\p@enumii\theenumii}%
  \def\p@enumiv{\p@enumiii\theenumiii}%
  \expandafter\es@enumerate\es@enum}
\es@optlayout{noitemize}{%
 \def\es@itemize#1#2#3#4{%
  \def\labelitemi{\es@listquot#1}%
  \def\labelitemii{\es@listquot#2}%
  \def\labelitemiii{\es@listquot#3}%
  \def\labelitemiv{\es@listquot#4}}%
  \expandafter\es@itemize\es@item}
\let\esromanindex\@secondoftwo
\es@ifoption{ucroman}
 {\def\es@romandef{%
   \def\esromanindex##1##2{##1{\uppercase{##2}}}%
   \def\@roman{\@Roman}}}
 {\def\es@romandef{%
   \def\esromanindex##1##2{##1{\es@scroman{##2}}}%
   \def\@roman##1{\es@roman{\number##1}}%
   \def\es@roman##1{\es@scroman{\romannumeral##1}}%
   \DeclareRobustCommand\es@scroman{\es@xlsc\uppercase\@firstofone}}}
\es@optlayout{lcroman}{\es@romandef}
\newcommand\spanishlcroman{\def\@roman##1{\romannumeral##1}}
\newcommand\spanishucroman{\def\@roman{\@Roman}}
\newcommand\spanishscroman{\def\@roman##1{\es@roman{\romannumeral##1}}}
\es@optlayout{noindentfirst}{%
 \let\@afterindentfalse\@afterindenttrue
 \@afterindenttrue}
\es@optlayout{nosectiondot}{%
 \def\@seccntformat#1{\csname the#1\endcsname.\quad}%
 \def\numberline#1{\hb@xt@\@tempdima{#1\if&#1&\else.\fi\hfil}}}
\es@ifoption{nolayout}{\let\layoutspanish\relax}{}
\es@ifoption{sloppy}{\let\textspanish\relax\let\mathspanish\relax}{}
\es@ifoption{delayed}{}{\def\es@layoutspanish{\layoutspanish}}
\es@ifoption{preindex}{\AtEndOfPackage{\RequirePackage{romanidx}}}{}
%    \end{macrocode}
%    
%    We need to execute the following code when babel has been
%    run, in order to see if |spanish| is the main language.
%    
%    \begin{macrocode}
\AtEndOfPackage{%
\let\es@activeacute\@undefined
\def\bbl@tempa{spanish}%
\ifx\bbl@main@language\bbl@tempa
 \@nameuse{es@layoutspanish}%
 \addto\es@select{%
  \@ifstar{\PackageError{spanish}%
   {Old syntax--use es-nolayout}%
   {If you don't want changes in layout\MessageBreak
   use the es-nolayout package option}}%
   {}}%
 \AtBeginDocument{\layoutspanish}%
\fi
\selectspanish}
\fi
%    \end{macrocode}
%    
%    After restoring the catcode of |~| and setting the minimal
%    values for hyphenation, the |.ldf| is finished.
%
%    \begin{macrocode}
\es@savedcatcodes
\providehyphenmins{\CurrentOption}{\tw@\tw@}
\ifes@latex\else
 \es@select
\fi
\ldf@finish{spanish}
\csname activatequoting\endcsname
%</code>
%    \end{macrocode}
%    That's all in the main file. 
%
%    The |spanish| option writes a macro in the page field of
%    \textit{MakeIndex} in entries with small caps number, and they
%    are rejected. This program is a preprocessor which moves this
%    macro to the entry field. It can be called from the main 
%    document as a package or with the package option |es-preindex|.
%
%    \begin{macrocode}
%<*indexes>
\makeatletter

\@ifundefined{es@idxfile}
  {\def\spanishindexchars#1#2#3{%
     \edef\es@encap{`\expandafter\noexpand\csname\string#1\endcsname}%
     \edef\es@openrange{`\expandafter\noexpand\csname\string#2\endcsname}%
     \edef\es@closerange{`\expandafter\noexpand\csname\string#3\endcsname}}%
   \spanishindexchars{|}{(}{)}%
   \ifx\documentclass\@twoclasseserror
      \edef\es@idxfile{\jobname}%
      \AtEndDocument{%
        \addto\@defaultsubs{%
          \immediate\closeout\@indexfile
          \input{romanidx.sty}}}%
      \expandafter\endinput
   \fi}{}

\newcount\es@converted
\newcount\es@processed

\def\es@split@file#1.#2\@@{#1}
\def\es@split@ext#1.#2\@@{#2}

\@ifundefined{es@idxfile}
  {\typein[\answer]{^^JArchivo que convertir^^J%
   (extension por omision .idx):}}
  {\let\answer\es@idxfile}

\@expandtwoargs\in@{.}{\answer}
\ifin@
  \edef\es@input@file{\expandafter\es@split@file\answer\@@}
  \edef\es@input@ext{\expandafter\es@split@ext\answer\@@}
\else
  \edef\es@input@file{\answer}
  \def\es@input@ext{idx}
\fi

\@ifundefined{es@idxfile}
  {\typein[\answer]{^^JArchivo de destino^^J%
     (archivo por omision: \es@input@file.eix,^^J%
      extension por omision .eix):}}
  {\let\answer\es@idxfile}
\ifx\answer\@empty
  \edef\es@output{\es@input@file.eix}
\else
  \@expandtwoargs\in@{.}{\answer}
  \ifin@
     \edef\es@output{\answer}
  \else
     \edef\es@output{\answer.eix}
  \fi
\fi

\@ifundefined{es@idxfile}
  {\typein[\answer]{%
   ^^J?Se ha usado algun esquema especial de controles^^J%
   de MakeIndex para encap, open_range o close_range?^^J%
   [s/n] (n por omision)}}
  {\def\answer{n}}

\if s\answer
  \typein[\answer]{^^JCaracter para 'encap'^^J%
    (\string| por omision)}
  \ifx\answer\@empty\else
    \edef\es@encap{%
      `\expandafter\noexpand\csname\expandafter\string\answer\endcsname}
  \fi
  \typein[\answer]{^^JCaracter para 'open_range'^^J%
    (\string( por omision)}
  \ifx\answer\@empty\else
    \edef\es@openrange{%
      `\expandafter\noexpand\csname\expandafter\string\answer\endcsname}
  \fi
  \typein[\answer]{^^JCaracter para 'close_range'^^J%
    (\string) por omision)}
  \ifx\answer\@empty\else
    \edef\es@closerange{%
      `\expandafter\noexpand\csname\expandafter\string\answer\endcsname}
  \fi
\fi

\newwrite\es@indexfile
\immediate\openout\es@indexfile=\es@output

\newif\ifes@encapsulated

\def\es@scroman#1{#1}
\edef\es@slash{\expandafter\@gobble\string\\}

\def\indexentry{%
  \begingroup
  \@sanitize
  \es@indexentry}

\begingroup

\catcode`\|=12 \lccode`\|=\es@encap\relax
\catcode`\(=12 \lccode`\(=\es@openrange\relax
\catcode`\)=12 \lccode`\)=\es@closerange\relax

\lowercase{
\gdef\es@indexentry#1{%
  \endgroup
  \advance\es@processed\@ne
  \es@encapsulatedfalse
  \es@bar@idx#1|\@@
  \es@idxentry}%
}

\lowercase{
\gdef\es@idxentry#1{%
  \in@{\es@scroman}{#1}%
  \ifin@
    \advance\es@converted\@ne
    \immediate\write\es@indexfile{%
      \string\indexentry{\es@b|\ifes@encapsulated\es@p\fi esromanindex%
        {\ifx\es@a\@empty\else\es@slash\es@a\fi}}{#1}}%
  \else
    \immediate\write\es@indexfile{%
      \string\indexentry{\es@b\ifes@encapsulated|\es@p\es@a\fi}{#1}}%
  \fi}
}

\lowercase{
\gdef\es@bar@idx#1|#2\@@{%
  \def\es@b{#1}\def\es@a{#2}%
  \ifx\es@a\@empty\else\es@encapsulatedtrue\es@bar@eat#2\fi}
}

\lowercase{
\gdef\es@bar@eat#1#2|{\def\es@p{#1}\def\es@a{#2}%
  \edef\es@t{(}\ifx\es@t\es@p
  \else\edef\es@t{)}\ifx\es@t\es@p
  \else
    \edef\es@a{\es@p\es@a}\let\es@p\@empty%
  \fi\fi}
}

\endgroup

\input \es@input@file.\es@input@ext

\immediate\closeout\es@indexfile

\typeout{*****************}
\typeout{Se ha procesado: \es@input@file.\es@input@ext }
\typeout{Lineas leidas: \the\es@processed}
\typeout{Lineas convertidas: \the\es@converted}
\typeout{Resultado en: \es@output}
\ifnum\es@converted>\z@
  \typeout{Genere el indice a partir de ese archivo}
\else
  \typeout{No se ha convertido nada. Se puede generar}
  \typeout{el .ind  directamente de \es@input@file.\es@input@ext}
\fi
\typeout{*****************}

\@ifundefined{es@sdef}{\@@end}{}

\endinput
%</indexes>
%    \end{macrocode}
%
% \Finale
%
%%
%% \CharacterTable
%%  {Upper-case    \A\B\C\D\E\F\G\H\I\J\K\L\M\N\O\P\Q\R\S\T\U\V\W\X\Y\Z
%%   Lower-case    \a\b\c\d\e\f\g\h\i\j\k\l\m\n\o\p\q\r\s\t\u\v\w\x\y\z
%%   Digits        \0\1\2\3\4\5\6\7\8\9
%%   Exclamation   \!     Double quote  \"     Hash (number) \#
%%   Dollar        \$     Percent       \%     Ampersand     \&
%%   Acute accent  \'     Left paren    \(     Right paren   \)
%%   Asterisk      \*     Plus          \+     Comma         \,
%%   Minus         \-     Point         \.     Solidus       \/
%%   Colon         \:     Semicolon     \;     Less than     \<
%%   Equals        \=     Greater than  \>     Question mark \?
%%   Commercial at \@     Left bracket  \[     Backslash     \\
%%   Right bracket \]     Circumflex    \^     Underscore    \_
%%   Grave accent  \`     Left brace    \{     Vertical bar  \|
%%   Right brace   \}     Tilde         \~}
%%
\endinput





}
\DeclareOption{swedish}{% \iffalse meta-comment
%
% Copyright 1989-2005 Johannes L. Braams and any individual authors
% listed elsewhere in this file.  All rights reserved.
% 
% This file is part of the Babel system.
% --------------------------------------
% 
% It may be distributed and/or modified under the
% conditions of the LaTeX Project Public License, either version 1.3
% of this license or (at your option) any later version.
% The latest version of this license is in
%   http://www.latex-project.org/lppl.txt
% and version 1.3 or later is part of all distributions of LaTeX
% version 2003/12/01 or later.
% 
% This work has the LPPL maintenance status "maintained".
% 
% The Current Maintainer of this work is Johannes Braams.
% 
% The list of all files belonging to the Babel system is
% given in the file `manifest.bbl. See also `legal.bbl' for additional
% information.
% 
% The list of derived (unpacked) files belonging to the distribution
% and covered by LPPL is defined by the unpacking scripts (with
% extension .ins) which are part of the distribution.
% \fi
% \CheckSum{272}
% \iffalse
%    Tell the \LaTeX\ system who we are and write an entry on the
%    transcript.
%<*dtx>
\ProvidesFile{swedish.dtx}
%</dtx>
%<code>\ProvidesLanguage{swedish}
%\fi
%\ProvidesFile{swedish.dtx}
        [2005/03/31 v2.3d Swedish support from the babel system]
%\iffalse
%% File `swedish.dtx'
%% Babel package for LaTeX version 2e
%% Copyright (C) 1989 - 2005
%%           by Johannes Braams, TeXniek
%
%% Please report errors to: J.L. Braams
%%                          babel at braams.cistron.nl
%
%    This file is part of the babel system, it provides the source
%    code for the Swedish language definition file.  A contribution
%    was made by Sten Hellman HELLMAN@CERNVM.CERN.CH
%
%    Further enhancements for version 2.0 were provided by 
%    Erik "Osthols <erik\_osthols@yahoo.com>
%<*filedriver>
\documentclass{ltxdoc}
\newcommand*\TeXhax{\TeX hax}
\newcommand*\babel{\textsf{babel}}
\newcommand*\langvar{$\langle \it lang \rangle$}
\newcommand*\note[1]{}
\newcommand*\Lopt[1]{\textsf{#1}}
\newcommand*\file[1]{\texttt{#1}}
\begin{document}
 \DocInput{swedish.dtx}
\end{document}
%</filedriver>
%\fi
% \GetFileInfo{swedish.dtx}
%
% \changes{swedish-1.0a}{1991/07/15}{Renamed \file{babel.sty} in
%    \file{babel.com}}
% \changes{swedish-1.1}{1992/02/16}{Brought up-to-date with babel 3.2a}
% \changes{swedish-1.2}{1994/02/27}{Update for LaTeX2e}
% \changes{swedish-1.3d}{1994/06/26}{Removed the use of \cs{filedate}
%    and moved identification after the loading of \file{babel.def}}
% \changes{swedish-1.3e}{1995/05/28}{Update for release 3.5}
% \changes{swedish-2.0}{1996/01/24}{Introduced active double quote}
% \changes{swedish-2.1}{1996/10/10}{Replaced \cs{undefined} with
%    \cs{@undefined} and \cs{empty} with \cs{@empty} for consistency
%    with \LaTeX, moved the definition of \cs{atcatcode} right to the
%    beginning.}
% \changes{swedish-2.3d}{2001/11/16}{Fixed a \cs{changes} entry}
%
%  \section{The Swedish language}
%
%    The file \file{\filename}\footnote{The file described in this
%    section has version number \fileversion\ and was last revised on
%    \filedate. Contributions were made by Sten Hellman
%    (\texttt{HELLMAN@CERNVM.CERN.CH}) and Erik \"Osthols
%    (\texttt{erik\_osthols@yahoo.com}).} defines all the
%    language-specific macros for the Swedish language. This
%    file has borrowed heavily from |finnish.dtx| and |germanb.dtx|.
%
%    For this language the character |"| is made active. In
%    table~\ref{tab:swedish-quote} an overview is given of its
%    purpose. The vertical placement of the "umlaut" in some letters
%    can be controlled this way.
%    \begin{table}[htb]
%     \begin{center}
%     \begin{tabular}{lp{8cm}}
%      |"a| & Gives \"a, also implemented for |"A|, |"o| and |"O|. \\
%      |"w|, |"W| & gives {\aa} and {\AA}.                         \\
%      |"ff| & for |ff| to be hyphenated as |ff-f|. Used for compound
%              words, such as \texttt{stra|"|ff{\aa}nge}, which
%              should be hyphenated as \texttt{straff-f{\aa}nge}.
%              This is also implemented for b, d, f, g, l, m, n,
%              p, r, s, and t.                                     \\
%      \verb="|= & disable ligature at this position. This should be 
%              used for compound words, such as
%              ``\texttt{stra|"|ffinr\"attning}'',
%              which should not have the ligature ``ffi''.         \\
%      |"-| & an explicit hyphen sign, allowing hyphenation
%             in the rest of the word, such as e. g. in
%             ``x|"-|axeln''.                                      \\
%      |""| & like |"-|, but producing no hyphen sign
%             (for words that should break at some sign such as
%             \texttt{och/|""|eller}).                             \\
%      |"~| & for an explicit hyphen without a breakpoint; useful for
%             expressions such as ``2|"~|3 veckor'' where no linebreak
%             is desirable.                                        \\
%      |"=| & an explicit hyphen sign allowing subsequent hyphenation,
%             for expressions such as ``studiebidrag och \newline
%             -l{\aa}n''.                                          \\
%      |\-| & like the old |\-|, but allowing hyphenation
%             in the rest of the word.                             \\
%     \end{tabular}
%     \caption{The extra definitions made
%              by \file{swedish.sty}}\label{tab:swedish-quote}
%     \end{center}
%    \end{table}
%
%    Two variations for formatting of dates are added. \cs{datesymd}
%    makes \cs{today} output dates formatted as YYYY-MM-DD, which is
%    commonly used in Sweden today. \cs{datesdmy} formats the date as
%    D/M YYYY, which is also very common in Sweden. These commands
%    should be issued after \cs{begin{document}}.
%
% \StopEventually{}
%
%    The macro |\LdfInit| takes care of preventing that this file is
%    loaded more than once, checking the category code of the
%    \texttt{@} sign, etc.
% \changes{swedish-2.1}{1996/11/03}{Now use \cs{LdfInit} to perform
%    initial checks} 
%    \begin{macrocode}
%<*code>
\LdfInit{swedish}\captionsswedish
%    \end{macrocode}
%
%    When this file is read as an option, i.e. by the |\usepackage|
%    command, \texttt{swedish} will be an `unknown' language in which
%    case we have to make it known. So we check for the existence of
%    |\l@swedish| to see whether we have to do something here.
%
% \changes{swedish-1.0c}{1991/10/29}{Removed use of \cs{@ifundefined}}
% \changes{swedish-1.1}{1992/02/16}{Added a warning when no hyphenation
%    patterns were loaded.}
% \changes{swedish-1.3d}{1994/06/26}{Now use \cs{@nopatterns} to
%    producew the warning}
%    \begin{macrocode}
\ifx\l@swedish\@undefined
    \@nopatterns{Swedish}
    \adddialect\l@swedish0\fi
%    \end{macrocode}
%
%    The next step consists of defining commands to switch to the
%    Swedish language. The reason for this is that a user might want
%    to switch back and forth between languages.
%
% \begin{macro}{\captionsswedish}
%    The macro |\captionsswedish| defines all strings used in the four
%    standard documentclasses provided with \LaTeX.
% \changes{swedish-1.0b}{1991/08/21}{removed type in definition of
%    \cs{contentsname}}
% \changes{swedish-1.0b}{1991/08/21}{added definition for \cs{enclname}}
% \changes{swedish-1.0b}{1991/08/21}{made definition of \cs{refname}
%    pluralis}
% \changes{swedish-1.1}{1992/02/16}{Added \cs{seename}, \cs{alsoname}
%    and \cs{prefacename}}
% \changes{swedish-1.1}{1993/07/15}{\cs{headpagename} should be
%    \cs{pagename}}
% \changes{swedish-1.1b}{1993/09/16}{Added translations}
% \changes{swedish-1.3d}{1994/07/27}{Changed \cs{aa} to
%    \cs{csname}\texttt{ aa}\cs{endcsname}, to make \cs{uppercase} do
%    the right thing}
% \changes{swedish-1.3f}{1995/07/04}{Added \cs{proofname} for
%    AMS-\LaTeX}
%    \begin{macrocode}
\addto\captionsswedish{%
  \def\prefacename{F\"orord}%
  \def\refname{Referenser}%
  \def\abstractname{Sammanfattning}%
  \def\bibname{Litteraturf\"orteckning}%
  \def\chaptername{Kapitel}%
  \def\appendixname{Bilaga}%
  \def\contentsname{Inneh\csname aa\endcsname ll}%
  \def\listfigurename{Figurer}%
  \def\listtablename{Tabeller}%
  \def\indexname{Sakregister}%
  \def\figurename{Figur}%
  \def\tablename{Tabell}%
  \def\partname{Del}%
%    \end{macrocode}
% \changes{swedish-2.3a}{2000/01/19}{Added full stop after ``Bil''}
%    \begin{macrocode}
  \def\enclname{Bil.}%
  \def\ccname{Kopia f\"or k\"annedom}%
  \def\headtoname{Till}% in letter
  \def\pagename{Sida}%
  \def\seename{se}%
%    \end{macrocode}
% \changes{swedish-1.3e}{1995/05/29}{Changed \cs{alsoname} from
%    `\texttt{se ocks\aa}'}
%    \begin{macrocode}
  \def\alsoname{se \"aven}%
%    \end{macrocode}
% \changes{swedish-1.3g}{1995/11/03}{Replaced `Proof' by its
%    translation} 
% \changes{swedish-2.3b}{2000/09/26}{Added \cs{glossaryname}}
% \changes{swedish-2.3c}{2001/03/12}{Provided translation for
%    Glossary}
%    \begin{macrocode}
  \def\proofname{Bevis}%
  \def\glossaryname{Ordlista}%
  }%
%    \end{macrocode}
% \end{macro}
%
% \begin{macro}{\dateswedish}
%    The macro |\dateswedish| redefines the command |\today| to
%    produce Swedish dates.
% \changes{swedish-2.2}{1997/10/01}{Use \cs{edef} to define
%    \cs{today} to save memory}
% \changes{swedish-2.2}{1998/03/28}{use \cs{def} instead of
%    \cs{edef}}
%    \begin{macrocode}
\def\dateswedish{%
  \def\today{%
    \number\day~\ifcase\month\or
    januari\or februari\or mars\or april\or maj\or juni\or
    juli\or augusti\or september\or oktober\or november\or
    december\fi
    \space\number\year}}
%    \end{macrocode}
% \end{macro}
%
%  \begin{macro}{\datesymd}
% \changes{swedish-2.3a}{2000/01/20}{Command added}
%    The macro |\datesymd| redefines the command |\today| to
%    produce dates in the format YYYY-MM-DD, common in Sweden.
%    \begin{macrocode}
\def\datesymd{%
  \def\today{\number\year-\two@digits\month-\two@digits\day}%
  }
%    \end{macrocode}
%  \end{macro}
%
%  \begin{macro}{\datesdmy}
% \changes{swedish-2.3a}{2000/01/20}{Command added}
%    The macro |\datesdmy| redefines the command |\today| to
%    produce Swedish dates in the format DD/MM YYYY, also common in
%    Sweden.
%    \begin{macrocode}
\def\datesdmy{%
  \def\today{\number\day/\number\month\space\number\year}%
  }
%    \end{macrocode}
%  \end{macro}
%
%  \begin{macro}{\swedishhyphenmins}
%    The swedish hyphenation patterns can be used with |\lefthyphenmin|
%    set to~2 and |\righthyphenmin| set to~2.
% \changes{swedish-1.3e}{1995/06/02}{use \cs{swedishhyphenmins} to
%    store the correct values}
%    \begin{macrocode}
\providehyphenmins{swedish}{\tw@\tw@}
%    \end{macrocode}
%  \end{macro}
%
% \begin{macro}{\extrasswedish}
% \changes{swedish-1.3e}{1995/05/29}{Added \cs{bbl@frenchspacing}}
% \begin{macro}{\noextrasswedish}
% \changes{swedish-1.3e}{1995/05/29}{Added \cs{bbl@nonfrenchspacing}}
%    The macro |\extrasswedish| performs all the extra definitions
%    needed for the Swedish language. The macro |\noextrasswedish| is
%    used to cancel the actions of |\extrasswedish|.
%
%    For Swedish texts |\frenchspacing| should be in effect.  We make
%    sure this is the case and reset it if necessary.
%
%    \begin{macrocode}
\addto\extrasswedish{\bbl@frenchspacing}
\addto\noextrasswedish{\bbl@nonfrenchspacing}
%    \end{macrocode}
%
%    For Swedish the \texttt{"} character is made active. This is done
%    once, later on its definition may vary.
% \changes{swedish-2.0}{1996/01/24}{Added active double quote
%    character} 
%
%    \begin{macrocode}
\initiate@active@char{"}
\addto\extrasswedish{\languageshorthands{swedish}}
\addto\extrasswedish{\bbl@activate{"}}
%    \end{macrocode}
%    Don't forget to turn the shorthands off again.
% \changes{swedish-2.2b}{1999/12/17}{Deactivate shorthands ouside of
%    Swedish}
%    \begin{macrocode}
\addto\noextrasswedish{\bbl@deactivate{"}}
%    \end{macrocode}
%    The ``umlaut'' accent macro |\"| is changed to lower the
%    ``umlaut'' dots. The redefinition is done with the help of
%    |\umlautlow|.
%^^A    PROBLEM:
%^^A    What happens to ``umlaut''
%^^A    characters entered using e. g. latin input encoding? They should
%^^A    be lowered as well.
%    \begin{macrocode}
\addto\extrasswedish{\babel@save\"\umlautlow}
\addto\noextrasswedish{\umlauthigh}
%    \end{macrocode}
% \end{macro}
% \end{macro}
%
%    The code above is necessary because we need an extra active
%    character. This character is then used as indicated in
%    table~\ref{tab:swedish-quote}.
%
%    To be able to define the function of |"|, we first define a
%    couple of `support' macros.
%
%  \begin{macro}{\dq}
%    We save the original double quote character in |\dq| to keep
%    it available, the math accent |\"| can now be typed as |"|.
%    \begin{macrocode}
\begingroup \catcode`\"12
\def\x{\endgroup
  \def\@SS{\mathchar"7019 }
  \def\dq{"}}
\x
%    \end{macrocode}
%  \end{macro}
%
%    Now we can define the doublequote macros: the umlauts and {\aa}.
% \changes{swedish-2.3a}{2000/01/20}{added \cs{allowhyphens}}
%    \begin{macrocode}
\declare@shorthand{swedish}{"w}{\textormath{{\aa}\allowhyphens}{\ddot w}}
\declare@shorthand{swedish}{"a}{\textormath{\"{a}\allowhyphens}{\ddot a}}
\declare@shorthand{swedish}{"o}{\textormath{\"{o}\allowhyphens}{\ddot o}}
\declare@shorthand{swedish}{"W}{\textormath{{\AA}\allowhyphens}{\ddot W}}
\declare@shorthand{swedish}{"A}{\textormath{\"{A}\allowhyphens}{\ddot A}}
\declare@shorthand{swedish}{"O}{\textormath{\"{O}\allowhyphens}{\ddot O}}
%    \end{macrocode}
%    discretionary commands
%^^A PROBLEM:
%^^A Will these allow subsequent hyphenation? Perhaps we need to
%^^A add \cs{allowhyphens} here as well?
%    \begin{macrocode}
\declare@shorthand{swedish}{"b}{\textormath{\bbl@disc b{bb}}{b}}
\declare@shorthand{swedish}{"B}{\textormath{\bbl@disc B{BB}}{B}}
\declare@shorthand{swedish}{"d}{\textormath{\bbl@disc d{dd}}{d}}
\declare@shorthand{swedish}{"D}{\textormath{\bbl@disc D{DD}}{D}}
\declare@shorthand{swedish}{"f}{\textormath{\bbl@disc f{ff}}{f}}
\declare@shorthand{swedish}{"F}{\textormath{\bbl@disc F{FF}}{F}}
\declare@shorthand{swedish}{"g}{\textormath{\bbl@disc g{gg}}{g}}
\declare@shorthand{swedish}{"G}{\textormath{\bbl@disc G{GG}}{G}}
\declare@shorthand{swedish}{"l}{\textormath{\bbl@disc l{ll}}{l}}
\declare@shorthand{swedish}{"L}{\textormath{\bbl@disc L{LL}}{L}}
\declare@shorthand{swedish}{"m}{\textormath{\bbl@disc m{mm}}{m}}
\declare@shorthand{swedish}{"M}{\textormath{\bbl@disc M{MM}}{M}}
\declare@shorthand{swedish}{"n}{\textormath{\bbl@disc n{nn}}{n}}
\declare@shorthand{swedish}{"N}{\textormath{\bbl@disc N{NN}}{N}}
\declare@shorthand{swedish}{"p}{\textormath{\bbl@disc p{pp}}{p}}
\declare@shorthand{swedish}{"P}{\textormath{\bbl@disc P{PP}}{P}}
\declare@shorthand{swedish}{"r}{\textormath{\bbl@disc r{rr}}{r}}
\declare@shorthand{swedish}{"R}{\textormath{\bbl@disc R{RR}}{R}}
\declare@shorthand{swedish}{"s}{\textormath{\bbl@disc s{ss}}{s}}
\declare@shorthand{swedish}{"S}{\textormath{\bbl@disc S{SS}}{S}}
\declare@shorthand{swedish}{"t}{\textormath{\bbl@disc t{tt}}{t}}
\declare@shorthand{swedish}{"T}{\textormath{\bbl@disc T{TT}}{T}}
%    \end{macrocode}
%    and some additional commands:
% \def\indexeqs{=}
% \changes{swedish-2.3a}{2000/01/28}{changed definition of
%    \texttt{"\protect\indexeqs}, \cs{-} and \texttt{"~}}
%    \begin{macrocode}
\declare@shorthand{swedish}{"-}{\nobreak-\bbl@allowhyphens}
\declare@shorthand{swedish}{"|}{%
  \textormath{\nobreak\discretionary{-}{}{\kern.03em}%
              \bbl@allowhyphens}{}}
\declare@shorthand{swedish}{""}{\hskip\z@skip}
\declare@shorthand{swedish}{"~}{%
  \textormath{\leavevmode\hbox{-}\bbl@allowhyphens}{-}}
\declare@shorthand{swedish}{"=}{\hbox{-}\allowhyphens}
%    \end{macrocode}
%
%  \begin{macro}{\-}
%    Redefinition of |\-|. The new version of |\-| should indicate an
%    extra hyphenation position, while allowing other hyphenation
%    positions to be generated  automatically. The standard behaviour
%    of \TeX\ in this respect is very unfortunate for languages such
%    as Dutch, Finnish, German and Swedish, where long compound words
%    are quite normal and all one needs is a means to indicate an
%    extra hyphenation position on top of the ones that \TeX\ can
%    generate from the hyphenation patterns.
%    \begin{macrocode}
\addto\extrasswedish{\babel@save\-}
\addto\extrasswedish{\def\-{\allowhyphens
                          \discretionary{-}{}{}\allowhyphens}}
%    \end{macrocode}
%  \end{macro}
%
%    The macro |\ldf@finish| takes care of looking for a
%    configuration file, setting the main language to be switched on
%    at |\begin{document}| and resetting the category code of
%    \texttt{@} to its original value.
% \changes{swedish-2.1}{1996/11/03}{Now use \cs{ldf@finish} to wrap up}
%    \begin{macrocode}
\ldf@finish{swedish}
%</code>
%    \end{macrocode}
%
%\Finale
%%
%% \CharacterTable
%%  {Upper-case    \A\B\C\D\E\F\G\H\I\J\K\L\M\N\O\P\Q\R\S\T\U\V\W\X\Y\Z
%%   Lower-case    \a\b\c\d\e\f\g\h\i\j\k\l\m\n\o\p\q\r\s\t\u\v\w\x\y\z
%%   Digits        \0\1\2\3\4\5\6\7\8\9
%%   Exclamation   \!     Double quote  \"     Hash (number) \#
%%   Dollar        \$     Percent       \%     Ampersand     \&
%%   Acute accent  \'     Left paren    \(     Right paren   \)
%%   Asterisk      \*     Plus          \+     Comma         \,
%%   Minus         \-     Point         \.     Solidus       \/
%%   Colon         \:     Semicolon     \;     Less than     \<
%%   Equals        \=     Greater than  \>     Question mark \?
%%   Commercial at \@     Left bracket  \[     Backslash     \\
%%   Right bracket \]     Circumflex    \^     Underscore    \_
%%   Grave accent  \`     Left brace    \{     Vertical bar  \|
%%   Right brace   \}     Tilde         \~}
%%
\endinput
}
%^^A\DeclareOption{tamil}{\input{tamil.ldf}}
\DeclareOption{turkish}{%%
%% This file will generate fast loadable files and documentation
%% driver files from the doc files in this package when run through
%% LaTeX or TeX.
%%
%% Copyright 1989-2005 Johannes L. Braams and any individual authors
%% listed elsewhere in this file.  All rights reserved.
%% 
%% This file is part of the Babel system.
%% --------------------------------------
%% 
%% It may be distributed and/or modified under the
%% conditions of the LaTeX Project Public License, either version 1.3
%% of this license or (at your option) any later version.
%% The latest version of this license is in
%%   http://www.latex-project.org/lppl.txt
%% and version 1.3 or later is part of all distributions of LaTeX
%% version 2003/12/01 or later.
%% 
%% This work has the LPPL maintenance status "maintained".
%% 
%% The Current Maintainer of this work is Johannes Braams.
%% 
%% The list of all files belonging to the LaTeX base distribution is
%% given in the file `manifest.bbl. See also `legal.bbl' for additional
%% information.
%% 
%% The list of derived (unpacked) files belonging to the distribution
%% and covered by LPPL is defined by the unpacking scripts (with
%% extension .ins) which are part of the distribution.
%%
%% --------------- start of docstrip commands ------------------
%%
\def\filedate{1999/04/11}
\def\batchfile{turkish.ins}
\input docstrip.tex

{\ifx\generate\undefined
\Msg{**********************************************}
\Msg{*}
\Msg{* This installation requires docstrip}
\Msg{* version 2.3c or later.}
\Msg{*}
\Msg{* An older version of docstrip has been input}
\Msg{*}
\Msg{**********************************************}
\errhelp{Move or rename old docstrip.tex.}
\errmessage{Old docstrip in input path}
\batchmode
\csname @@end\endcsname
\fi}

\declarepreamble\mainpreamble
This is a generated file.

Copyright 1989-2005 Johannes L. Braams and any individual authors
listed elsewhere in this file.  All rights reserved.

This file was generated from file(s) of the Babel system.
---------------------------------------------------------

It may be distributed and/or modified under the
conditions of the LaTeX Project Public License, either version 1.3
of this license or (at your option) any later version.
The latest version of this license is in
  http://www.latex-project.org/lppl.txt
and version 1.3 or later is part of all distributions of LaTeX
version 2003/12/01 or later.

This work has the LPPL maintenance status "maintained".

The Current Maintainer of this work is Johannes Braams.

This file may only be distributed together with a copy of the Babel
system. You may however distribute the Babel system without
such generated files.

The list of all files belonging to the Babel distribution is
given in the file `manifest.bbl'. See also `legal.bbl for additional
information.

The list of derived (unpacked) files belonging to the distribution
and covered by LPPL is defined by the unpacking scripts (with
extension .ins) which are part of the distribution.
\endpreamble

\declarepreamble\fdpreamble
This is a generated file.

Copyright 1989-2005 Johannes L. Braams and any individual authors
listed elsewhere in this file.  All rights reserved.

This file was generated from file(s) of the Babel system.
---------------------------------------------------------

It may be distributed and/or modified under the
conditions of the LaTeX Project Public License, either version 1.3
of this license or (at your option) any later version.
The latest version of this license is in
  http://www.latex-project.org/lppl.txt
and version 1.3 or later is part of all distributions of LaTeX
version 2003/12/01 or later.

This work has the LPPL maintenance status "maintained".

The Current Maintainer of this work is Johannes Braams.

This file may only be distributed together with a copy of the Babel
system. You may however distribute the Babel system without
such generated files.

The list of all files belonging to the Babel distribution is
given in the file `manifest.bbl'. See also `legal.bbl for additional
information.

In particular, permission is granted to customize the declarations in
this file to serve the needs of your installation.

However, NO PERMISSION is granted to distribute a modified version
of this file under its original name.

\endpreamble

\keepsilent

\usedir{tex/generic/babel} 

\usepreamble\mainpreamble
\generate{\file{turkish.ldf}{\from{turkish.dtx}{code}}
          }
\usepreamble\fdpreamble

\ifToplevel{
\Msg{***********************************************************}
\Msg{*}
\Msg{* To finish the installation you have to move the following}
\Msg{* files into a directory searched by TeX:}
\Msg{*}
\Msg{* \space\space All *.def, *.fd, *.ldf, *.sty}
\Msg{*}
\Msg{* To produce the documentation run the files ending with}
\Msg{* '.dtx' and `.fdd' through LaTeX.}
\Msg{*}
\Msg{* Happy TeXing}
\Msg{***********************************************************}
}
 
\endinput
}
\DeclareOption{ukrainian}{% \iffalse meta-comment
%
% Copyright 1989-2008 Johannes L. Braams and any individual authors
% listed elsewhere in this file.  All rights reserved.
% 
% This file is part of the Babel system.
% --------------------------------------
% 
% It may be distributed and/or modified under the
% conditions of the LaTeX Project Public License, either version 1.3
% of this license or (at your option) any later version.
% The latest version of this license is in
%   http://www.latex-project.org/lppl.txt
% and version 1.3 or later is part of all distributions of LaTeX
% version 2003/12/01 or later.
% 
% This work has the LPPL maintenance status "maintained".
% 
% The Current Maintainer of this work is Johannes Braams.
% 
% The list of all files belonging to the Babel system is
% given in the file `manifest.bbl. See also `legal.bbl' for additional
% information.
% 
% The list of derived (unpacked) files belonging to the distribution
% and covered by LPPL is defined by the unpacking scripts (with
% extension .ins) which are part of the distribution.
% \fi
% \CheckSum{1472}
%
% \iffalse
%    Tell the \LaTeX\ system who we are and write an entry on the
%    transcript.
%<*dtx>
\ProvidesFile{ukraineb.dtx}
%</dtx>
%<code>\ProvidesLanguage{ukraineb}
        [2008/03/21 v1.1l Ukrainian support from the babel system]
%
%% File `ukraineb.dtx'
%% Babel package for LaTeX version 2e
%% Copyright (C) 1989 - 2008
%%           by Johannes Braams, TeXniek
%
%% ukraineb Language Definition File
%% Copyright (C) 1997 - 2008
%%           by Andrij Shvaika ashv at icmp.lviv.ua
%
%% derived from the Russianb Language Definition File
%% Copyright (C) 1995 - 2008
%%           by Olga Lapko cyrtug at mir.msk.su
%%              Johannes Braams, TeXniek
% adapted to the new T2 and X2 Cyrillic encodings
%           by Vladimir Volovich TeX at vvv.vsu.ru
%              Werner Lemberg wl at gnu.org
%
%% Please report errors to: J.L. Braams
%%                          babel at braams.xs4all.nl
%
%<*filedriver>
\documentclass{ltxdoc}
\newcommand\TeXhax{\TeX hax}
\newcommand\babel{\textsf{babel}}
\newcommand\langvar{$\langle \it lang \rangle$}
\newcommand\note[1]{}
\newcommand\Lopt[1]{\textsf{#1}}
\newcommand\file[1]{\texttt{#1}}
\newcommand\pkg[1]{\texttt{#1}}
\begin{document}
 \DocInput{ukraineb.dtx}
\end{document}
%</filedriver>
%\fi
% \GetFileInfo{ukraineb.dtx}
%
% \changes{ukraineb-1.1e}{1999/08/19}{replaced all \cs{penalty}\cs{@M}
%    with \cs{nobreak}}
%
%  \section{The Ukrainian language}
%
%    The file \file{\filename}\footnote{The file described in this
%    section has version number \fileversion.
%    This file was derived from the \file{russianb.dtx} version 1.1g.}
%    defines all the language-specific macros for the Ukrainian
%    language. It needs the file \file{cyrcod} for success documentation
%    with Ukrainian encodings (see below).
%
%    For this language the character |"| is made active. In
%    table~\ref{tab:ukrainian-quote} an overview is given of its
%    purpose. 
%
%    \begin{table}[htb]
%      \begin{center}
%      \begin{tabular}{lp{8cm}}
%       \verb="|= & disable ligature at this position.               \\
%       |"-| & an explicit hyphen sign, allowing hyphenation
%                   in the rest of the word.                         \\
%       |"---| & Cyrillic emdash in plain text.                      \\
%       |"--~| & Cyrillic emdash in compound names (surnames).       \\
%       |"--*| & Cyrillic emdash for denoting direct speech.         \\
%       |""| & like |"-|, but producing no hyphen sign
%                   (for compund words with hyphen, e.g.\ |x-""y|
%                   or some other signs  as ``disable/enable'').     \\
%       |"~| & for a compound word mark without a breakpoint.        \\
%       |"=| & for a compound word mark with a breakpoint, allowing
%              hyphenation in the composing words.                   \\
%       |",| & thinspace for initials with a breakpoint
%               in following surname.                                \\
%       |"`| & for German left double quotes
%                   (looks like ,\kern-0.08em,).                     \\
%       |"'| & for German right double quotes (looks like ``).       \\%^^A''
%       |"<| & for French left double quotes (looks like $<\!\!<$).  \\
%       |">| & for French right double quotes (looks like $>\!\!>$). \\
%      \end{tabular}
%      \caption{The extra definitions made
%               by \file{ukraineb}}\label{tab:ukrainian-quote}
%      \end{center}
%    \end{table}
%
%    The quotes in table~\ref{tab:ukrainian-quote} (see, also
%    table~\ref{tab:russian-quote}) can also be typeset by using the commands
%    in table~\ref{tab:umore-quote} (see, also table~\ref{tab:rmore-quote}).
%
%    \begin{table}[htb]
%      \begin{center}
%      \begin{tabular}{lp{8cm}}
%       |\cdash---| & Cyrillic emdash in plain text.                    \\
%       |\cdash--~| & Cyrillic emdash in compound names (surnames).     \\
%       |\cdash--*| & Cyrillic emdash for denoting direct speech.       \\
%       |\glqq| & for German left double quotes
%                    (looks like ,\kern-0.08em,).                       \\
%       |\grqq| & for German right double quotes (looks like ``).       \\%^^A''
%       |\flqq| & for French left double quotes (looks like $<\!\!<$).  \\
%       |\frqq| & for French right double quotes (looks like $>\!\!>$). \\
%       |\dq|   & the original quotes character (|"|).                  \\
%      \end{tabular}
%      \caption{More commands which produce quotes, defined
%               by \babel}\label{tab:umore-quote}
%      \end{center}
%    \end{table}
%
%    The French quotes are also available as ligatures `|<<|' and `|>>|' in
%    8-bit Cyrillic font encodings (\texttt{LCY}, \texttt{X2}, \texttt{T2*})
%    and as `|<|' and `|>|' characters in 7-bit Cyrillic font encodings
%    (\texttt{OT2} and \texttt{LWN}).
%
%    The quotation marks traditionally used in Ukrainian and Russian
%    languages were borrowed from other languages (e.g. French and German)
%    so they keep their original names.
%
% \StopEventually{}
%
%    The macro |\LdfInit| takes care of preventing that this file is loaded
%    more than once, checking the category code of the \texttt{@} sign, etc.
%
%    \begin{macrocode}
%<*code>
\LdfInit{ukrainian}{captionsukrainian}
%    \end{macrocode}
%
%    When this file is read as an option, i.e., by the |\usepackage|
%    command, \texttt{ukraineb} will be an `unknown' language, in which case
%    we have to make it known. So we check for the existence of |\l@ukrainian|
%    to see whether we have to do something here.
%
%    \begin{macrocode}
\ifx\l@ukrainian\@undefined
  \@nopatterns{Ukrainian}
  \adddialect\l@ukrainian0
\fi
%    \end{macrocode}
%
%  \begin{macro}{\latinencoding}
%
%    We need to know the encoding for text that is supposed to be which is
%    active at the end of the \babel\ package. If the \pkg{fontenc} package
%    is loaded later, then\ldots too bad!
%
%    \begin{macrocode}
\let\latinencoding\cf@encoding
%    \end{macrocode}
%
%  \end{macro}
%
%    The user may choose between different available Cyrillic
%    encodings---e.g., \texttt{X2}, \texttt{LCY}, or \texttt{LWN}.\@
%    Hopefully, \texttt{X2} will eventually replace the two latter encodings
%    (\texttt{LCY} and \texttt{LWN}).\@ If the user wants to use another
%    font encoding than the default (\texttt{T2A}), he has to load the
%    corresponding file \emph{before} \file{ukraineb.sty}. This may be done
%    in the following way:
%
%    \begin{verbatim}
%      % override the default X2 encoding used in Babel
%      \usepackage[LCY,OT1]{fontenc}
%      \usepackage[english,ukrainian]{babel}
%    \end{verbatim}
%    \unskip
%
%    Note: for the Ukrainian language, the \texttt{T2A} encoding is better than
%    \texttt{X2}, because \texttt{X2} does not contain Latin letters, and
%    users should be very careful to switch the language every time they
%    want to typeset a Latin word inside a Ukrainian phrase or vice versa.
%
%    We parse the |\cdp@list| containing the encodings known to \LaTeX\ in
%    the order they were loaded. We set the |\cyrillicencoding| to the
%    \emph{last} loaded encoding in the list of supported Cyrillic
%    encodings: \texttt{OT2}, \texttt{LWN}, \texttt{LCY}, \texttt{X2},
%    \texttt{T2C}, \texttt{T2B}, \texttt{T2A}, if any.
%
%    \begin{macrocode}
\def\reserved@a#1#2{%
   \edef\reserved@b{#1}%
   \edef\reserved@c{#2}%
   \ifx\reserved@b\reserved@c
     \let\cyrillicencoding\reserved@c
   \fi}
\def\cdp@elt#1#2#3#4{%
   \reserved@a{#1}{OT2}%
   \reserved@a{#1}{LWN}%
   \reserved@a{#1}{LCY}%
   \reserved@a{#1}{X2}%
   \reserved@a{#1}{T2C}%
   \reserved@a{#1}{T2B}%
   \reserved@a{#1}{T2A}}
\cdp@list
%    \end{macrocode}
%
%    Now, if |\cyrillicencoding| is undefined, then the user did not load
%    any of supported encodings. So, we have to set |\cyrillicencoding| to
%    some default value. We test the presence of the encoding definition
%    files in the order from less preferable to more preferable encodings.
%    We use the lowercase names (i.e., \file{lcyenc.def} instead of
%    \file{LCYenc.def}).
%
%    \begin{macrocode}
\ifx\cyrillicencoding\undefined
  \IfFileExists{ot2enc.def}{\def\cyrillicencoding{OT2}}\relax
  \IfFileExists{lwnenc.def}{\def\cyrillicencoding{LWN}}\relax
  \IfFileExists{lcyenc.def}{\def\cyrillicencoding{LCY}}\relax
  \IfFileExists{x2enc.def}{\def\cyrillicencoding{X2}}\relax
  \IfFileExists{t2cenc.def}{\def\cyrillicencoding{T2C}}\relax
  \IfFileExists{t2benc.def}{\def\cyrillicencoding{T2B}}\relax
  \IfFileExists{t2aenc.def}{\def\cyrillicencoding{T2A}}\relax
%    \end{macrocode}
%
%    If |\cyrillicencoding| is still undefined, then the user seems not to
%    have a properly installed distribution. A fatal error.
%
%    \begin{macrocode}
  \ifx\cyrillicencoding\undefined
    \PackageError{babel}%
      {No Cyrillic encoding definition files were found}%
      {Your installation is incomplete.\MessageBreak
       You need at least one of the following files:\MessageBreak
       \space\space
       x2enc.def, t2aenc.def, t2benc.def, t2cenc.def,\MessageBreak
       \space\space
       lcyenc.def, lwnenc.def, ot2enc.def.}%
  \else
%    \end{macrocode}
%
%    We avoid |\usepackage[\cyrillicencoding]{fontenc}| because we don't
%    want to force the switch of |\encodingdefault|.
%
%    \begin{macrocode}
    \lowercase
      \expandafter{\expandafter\input\cyrillicencoding enc.def\relax}%
  \fi
\fi
%    \end{macrocode}
%
%    \begin{verbatim}
%      \PackageInfo{babel}
%        {Using `\cyrillicencoding' as a default Cyrillic encoding}%
%    \end{verbatim}
%    \unskip
%
%    \begin{macrocode}
\DeclareRobustCommand{\Ukrainian}{%
  \fontencoding\cyrillicencoding\selectfont
  \let\encodingdefault\cyrillicencoding
  \expandafter\set@hyphenmins\ukrainianhyphenmins
  \language\l@ukrainian}%
\DeclareRobustCommand{\English}{%
  \fontencoding\latinencoding\selectfont
  \let\encodingdefault\latinencoding
  \expandafter\set@hyphenmins\englishhyphenmins
  \language\l@english}%
\let\Ukr\Ukrainian
\let\Eng\English
\let\cyrillictext\Ukrainian
\let\cyr\Ukrainian
%    \end{macrocode}
%
%    Since the \texttt{X2} encoding does not contain Latin letters, we
%    should make some redefinitions of \LaTeX\ macros which implicitly
%    produce Latin letters.
%
%    \begin{macrocode}
\expandafter\ifx\csname T@X2\endcsname\relax\else
%    \end{macrocode}
%
%    We put |\latinencoding| in braces to avoid problems with
%    |\@alph| inside minipages (e.g., footnotes inside minipages) where
%    |\@alph| is expanded and we get for example `|\fontencoding OT1|'
%    (|\fontencoding| is robust).
%
%    \begin{macrocode}
  \def\@alph#1{{\fontencoding{\latinencoding}\selectfont
    \ifcase#1\or
      a\or b\or c\or d\or e\or f\or g\or h\or
      i\or j\or k\or l\or m\or n\or o\or p\or
      q\or r\or s\or t\or u\or v\or w\or x\or
      y\or z\else\@ctrerr\fi}}%
  \def\@Alph#1{{\fontencoding{\latinencoding}\selectfont
    \ifcase#1\or
      A\or B\or C\or D\or E\or F\or G\or H\or
      I\or J\or K\or L\or M\or N\or O\or P\or
      Q\or R\or S\or T\or U\or V\or W\or X\or
      Y\or Z\else\@ctrerr\fi}}%
%    \end{macrocode}
%
%    Unfortunately, the commands |\AA| and |\aa| are not encoding dependent
%    in \LaTeX\ (unlike e.g., |\oe| or |\DH|). They are defined as |\r{A}| and
%    |\r{a}|. This leads to unpredictable results when the font encoding
%    does not contain the Latin letters `A' and `a' (like \texttt{X2}).
%
%    \begin{macrocode}
  \DeclareTextSymbolDefault{\AA}{OT1}
  \DeclareTextSymbolDefault{\aa}{OT1}
  \DeclareTextCommand{\aa}{OT1}{\r a}
  \DeclareTextCommand{\AA}{OT1}{\r A}
\fi
%    \end{macrocode}
%
%    The following block redefines the character class of uppercase Greek
%    letters and some accents, if it is equal to 7 (variable family), to
%    avoid incorrect results if the font encoding in some math family does
%    not contain these characters in places of OT1 encoding. The code was
%    taken from |amsmath.dtx|. See comments and further explanation there.
%
% \changes{ukraineb-1.1i}{2001/02/21}{As this code generates a
%    textfont 7 error it is commented out for now.}
%    \begin{macrocode}
% \begingroup\catcode`\"=12
% % uppercase greek letters:
% \def\@tempa#1{\expandafter\@tempb\meaning#1\relax\relax\relax\relax
%   "0000\@nil#1}
% \def\@tempb#1"#2#3#4#5#6\@nil#7{%
%   \ifnum"#2=7 \count@"1#3#4#5\relax
%     \ifnum\count@<"1000 \else \global\mathchardef#7="0#3#4#5\relax \fi
%   \fi}
% \@tempa\Gamma\@tempa\Delta\@tempa\Theta\@tempa\Lambda\@tempa\Xi
% \@tempa\Pi\@tempa\Sigma\@tempa\Upsilon\@tempa\Phi\@tempa\Psi
% \@tempa\Omega
% % some accents:
% \def\@tempa#1#2\@nil{\def\@tempc{#1}}\def\@tempb{\mathaccent}
% \expandafter\@tempa\hat\relax\relax\@nil
% \ifx\@tempb\@tempc
%   \def\@tempa#1\@nil{#1}%
%   \def\@tempb#1{\afterassignment\@tempa\mathchardef\@tempc=}%
%   \def\do#1"#2{}
%   \def\@tempd#1{\expandafter\@tempb#1\@nil
%     \ifnum\@tempc>"FFF
%       \xdef#1{\mathaccent"\expandafter\do\meaning\@tempc\space}%
%     \fi}
%   \@tempd\hat\@tempd\check\@tempd\tilde\@tempd\acute\@tempd\grave
%   \@tempd\dot\@tempd\ddot\@tempd\breve\@tempd\bar
% \fi
% \endgroup
%    \end{macrocode}
%
%    The user must use the \pkg{inputenc} package when any 8-bit Cyrillic
%    font encoding is used, selecting one of the Cyrillic input encodings.
%    We do not assume any default input encoding, so the user should
%    explicitly call the \pkg{inputenc} package by |\usepackage{inputenc}|.
%    We also removed |\AtBeginDocument|, so \pkg{inputenc} should be used
%    before \babel.
%
% \changes{ukraineb-1.1f}{1999/08/27}{Made not using inputenc a
%    warning instead of an error}
%    \begin{macrocode}
\@ifpackageloaded{inputenc}{}{%
  \def\reserved@a{LWN}%
  \ifx\reserved@a\cyrillicencoding\else
    \def\reserved@a{OT2}%
    \ifx\reserved@a\cyrillicencoding\else
      \PackageWarning{babel}%
        {No input encoding specified for Ukrainian language}
  \fi\fi}
%    \end{macrocode}
%
%    Now we define two commands that offer the possibility to switch between
%    Cyrillic and Roman encodings.
%
%  \begin{macro}{\cyrillictext}
%  \begin{macro}{\latintext}
%
%    The command |\cyrillictext| will switch from Latin font encoding to the
%    Cyrillic font encoding, the command |\latintext| switches back. This
%    assumes that the `normal' font encoding is a Latin one. These commands
%    are \emph{declarations}, for shorter peaces of text the commands
%    |\textlatin| and |\textcyrillic| can be used.
%
% \changes{ukraineb-1.1j}{2003/10/12}{\cs{latintext} is already
%    defined by the core of \babel}
%    \begin{macrocode}
%\DeclareRobustCommand{\latintext}{%
%  \fontencoding{\latinencoding}\selectfont
%  \def\encodingdefault{\latinencoding}}
\let\lat\latintext
%    \end{macrocode}
%
%  \end{macro}
%  \end{macro}
%
%  \begin{macro}{\textcyrillic}
%  \begin{macro}{\textlatin}
%
%    These commands take an argument which is then typeset using the
%    requested font encoding.
% \changes{ukraineb-1.1j}{2003/10/12}{\cs{latintext} is already
%    defined by the core of \babel}
%    \begin{macrocode}
\DeclareTextFontCommand{\textcyrillic}{\cyrillictext}
%\DeclareTextFontCommand{\textlatin}{\latintext}
%    \end{macrocode}
%
%  \end{macro}
%  \end{macro}
%
%    We make the \TeX
%    \begin{macrocode}
%\ifx\ltxTeX\undefined\let\ltxTeX\TeX\fi
%\ProvideTextCommandDefault{\TeX}{\textlatin{\ltxTeX}}
%    \end{macrocode}
%    and \LaTeX\ logos encoding independent.
%    \begin{macrocode}
%\ifx\ltxLaTeX\undefined\let\ltxLaTeX\LaTeX\fi
%\ProvideTextCommandDefault{\LaTeX}{\textlatin{\ltxLaTeX}}
%    \end{macrocode}
%
%    The next step consists of defining commands to switch to (and
%    from) the Ukrainian language.
%
% \begin{macro}{\captionsukrainian}
%
%    The macro |\captionsukrainian| defines all strings used in the four
%    standard document classes provided with \LaTeX. The two commands |\cyr|
%    and |\lat| activate Cyrillic resp.\ Latin encoding.
% \changes{ukraineb-1.1d}{1999/04/03}{replace \cs{CYRUKRI} with
%    \cs{CYRII} in \cs{authorname} }
% \changes{ukraineb-1.1g}{2000/09/20}{Added \cs{glossaryname}}
% \changes{ukraineb-1.1h}{2001/02/13}{Added translation for
%    `Glossary'}
%    \begin{macrocode}
\addto\captionsukrainian{%
  \def\prefacename{{\cyr\CYRV\cyrs\cyrt\cyru\cyrp}}%
% \def\prefacename{{\cyr\CYRP\cyre\cyrr\cyre\cyrd\cyrm\cyro\cyrv\cyra}}%
  \def\refname{%
    {\cyr\CYRL\cyrii\cyrt\cyre\cyrr\cyra\cyrt\cyru\cyrr\cyra}}%
%  \def\refname{%
%    {\cyr\CYRP\cyre\cyrr\cyre\cyrl\cyrii\cyrk
%         \ \cyrp\cyro\cyrs\cyri\cyrl\cyra\cyrn\cyrsftsn}}%
  \def\abstractname{%
    {\cyr\CYRA\cyrn\cyro\cyrt\cyra\cyrc\cyrii\cyrya}}%
%  \def\abstractname{{\cyr\CYRR\cyre\cyrf\cyre\cyrr\cyra\cyrt}}%
  \def\bibname{%
    {\cyr\CYRB\cyrii\cyrb\cyrl\cyrii\cyro\cyrgup\cyrr\cyra\cyrf\cyrii\cyrya}}%
% \def\bibname{{\cyr\CYRL\cyrii\cyrt\cyre\cyrr\cyra\cyrt\cyru\cyrr\cyra}}%
  \def\chaptername{{\cyr\CYRR\cyro\cyrz\cyrd\cyrii\cyrl}}%
%  \def\chaptername{{\cyr\CYRG\cyrl\cyra\cyrv\cyra}}%
  \def\appendixname{{\cyr\CYRD\cyro\cyrd\cyra\cyrt\cyro\cyrk}}%
  \def\contentsname{{\cyr\CYRZ\cyrm\cyrii\cyrs\cyrt}}%
  \def\listfigurename{{\cyr\CYRP\cyre\cyrr\cyre\cyrl\cyrii\cyrk
         \ \cyrii\cyrl\cyryu\cyrs\cyrt\cyrr\cyra\cyrc\cyrii\cyrishrt}}%
  \def\listtablename{{\cyr\CYRP\cyre\cyrr\cyre\cyrl\cyrii\cyrk
         \ \cyrt\cyra\cyrb\cyrl\cyri\cyrc\cyrsftsn}}%
  \def\indexname{{\cyr\CYRP\cyro\cyrk\cyra\cyrzh\cyrch\cyri\cyrk}}%
  \def\authorname{{\cyr\CYRII\cyrm\cyre\cyrn\cyrn\cyri\cyrishrt
         \ \cyrp\cyro\cyrk\cyra\cyrzh\cyrch\cyri\cyrk}}%
  \def\figurename{{\cyr\CYRR\cyri\cyrs.}}%
%  \def\figurename{\cyr\CYRR\cyri\cyrs\cyru\cyrn\cyro\cyrk}}%
  \def\tablename{{\cyr\CYRT\cyra\cyrb\cyrl.}}%
%  \def\tablename{\cyr\CYRT\cyra\cyrb\cyrl\cyri\cyrc\cyrya}}%
  \def\partname{{\cyr\CYRCH\cyra\cyrs\cyrt\cyri\cyrn\cyra}}%
  \def\enclname{{\cyr\cyrv\cyrk\cyrl\cyra\cyrd\cyrk\cyra}}%
  \def\ccname{{\cyr\cyrk\cyro\cyrp\cyrii\cyrya}}%
  \def\headtoname{{\cyr\CYRD\cyro}}%
  \def\pagename{{\cyr\cyrs.}}%
%  \def\pagename{{\cyr\cyrs\cyrt\cyro\cyrr\cyrii\cyrn\cyrk\cyra}}%
  \def\seename{{\cyr\cyrd\cyri\cyrv.}}%
  \def\alsoname{{\cyr\cyrd\cyri\cyrv.\ \cyrt\cyra\cyrk\cyro\cyrzh}}
  \def\proofname{{\cyr\CYRD\cyro\cyrv\cyre\cyrd\cyre\cyrn\cyrn\cyrya}}%
  \def\glossaryname{{\cyr\CYRS\cyrl\cyro\cyrv\cyrn\cyri\cyrk\ %
                   \cyrt\cyre\cyrr\cyrm\cyrii\cyrn\cyrii\cyrv}}%
  }
%    \end{macrocode}
%
% \end{macro}
%
% \begin{macro}{\dateukrainian}
%
%    The macro |\dateukrainian| redefines the command |\today| to produce
%    Ukrainian dates.
%
%    \begin{macrocode}
\def\dateukrainian{%
  \def\today{\number\day~\ifcase\month\or
    \cyrs\cyrii\cyrch\cyrn\cyrya\or
    \cyrl\cyryu\cyrt\cyro\cyrg\cyro\or
    \cyrb\cyre\cyrr\cyre\cyrz\cyrn\cyrya\or
    \cyrk\cyrv\cyrii\cyrt\cyrn\cyrya\or
    \cyrt\cyrr\cyra\cyrv\cyrn\cyrya\or
    \cyrch\cyre\cyrr\cyrv\cyrn\cyrya\or
    \cyrl\cyri\cyrp\cyrn\cyrya\or
    \cyrs\cyre\cyrr\cyrp\cyrn\cyrya\or
    \cyrv\cyre\cyrr\cyre\cyrs\cyrn\cyrya\or
    \cyrzh\cyro\cyrv\cyrt\cyrn\cyrya\or
    \cyrl\cyri\cyrs\cyrt\cyro\cyrp\cyra\cyrd\cyra\or
    \cyrg\cyrr\cyru\cyrd\cyrn\cyrya\fi
    \space\number\year~\cyrr.}}
%    \end{macrocode}
%
% \end{macro}
%
% \begin{macro}{\extrasukrainian}
%
%    The macro |\extrasukrainian| will perform all the extra definitions
%    needed for the Ukrainian language. The macro |\noextrasukrainian|
%    is used to cancel the actions of |\extrasukrainian|.
%
%    The first action we define is to switch on the selected Cyrillic
%    encoding whenever we enter `ukrainian'.
%
%    \begin{macrocode}
\addto\extrasukrainian{\cyrillictext}
%    \end{macrocode}
%
%    When the encoding definition file was processed by \LaTeX\ the current
%    font encoding is stored in |\latinencoding|, assuming that \LaTeX\ uses
%    \texttt{T1} or \texttt{OT1} as default. Therefore we switch back to
%    |\latinencoding| whenever the Ukrainian language is no longer `active'.
%
%    \begin{macrocode}
\addto\noextrasukrainian{\latintext}
%    \end{macrocode}
%
%    Next we must allow hyphenation in the Ukrainian words with apostrophe
%    whenever we enter `ukrainian'. This solution was proposed by
%    Vladimir Volovich <vvv@vvv.vsu.ru>
%
%    \begin{macrocode}
\addto\extrasukrainian{\lccode`\'=`\'}
\addto\noextrasukrainian{\lccode`\'=0}
%    \end{macrocode}
%
%  \begin{macro}{\verbatim@font}
%
%    In order to get both Latin and Cyrillic letters in verbatim text we
%    need to change the definition of an internal \LaTeX\ command somewhat:
%
%    \begin{macrocode}
%\def\verbatim@font{%
%  \let\encodingdefault\latinencoding
%  \normalfont\ttfamily
%  \expandafter\def\csname\cyrillicencoding-cmd\endcsname##1##2{%
%    \ifx\protect\@typeset@protect
%      \begingroup\UseTextSymbol\cyrillicencoding##1\endgroup
%    \else\noexpand##1\fi}}
%    \end{macrocode}
%
%  \end{macro}
%
%    The category code of the characters `\texttt{:}', `\texttt{;}',
%    `\texttt{!}', and `\texttt{?}' is made |\active| to insert a little
%    white space.
%
%    For Ukrainian (as well as for Russian and German) the \texttt{"}
%    character also is made active.
%
%    Note: It is \emph{very} questionable whether the Russian typesetting
%    tradition requires additional spacing before those punctuation signs.
%    Therefore, we make the corresponding code optional. If you need it,
%    then define the \texttt{frenchpunct} docstrip option in
%    \file{babel.ins}.
%
%    Borrowed from french.
%    Some users dislike automatic insertion of a space before
%    `double punctuation', and prefer to decide themselves whether a
%    space should be added or not; so a hook |\NoAutoSpaceBeforeFDP|
%    is provided: if this command is added (in file |ukraineb.cfg|, or
%    anywhere in a document) |ukraineb| will respect your typing, and
%    introduce a suitable space before `double punctuation' \emph{if
%    and only if} a space is typed in the source file before those
%    signs.
%
%    The command |\AutoSpaceBeforeFDP| switches back to the
%    default behavior of |ukraineb|.
%
%    \begin{macrocode}
%<*frenchpunct>
\initiate@active@char{:}
\initiate@active@char{;}
%</frenchpunct>
%<*frenchpunct|spanishligs>
\initiate@active@char{!}
\initiate@active@char{?}
%</frenchpunct|spanishligs>
\initiate@active@char{"}
%    \end{macrocode}
%
%    The code above is necessary because we need extra active characters.
%    The character |"| is used as indicated in
%    table~\ref{tab:ukrainian-quote}.
%
%    We specify that the Ukrainian group of shorthands should be used.
%
%    \begin{macrocode}
\addto\extrasukrainian{\languageshorthands{ukrainian}}
%    \end{macrocode}
%
%    These characters are `turned on' once, later their definition may
%    vary.
%
%    \begin{macrocode}
\addto\extrasukrainian{%
%<frenchpunct>  \bbl@activate{:}\bbl@activate{;}%
%<frenchpunct|spanishligs>  \bbl@activate{!}\bbl@activate{?}%
  \bbl@activate{"}}
\addto\noextrasukrainian{%
%<frenchpunct>  \bbl@deactivate{:}\bbl@deactivate{;}%
%<frenchpunct|spanishligs>  \bbl@deactivate{!}\bbl@deactivate{?}%
  \bbl@deactivate{"}}
%    \end{macrocode}
%
%   The \texttt{X2} and \texttt{T2*} encodings do not contain
%   |spanish_shriek| and |spanish_query| symbols; as a consequence, the
%   ligatures `|?`|' and `|!`|' do not work with them (these characters are
%   useless for Cyrillic texts anyway). But we define the shorthands to
%   emulate these ligatures (optionally).
%
%   We do not use |\latinencoding| here (but instead explicitly use
%   \texttt{OT1}) because the user may choose \texttt{T2A} to be the primary
%   encoding, but it does not contain these characters.
%
%    \begin{macrocode}
%<*spanishligs>
\declare@shorthand{ukrainian}{?`}{\UseTextSymbol{OT1}\textquestiondown}
\declare@shorthand{ukrainian}{!`}{\UseTextSymbol{OT1}\textexclamdown}
%</spanishligs>
%    \end{macrocode}
%
% \begin{macro}{\ukrainian@sh@;@}
% \begin{macro}{\ukrainian@sh@:@}
% \begin{macro}{\ukrainian@sh@!@}
% \begin{macro}{\ukrainian@sh@?@}
%
%    We have to reduce the amount of white space before \texttt{;},
%    \texttt{:} and \texttt{!}. This should only happen in horizontal mode,
%    hence the test with |\ifhmode|.
%
%    \begin{macrocode}
%<*frenchpunct>
\declare@shorthand{ukrainian}{;}{%
  \ifhmode
%    \end{macrocode}
%
%    In horizontal mode we check for the presence of a `space', `unskip' if
%    it exists and place a |0.1em| kerning.
%
%    \begin{macrocode}
    \ifdim\lastskip>\z@
      \unskip\nobreak\kern.1em
    \else
%    \end{macrocode}
%    If no space has been typed, we add |\FDP@thinspace|
%    which will be
%    defined, up to the user's wishes, as an automatic added
%    thinspace, or as |\@empty|.
%
%    \begin{macrocode}
        \FDP@thinspace
    \fi
  \fi
%    \end{macrocode}
%
%    Now we can insert a `|;|' character.
%
%    \begin{macrocode}
  \string;}
%    \end{macrocode}
%
%    The other definitions are very similar.
%
%    \begin{macrocode}
\declare@shorthand{ukrainian}{:}{%
  \ifhmode
    \ifdim\lastskip>\z@
      \unskip\nobreak\kern.1em
    \else
        \FDP@thinspace
    \fi
  \fi
  \string:}
%    \end{macrocode}
%
%    \begin{macrocode}
\declare@shorthand{ukrainian}{!}{%
  \ifhmode
    \ifdim\lastskip>\z@
      \unskip\nobreak\kern.1em
    \else
        \FDP@thinspace
    \fi
  \fi
  \string!}
%    \end{macrocode}
%
%    \begin{macrocode}
\declare@shorthand{ukrainian}{?}{%
  \ifhmode
    \ifdim\lastskip>\z@
      \unskip\nobreak\kern.1em
    \else
        \FDP@thinspace
    \fi
  \fi
  \string?}
%    \end{macrocode}
%
% \end{macro}
% \end{macro}
% \end{macro}
% \end{macro}
%
%
%  \begin{macro}{\AutoSpaceBeforeFDP}
%  \begin{macro}{\NoAutoSpaceBeforeFDP}
%  \begin{macro}{\FDP@thinspace}
%    |\FDP@thinspace| is defined as unbreakable
%    spaces if |\AutoSpaceBeforeFDP| is activated or as |\@empty| if
%    |\NoAutoSpaceBeforeFDP| is in use.
%    The default is |\AutoSpaceBeforeFDP|.
%    \begin{macrocode}
\def\AutoSpaceBeforeFDP{%
      \def\FDP@thinspace{\nobreak\kern.1em}}
\def\NoAutoSpaceBeforeFDP{\let\FDP@thinspace\@empty}
\AutoSpaceBeforeFDP
%    \end{macrocode}
%  \end{macro}
%  \end{macro}
%  \end{macro}
%
%  \begin{macro}{\FDPon}
%  \begin{macro}{\FDPoff}
%
%     The next macros allow to switch on/off activeness of double
%     punctuation signs.
%
%    \begin{macrocode}
\def\FDPon{\bbl@activate{:}%
        \bbl@activate{;}%
        \bbl@activate{?}%
        \bbl@activate{!}}
\def\FDPoff{\bbl@deactivate{:}%
        \bbl@deactivate{;}%
        \bbl@deactivate{?}%
        \bbl@deactivate{!}}
%    \end{macrocode}
%  \end{macro}
%  \end{macro}
%
%  \begin{macro}{\system@sh@:@}
%  \begin{macro}{\system@sh@!@}
%  \begin{macro}{\system@sh@?@}
%  \begin{macro}{\system@sh@;@}
%
%    When the active characters appear in an environment where their
%    Ukrainian behaviour is not wanted they should give an `expected'
%    result. Therefore we define shorthands at system level as well.
%
%    \begin{macrocode}
\declare@shorthand{system}{:}{\string:}
\declare@shorthand{system}{;}{\string;}
%</frenchpunct>
%<*frenchpunct&!spanishligs>
\declare@shorthand{system}{!}{\string!}
\declare@shorthand{system}{?}{\string?}
%</frenchpunct&!spanishligs>
%    \end{macrocode}
%
%  \end{macro}
%  \end{macro}
%  \end{macro}
%  \end{macro}
%
%    To be able to define the function of `|"|', we first define a couple of
%    `support' macros.
%
%  \begin{macro}{\dq}
%
%    We save the original double quote character in |\dq| to keep it
%    available, the math accent |\"| can now be typed as `|"|'.
%
%    \begin{macrocode}
\begingroup \catcode`\"12
\def\reserved@a{\endgroup
  \def\@SS{\mathchar"7019 }
  \def\dq{"}}
\reserved@a
%    \end{macrocode}
%
%  \end{macro}
%
%    Now we can define the doublequote macros: german and french quotes.
%    We use definitions of these quotes made in babel.sty.
%    The french quotes are contained in the \texttt{T2*} encodings.
%
%    \begin{macrocode}
\declare@shorthand{ukrainian}{"`}{\glqq}
\declare@shorthand{ukrainian}{"'}{\grqq}
\declare@shorthand{ukrainian}{"<}{\flqq}
\declare@shorthand{ukrainian}{">}{\frqq}
%    \end{macrocode}
%
%    Some additional commands:
%
%    \begin{macrocode}
\declare@shorthand{ukrainian}{""}{\hskip\z@skip}
\declare@shorthand{ukrainian}{"~}{\textormath{\leavevmode\hbox{-}}{-}}
\declare@shorthand{ukrainian}{"=}{\nobreak-\hskip\z@skip}
\declare@shorthand{ukrainian}{"|}{%
  \textormath{\nobreak\discretionary{-}{}{\kern.03em}%
              \allowhyphens}{}}
%    \end{macrocode}
%
%    The next two macros for |"-| and |"---| are somewhat different.
%    We must check whether the second token is a hyphen character:
%
%    \begin{macrocode}
\declare@shorthand{ukrainian}{"-}{%
%    \end{macrocode}
%
%    If the next token is `|-|', we typeset an emdash, otherwise a hyphen
%    sign:
%
%    \begin{macrocode}
  \def\ukrainian@sh@tmp{%
    \if\ukrainian@sh@next-\expandafter\ukrainian@sh@emdash
    \else\expandafter\ukrainian@sh@hyphen\fi
  }%
%    \end{macrocode}
%
%    \TeX\ looks for the next token after the first `|-|': the meaning of
%    this token is written to |\ukrainian@sh@next| and |\ukrainian@sh@tmp| is
%    called.
%
%    \begin{macrocode}
  \futurelet\ukrainian@sh@next\ukrainian@sh@tmp}
%    \end{macrocode}
%
%    Here are the definitions of hyphen and emdash. First the hyphen:
%
%    \begin{macrocode}
\def\ukrainian@sh@hyphen{%
  \nobreak\-\bbl@allowhyphens}
%    \end{macrocode}
%
%    For the emdash definition, there are the two parameters: we must `eat'
%    two last hyphen signs of our emdash\dots :
%    \begin{macrocode}
\def\ukrainian@sh@emdash#1#2{\cdash-#1#2}
%    \end{macrocode}
%  \begin{macro}{\cdash}
%    \dots\ these two parameters are useful for another macro:
%    |\cdash|:
%    \begin{macrocode}
%\ifx\cdash\undefined % should be defined earlier
\def\cdash#1#2#3{\def\tempx@{#3}%
\def\tempa@{-}\def\tempb@{~}\def\tempc@{*}%
 \ifx\tempx@\tempa@\@Acdash\else
  \ifx\tempx@\tempb@\@Bcdash\else
   \ifx\tempx@\tempc@\@Ccdash\else
    \errmessage{Wrong usage of cdash}\fi\fi\fi}
%    \end{macrocode}
%   second parameter (or third for |\cdash|) shows what kind of emdash
%   to create in next step
%      \begin{center}
%      \begin{tabular}{@{}p{.1\hsize}@{}p{.9\hsize}@{}}
%       |"---| & ordinary (plain) Cyrillic emdash inside text:
%       an unbreakable thinspace will be inserted before only in case of
%       a \textit{space} before the dash (it is necessary for dashes after
%       display maths formulae: there could be lists, enumerations etc.\
%       started with ``--- where $a$ is ...'' i.e., the dash starts a line).
%       (Firstly there were planned rather soft rules for user: he may put
%       a space before the dash or not.  But it is difficult to place this
%       thinspace automatically, i.e., by checking modes because after
%       display formulae \TeX{} uses horizontal mode.  Maybe there is a
%       misunderstanding?  Maybe there is another way?)  After a dash
%       a breakable thinspace is always placed; \\
%   \end{tabular}
%   \end{center}
%    \begin{macrocode}
% What is more grammatically: .2em or .2\fontdimen6\font ?
\def\@Acdash{\ifdim\lastskip>\z@\unskip\nobreak\hskip.2em\fi
  \cyrdash\hskip.2em\ignorespaces}%
%    \end{macrocode}
%      \begin{center}
%      \begin{tabular}{@{}p{.1\hsize}@{}p{.9\hsize}@{}}
%       |"--~| & emdash in compound names or surnames
%       (like Mendeleev--Klapeiron); this dash has no space characters
%       around; after the dash some space is added
%       |\exhyphenalty| \\
%   \end{tabular}
%   \end{center}
%    \begin{macrocode}
\def\@Bcdash{\leavevmode\ifdim\lastskip>\z@\unskip\fi
 \nobreak\cyrdash\penalty\exhyphenpenalty\hskip\z@skip\ignorespaces}%
%    \end{macrocode}
%      \begin{center}
%      \begin{tabular}{@{}p{.1\hsize}@{}p{.9\hsize}@{}}
%       |"--*| & for denoting direct speech (a space like |\enskip|
%       must follow the emdash); \\
%   \end{tabular}
%   \end{center}
%    \begin{macrocode}
\def\@Ccdash{\leavevmode
 \nobreak\cyrdash\nobreak\hskip.35em\ignorespaces}%
%\fi
%    \end{macrocode}
%  \end{macro}
%
%  \begin{macro}{\cyrdash}
%   Finally the macro for ``body'' of the Cyrillic emdash.
%   The |\cyrdash| macro will be defined in case this macro hasn't been
%   defined in a fontenc file.  For T2* fonts, cyrdash will be placed in
%   the code of the English emdash thus it uses ligature |---|.
%    \begin{macrocode}
% Is there an IF necessary?
\ifx\cyrdash\undefined
  \def\cyrdash{\hbox to.8em{--\hss--}}
\fi
%    \end{macrocode}
%  \end{macro}
%
%    Here a really new macro---to place thinspace between initials.
%    This macro used instead of |\,| allows hyphenation in the following
%    surname.
%
%    \begin{macrocode}
\declare@shorthand{ukrainian}{",}{\nobreak\hskip.2em\ignorespaces}
%    \end{macrocode}
%
%  \begin{macro}{\mdqon}
%  \begin{macro}{\mdqoff}
%    All that's left to do now is to  define a couple of commands
%    for |"|.
%    \begin{macrocode}
\def\mdqon{\bbl@activate{"}}
\def\mdqoff{\bbl@deactivate{"}}
%    \end{macrocode}
%  \end{macro}
%  \end{macro}
%
%    The Ukrainian hyphenation patterns can be used with |\lefthyphenmin|
%    and |\righthyphenmin| set to~2.
% \changes{ukraineb-1.1g}{2000/09/22}{Now use \cs{providehyphenmins} to
%    provide a default value}
%    \begin{macrocode}
\providehyphenmins{\CurrentOption}{\tw@\tw@}
% temporary hack:
\ifx\englishhyphenmins\undefined
  \def\englishhyphenmins{\tw@\thr@@}
\fi
%    \end{macrocode}
%
%    Now the action |\extrasukrainian| has to execute is to make sure that the
%    command |\frenchspacing| is in effect. If this is not the case the
%    execution of |\noextrasukrainian| will switch it off again.
%
%    \begin{macrocode}
\addto\extrasukrainian{\bbl@frenchspacing}
\addto\noextrasukrainian{\bbl@nonfrenchspacing}
%    \end{macrocode}
%
% \end{macro}
%
%    Next we add a new enumeration style for Ukrainian manuscripts with
%    Cyrillic letters, and later on we define some math operator names in
%    accordance with Ukrainian and Russian typesetting traditions.
%
%  \begin{macro}{\Asbuk}
%
%    We begin by defining |\Asbuk| which works like |\Alph|, but produces
%    (uppercase) Cyrillic letters intead of Latin ones. The letters CYRGUP,
%    and SFTSN are skipped, as usual for such enumeration.
%
%    \begin{macrocode}
\def\Asbuk#1{\expandafter\@Asbuk\csname c@#1\endcsname}
\def\@Asbuk#1{\ifcase#1\or
  \CYRA\or\CYRB\or\CYRV\or\CYRG\or\CYRD\or\CYRE\or\CYRIE\or
  \CYRZH\or\CYRZ\or\CYRI\or\CYRII\or\CYRYI\or\CYRISHRT\or
  \CYRK\or\CYRL\or\CYRM\or\CYRN\or\CYRO\or\CYRP\or\CYRR\or
  \CYRS\or\CYRT\or\CYRU\or\CYRF\or\CYRH\or\CYRC\or\CYRCH\or
  \CYRSH\or\CYRSHCH\or\CYRYU\or\CYRYA\else\@ctrerr\fi}
%    \end{macrocode}
%
%  \end{macro}
%
%  \begin{macro}{\asbuk}
%
%    The macro |\asbuk| is similar to |\alph|; it produces lowercase
%    Ukrainian letters.
%
%    \begin{macrocode}
\def\asbuk#1{\expandafter\@asbuk\csname c@#1\endcsname}
\def\@asbuk#1{\ifcase#1\or
  \cyra\or\cyrb\or\cyrv\or\cyrg\or\cyrd\or\cyre\or\cyrie\or
  \cyrzh\or\cyrz\or\cyri\or\cyrii\or\cyryi\or\cyrishrt\or
  \cyrk\or\cyrl\or\cyrm\or\cyrn\or\cyro\or\cyrp\or\cyrr\or
  \cyrs\or\cyrt\or\cyru\or\cyrf\or\cyrh\or\cyrc\or\cyrch\or
  \cyrsh\or\cyrshch\or\cyryu\or\cyrya\else\@ctrerr\fi}
%    \end{macrocode}
%
%  \end{macro}
%
%    Set up default Cyrillic math alphabets. The math groups for cyrillic
%    letters are defined in the encoding definition files. First, declare
%    a new alphabet for symbols, |\cyrmathrm|, based on the symbol font
%    for Cyrillic letters defined in the encoding definition file. Note,
%    that by default Cyrillic letters are taken from upright font in math
%    mode (unlike Latin letters).
%    \begin{macrocode}
%\RequirePackage{textmath}
\@ifundefined{sym\cyrillicencoding letters}{}{%
\SetSymbolFont{\cyrillicencoding letters}{bold}\cyrillicencoding
  \rmdefault\bfdefault\updefault
\DeclareSymbolFontAlphabet\cyrmathrm{\cyrillicencoding letters}
%    \end{macrocode}
%    And we need a few commands to be able to switch to different variants.
%    \begin{macrocode}
\DeclareMathAlphabet\cyrmathbf\cyrillicencoding
  \rmdefault\bfdefault\updefault
\DeclareMathAlphabet\cyrmathsf\cyrillicencoding
  \sfdefault\mddefault\updefault
\DeclareMathAlphabet\cyrmathit\cyrillicencoding
  \rmdefault\mddefault\itdefault
\DeclareMathAlphabet\cyrmathtt\cyrillicencoding
  \ttdefault\mddefault\updefault
%
\SetMathAlphabet\cyrmathsf{bold}\cyrillicencoding
  \sfdefault\bfdefault\updefault
\SetMathAlphabet\cyrmathit{bold}\cyrillicencoding
  \rmdefault\bfdefault\itdefault
}
%    \end{macrocode}
%
%    Some math functions in Ukrainian and Russian math books have other
%    names: e.g., \texttt{sinh} in Russian is written as \texttt{sh} etc.
%    So we define a number of new math operators.
%
%    |\sinh|:
%    \begin{macrocode}
\def\sh{\mathop{\operator@font sh}\nolimits}
%    \end{macrocode}
%    |\cosh|:
%    \begin{macrocode}
\def\ch{\mathop{\operator@font ch}\nolimits}
%    \end{macrocode}
%    |\tan|:
%    \begin{macrocode}
\def\tg{\mathop{\operator@font tg}\nolimits}
%    \end{macrocode}
%    |\arctan|:
%    \begin{macrocode}
\def\arctg{\mathop{\operator@font arctg}\nolimits}
%    \end{macrocode}
%    arcctg:
%    \begin{macrocode}
\def\arcctg{\mathop{\operator@font arcctg}\nolimits}
%    \end{macrocode}
%    The following macro conflicts with |\th| defined in Latin~1 encoding:
%
%    |\tanh|:
% \changes{ukraineb-1.1k}{2004/05/21}{Change definition of \cs{th}
%    only for this language}
%    \begin{macrocode}
\addto\extrasrussian{%
  \babel@save{\th}%
  \let\ltx@th\th
  \def\th{\textormath{\ltx@th}%
                     {\mathop{\operator@font th}\nolimits}}%
  }
%    \end{macrocode}
%    |\cot|:
%    \begin{macrocode}
\def\ctg{\mathop{\operator@font ctg}\nolimits}
%    \end{macrocode}
%    |\coth|:
%    \begin{macrocode}
\def\cth{\mathop{\operator@font cth}\nolimits}
%    \end{macrocode}
%    |\csc|:
%    \begin{macrocode}
\def\cosec{\mathop{\operator@font cosec}\nolimits}
%    \end{macrocode}
%
%    And finally some other Ukrainian and Russian mathematical symbols:
%    \begin{macrocode}
\def\Prob{\mathop{\kern\z@\mathsf{P}}\nolimits}
\def\Variance{\mathop{\kern\z@\mathsf{D}}\nolimits}
\def\nsd{\mathop{\cyrmathrm{\cyrn.\cyrs.\cyrd.}}\nolimits}
\def\nsk{\mathop{\cyrmathrm{\cyrn.\cyrs.\cyrk.}}\nolimits}
\def\NSD{\mathop{\cyrmathrm{\CYRN\CYRS\CYRD}}\nolimits}
\def\NSK{\mathop{\cyrmathrm{\CYRN\CYRS\CYRK}}\nolimits}
  \def\nod{\mathop{\cyrmathrm{\cyrn.\cyro.\cyrd.}}\nolimits}    % ??????
  \def\nok{\mathop{\cyrmathrm{\cyrn.\cyro.\cyrk.}}\nolimits}    % ??????
  \def\NOD{\mathop{\cyrmathrm{\CYRN\CYRO\CYRD}}\nolimits}       % ??????
  \def\NOK{\mathop{\cyrmathrm{\CYRN\CYRO\CYRK}}\nolimits}       % ??????
\def\Proj{\mathop{\cyrmathrm{\CYRP\cyrr}}\nolimits}
%    \end{macrocode}
%
% This is for compatibility with older Ukrainian packages.
%    \begin{macrocode}
\DeclareRobustCommand{\No}{%
   \ifmmode{\nfss@text{\textnumero}}\else\textnumero\fi}
%    \end{macrocode}
%
%    The macro |\ldf@finish| takes care of looking for a configuration file,
%    setting the main language to be switched on at |\begin{document}| and
%    resetting the category code of \texttt{@} to its original value.
%
%    \begin{macrocode}
\ldf@finish{ukrainian}
%</code>
%    \end{macrocode}
%
% \Finale
%%
%% \CharacterTable
%%  {Upper-case    \A\B\C\D\E\F\G\H\I\J\K\L\M\N\O\P\Q\R\S\T\U\V\W\X\Y\Z
%%   Lower-case    \a\b\c\d\e\f\g\h\i\j\k\l\m\n\o\p\q\r\s\t\u\v\w\x\y\z
%%   Digits        \0\1\2\3\4\5\6\7\8\9
%%   Exclamation   \!     Double quote  \"     Hash (number) \#
%%   Dollar        \$     Percent       \%     Ampersand     \&
%%   Acute accent  \'     Left paren    \(     Right paren   \)
%%   Asterisk      \*     Plus          \+     Comma         \,
%%   Minus         \-     Point         \.     Solidus       \/
%%   Colon         \:     Semicolon     \;     Less than     \<
%%   Equals        \=     Greater than  \>     Question mark \?
%%   Commercial at \@     Left bracket  \[     Backslash     \\
%%   Right bracket \]     Circumflex    \^     Underscore    \_
%%   Grave accent  \`     Left brace    \{     Vertical bar  \|
%%   Right brace   \}     Tilde         \~}
%%
\endinput
}
\DeclareOption{uppersorbian}{% \iffalse meta-commen

% Copyright 1989-2008 Johannes L. Braams and any individual author
% listed elsewhere in this file.  All rights reserved
%
% This file is part of the Babel system
% -------------------------------------
%
% It may be distributed and/or modified under th
% conditions of the LaTeX Project Public License, either version 1.
% of this license or (at your option) any later version
% The latest version of this license is i
%   http://www.latex-project.org/lppl.tx
% and version 1.3 or later is part of all distributions of LaTe
% version 2003/12/01 or later
%
% This work has the LPPL maintenance status "maintained"
%
% The Current Maintainer of this work is Johannes Braams
%
% The list of all files belonging to the Babel system i
% given in the file `manifest.bbl. See also `legal.bbl' for additiona
% information
%
% The list of derived (unpacked) files belonging to the distributio
% and covered by LPPL is defined by the unpacking scripts (wit
% extension .ins) which are part of the distribution
% \f
% \CheckSum{344
% \iffals
%    Tell the \LaTeX\ system who we are and write an entry on th
%    transcript
%<*dtx
\ProvidesFile{usorbian.dtx
%</dtx
%<code>\ProvidesLanguage{usorbian
%\f
%\ProvidesFile{usorbian.dtx
        [2008/03/17 v1.0k Upper Sorbian support from the babel system
%\iffals
%% File `usorbian.dtx
%% Babel package for LaTeX version 2
%% Copyright (C) 1989 - 200
%%           by Johannes Braams, TeXnie

%% Upper Sorbian Language Definition Fil
%% Copyright (C) 1994 - 200
%%           by Eduard Werne
%           Werner, Eduard"
%           Serbski institut z. t.
%           Dw\'orni\v{s}\'cowa
%           02625 Budy\v{s}in/Bautze
%           Germany"
%           (??)3591 497223"
%           edi at kaihh.hanse.de"

%% Please report errors to: Eduard Werner edi at kaihh.hanse.d
%
%    This file is part of the babel system, it provides the sourc
%    code for the Upper Sorbian definition file
%<*filedriver
\documentclass{ltxdoc
\newcommand*\TeXhax{\TeX hax
\newcommand*\babel{\textsf{babel}
\newcommand*\langvar{$\langle \it lang \rangle$
\newcommand*\note[1]{
\newcommand*\Lopt[1]{\textsf{#1}
\newcommand*\file[1]{\texttt{#1}
\newfont{\logo}{logo10
\newcommand*\MF{{\logo METAFONT}
\begin{document
 \DocInput{usorbian.dtx
\end{document
%</filedriver
%\f
% \GetFileInfo{usorbian.dtx

% \changes{usorbian-0.1}{1994/10/10}{First version
% \changes{usorbian-0.1b}{1994/10/18}{Made it possible to run throug
%    \LaTeX; added \cs{MF} and removed extra \cs{end{macro}}
% \changes{usorbian-1.0d}{1996/07/13}{Replaced \cs{undefined} wit
%    \cs{@undefined} and \cs{empty} with \cs{@empty} for consistenc
%    with \LaTeX
% \changes{usorbian-1.0e}{1996/10/10}{Moved the definition o
%    \cs{atcatcode} right to the beginning.

%  \section{The Upper Sorbian language

%    The file \file{\filename}\footnote{The file described in thi
%    section has version number \fileversion\ and was last revised o
%    \filedate.  It was written by Eduard Werne
%    (\texttt{edi@kaihh.hanse.de}).}  It defines all th
%    language-specific macros for Upper Sorbian

% \StopEventually{

%    The macro |\LdfInit| takes care of preventing that this file i
%    loaded more than once, checking the category code of th
%    \texttt{@} sign, etc
% \changes{usorbian-1.0e}{1996/11/03}{Now use \cs{LdfInit} to perfor
%    initial checks}
% \changes{usorbian-1.0j}{2007/10/19}{This file can be loaded unde
%    more than one name.
%    \begin{macrocode
%<*code
\LdfInit\CurrentOption{date\CurrentOption
%    \end{macrocode

%    When this file is read as an option, i.e. by the |\usepackage
%    command, \texttt{usorbian} will be an `unknown' language, in whic
%    case we have to make it known. So we check for the existence o
%    |\l@usorbian| to see whether we have to do something here.
% \changes{usorbian-1.0j}{2007/10/19}{Check for the optio
%    lowersorbian
%    A
%    \babel\ also knwos the option \Lopt{uppersorbian} we have t
%    check that as well

%    \begin{macrocode
\ifx\l@uppersorbian\@undefine
  \ifx\l@usorbian\@undefine
    \@nopatterns{Usorbian
    \adddialect\l@usorbian\z
    \let\l@uppersorbian\l@usorbia
  \els
    \let\l@uppersorbian\l@usorbia
  \f
\els
  \let\l@usorbian\l@uppersorbia
\f
%    \end{macrocode

%    The next step consists of defining commands to switch to (an
%    from) the Upper Sorbian language

% \begin{macro}{\captionsusorbian
%    The macro |\captionsusorbian| defines all strings used in the fou
%    standard documentclasses provided with \LaTeX
% \changes{usorbian-0.1c}{1994/11/27}{Removed two typos (Kapitel an
%    Dodatki)
% \changes{usorbian-1.0b}{1995/07/04}{Added \cs{proofname} fo
%    AMS-\LaTeX
% \changes{usorbian-1.0i}{2000/09/22}{Added \cs{glossaryname}
% \changes{usorbian-1.0j}{2007/10/19}{Make this work for more than on
%    option name
%    \begin{macrocode
\@namedef{captions\CurrentOption}{
  \def\prefacename{Zawod}
  \def\refname{Referency}
  \def\abstractname{Abstrakt}
  \def\bibname{Literatura}
  \def\chaptername{Kapitl}
  \def\appendixname{Dodawki}
  \def\contentsname{Wobsah}
  \def\listfigurename{Zapis wobrazow}
  \def\listtablename{Zapis tabulkow}
  \def\indexname{Indeks}
  \def\figurename{Wobraz}
  \def\tablename{Tabulka}
  \def\partname{D\'z\v el}
  \def\enclname{P\v r\l oha}
  \def\ccname{CC}
  \def\headtoname{Komu}
  \def\pagename{Strona}
  \def\seename{hl.}
  \def\alsoname{hl.~te\v z
  \def\proofname{Proof}%  <-- needs translatio
  \def\glossaryname{Glossary}% <-- Needs translatio
  }
%    \end{macrocode
% \end{macro

% \begin{macro}{\newdateusorbian
%    The macro |\newdateusorbian| redefines the command |\today| t
%    produce Upper Sorbian dates
% \changes{usorbian-1.0g}{1997/10/01}{Use \cs{edef} to defin
%    \cs{today} to save memory
% \changes{usorbian-1.0g}{1998/03/28}{use \cs{def} instead o
%    \cs{edef}
% \changes{usorbian-1.0j}{2007/10/19}{Make this work for more than on
%    option name
%    \begin{macrocode
\@namedef{newdate\CurrentOption}{
  \def\today{\number\day.~\ifcase\month\o
    januara\or februara\or m\v erca\or apryla\or meje\or junija\o
    julija\or awgusta\or septembra\or oktobra\o
    nowembra\or decembra\f
    \space \number\year}
%    \end{macrocode
% \end{macro

% \begin{macro}{\olddateusorbian
%    The macro |\olddateusorbian| redefines the command |\today| t
%    produce old-style Upper Sorbian dates
% \changes{usorbian-1.0g}{1997/10/01}{Use \cs{edef} to defin
%    \cs{today} to save memory
% \changes{usorbian-1.0g}{1998/03/28}{use \cs{def} instead o
%    \cs{edef}
% \changes{usorbian-1.0j}{2007/10/19}{Make this work for more than on
%    option name
%    \begin{macrocode
\@namedef{olddate\CurrentOption}{
  \def\today{\number\day.~\ifcase\month\o
    wulkeho r\'o\v zka\or ma\l eho r\'o\v zka\or nal\v etnika\o
    jutrownika\or r\'o\v zownika\or  sma\v znika\or pra\v znika\o
    \v znjenca\or po\v znjenca\or winowca\or nazymnika\o
    hodownika\fi \space \number\year}
%    \end{macrocode
% \end{macro

%    The default will be the new-style dates
% \changes{usorbian-1.0j}{2007/10/19}{Make this work for more than on
%    option name
%    \begin{macrocode
\expandafter\let\csname date\CurrentOption\expandafter\endcsnam
                \csname newdate\CurrentOption\endcsnam
%    \end{macrocode

% \begin{macro}{\extrasusorbian
%    The macro |\extrasusorbian| will perform all the extr
%    definitions needed for the Upper Sorbian language. It's pirate
%    from |germanb.sty|.  The macro |\noextrasusorbian| is used t
%    cancel the actions of |\extrasusorbian|

%    Because for Upper Sorbian (as well as for Dutch) the \texttt{"
%    character is made active. This is done once, later on it
%    definition may vary
% \changes{usorbian-1.0j}{2007/10/19}{Make this work for more than on
%    option name
%    \begin{macrocode
\initiate@active@char{"
\@namedef{extras\CurrentOption}{\languageshorthands{usorbian}
\expandafter\addto\csname extras\CurrentOption\endcsname{
  \bbl@activate{"}
%    \end{macrocode
%    Don't forget to turn the shorthands off again
% \changes{usorbian-1.0h}{1999/12/17}{Deactivate shorthands ouside o
%    Upper Sorbian
%    \begin{macrocode
\expandafter\addto\csname extras\CurrentOption\endcsname{
  \bbl@deactivate{"}
%    \end{macrocode

%    In order for \TeX\ to be able to hyphenate German Upper Sorbia
%    words which contain `\ss' we have to give the character a nonzer
%    |\lccode| (see Appendix H, the \TeX book). As some of the othe
%    language definitions turn the character |^| into a shorthand w
%    need to make sure that it has it's orginial definition here
% \changes{usorbian-1.0k}{2008/03/17}{Make sure the caret has th
%    right \cs{catcdoe}}
%    \begin{macrocode
\begingroup \catcode`\^
\def\x{\endgrou
  \expandafter\addto\csname extras\CurrentOption\endcsname{
    \babel@savevariable{\lccode`\^^Y}
    \lccode`\^^Y`\^^Y}
\
%    \end{macrocode
%    The umlaut accent macro |\"| is changed to lower the umlaut dots
%    The redefinition is done with the help of |\umlautlow|
%    \begin{macrocode
\expandafter\addto\csname extras\CurrentOption\endcsname{
  \babel@save\"\umlautlow
\expandafter\addto\csname noextras\CurrentOption\endcsname{
  \umlauthigh
%    \end{macrocode
%    The Upper Sorbian hyphenation patterns can be used wit
%    |\lefthyphenmin| and |\righthyphenmin| set to~2
% \changes{usorbian-1.0i}{2000/09/22}{Now use \cs{providehyphenmins} t
%    provide a default value
%    \begin{macrocode
\providehyphenmins{\CurrentOption}{\tw@\tw@
%    \end{macrocode
% \end{macro

% \changes{usorbian-1.0a}{1995/05/27}{Removed stuff that has bee
%    moved to \file{babel.def}

%  \begin{macro}{\dq
%    We save the original double quote character in |\dq| to keep i
%    available, the math accent |\"| can now be typed as |"|.  Also w
%    store the original meaning of the command |\"| for future use
%    \begin{macrocode
\begingroup \catcode`\"1
\def\x{\endgrou
  \def\@SS{\mathchar"7019
  \def\dq{"}
\
%    \end{macrocode
% \end{macro

%    Now we can define the doublequote macros: the umlauts
%    \begin{macrocode
\declare@shorthand{usorbian}{"a}{\textormath{\"{a}}{\ddot a}
\declare@shorthand{usorbian}{"o}{\textormath{\"{o}}{\ddot o}
\declare@shorthand{usorbian}{"u}{\textormath{\"{u}}{\ddot u}
\declare@shorthand{usorbian}{"A}{\textormath{\"{A}}{\ddot A}
\declare@shorthand{usorbian}{"O}{\textormath{\"{O}}{\ddot O}
\declare@shorthand{usorbian}{"U}{\textormath{\"{U}}{\ddot U}
%    \end{macrocode
%    tremas
%    \begin{macrocode
\declare@shorthand{usorbian}{"e}{\textormath{\"{e}}{\ddot e}
\declare@shorthand{usorbian}{"E}{\textormath{\"{E}}{\ddot E}
\declare@shorthand{usorbian}{"i}{\textormath{\"{\i}}{\ddot\imath}
\declare@shorthand{usorbian}{"I}{\textormath{\"{I}}{\ddot I}
%    \end{macrocode
%    usorbian es-zet (sharp s)
%    \begin{macrocode
\declare@shorthand{usorbian}{"s}{\textormath{\ss{}}{\@SS{}}
\declare@shorthand{usorbian}{"S}{SS
%    \end{macrocode
%    german and french quotes
% \changes{usorbian-1.0f}{1997/04/03}{Removed empty groups afte
%    double quote and guillemot characters
%    \begin{macrocode
\declare@shorthand{usorbian}{"`}{
  \textormath{\quotedblbase}{\mbox{\quotedblbase}}
\declare@shorthand{usorbian}{"'}{
  \textormath{\textquotedblleft}{\mbox{\textquotedblleft}}
\declare@shorthand{usorbian}{"<}{
  \textormath{\guillemotleft}{\mbox{\guillemotleft}}
\declare@shorthand{usorbian}{">}{
  \textormath{\guillemotright}{\mbox{\guillemotright}}
%    \end{macrocode
%    discretionary command
% \changes{usorbian-1.0c}{1996/01/24}{Now use \cs{bbl@disc}
%    \begin{macrocode
\declare@shorthand{usorbian}{"c}{\textormath{\bbl@disc ck}{c}
\declare@shorthand{usorbian}{"C}{\textormath{\bbl@disc CK}{C}
\declare@shorthand{usorbian}{"f}{\textormath{\bbl@disc f{ff}}{f}
\declare@shorthand{usorbian}{"F}{\textormath{\bbl@disc F{FF}}{F}
\declare@shorthand{usorbian}{"l}{\textormath{\bbl@disc l{ll}}{l}
\declare@shorthand{usorbian}{"L}{\textormath{\bbl@disc L{LL}}{L}
\declare@shorthand{usorbian}{"m}{\textormath{\bbl@disc m{mm}}{m}
\declare@shorthand{usorbian}{"M}{\textormath{\bbl@disc M{MM}}{M}
\declare@shorthand{usorbian}{"n}{\textormath{\bbl@disc n{nn}}{n}
\declare@shorthand{usorbian}{"N}{\textormath{\bbl@disc N{NN}}{N}
\declare@shorthand{usorbian}{"p}{\textormath{\bbl@disc p{pp}}{p}
\declare@shorthand{usorbian}{"P}{\textormath{\bbl@disc P{PP}}{P}
\declare@shorthand{usorbian}{"t}{\textormath{\bbl@disc t{tt}}{t}
\declare@shorthand{usorbian}{"T}{\textormath{\bbl@disc T{TT}}{T}
%    \end{macrocode
%    and some additional commands
%    \begin{macrocode
\declare@shorthand{usorbian}{"-}{\nobreak\-\bbl@allowhyphens
\declare@shorthand{usorbian}{"|}{
  \textormath{\nobreak\discretionary{-}{}{\kern.03em}
              \allowhyphens}{}
\declare@shorthand{usorbian}{""}{\hskip\z@skip
%    \end{macrocode

%  \begin{macro}{\mdqon
%  \begin{macro}{\mdqoff
%  \begin{macro}{\ck
%    All that's left to do now is to  define a couple of command
%    for reasons of compatibility with \file{german.sty}
% \changes{usorbian-1.0g}{1998/06/07}{Now use \cs{shorthandon} an
%    \cs{shorthandoff}}
%    \begin{macrocode
\def\mdqon{\shorthandon{"}
\def\mdqoff{\shorthandoff{"}
\def\ck{\allowhyphens\discretionary{k-}{k}{ck}\allowhyphens
%    \end{macrocode
%  \end{macro
%  \end{macro
%  \end{macro

%    The macro |\ldf@finish| takes care of looking for
%    configuration file, setting the main language to be switched o
%    at |\begin{document}| and resetting the category code o
%    \texttt{@} to its original value
% \changes{usorbian-1.0e}{1996/11/03}{Now use \cs{ldf@finish} to wra
%    up}
% \changes{usorbian-1.0j}{2007/10/19}{Make this work for more than on
%    option name
%    \begin{macrocode
\ldf@finish\CurrentOptio
%</code
%    \end{macrocode

% \Final
%
%% \CharacterTabl
%%  {Upper-case    \A\B\C\D\E\F\G\H\I\J\K\L\M\N\O\P\Q\R\S\T\U\V\W\X\Y\
%%   Lower-case    \a\b\c\d\e\f\g\h\i\j\k\l\m\n\o\p\q\r\s\t\u\v\w\x\y\
%%   Digits        \0\1\2\3\4\5\6\7\8\
%%   Exclamation   \!     Double quote  \"     Hash (number) \
%%   Dollar        \$     Percent       \%     Ampersand     \
%%   Acute accent  \'     Left paren    \(     Right paren   \
%%   Asterisk      \*     Plus          \+     Comma         \
%%   Minus         \-     Point         \.     Solidus       \
%%   Colon         \:     Semicolon     \;     Less than     \
%%   Equals        \=     Greater than  \>     Question mark \
%%   Commercial at \@     Left bracket  \[     Backslash     \
%%   Right bracket \]     Circumflex    \^     Underscore    \
%%   Grave accent  \`     Left brace    \{     Vertical bar  \
%%   Right brace   \}     Tilde         \~
%
\endinpu
}
\DeclareOption{welsh}{% \iffalse meta-comment
%
% Copyright 1989-2005 Johannes L. Braams and any individual authors
% listed elsewhere in this file.  All rights reserved.
% 
% This file is part of the Babel system.
% --------------------------------------
% 
% It may be distributed and/or modified under the
% conditions of the LaTeX Project Public License, either version 1.3
% of this license or (at your option) any later version.
% The latest version of this license is in
%   http://www.latex-project.org/lppl.txt
% and version 1.3 or later is part of all distributions of LaTeX
% version 2003/12/01 or later.
% 
% This work has the LPPL maintenance status "maintained".
% 
% The Current Maintainer of this work is Johannes Braams.
% 
% The list of all files belonging to the Babel system is
% given in the file `manifest.bbl. See also `legal.bbl' for additional
% information.
% 
% The list of derived (unpacked) files belonging to the distribution
% and covered by LPPL is defined by the unpacking scripts (with
% extension .ins) which are part of the distribution.
% \fi
% \CheckSum{97}
%
% \iffalse
%    Tell the \LaTeX\ system who we are and write an entry on the
%    transcript.
%<*dtx>
\ProvidesFile{welsh.dtx}
%</dtx>
%<code>\ProvidesLanguage{welsh}
%\fi
%\ProvidesFile{welsh.dtx}
        [2005/03/31 v1.0d Welsh support from the babel system]
%\iffalse
%% File `welsh.dtx'
%% Babel package for LaTeX version 2e
%% Copyright (C) 1989 -- 2005
%%           by Johannes Braams, TeXniek
%
%% Please report errors to: J.L. Braams
%%                          babel at braams.cistron.nl
%
%    This file is part of the babel system, it provides the source
%    code for the Welsh language definition file.
%<*filedriver>
\documentclass{ltxdoc}
\newcommand*{\TeXhax}{\TeX hax}
\newcommand*{\babel}{\textsf{babel}}
\newcommand*{\langvar}{$\langle \mathit lang \rangle$}
\newcommand*{\note}[1]{}
\newcommand*{\Lopt}[1]{\textsf{#1}}
\newcommand*{\file}[1]{\texttt{#1}}
\begin{document}
 \DocInput{welsh.dtx}
\end{document}
%</filedriver>
%\fi
% \GetFileInfo{welsh.dtx}
%
%  \section{The Welsh language}
%
%    The file \file{\filename}\footnote{The file described in this
%    section has version number \fileversion\ and was last revised on
%    \filedate.}  defines all the language definition macros for the
%    Welsh language.
%
%    For this language currently no special definitions are needed or
%    available.
%
% \StopEventually{}
%
%    The macro |\ldf@init| takes care of preventing that this file is
%    loaded more than once, checking the category code of the
%    \texttt{@} sign, etc.
%    \begin{macrocode}
%<*code>
\LdfInit{welsh}{captionswelsh}
%    \end{macrocode}
%
%    When this file is read as an option, i.e. by the |\usepackage|
%    command, \texttt{welsh} could be an `unknown' language in
%    which case we have to make it known.  So we check for the
%    existence of |\l@welsh| to see whether we have to do
%    something here.
%
%    \begin{macrocode}
\ifx\undefined\l@welsh
  \@nopatterns{welsh}
  \adddialect\l@welsh0\fi
%    \end{macrocode}
%    The next step consists of defining commands to switch to (and
%    from) the Welsh language.
%
%  \begin{macro}{\welshhyphenmins}
%    This macro is used to store the correct values of the hyphenation
%    parameters |\lefthyphenmin| and |\righthyphenmin|.
% \changes{welsh-1.0c}{2000/09/22}{Now use \cs{providehyphenmins} to
%    provide a default value}
%    \begin{macrocode}
\providehyphenmins{\CurrentOption}{\tw@\thr@@}
%    \end{macrocode}
%  \end{macro}
%
% \begin{macro}{\captionswelsh}
%    The macro |\captionswelsh| defines all strings used in the
%    four standard documentclasses provided with \LaTeX.
% \changes{welsh-1.0c}{2000/09/20}{Added \cs{glossaryname}}
% \changes{welsh-1.0d}{2005/03/31}{Provided the translation for
%    Glossary} 
%    \begin{macrocode}
\def\captionswelsh{%
  \def\prefacename{Rhagair}%
  \def\refname{Cyfeiriadau}%   
  \def\abstractname{Crynodeb}% 
  \def\bibname{Llyfryddiaeth}%
  \def\chaptername{Pennod}%
  \def\appendixname{Atodiad}%
  \def\contentsname{Cynnwys}%
  \def\listfigurename{Rhestr Ddarluniau}%
  \def\listtablename{Rhestr Dablau}%
  \def\indexname{Mynegai}%
  \def\figurename{Darlun}%
  \def\tablename{Taflen}%
  \def\partname{Rhan}%
  \def\enclname{amgae\"edig}%
  \def\ccname{cop\"\i au}%
  \def\headtoname{At}%  % `at' on letters meaning `to ( a person)'
                        % `to (a place)' is `i' in Welsh
  \def\pagename{tudalen}%
  \def\seename{gweler}%
  \def\alsoname{gweler hefyd}%
  \def\proofname{Prawf}%
  \def\glossaryname{Rhestr termau}%
  }
%    \end{macrocode}
% \end{macro}
%
% \begin{macro}{\datewelsh}
%    The macro |\datewelsh| redefines the command |\today| to
%    produce welsh dates.
% \changes{welsh-1.0b}{1997/10/01}{Use \cs{edef} to define
%    \cs{today} to save memory}
% \changes{welsh-1.0b}{1998/03/28}{use \cs{def} instead of
%    \cs{edef}}
% \changes{welsh-1.0d}{2005/03/31}{removed `a viz' from the definition
%    of \cs{today}} 
%    \begin{macrocode}
\def\datewelsh{%
  \def\today{\ifnum\day=1\relax 1\/$^{\mathrm{a\tilde{n}}}$\else
    \number\day\fi\space\ifcase\month\or
    Ionawr\or Chwefror\or Mawrth\or Ebrill\or
    Mai\or Mehefin\or Gorffennaf\or Awst\or
    Medi\or Hydref\or Tachwedd\or Rhagfyr\fi
  \space\number\year}}
%    \end{macrocode}
% \end{macro}
%
% \begin{macro}{\extraswelsh}
% \begin{macro}{\noextraswelsh}
%    The macro |\extraswelsh| will perform all the extra
%    definitions needed for the welsh language. The macro
%    |\noextraswelsh| is used to cancel the actions of
%    |\extraswelsh|.  For the moment these macros are empty but
%    they are defined for compatibility with the other
%    language definition files.
%
%    \begin{macrocode}
\addto\extraswelsh{}
\addto\noextraswelsh{}
%    \end{macrocode}
% \end{macro}
% \end{macro}
%
%    The macro |\ldf@finish| takes care of looking for a
%    configuration file, setting the main language to be switched on
%    at |\begin{document}| and resetting the category code of
%    \texttt{@} to its original value.
%    \begin{macrocode}
\ldf@finish{welsh}
%</code>
%    \end{macrocode}
%
% \Finale
%\endinput
%% \CharacterTable
%%  {Upper-case    \A\B\C\D\E\F\G\H\I\J\K\L\M\N\O\P\Q\R\S\T\U\V\W\X\Y\Z
%%   Lower-case    \a\b\c\d\e\f\g\h\i\j\k\l\m\n\o\p\q\r\s\t\u\v\w\x\y\z
%%   Digits        \0\1\2\3\4\5\6\7\8\9
%%   Exclamation   \!     Double quote  \"     Hash (number) \#
%%   Dollar        \$     Percent       \%     Ampersand     \&
%%   Acute accent  \'     Left paren    \(     Right paren   \)
%%   Asterisk      \*     Plus          \+     Comma         \,
%%   Minus         \-     Point         \.     Solidus       \/
%%   Colon         \:     Semicolon     \;     Less than     \<
%%   Equals        \=     Greater than  \>     Question mark \?
%%   Commercial at \@     Left bracket  \[     Backslash     \\
%%   Right bracket \]     Circumflex    \^     Underscore    \_
%%   Grave accent  \`     Left brace    \{     Vertical bar  \|
%%   Right brace   \}     Tilde         \~}
%%
}
%    \end{macrocode}
% \changes{babel~3.6e}{1997/01/08}{Added options \Lopt{UKenglish} and
%    \Lopt{USenglish}}
%    \begin{macrocode}
\DeclareOption{UKenglish}{%%
%% This file will generate fast loadable files and documentation
%% driver files from the doc files in this package when run through
%% LaTeX or TeX.
%%
%% Copyright 1989-2005 Johannes L. Braams and any individual authors
%% listed elsewhere in this file.  All rights reserved.
%% 
%% This file is part of the Babel system.
%% --------------------------------------
%% 
%% It may be distributed and/or modified under the
%% conditions of the LaTeX Project Public License, either version 1.3
%% of this license or (at your option) any later version.
%% The latest version of this license is in
%%   http://www.latex-project.org/lppl.txt
%% and version 1.3 or later is part of all distributions of LaTeX
%% version 2003/12/01 or later.
%% 
%% This work has the LPPL maintenance status "maintained".
%% 
%% The Current Maintainer of this work is Johannes Braams.
%% 
%% The list of all files belonging to the LaTeX base distribution is
%% given in the file `manifest.bbl. See also `legal.bbl' for additional
%% information.
%% 
%% The list of derived (unpacked) files belonging to the distribution
%% and covered by LPPL is defined by the unpacking scripts (with
%% extension .ins) which are part of the distribution.
%%
%% --------------- start of docstrip commands ------------------
%%
\def\filedate{1999/04/11}
\def\batchfile{english.ins}
\input docstrip.tex

{\ifx\generate\undefined
\Msg{**********************************************}
\Msg{*}
\Msg{* This installation requires docstrip}
\Msg{* version 2.3c or later.}
\Msg{*}
\Msg{* An older version of docstrip has been input}
\Msg{*}
\Msg{**********************************************}
\errhelp{Move or rename old docstrip.tex.}
\errmessage{Old docstrip in input path}
\batchmode
\csname @@end\endcsname
\fi}

\declarepreamble\mainpreamble
This is a generated file.

Copyright 1989-2005 Johannes L. Braams and any individual authors
listed elsewhere in this file.  All rights reserved.

This file was generated from file(s) of the Babel system.
---------------------------------------------------------

It may be distributed and/or modified under the
conditions of the LaTeX Project Public License, either version 1.3
of this license or (at your option) any later version.
The latest version of this license is in
  http://www.latex-project.org/lppl.txt
and version 1.3 or later is part of all distributions of LaTeX
version 2003/12/01 or later.

This work has the LPPL maintenance status "maintained".

The Current Maintainer of this work is Johannes Braams.

This file may only be distributed together with a copy of the Babel
system. You may however distribute the Babel system without
such generated files.

The list of all files belonging to the Babel distribution is
given in the file `manifest.bbl'. See also `legal.bbl for additional
information.

The list of derived (unpacked) files belonging to the distribution
and covered by LPPL is defined by the unpacking scripts (with
extension .ins) which are part of the distribution.
\endpreamble

\declarepreamble\fdpreamble
This is a generated file.

Copyright 1989-2005 Johannes L. Braams and any individual authors
listed elsewhere in this file.  All rights reserved.

This file was generated from file(s) of the Babel system.
---------------------------------------------------------

It may be distributed and/or modified under the
conditions of the LaTeX Project Public License, either version 1.3
of this license or (at your option) any later version.
The latest version of this license is in
  http://www.latex-project.org/lppl.txt
and version 1.3 or later is part of all distributions of LaTeX
version 2003/12/01 or later.

This work has the LPPL maintenance status "maintained".

The Current Maintainer of this work is Johannes Braams.

This file may only be distributed together with a copy of the Babel
system. You may however distribute the Babel system without
such generated files.

The list of all files belonging to the Babel distribution is
given in the file `manifest.bbl'. See also `legal.bbl for additional
information.

In particular, permission is granted to customize the declarations in
this file to serve the needs of your installation.

However, NO PERMISSION is granted to distribute a modified version
of this file under its original name.

\endpreamble

\keepsilent

\usedir{tex/generic/babel} 

\usepreamble\mainpreamble
\generate{\file{english.ldf}{\from{english.dtx}{code}}
          }
\usepreamble\fdpreamble

\ifToplevel{
\Msg{***********************************************************}
\Msg{*}
\Msg{* To finish the installation you have to move the following}
\Msg{* files into a directory searched by TeX:}
\Msg{*}
\Msg{* \space\space All *.def, *.fd, *.ldf, *.sty}
\Msg{*}
\Msg{* To produce the documentation run the files ending with}
\Msg{* '.dtx' and `.fdd' through LaTeX.}
\Msg{*}
\Msg{* Happy TeXing}
\Msg{***********************************************************}
}
 
\endinput
}
\DeclareOption{USenglish}{%%
%% This file will generate fast loadable files and documentation
%% driver files from the doc files in this package when run through
%% LaTeX or TeX.
%%
%% Copyright 1989-2005 Johannes L. Braams and any individual authors
%% listed elsewhere in this file.  All rights reserved.
%% 
%% This file is part of the Babel system.
%% --------------------------------------
%% 
%% It may be distributed and/or modified under the
%% conditions of the LaTeX Project Public License, either version 1.3
%% of this license or (at your option) any later version.
%% The latest version of this license is in
%%   http://www.latex-project.org/lppl.txt
%% and version 1.3 or later is part of all distributions of LaTeX
%% version 2003/12/01 or later.
%% 
%% This work has the LPPL maintenance status "maintained".
%% 
%% The Current Maintainer of this work is Johannes Braams.
%% 
%% The list of all files belonging to the LaTeX base distribution is
%% given in the file `manifest.bbl. See also `legal.bbl' for additional
%% information.
%% 
%% The list of derived (unpacked) files belonging to the distribution
%% and covered by LPPL is defined by the unpacking scripts (with
%% extension .ins) which are part of the distribution.
%%
%% --------------- start of docstrip commands ------------------
%%
\def\filedate{1999/04/11}
\def\batchfile{english.ins}
\input docstrip.tex

{\ifx\generate\undefined
\Msg{**********************************************}
\Msg{*}
\Msg{* This installation requires docstrip}
\Msg{* version 2.3c or later.}
\Msg{*}
\Msg{* An older version of docstrip has been input}
\Msg{*}
\Msg{**********************************************}
\errhelp{Move or rename old docstrip.tex.}
\errmessage{Old docstrip in input path}
\batchmode
\csname @@end\endcsname
\fi}

\declarepreamble\mainpreamble
This is a generated file.

Copyright 1989-2005 Johannes L. Braams and any individual authors
listed elsewhere in this file.  All rights reserved.

This file was generated from file(s) of the Babel system.
---------------------------------------------------------

It may be distributed and/or modified under the
conditions of the LaTeX Project Public License, either version 1.3
of this license or (at your option) any later version.
The latest version of this license is in
  http://www.latex-project.org/lppl.txt
and version 1.3 or later is part of all distributions of LaTeX
version 2003/12/01 or later.

This work has the LPPL maintenance status "maintained".

The Current Maintainer of this work is Johannes Braams.

This file may only be distributed together with a copy of the Babel
system. You may however distribute the Babel system without
such generated files.

The list of all files belonging to the Babel distribution is
given in the file `manifest.bbl'. See also `legal.bbl for additional
information.

The list of derived (unpacked) files belonging to the distribution
and covered by LPPL is defined by the unpacking scripts (with
extension .ins) which are part of the distribution.
\endpreamble

\declarepreamble\fdpreamble
This is a generated file.

Copyright 1989-2005 Johannes L. Braams and any individual authors
listed elsewhere in this file.  All rights reserved.

This file was generated from file(s) of the Babel system.
---------------------------------------------------------

It may be distributed and/or modified under the
conditions of the LaTeX Project Public License, either version 1.3
of this license or (at your option) any later version.
The latest version of this license is in
  http://www.latex-project.org/lppl.txt
and version 1.3 or later is part of all distributions of LaTeX
version 2003/12/01 or later.

This work has the LPPL maintenance status "maintained".

The Current Maintainer of this work is Johannes Braams.

This file may only be distributed together with a copy of the Babel
system. You may however distribute the Babel system without
such generated files.

The list of all files belonging to the Babel distribution is
given in the file `manifest.bbl'. See also `legal.bbl for additional
information.

In particular, permission is granted to customize the declarations in
this file to serve the needs of your installation.

However, NO PERMISSION is granted to distribute a modified version
of this file under its original name.

\endpreamble

\keepsilent

\usedir{tex/generic/babel} 

\usepreamble\mainpreamble
\generate{\file{english.ldf}{\from{english.dtx}{code}}
          }
\usepreamble\fdpreamble

\ifToplevel{
\Msg{***********************************************************}
\Msg{*}
\Msg{* To finish the installation you have to move the following}
\Msg{* files into a directory searched by TeX:}
\Msg{*}
\Msg{* \space\space All *.def, *.fd, *.ldf, *.sty}
\Msg{*}
\Msg{* To produce the documentation run the files ending with}
\Msg{* '.dtx' and `.fdd' through LaTeX.}
\Msg{*}
\Msg{* Happy TeXing}
\Msg{***********************************************************}
}
 
\endinput
}
%    \end{macrocode}
%
%    For all those languages for which the option name is the same as
%    the name of the language specific file we specify a default
%    option, which tries to load the file specified. If this doesn't
%    succeed an error is signalled.
% \changes{babel~3.6i}{1997/03/12}{Added default option}
%    \begin{macrocode}
\DeclareOption*{%
  \InputIfFileExists{\CurrentOption.ldf}{}{%
    \PackageError{babel}{%
      Language definition file \CurrentOption.ldf not found}{%
      Maybe you misspelled the language option?}}%
  }
%    \end{macrocode}
%    Another way to extend the list of `known' options for \babel\ is
%    to create the file \file{bblopts.cfg} in which one can add
%    option declarations.
% \changes{babel~3.6i}{1997/03/15}{Added the possibility to have a
%    \file{bblopts.cfg} file with option declarations.}
%    \begin{macrocode}
\InputIfFileExists{bblopts.cfg}{%
  \typeout{*************************************^^J%
           * Local config file bblopts.cfg used^^J%
           *}%
  }{}
%    \end{macrocode}
%
%    Apart from all the language options we also have a few options
%    that influence the behaviour of language definition files.
%
%    The following options don't do anything themselves, they are just
%    defined in order to make it possible for language definition
%    files to check if one of them was specified by the user.
% \changes{babel~3.5d}{1995/07/04}{Added options to influence
%    behaviour of active acute and grave accents}
%    \begin{macrocode}
\DeclareOption{activeacute}{}
\DeclareOption{activegrave}{}
%    \end{macrocode}
%    The next option tells \babel\ to leave shorthand characters
%    active at the end of processing the package. This is \emph{not}
%    the default as it can cause problems with other packages, but for
%    those who want to use the shorthand characters in the preamble of
%    their documents this can help.
% \changes{babel~3.6f}{1997/01/14}{Added option
%    \Lopt{KeepShorthandsActive}}
% \changes{babel~3.7a}{1997/03/21}{No longer define the control
%    sequence \cs{KeepShorthandsActive}}
%    \begin{macrocode}
\DeclareOption{KeepShorthandsActive}{}
%    \end{macrocode}
%
%    The options have to be processed in the order in which the user
%    specified them:
%    \begin{macrocode}
\ProcessOptions*
%    \end{macrocode}
% \changes{babel~3.7c}{1999/03/13}{Added an error message for when no
%    language option was specified}
%    In order to catch the case where the user forgot to specify a
%    language we check whether |\bbl@main@language|, has become
%    defined. If not, no language has been loaded and an error
%    message is displayed.
% \changes{babel~3.7c}{1999/04/09}{No longer us a redefinition of an
%    internal macro, just check \cs{bbl@main@language} and load
%    \file{babel.def}}
%    \begin{macrocode}
\ifx\bbl@main@language\@undefined
  \PackageError{babel}{%
    You haven't specified a language option}{%
    You need to specify a language, either as a global
    option\MessageBreak
    or as an optional argument to the \string\usepackage\space
    command; \MessageBreak
    You shouldn't try to proceed from here, type x to quit.}
%    \end{macrocode}
%    To prevent undefined command errors when the user insists on
%    continuing we load \file{babel.def} here. He should expect more
%    errors though.
%    \begin{macrocode}
  %%
%% This file will generate fast loadable files and documentation
%% driver files from the doc files in this package when run through
%% LaTeX or TeX.
%%
%% Copyright 2012-2021 Javier Bezos and Johannes L. Braams.
%% Copyright 1989-2008 Johannes L. Braams and any individual authors
%% listed elsewhere in this file.  All rights reserved.
%% 
%% This file is part of the Babel system.
%% --------------------------------------
%% 
%% It may be distributed and/or modified under the
%% conditions of the LaTeX Project Public License, either version 1.3
%% of this license or (at your option) any later version.
%% The latest version of this license is in
%%   http://www.latex-project.org/lppl.txt
%% and version 1.3 or later is part of all distributions of LaTeX
%% version 2003/12/01 or later.
%% 
%% This work has the LPPL maintenance status "maintained".
%% 
%% The Current Maintainer of this work is Javier Bezos.
%% 
%% The list of derived (unpacked) files belonging to the distribution
%% and covered by LPPL is defined by the unpacking scripts (with
%% extension .ins) which are part of the distribution.
%%
\def\filedate{2021/04/13}
\def\batchfile{babel.ins}
\input docstrip.tex

{\ifx\generate\undefined
\Msg{**********************************************}
\Msg{*}
\Msg{* This installation requires docstrip}
\Msg{* version 2.3c or later.}
\Msg{*}
\Msg{* An older version of docstrip has been input}
\Msg{*}
\Msg{**********************************************}
\errhelp{Move or rename old docstrip.tex.}
\errmessage{Old docstrip in input path}
\batchmode
\csname @@end\endcsname
\fi}

% Modify docstrip. A pseudo-guard is defined to set variables:
% <<name=value>>. These variables are used with <@name@> Two further
% pseudo-guards define "block" variables: <<*name>> and <</name>>
% delimite the lines to be retrived with <@name@>. Note the verbatim
% guard is overridden, but it's not used here.  This is done in two
% passes: 1) with saving true, there is a dummy pass, generating
% nothing, but blocks are read and saved; 2) with saving false, blocks
% are always ignored, but replacing <@name@>.  While <@name@> can be
% used freely outside <<>>, it's only allowed inside <<>> if
% previously defined. Deeper nesting is not allowed.

\def\replaceVar#1<@#2{%
  #1%
  \ifx\endLine#2\else
    \expandafter\replaceVarX\expandafter#2%
  \fi}

\def\replaceVarX#1@>{\csname #1Var\endcsname\replaceVar}

\def\normalLine#1\endLine{%
  \advance\codeLinesPassed\@ne
  \maybeMsg{.}%
  \edef\inLine{\replaceVar#1<@\endLine}%
  \let\do\putline@do
  \ifcollect
    \xdef\varCollect{\varCollect^^J\inLine}%
  \else
    \activefiles
  \fi}
  
\newif\ifcollect
\newif\ifsaving

\def\verbOption<#1#2>>#3{%
  \ifx#1*%
    \maybeMsg{<<*#2>>}%
    \global\collecttrue
    \gdef\varCollect##1{}%
  \else\ifx#1/%
    \global\collectfalse
    \ifsaving
      \expandafter\ifx\csname #2Var\endcsname\relax
        \global\expandafter\let\csname #2Var\endcsname\varCollect
      \else
        \toks@\expandafter\expandafter\expandafter{%
          \csname #2Var\expandafter\endcsname\expandafter^^J%
          \varCollect}%
        \expandafter\xdef\csname #2Var\endcsname{\the\toks@}%
      \fi
    \fi
  \else
    \varOptionI#1#2>%
  \fi\fi}

\def\varOptionI#1=#2>{%
  \maybeMsg{<<#1=#2>>}%
  \ifsaving
    \expandafter\gdef\csname #1Var\endcsname{#2}%
  \fi}

% Preambles

\declarepreamble\mainpreamble
\endpreamble

\def\MetaPrefix{--}
\declarepreamble\luapreamble
\endpreamble
\let\MetaPrefix\DoubleperCent

\keepsilent
\askonceonly

\usedir{tex/generic/babel}

\usepreamble\mainpreamble

% Dummy, it just read "modules" to be used when generating
% the actual file. There must be a better way.
\savingtrue
\generate{\usepreamble\empty
          \usepostamble\empty
          \file{dummy.log}{\from{babel.dtx}{dummy}}}
\savingfalse

\generate{\file{babel.sty}{\from{babel.dtx}{package}}
          \file{babel.def}{\from{babel.dtx}{core}}
          \file{switch.def}{\from{babel.dtx}{kernel}}
          \file{hyphen.cfg}{\from{babel.dtx}{patterns}}
          \file{nil.ldf}{\from{babel.dtx}{nil}}
          \file{xebabel.def}{\from{babel.dtx}{xetex}}
          \file{luababel.def}{\from{babel.dtx}{luatex}}
          \file{txtbabel.def}{\from{babel.dtx}{texxet}}
          }

% Support for plain users
\generate{\file{plain.def}{\from{babel.dtx}{plain}}
          \file{bplain.tex}{\from{babel.dtx}{bplain}}
          \file{blplain.tex}{\from{babel.dtx}{blplain}}
         }

% compatibility files
\def\compatfile#1{\file{#1.sty}{\from{bbcompat.dtx}{styfile,#1}}}

\generate{%
         \compatfile{esperanto}
         \compatfile{afrikaans}
         \compatfile{dutch}
         \compatfile{american}
         \compatfile{british}
         \compatfile{english}
         \compatfile{UKenglish}
         \compatfile{USenglish}
         \compatfile{germanb}
         \compatfile{austrian}
         \compatfile{ngermanb}
         \compatfile{naustrian}
         \compatfile{irish}
         \compatfile{scottish}
         \compatfile{welsh}
         \compatfile{breton}
         }
\generate{%
         \compatfile{francais}
         \compatfile{italian}
         \compatfile{portuges}
         \compatfile{spanish}
         \compatfile{catalan}
         \compatfile{galician}
         \compatfile{danish}
         \compatfile{norsk}
         \compatfile{swedish}
         \compatfile{finnish}
         \compatfile{magyar}
         \compatfile{greek}
         \compatfile{croatian}
         \compatfile{czech}
         \compatfile{slovak}
         \compatfile{polish}
         }
\generate{%
         \compatfile{estonian}
         \compatfile{romanian}
         \compatfile{slovene}
         \compatfile{russianb}
         \compatfile{ukraineb}
         \compatfile{turkish}
         \compatfile{lsorbian}
         \compatfile{usorbian}
         \compatfile{bahasa}
         \compatfile{hebrew}
         \compatfile{basque}
         \compatfile{latin}
         \compatfile{icelandic}
         \compatfile{serbian}
         \compatfile{bulgarian}
         }
\generate{%
         \compatfile{samin}
         \compatfile{interlingua}
         \compatfile{albanian}
         \compatfile{bahasam}
         }

% MakeIndex style files

\usedir{makeindex/babel}

\generate{\file{bbind.ist}{\from{bbidxglo.dtx}{idx}}
          \file{bbglo.ist}{\from{bbidxglo.dtx}{glo}}}

% lua code

\def\MetaPrefix{--}
\usepreamble\luapreamble
\nopostamble
\generate{\file{babel-data-bidi.lua}{\from{babel.dtx}{bididata}}}
\generate{\file{babel-data-cjk.lua}{\from{babel.dtx}{cjkdata}}}
\generate{\file{babel-bidi-basic-r.lua}{\from{babel.dtx}{basic-r}}}
\generate{\file{babel-bidi-basic.lua}{\from{babel.dtx}{basic}}}

\Msg{***********************************************************}
\Msg{*}
\Msg{* To finish the installation you have to move all the files}
\Msg{* with names ending in .ldf, .sty, .def or .lua into a}
\Msg{* directory searched by TeX}
\Msg{*}
\Msg{* For making a format the following files have to be in a}
\Msg{* directory which is searched by IniTeX:}
\Msg{* \space\space hyphen.cfg}
\Msg{* \space\space language.dat}
\Msg{* \space\space and files with hyphenation patterns}
\Msg{*}
\Msg{* To produce source listings you can run babel.dtx}
\Msg{* through LuaLaTeX. Deja Vu fonts are required.}
\Msg{*}
\Msg{*}
\Msg{* Happy TeXing}
\Msg{*}
\Msg{***********************************************************}



\fi
%    \end{macrocode}
%
%  \begin{macro}{\substitutefontfamily}
%    The command |\substitutefontfamily| creates an \file{.fd} file on
%    the fly. The first argument is an encoding mnemonic, the second
%    and third arguments are font family names.
% \changes{babel~3.7j}{2003/06/15}{create file with lowercase name}
%    \begin{macrocode}
\def\substitutefontfamily#1#2#3{%
  \lowercase{\immediate\openout15=#1#2.fd\relax}%
  \immediate\write15{%
    \string\ProvidesFile{#1#2.fd}%
    [\the\year/\two@digits{\the\month}/\two@digits{\the\day}
     \space generated font description file]^^J
    \string\DeclareFontFamily{#1}{#2}{}^^J
    \string\DeclareFontShape{#1}{#2}{m}{n}{<->ssub * #3/m/n}{}^^J
    \string\DeclareFontShape{#1}{#2}{m}{it}{<->ssub * #3/m/it}{}^^J
    \string\DeclareFontShape{#1}{#2}{m}{sl}{<->ssub * #3/m/sl}{}^^J
    \string\DeclareFontShape{#1}{#2}{m}{sc}{<->ssub * #3/m/sc}{}^^J
    \string\DeclareFontShape{#1}{#2}{b}{n}{<->ssub * #3/bx/n}{}^^J
    \string\DeclareFontShape{#1}{#2}{b}{it}{<->ssub * #3/bx/it}{}^^J
    \string\DeclareFontShape{#1}{#2}{b}{sl}{<->ssub * #3/bx/sl}{}^^J
    \string\DeclareFontShape{#1}{#2}{b}{sc}{<->ssub * #3/bx/sc}{}^^J
    }%
  \closeout15
  }
%    \end{macrocode}
%    This command should only be used in the preamble of a document.
%    \begin{macrocode}
\@onlypreamble\substitutefontfamily
%    \end{macrocode}
%  \end{macro}
%
%    \begin{macrocode}
%</package>
%    \end{macrocode}
%
% \section{The Kernel of Babel}
%
%    The kernel of the \babel\ system is stored in either
%    \file{hyphen.cfg} or \file{switch.def} and \file{babel.def}. The
%    file \file{hyphen.cfg} is a file that can be loaded into the
%    format, which is necessary when you want to be able to switch
%    hyphenation patterns. The file \file{babel.def} contains some
%    \TeX\ code that can be read in at run time. When \file{babel.def}
%    is loaded it checks if \file{hyphen.cfg} is in the format; if
%    not the file \file{switch.def} is loaded.
%
%    Because plain \TeX\ users might want to use some of the features
%    of the \babel{} system too, care has to be taken that plain \TeX\
%    can process the files. For this reason the current format will
%    have to be checked in a number of places. Some of the code below
%    is common to plain \TeX\ and \LaTeX, some of it is for the
%    \LaTeX\ case only.
%
%    When the command |\AtBeginDocument| doesn't exist we assume that
%    we are dealing with a plain-based format. In that case the file
%    \file{plain.def} is needed.
%
%    \begin{macrocode}
%<*kernel|core>
\ifx\AtBeginDocument\@undefined
%    \end{macrocode}
%    But we need to use the second part of \file{plain.def} (when we
%    load it from \file{switch.def}) which we can do by defining
%    |\adddialect|.
% \changes{babel~3.7c}{1999/04/20}{define \cs{adddialect} before
%    loading \file{plain.def} here}
%    \begin{macrocode}
%<kernel&!patterns>  \def\adddialect{}
  \input plain.def\relax
\fi
%</kernel|core>
%    \end{macrocode}
%
%    Check the presence of the command |\iflanguage|, if it is
%    undefined read the file \file{switch.def}.
% \changes{babel~3.0d}{1991/10/29}{Removed use of \cs{@ifundefined}}
%    \begin{macrocode}
%<*core>
\ifx\iflanguage\@undefined
  \input switch.def\relax
\fi
%</core>
%    \end{macrocode}
% \changes{babel~3.6a}{1996/11/02}{Removed \cs{babel@core@loaded}, no
%    longer needed with the advent of \cs{LdfInit}}
%
%  \subsection{Encoding issues (part 1)}
%
%    The first thing we need to do is to determine, at
%    |\begin{document}|, which latin fontencoding to use.
%
%  \begin{macro}{\latinencoding}
% \changes{babel~3.6i}{1997/03/15}{Macro added, moved from
%    \file{.ldf} files}
%    When text is being typeset in an encoding other than `latin'
%    (\texttt{OT1} or \texttt{T1}), it would be nice to still have
%    Roman numerals come out in the Latin encoding.
%    So we first assume that the current encoding at the end
%    of processing the package is the Latin encoding.
%    \begin{macrocode}
%<*core>
\AtEndOfPackage{\edef\latinencoding{\cf@encoding}}
%    \end{macrocode}
%    But this might be overruled with a later loading of the package
%    \pkg{fontenc}. Therefor we check at the execution of
%    |\begin{document}| whether it was loaded with the \Lopt{T1}
%    option. The normal way to do this (using |\@ifpackageloaded|) is
%    disabled for this package. Now we have to revert to parsing the
%    internal macro |\@filelist| which contains all the filenames
%    loaded.
% \changes{babel~3.6k}{1999/03/15}{Use T1 encoding when it is a known
%    encoding}
% \changes{babel~3.6m}{1999/04/06}{Can't use \cs{@ifpackageloaded}
%    need to parse \cs{@filelist}}
% \changes{babel~3.6n}{1999/04/07}{moved checking for fontenc right to
%    the top of \file{babel.sty}}
% \changes{babel~3.6n}{1999/04/07}{Added a check for `manual' selection
%    of \texttt{T1} encoding, without loading \pkg{fontenc}}
% \changes{babel~3.6q}{1999/04/12}{Better solution then parsing
%    \cs{@filelist}, use \cs{@ifl@aded}}
% \changes{babel~3.6u}{1999/04/20}{Moved this code to
%    \file{babel.def}}
%    \begin{macrocode}
\AtBeginDocument{%
  \gdef\latinencoding{OT1}%
  \ifx\cf@encoding\bbl@t@one
    \xdef\latinencoding{\bbl@t@one}%
  \else
    \@ifl@aded{def}{t1enc}{\xdef\latinencoding{\bbl@t@one}}{}%
  \fi
  }
%    \end{macrocode}
%  \end{macro}
%
%  \begin{macro}{\latintext}
% \changes{babel~3.6i}{1997/03/15}{Macro added, moved from
%    \file{.ldf} files}
%    Then we can define the command |\latintext| which is a
%    declarative switch to a latin font-encoding.
%    \begin{macrocode}
\DeclareRobustCommand{\latintext}{%
  \fontencoding{\latinencoding}\selectfont
  \def\encodingdefault{\latinencoding}}
%    \end{macrocode}
%  \end{macro}
%
%  \begin{macro}{\textlatin}
% \changes{babel~3.6i}{1997/03/15}{Macro added, moved from
%    \file{.ldf} files}
% \changes{babel~3.7j}{2003/03/19}{added \cs{leavevmode} to prevent a
%    paragraph starting \emph{inside} the group}
% \changes{babel~3.7k}{2003/10/12}{Use \cs{DeclareTextFontComand}}
%    This command takes an argument which is then typeset using the
%    requested font encoding. In order to avoid many encoding switches
%    it operates in a local scope.
%    \begin{macrocode}
\ifx\@undefined\DeclareTextFontCommand
  \DeclareRobustCommand{\textlatin}[1]{\leavevmode{\latintext #1}}
\else
  \DeclareTextFontCommand{\textlatin}{\latintext}
\fi
%</core>
%    \end{macrocode}
%  \end{macro}
%
%    We also need to redefine a number of commands to ensure that the
%    right font encoding is used, but this can't be done before
%    \file{babel.def} is loaded.
% \changes{babel~3.6o}{1999/04/07}{Moved the rest of the font encoding
%    related definitions to their original place}
%
% \subsection{Multiple languages}
%
%    With \TeX\ version~3.0 it has become possible to load hyphenation
%    patterns for more than one language. This means that some extra
%    administration has to be taken care of.  The user has to know for
%    which languages patterns have been loaded, and what values of
%    |\language| have been used.
%
%    Some discussion has been going on in the \TeX\ world about how to
%    use |\language|. Some have suggested to set a fixed standard,
%    i.\,e., patterns for each language should \emph{always} be loaded
%    in the same location. It has also been suggested to use the
%    \textsc{iso} list for this purpose. Others have pointed out that
%    the \textsc{iso} list contains more than 256~languages, which
%    have \emph{not} been numbered consecutively.
%
%    I think the best way to use |\language|, is to use it
%    dynamically.  This code implements an algorithm to do so. It uses
%    an external file in which the person who maintains a \TeX\
%    environment has to record for which languages he has hyphenation
%    patterns \emph{and} in which files these are stored\footnote{This
%    is because different operating systems sometimes use \emph{very}
%    different file-naming conventions.}. When hyphenation exceptions
%    are stored in a separate file this can be indicated by naming
%    that file \emph{after} the file with the hyphenation patterns.
%
%    This ``configuration file'' can contain empty lines and comments,
%    as well as lines which start with an equals (\texttt{=})
%    sign. Such a line will instruct \LaTeX\ that the hyphenation
%    patterns just processed have to be known under an alternative
%    name. Here is an example:
%  \begin{verbatim}
%    % File    : language.dat
%    % Purpose : tell iniTeX what files with patterns to load.
%    english    english.hyphenations
%    =british
%
%    dutch      hyphen.dutch exceptions.dutch % Nederlands
%    german hyphen.ger
%  \end{verbatim}
%
%    As the file \file{switch.def} needs to be read only once, we
%    check whether it was read before.  If it was, the command
%    |\iflanguage| is already defined, so we can stop processing.
%    \begin{macrocode}
%<*kernel>
%<*!patterns>
\expandafter\ifx\csname iflanguage\endcsname\relax \else
\expandafter\endinput
\fi
%</!patterns>
%    \end{macrocode}
%
%  \begin{macro}{\language}
%    Plain \TeX\ version~3.0 provides the primitive |\language| that
%    is used to store the current language. When used with a pre-3.0
%    version this function has to be implemented by allocating a
%    counter.
%    \begin{macrocode}
\ifx\language\@undefined
  \csname newcount\endcsname\language
\fi
%    \end{macrocode}
%  \end{macro}
%
%  \begin{macro}{\last@language}
%    Another counter is used to store the last language defined.  For
%    pre-3.0 formats an extra counter has to be allocated,
%    \begin{macrocode}
\ifx\newlanguage\@undefined
  \csname newcount\endcsname\last@language
%    \end{macrocode}
%    plain \TeX\ version 3.0 uses |\count 19| for this purpose.
%    \begin{macrocode}
\else
  \countdef\last@language=19
\fi
%    \end{macrocode}
%  \end{macro}
%
%  \begin{macro}{\addlanguage}
%
%    To add languages to \TeX's memory plain \TeX\ version~3.0
%    supplies |\newlanguage|, in a pre-3.0 environment a similar macro
%    has to be provided. For both cases a new macro is defined here,
%    because the original |\newlanguage| was defined to be |\outer|.
%
%    For a format based on plain version~2.x, the definition of
%    |\newlanguage| can not be copied because |\count 19| is used for
%    other purposes in these formats. Therefor |\addlanguage| is
%    defined using a definition based on the macros used to define
%    |\newlanguage| in plain \TeX\ version~3.0.
% \changes{babel~3.2}{1991/11/11}{Added a \texttt{\%}, removed
%    \texttt{by}}
%    \begin{macrocode}
\ifx\newlanguage\@undefined
  \def\addlanguage#1{%
    \global\advance\last@language \@ne
    \ifnum\last@language<\@cclvi
    \else
        \errmessage{No room for a new \string\language!}%
    \fi
    \global\chardef#1\last@language
    \wlog{\string#1 = \string\language\the\last@language}}
%    \end{macrocode}
%
%    For formats based on plain version~3.0 the definition of
%    |\newlanguage| can be simply copied, removing |\outer|.
%
%    \begin{macrocode}
\else
  \def\addlanguage{\alloc@9\language\chardef\@cclvi}
\fi
%    \end{macrocode}
%  \end{macro}
%
%  \begin{macro}{\adddialect}
%    The macro |\adddialect| can be used to add the name of a dialect
%    or variant language, for which an already defined hyphenation
%    table can be used.
% \changes{babel~3.2}{1991/11/11}{Added \cs{relax}}
%    \begin{macrocode}
\def\adddialect#1#2{%
    \global\chardef#1#2\relax
    \wlog{\string#1 = a dialect from \string\language#2}}
%    \end{macrocode}
%  \end{macro}
%
%  \begin{macro}{\iflanguage}
%    Users might want to test (in a private package for instance)
%    which language is currently active. For this we provide a test
%    macro, |\iflanguage|, that has three arguments.  It checks
%    whether the first argument is a known language. If so, it
%    compares the first argument with the value of |\language|. Then,
%    depending on the result of the comparison, it executes either the
%    second or the third argument.
% \changes{babel~3.0a}{1991/05/29}{Added \cs{@bsphack} and
%    \cs{@esphack}}
% \changes{babel~3.0c}{1991/07/21}{Added comment character after
%    \texttt{\#2}}
% \changes{babel~3.0d}{1991/08/08}{Removed superfluous
%    \cs{expandafter}}
% \changes{babel~3.0d}{1991/10/07}{Removed space hacks and use of
%    \cs{@ifundefined}}
% \changes{babel~3.2}{1991/11/11}{Rephrased \cs{ifnum} test}
% \changes{babel~3.7a}{1998/06/10}{Now evaluate the \cs{ifnum} test
%    \emph{after} the \cs{fi} from the \cs{ifx} test and use
%    \cs{@firstoftwo} and \cs{@secondoftwo}}
% \changes{babel~3.7b}{1998/06/29}{Slight enhancement: added braces
%    around first argument of \cs{bbl@afterfi}}
%    \begin{macrocode}
\def\iflanguage#1{%
  \expandafter\ifx\csname l@#1\endcsname\relax
    \@nolanerr{#1}%
  \else
    \bbl@afterfi{\ifnum\csname l@#1\endcsname=\language
      \expandafter\@firstoftwo
    \else
      \expandafter\@secondoftwo
    \fi}%
  \fi}
%    \end{macrocode}
%  \end{macro}
%
%  \begin{macro}{\selectlanguage}
%    The macro |\selectlanguage| checks whether the language is
%    already defined before it performs its actual task, which is to
%    update |\language| and activate language-specific definitions.
%
%    To allow the call of |\selectlanguage| either with a control
%    sequence name or with a simple string as argument, we have to use
%    a trick to delete the optional escape character.
%
%    To convert a control sequence to a string, we use the |\string|
%    primitive.  Next we have to look at the first character of this
%    string and compare it with the escape character.  Because this
%    escape character can be changed by setting the internal integer
%    |\escapechar| to a character number, we have to compare this
%    number with the character of the string.  To do this we have to
%    use \TeX's backquote notation to specify the character as a
%    number.
%
%    If the first character of the |\string|'ed argument is the
%    current escape character, the comparison has stripped this
%    character and the rest in the `then' part consists of the rest of
%    the control sequence name.  Otherwise we know that either the
%    argument is not a control sequence or |\escapechar| is set to a
%    value outside of the character range~$0$--$255$.
%
%    If the user gives an empty argument, we provide a default
%    argument for |\string|.  This argument should expand to nothing.
%
% \changes{babel~3.0c}{1991/06/06}{Made \cs{selectlanguage}
%    robust}
% \changes{babel~3.2}{1991/11/11}{Modified to allow arguments that
%    start with an escape character}
% \changes{babel~3.2a}{1991/11/17}{Simplified the modification to
%    allow the use in a \cs{write} command}
% \changes{babel~3.5b}{1995/05/13}{Store the name of the current
%    language in a control sequence instead of passing the whole macro
%    construct to strip the escape character in the argument of
%    \cs{selectlanguage }.}
% \changes{babel~3.5f}{1995/08/30}{Added a missing percent character}
% \changes{babel~3.5f}{1995/11/16}{Moved check for escape character
%    one level down in the expansion}
%    \begin{macrocode}
\edef\selectlanguage{%
  \noexpand\protect
  \expandafter\noexpand\csname selectlanguage \endcsname
  }
%    \end{macrocode}
%    Because the command |\selectlanguage| could be used in a moving
%    argument it expands to \verb*=\protect\selectlanguage =.
%    Therefor, we have to make sure that a macro |\protect| exists.
%    If it doesn't it is |\let| to |\relax|.
%    \begin{macrocode}
\ifx\@undefined\protect\let\protect\relax\fi
%    \end{macrocode}
%    As \LaTeX$\:$2.09 writes to files \textit{expanded} whereas
%    \LaTeXe\ takes care \textit{not} to expand the arguments of
%    |\write| statements we need to be a bit clever about the way we
%    add information to \file{.aux} files. Therefor we introduce the
%    macro |\xstring| which should expand to the right amount of
%    |\string|'s.
%    \begin{macrocode}
\ifx\documentclass\@undefined
  \def\xstring{\string\string\string}
\else
  \let\xstring\string
\fi
%    \end{macrocode}
%
% \changes{babel~3.5b}{1995/03/04}{Changed the name of the internal
%    macro to \cs{selectlanguage }.}
% \changes{babel~3.5b}{1995/03/05}{Added an extra level of expansion to
%    separate the switching mechanism from writing to aux files}
% \changes{babel~3.7f}{2000/09/25}{Use \cs{aftergroup} to keep the
%    language grouping correct in auxiliary files {PR3091}}
%    Since version 3.5 \babel\ writes entries to the auxiliary files in
%    order to typeset table of contents etc. in the correct language
%    environment.
%  \begin{macro}{\bbl@pop@language}
%    \emph{But} when the language change happens \emph{inside} a group
%    the end of the group doesn't write anything to the auxiliary
%    files. Therefor we need \TeX's |aftergroup| mechanism to help
%    us. The command |\aftergroup| stores the token immediately
%    following it to be executed when the current group is closed. So
%    we define a temporary control sequence |\bbl@pop@language| to be
%    executed at the end of the group. It calls |\bbl@set@language|
%    with the name of the current language as its argument.
%
% \changes{babel~3.7j}{2003/03/18}{Introduce the language stack
%    mechanism}
%  \begin{macro}{\bbl@language@stack}
%    The previous solution works for one level of nesting groups, but
%    as soon as more levels are used it is no longer adequate. For
%    that case we need to keep track of the nested languages using a
%    stack mechanism. This stack is called |\bbl@language@stack| and
%    initially empty.
%    \begin{macrocode}
\xdef\bbl@language@stack{}
%    \end{macrocode}
%    When using a stack we need a mechanism to push an element on the
%    stack and to retrieve the information afterwards.
%  \begin{macro}{\bbl@push@language}
%  \begin{macro}{\bbl@pop@language}
%    The stack is simply a list of languagenames, separated with a `+'
%    sign; the push function can be simple:
%    \begin{macrocode}
\def\bbl@push@language{%
  \xdef\bbl@language@stack{\languagename+\bbl@language@stack}%
  }
%    \end{macrocode}
%    Retrieving information from the stack is a little bit less simple,
%    as we need to remove the element from the stack while storing it
%    in the macro |\languagename|. For this we first define a helper function.
%  \begin{macro}{\bbl@pop@lang}
%    This macro stores its first element (which is delimited by the
%    `+'-sign) in |\languagename| and stores the rest of the string
%    (delimited by `-') in its third argument.
%    \begin{macrocode}
\def\bbl@pop@lang#1+#2-#3{%
  \def\languagename{#1}\xdef#3{#2}%
  }
%    \end{macrocode}
%  \end{macro}
%    The reason for the somewhat weird arrangement of arguments to the
%    helper function is the fact it is called in the following way:
%    \begin{macrocode}
\def\bbl@pop@language{%
  \expandafter\bbl@pop@lang\bbl@language@stack-\bbl@language@stack
%    \end{macrocode}
%    This means that before |\bbl@pop@lang| is executed \TeX\ first
%    \emph{expands} the stack, stored in |\bbl@language@stack|. The
%    result of that is that the argument string of |\bbl@pop@lang|
%    contains one or more language names, each followed by a `+'-sign
%    (zero language names won't occur as this macro will only be
%    called after something has been pushed on the stack) followed by
%    the `-'-sign and finally the reference to the stack.
%    \begin{macrocode}$$
  \expandafter\bbl@set@language\expandafter{\languagename}%
  }
%    \end{macrocode}
%    Once the name of the previous language is retrieved from the stack,
%    it is fed to |\bbl@set@language| to do the actual work of
%    switching everything that needs switching.
%  \end{macro}
%  \end{macro}
%  \end{macro}
%
% \changes{babel~3.7j}{2003/03/18}{Now use the language stack mechanism}
%    \begin{macrocode}
\expandafter\def\csname selectlanguage \endcsname#1{%
  \bbl@push@language
  \aftergroup\bbl@pop@language
  \bbl@set@language{#1}}
%    \end{macrocode}
% \changes{babel~3.7m}{2003/11/12}{Removed the superfluous empty
%    definition of \cs{bbl@pop@language}}
%  \end{macro}
%
%  \begin{macro}{\bbl@set@language}
% \changes{babel~3.5f}{1995/11/16}{Now also define \cs{languagename}
%    at this level}
% \changes{babel~3.7f}{2000/09/25}{Macro \cs{bbl@set@language}
%    introduced}
%    The macro |\bbl@set@language| takes care of switching the language
%    environment \emph{and} of writing entries on the auxiliary files.
%    \begin{macrocode}
\def\bbl@set@language#1{%
  \edef\languagename{%
    \ifnum\escapechar=\expandafter`\string#1\@empty
    \else \string#1\@empty\fi}%
  \select@language{\languagename}%
%    \end{macrocode}
%    We also write a command to change the current language in the
%    auxiliary files.
% \changes{babel~3.5a}{1995/02/17}{write the language change to the
%    auxiliary files}
%    \begin{macrocode}
  \if@filesw
    \protected@write\@auxout{}{\string\select@language{\languagename}}%
    \addtocontents{toc}{\xstring\select@language{\languagename}}%
    \addtocontents{lof}{\xstring\select@language{\languagename}}%
    \addtocontents{lot}{\xstring\select@language{\languagename}}%
  \fi}
%    \end{macrocode}
%  \end{macro}
%
%    First, check if the user asks for a known language. If so,
%    update the value of |\language| and call |\originalTeX|
%    to bring \TeX\ in a certain pre-defined state.
% \changes{babel~3.0a}{1991/05/29}{Added \cs{@bsphack} and
%    \cs{@esphack}}
% \changes{babel~3.0d}{1991/08/08}{Removed superfluous
%    \cs{expandafter}}
% \changes{babel~3.0d}{1991/10/07}{Removed space hacks and use of
%    \cs{@ifundefined}}
% \changes{babel~3.2a}{1991/11/17}{Added \cs{relax} as first command
%    to stop an expansion if \cs{protect} is empty}
% \changes{babel~3.6a}{1996/11/07}{Check for the existence of
%    \cs{date...} instead of \cs{l@...}}
% \changes{babel~3.7m}{2003/11/16}{Check for the existence of both
%    \cs{l@...} and \cs{date...}}
% \changes{babel~3.8l}{2008/07/06}{Use \cs{bbl@patterns}}
%    \begin{macrocode}
\def\select@language#1{%
  \expandafter\ifx\csname l@#1\endcsname\relax
    \@nolanerr{#1}%
  \else
    \expandafter\ifx\csname date#1\endcsname\relax
      \@noopterr{#1}%
    \else
      \bbl@patterns{\languagename}%
      \originalTeX
%    \end{macrocode}
%    The name of the language is stored in the control sequence
%    |\languagename|. The contents of this control sequence could be
%    tested in the following way:
%  \begin{verbatim}
%    \edef\tmp{\string english}
%    \ifx\languagename\tmp
%        ...
%    \else
%        ...
%    \fi
%  \end{verbatim}
%    The construction with |\string| is necessary because
%    |\languagename| returns the name with characters of category code
%    \texttt{12} (other).  Then we have to \emph{re}define
%    |\originalTeX| to compensate for the things that have been
%    activated.  To save memory space for the macro definition of
%    |\originalTeX|, we construct the control sequence name for the
%    |\noextras|\langvar\ command at definition time by expanding the
%    |\csname| primitive.
% \changes{babel~3.0a}{1991/06/06}{Replaced \cs{gdef} with \cs{def}}
% \changes{babel~3.1}{1991/10/31}{\cs{originalTeX} should only be
%    executed once}
% \changes{babel~3.2a}{1991/11/17}{Added three \cs{expandafter}s
%    to save macro space for \cs{originalTeX}}
% \changes{babel~3.2a}{1991/11/20}{Moved definition of
%    \cs{originalTeX} before \cs{extras\langvar}}
% \changes{babel~3.2a}{1991/11/24}{Set \cs{originalTeX} to
%    \cs{empty}, because it should be expandable.}
%    \begin{macrocode}
      \expandafter\def\expandafter\originalTeX
          \expandafter{\csname noextras#1\endcsname
                       \let\originalTeX\@empty}%
%    \end{macrocode}
% \changes{babel~3.6d}{1997/01/07}{set the language shorthands to
%    `none' before switching on the extras}
%    \begin{macrocode}
      \languageshorthands{none}%
      \babel@beginsave
%    \end{macrocode}
%    Now activate the language-specific definitions. This is done by
%    constructing the names of three macros by concatenating three
%    words with the argument of |\selectlanguage|, and calling these
%    macros.
% \changes{babel~3.5b}{1995/05/13}{Separated the setting of the
%    hyphenmin values}
%    \begin{macrocode}
      \csname captions#1\endcsname
      \csname date#1\endcsname
      \csname extras#1\endcsname\relax
%    \end{macrocode}
%    The switching of the values of |\lefthyphenmin| and
%    |\righthyphenmin| is somewhat different. First we save their
%    current values, then we check if |\|\langvar|hyphenmins| is
%    defined. If it is not, we set default values (2 and 3), otherwise
%    the values in |\|\langvar|hyphenmins| will be used.
% \changes{babel~3.5b}{1995/06/05}{Addedd default setting of hyphenmin
%    parameters}
%    \begin{macrocode}
      \babel@savevariable\lefthyphenmin
      \babel@savevariable\righthyphenmin
      \expandafter\ifx\csname #1hyphenmins\endcsname\relax
        \set@hyphenmins\tw@\thr@@\relax
      \else
        \expandafter\expandafter\expandafter\set@hyphenmins
          \csname #1hyphenmins\endcsname\relax
      \fi
    \fi
  \fi}
%    \end{macrocode}
%  \end{macro}
%
%  \begin{environment}{otherlanguage}
%    The \Lenv{otherlanguage} environment can be used as an
%    alternative to using the |\selectlanguage| declarative
%    command. When you are typesetting a document which mixes
%    left-to-right and right-to-left typesetting you have to use this
%    environment in order to let things work as you expect them to.
%
%    The first thing this environment does is store the name of the
%    language in |\languagename|; it then calls
%    \verb*=\selectlanguage = to switch on everything that is needed for
%    this language The |\ignorespaces| command is necessary to hide
%    the environment when it is entered in horizontal mode.
% \changes{babel~3.5d}{1995/06/22}{environment added}
% \changes{babel~3.5e}{1995/07/07}{changed name}
% \changes{babel~3.7j}{2003/03/18}{rely on \cs{selectlanguage } to
%    keep track of the nesting}
%    \begin{macrocode}
\long\def\otherlanguage#1{%
  \csname selectlanguage \endcsname{#1}%
  \ignorespaces
  }
%    \end{macrocode}
%    The |\endotherlanguage| part of the environment calls
%    |\originalTeX| to restore (most of) the settings and tries to
%    hide itself when it is called in horizontal mode.
%    \begin{macrocode}
\long\def\endotherlanguage{%
  \originalTeX
  \global\@ignoretrue\ignorespaces
  }
%    \end{macrocode}
%  \end{environment}
%
%
%  \begin{environment}{otherlanguage*}
%    The \Lenv{otherlanguage} environment is meant to be used when a
%    large part of text from a different language needs to be typeset,
%    but without changing the translation of words such as `figure'.
%
%    This environment makes use of |\foreign@language|.
% \changes{babel~3.5f}{1996/05/29}{environment added}
% \changes{babel~3.6d}{1997/01/07}{Introduced \cs{foreign@language}}
%    \begin{macrocode}
\expandafter\def\csname otherlanguage*\endcsname#1{%
  \foreign@language{#1}%
  }
%    \end{macrocode}
%    At the end of the environment we need to switch off the extra
%    definitions. The grouping mechanism of the environment will take
%    care of resetting the correct hyphenation rules.
%    \begin{macrocode}
\expandafter\def\csname endotherlanguage*\endcsname{%
  \csname noextras\languagename\endcsname
  }
%    \end{macrocode}
%  \end{environment}
%
%  \begin{macro}{\foreignlanguage}
%    The |\foreignlanguage| command is another substitute for the
%    |\selectlanguage| command. This command takes two arguments, the
%    first argument is the name of the language to use for typesetting
%    the text specified in the second argument.
%
%    Unlike |\selectlanguage| this command doesn't switch
%    \emph{everything}, it only switches the hyphenation rules and the
%    extra definitions for the language specified. It does this within
%    a group and assumes the |\extras|\langvar\ command doesn't make
%    any |\global| changes. The coding is very similar to part of
%    |\selectlanguage|.
% \changes{babel~3.5d}{1995/06/22}{Macro added}
% \changes{babel~3.6d}{1997/01/07}{Introduced \cs{foreign@language}}
% \changes{babel~3.7a}{1998/03/12}{Added executing \cs{originalTeX}}
%    \begin{macrocode}
\def\foreignlanguage{\protect\csname foreignlanguage \endcsname}
\expandafter\def\csname foreignlanguage \endcsname#1#2{%
  \begingroup
    \originalTeX
    \foreign@language{#1}%
    #2%
    \csname noextras#1\endcsname
  \endgroup
  }
%    \end{macrocode}
%  \end{macro}
%
%  \begin{macro}{\foreign@language}
% \changes{babel~3.6d}{1997/01/07}{New macro}
%    This macro does the work for |\foreignlanguage| and the
%    \Lenv{otherlanguage*} environment.
%    \begin{macrocode}
\def\foreign@language#1{%
%    \end{macrocode}
%    First we need to store the name of the language and check that it
%    is a known language.
%    \begin{macrocode}
  \def\languagename{#1}%
  \expandafter\ifx\csname l@#1\endcsname\relax
    \@nolanerr{#1}%
  \else
%    \end{macrocode}
%    If it is we can select the proper hyphenation table and switch on
%    the extra definitions for this language.
% \changes{babel~3.6d}{1997/01/07}{set the language shorthands to
%    `none' before switching on the extras}
% \changes{babel~3.8l}{2008/07/06}{use \cs{bbl@patterns}}
%    \begin{macrocode}
    \bbl@patterns{\languagename}%
    \languageshorthands{none}%
%    \end{macrocode}
%    Then we set the left- and right hyphenmin variables.
% \changes{babel~3.6d}{1997/01/07}{Added \cs{relax} to prevent
%    disappearance of the first token after this command.}
%    \begin{macrocode}
    \csname extras#1\endcsname
    \expandafter\ifx\csname #1hyphenmins\endcsname\relax
      \set@hyphenmins\tw@\thr@@\relax
    \else
      \expandafter\expandafter\expandafter\set@hyphenmins
        \csname #1hyphenmins\endcsname\relax
    \fi
  \fi
  }
%    \end{macrocode}
%  \end{macro}
%
%  \begin{macro}{\bbl@patterns}
% \changes{babel~3.8l}{2008/07/06}{Macro added}
%    This macro selects the hyphenation patterns by changing the
%    \cs{language} register.  If special hyphenation patterns
%    are available specifically for the current font encoding,
%    use them instead of the default.
%    \begin{macrocode}
\def\bbl@patterns#1{%
  \language=\expandafter\ifx\csname l@#1:\f@encoding\endcsname\relax
    \csname l@#1\endcsname
  \else
    \csname l@#1:\f@encoding\endcsname
  \fi\relax
}
%    \end{macrocode}
%  \end{macro}
%
%  \begin{environment}{hyphenrules}
% \changes{babel~3.7e}{2000/01/28}{Added environment hyphenrules}
%    The environment \Lenv{hyphenrules} can be used to select
%    \emph{just} the hyphenation rules. This environment does
%    \emph{not} change |\languagename| and when the hyphenation rules
%    specified were not loaded it has no effect.
% \changes{babel~3.8j}{2008/03/16}{Also set the hyphenmin paramters to
%    the correct value (PR3997)} 
% \changes{babel~3.8l}{2008/07/06}{Use \cs{bbl@patterns}}
%    \begin{macrocode}
\def\hyphenrules#1{%
  \expandafter\ifx\csname l@#1\endcsname\@undefined
    \@nolanerr{#1}%
  \else
    \bbl@patterns{#1}%
    \languageshorthands{none}%
       \expandafter\ifx\csname #1hyphenmins\endcsname\relax
         \set@hyphenmins\tw@\thr@@\relax
       \else
         \expandafter\expandafter\expandafter\set@hyphenmins
         \csname #1hyphenmins\endcsname\relax
       \fi
  \fi
  }
\def\endhyphenrules{}
%    \end{macrocode}
%  \end{environment}
%
%  \begin{macro}{\providehyphenmins}
% \changes{babel~3.7f}{2000/02/18}{added macro}
%    The macro |\providehyphenmins| should be used in the language
%    definition files to provide a \emph{default} setting for the
%    hyphenation parameters |\lefthyphenmin| and |\righthyphenmin|. If
%    the macro |\|\langvar|hyphenmins| is already defined this command
%    has no effect.
%    \begin{macrocode}
\def\providehyphenmins#1#2{%
  \expandafter\ifx\csname #1hyphenmins\endcsname\relax
    \@namedef{#1hyphenmins}{#2}%
  \fi}
%    \end{macrocode}
%  \end{macro}
%
%  \begin{macro}{\set@hyphenmins}
%    This macro sets the values of |\lefthyphenmin| and
%    |\righthyphenmin|. It expects two values as its argument.
%    \begin{macrocode}
\def\set@hyphenmins#1#2{\lefthyphenmin#1\righthyphenmin#2}
%    \end{macrocode}
%  \end{macro}
%
%  \begin{macro}{\LdfInit}
% \changes{babel~3.6a}{1996/10/16}{Macro added}
%    This macro is defined in two versions. The first version is to be
%    part of the `kernel' of \babel, ie. the part that is loaded in
%    the format; the second version is defined in \file{babel.def}.
%    The version in the format just checks the category code of the
%    ampersand and then loads \file{babel.def}.
%    \begin{macrocode}
\def\LdfInit{%
  \chardef\atcatcode=\catcode`\@
  \catcode`\@=11\relax
  \input babel.def\relax
%    \end{macrocode}
%    The category code of the ampersand is restored and the macro
%    calls itself again with the new definition from
%    \file{babel.def}
%    \begin{macrocode}
  \catcode`\@=\atcatcode \let\atcatcode\relax
  \LdfInit}
%</kernel>
%    \end{macrocode}
%    The second version of this macro takes two arguments. The first
%    argument is the name of the language that will be defined in the
%    language definition file; the second argument is either a control
%    sequence or a string from which a control sequence should be
%    constructed. The existence of the control sequence indicates that
%    the file has been processed before.
%
%    At the start of processing a language definition file we always
%    check the category code of the ampersand. We make sure that it is
%    a `letter' during the processing of the file.
%    \begin{macrocode}
%<*core>
\def\LdfInit#1#2{%
  \chardef\atcatcode=\catcode`\@
  \catcode`\@=11\relax
%    \end{macrocode}
%    Another character that needs to have the correct category code
%    during processing of language definition files is the equals sign,
%    `=', because it is sometimes used in constructions with the
%    |\let| primitive. Therefor we store its current catcode and
%    restore it later on.
% \changes{babel~3.7o}{2003/11/26}{make sure the equals sign has its
%    default category code}
%    \begin{macrocode}
  \chardef\eqcatcode=\catcode`\=
  \catcode`\==12\relax
%    \end{macrocode}
%    Now we check whether we should perhaps stop the processing of
%    this file. To do this we first need to check whether the second
%    argument that is passed to |\LdfInit| is a control sequence. We
%    do that by looking at the first token after passing |#2| through
%    |string|. When it is equal to |\@backslashchar| we are dealing
%    with a control sequence which we can compare with |\@undefined|.
%    \begin{macrocode}
  \let\bbl@tempa\relax
  \expandafter\if\expandafter\@backslashchar
                 \expandafter\@car\string#2\@nil
    \ifx#2\@undefined
    \else
%    \end{macrocode}
%    If so, we call |\ldf@quit| (but after the end of this |\if|
%    construction) to set the main language, restore the category code
%    of the @-sign and call |\endinput|.
%    \begin{macrocode}
      \def\bbl@tempa{\ldf@quit{#1}}
    \fi
  \else
%    \end{macrocode}
%    When |#2| was \emph{not} a control sequence we construct one and
%    compare it with |\relax|.
%    \begin{macrocode}
    \expandafter\ifx\csname#2\endcsname\relax
    \else
      \def\bbl@tempa{\ldf@quit{#1}}
    \fi
  \fi
  \bbl@tempa
%    \end{macrocode}
%    Finally we check |\originalTeX|.
%    \begin{macrocode}
  \ifx\originalTeX\@undefined
    \let\originalTeX\@empty
  \else
    \originalTeX
  \fi}
%    \end{macrocode}
%  \end{macro}
%
%  \begin{macro}{\ldf@quit}
% \changes{babel~3.6a}{1996/10/29}{Macro added}
%    This macro interrupts the processing of a language definition file.
% \changes{babel~3.7o}{2003/11/26}{Also restore the category code of
%    the equals sign}
%    \begin{macrocode}
\def\ldf@quit#1{%
  \expandafter\main@language\expandafter{#1}%
  \catcode`\@=\atcatcode \let\atcatcode\relax
  \catcode`\==\eqcatcode \let\eqcatcode\relax
  \endinput
}
%    \end{macrocode}
%  \end{macro}
%
%  \begin{macro}{\ldf@finish}
% \changes{babel~3.6a}{1996/10/16}{Macro added}
%    This macro takes one argument. It is the name of the language
%    that was defined in the language definition file.
%
%    We load the local configuration file if one is present, we set
%    the main language (taking into account that the argument might be
%    a control sequence that needs to be expanded) and reset the
%    category code of the @-sign.
% \changes{babel~3.7o}{2003/11/26}{Also restore the category code of
%    the equals sign}
%    \begin{macrocode}
\def\ldf@finish#1{%
  \loadlocalcfg{#1}%
  \expandafter\main@language\expandafter{#1}%
  \catcode`\@=\atcatcode \let\atcatcode\relax
  \catcode`\==\eqcatcode \let\eqcatcode\relax
  }
%    \end{macrocode}
%  \end{macro}
%
%    After the preamble of the document the commands |\LdfInit|,
%    |\ldf@quit| and |\ldf@finish| are no longer needed. Therefor
%    they are turned into warning messages in \LaTeX.
%    \begin{macrocode}
\@onlypreamble\LdfInit
\@onlypreamble\ldf@quit
\@onlypreamble\ldf@finish
%    \end{macrocode}
%
%  \begin{macro}{\main@language}
% \changes{babel~3.5a}{1995/02/17}{Macro added}
% \changes{babel~3.6a}{1996/10/16}{\cs{main@language} now also sets
%    \cs{languagename} and \cs{l@languagename} for use by other
%    packages in the preamble of a document}
%  \begin{macro}{\bbl@main@language}
% \changes{babel~3.5a}{1995/02/17}{Macro added}
%    This command should be used in the various language definition
%    files. It stores its argument in |\bbl@main@language|; to be used
%    to switch to the correct language at the beginning of the
%    document.
% \changes{babel~3.8l}{2008/07/06}{Use \cs{bbl@patterns}}
%    \begin{macrocode}
\def\main@language#1{%
  \def\bbl@main@language{#1}%
  \let\languagename\bbl@main@language
  \bbl@patterns{\languagename}%
  }
%    \end{macrocode}
%    The default is to use English as the main language.
% \changes{babel~3.6c}{1997/01/05}{When \file{hyphen.cfg} is not
%    loaded in the format \cs{l@english} might not be defined; assume
%    english is language 0}
%    \begin{macrocode}
\ifx\l@english\@undefined
  \let\l@english\z@
\fi
\main@language{english}
%    \end{macrocode}
%    We also have to make sure that some code gets executed at the
%    beginning of the document.
%    \begin{macrocode}
\AtBeginDocument{%
  \expandafter\selectlanguage\expandafter{\bbl@main@language}}
%</core>
%    \end{macrocode}
%  \end{macro}
%  \end{macro}
%
%  \begin{macro}{\originalTeX}
%    The macro|\originalTeX| should be known to \TeX\ at this moment.
%    As it has to be expandable we |\let| it to |\@empty| instead of
%    |\relax|.
% \changes{babel~3.2a}{1991/11/24}{Set \cs{originalTeX} to
%    \cs{empty}, because it should be expandable.}
%    \begin{macrocode}
%<*kernel>
\ifx\originalTeX\@undefined\let\originalTeX\@empty\fi
%    \end{macrocode}
%    Because this part of the code can be included in a format, we
%    make sure that the macro which initialises the save mechanism,
%    |\babel@beginsave|, is not considered to be undefined.
%    \begin{macrocode}
\ifx\babel@beginsave\@undefined\let\babel@beginsave\relax\fi
%    \end{macrocode}
%  \end{macro}
%
%  \begin{macro}{\@nolanerr}
% \changes{babel~3.4e}{1994/06/25}{Use \cs{PackageError} in \LaTeXe\
%    mode}
%  \begin{macro}{\@nopatterns}
% \changes{babel~3.4e}{1994/06/25}{Macro added}
%    The \babel\ package will signal an error when a documents tries
%    to select a language that hasn't been defined earlier. When a
%    user selects a language for which no hyphenation patterns were
%    loaded into the format he will be given a warning about that
%    fact. We revert to the patterns for |\language|=0 in that case.
%    In most formats that will be (US)english, but it might also be
%    empty.
%  \begin{macro}{\@noopterr}
% \changes{babel~3.7m}{2003/11/16}{Macro added}
%    When the package was loaded without options not everything will
%    work as expected. An error message is issued in that case.
%
%    When the format knows about |\PackageError| it must be \LaTeXe,
%    so we can safely use its error handling interface. Otherwise
%    we'll have to `keep it simple'.
% \changes{babel~3.0d}{1991/10/07}{Added a percent sign to remove
%    unwanted white space}
% \changes{babel~3.5a}{1995/02/15}{Added \cs{@activated} to log active
%    characters}
% \changes{babel~3.5c}{1995/06/19}{Added missing closing brace}
%    \begin{macrocode}
\ifx\PackageError\@undefined
  \def\@nolanerr#1{%
    \errhelp{Your command will be ignored, type <return> to proceed}%
    \errmessage{You haven't defined the language #1\space yet}}
  \def\@nopatterns#1{%
    \message{No hyphenation patterns were loaded for}%
    \message{the language `#1'}%
    \message{I will use the patterns loaded for \string\language=0
          instead}}
  \def\@noopterr#1{%
    \errmessage{The option #1 was not specified in \string\usepackage}
    \errhelp{You may continue, but expect unexpected results}}
  \def\@activated#1{%
    \wlog{Package babel Info: Making #1 an active character}}
\else
  \newcommand*{\@nolanerr}[1]{%
    \PackageError{babel}%
                 {You haven't defined the language #1\space yet}%
        {Your command will be ignored, type <return> to proceed}}
  \newcommand*{\@nopatterns}[1]{%
    \PackageWarningNoLine{babel}%
        {No hyphenation patterns were loaded for\MessageBreak
          the language `#1'\MessageBreak
          I will use the patterns loaded for \string\language=0
          instead}}
  \newcommand*{\@noopterr}[1]{%
    \PackageError{babel}%
                 {You haven't loaded the option #1\space yet}%
             {You may proceed, but expect unexpected results}}
  \newcommand*{\@activated}[1]{%
    \PackageInfo{babel}{%
      Making #1 an active character}}
\fi
%    \end{macrocode}
%  \end{macro}
%  \end{macro}
%  \end{macro}
%
%    The following code is meant to be read by ini\TeX\ because it
%    should instruct \TeX\ to read hyphenation patterns. To this end
%    the \texttt{docstrip} option \texttt{patterns} can be used to
%    include this code in the file \file{hyphen.cfg}.
%    \begin{macrocode}
%<*patterns>
%    \end{macrocode}
%
% \changes{babel~3.5g}{1996/07/09}{Removed the use of
%    \cs{patterns@loaded} altogether}
%
%  \begin{macro}{\process@line}
% \changes{babel~3.5b}{1995/04/28}{added macro}
%    Each line in the file \file{language.dat} is processed by
%    |\process@line| after it is read. The first thing this macro does
%    is to check whether the line starts with \texttt{=}.
%    When the first token of a line is an \texttt{=}, the macro
%    |\process@synonym| is called; otherwise the macro
%    |\process@language| will continue.
% \changes{babel~3.5g}{1996/07/09}{Simplified code, removing
%    \cs{bbl@eq@}}
% \changes{babel~3.7c}{1999/04/09}{added an extra argument in order to
%    prevent a trailing space from becoming part of the control
%    sequence when defining a synonym (PR 2851)}
%    \begin{macrocode}
\def\process@line#1#2 #3/{%
  \ifx=#1
    \process@synonym#2 /
  \else
    \process@language#1#2 #3/%
  \fi
  }
%    \end{macrocode}
%  \end{macro}
%
%  \begin{macro}{\process@synonym}
% \changes{babel~3.5b}{1995/04/28}{added macro}
%    This macro takes care of the lines which start with an
%    \texttt{=}. It needs an empty token register to begin with.
%    \begin{macrocode}
\toks@{}
\def\process@synonym#1 /{%
  \ifnum\last@language=\m@ne
%    \end{macrocode}
%    When no languages have been loaded yet, the name following the
%    \texttt{=} will be a synonym for hyphenation register 0.
%    \begin{macrocode}
    \expandafter\chardef\csname l@#1\endcsname0\relax
    \wlog{\string\l@#1=\string\language0}
%    \end{macrocode}
%    As no hyphenation patterns are read in yet, we can not yet set
%    the hyphenmin parameters. Therefor a commands to do so is stored
%    in a token register and executed when the first pattern file has
%    been processed.
% \changes{babel~3.7c}{1999/04/27}{Use a token register to temporarily
%    store a command to set hyphenmin parameters for the synonym which
%    is defined \emph{before} the first pattern file is processed}
%    \begin{macrocode}
    \toks@\expandafter{\the\toks@
      \expandafter\let\csname #1hyphenmins\expandafter\endcsname
      \csname\languagename hyphenmins\endcsname}%
  \else
%    \end{macrocode}
%    Otherwise the name will be a synonym for the language loaded last.
%    \begin{macrocode}
    \expandafter\chardef\csname l@#1\endcsname\last@language
    \wlog{\string\l@#1=\string\language\the\last@language}
%    \end{macrocode}
%    We also need to copy the hyphenmin parameters for the synonym.
% \changes{babel~3.7c}{1999/04/22}{Now also store hyphenmin parameters
%    for language synonyms}
%    \begin{macrocode}
    \expandafter\let\csname #1hyphenmins\expandafter\endcsname
    \csname\languagename hyphenmins\endcsname
  \fi
  }
%    \end{macrocode}
%  \end{macro}
%
%  \begin{macro}{\process@language}
%    The macro |\process@language| is used to process a non-empty line
%    from the `configuration file'. It has three arguments, each
%    delimited by white space. The third argument is optional,
%    so a |/| character is expected to delimit the last
%    argument.  The first argument is the `name' of a language; the
%    second is the name of the file that contains the patterns. The
%    optional third argument is the name of a file containing
%    hyphenation exceptions.
%
%    The first thing to do is call |\addlanguage| to allocate a
%    pattern register and to make that register `active'.
% \changes{babel~3.0d}{1991/08/08}{Removed superfluous
%    \cs{expandafter}}
% \changes{babel~3.0d}{1991/08/21}{Reinserted \cs{expandafter}}
% \changes{babel~3.0d}{1991/10/27}{Added the collection of pattern
%    names.}
% \changes{babel~3.7c}{1999/04/22}{Also store \cs{languagename} for
%    possible later use in \cs{process@synonym}}
%    \begin{macrocode}
\def\process@language#1 #2 #3/{%
  \expandafter\addlanguage\csname l@#1\endcsname
  \expandafter\language\csname l@#1\endcsname
  \def\languagename{#1}%
%    \end{macrocode}
%    Then the `name' of the language that will be loaded now is
%    added to the token register |\toks8|. and finally
%    the pattern file is read.
%    \begin{macrocode}
  \global\toks8\expandafter{\the\toks8#1, }%
%    \end{macrocode}
% \changes{babel~3.7f}{2000/02/18}{Allow for the encoding to be used
%    as part of the language name} 
%    For some hyphenation patterns it is needed to load them with a
%    specific font encoding selected. This can be specified in the
%    file \file{language.dat} by adding for instance `\texttt{:T1}' to
%    the name of the language. The macro |\bbl@get@enc| extracts the
%    font encoding from the language name and stores it in
%    |\bbl@hyph@enc|.
%    \begin{macrocode}
  \begingroup
    \bbl@get@enc#1:\@@@
    \ifx\bbl@hyph@enc\@empty
    \else
      \fontencoding{\bbl@hyph@enc}\selectfont
    \fi
%    \end{macrocode}
%
% \changes{babel~3.4e}{1994/06/24}{Added code to detect assignments to
%    left- and righthyphenmin in the patternfile.}
%    Some pattern files contain assignments to |\lefthyphenmin| and
%    |\righthyphenmin|. \TeX\ does not keep track of these
%    assignments. Therefor we try to detect such assignments and
%    store them in the |\|\langvar|hyphenmins| macro. When no
%    assignments were made we provide a default setting.
%    \begin{macrocode}
    \lefthyphenmin\m@ne
%    \end{macrocode}
%    Some pattern files contain changes to the |\lccode| en |\uccode|
%    arrays. Such changes should remain local to the language;
%    therefor we process the pattern file in a group; the |\patterns|
%    command acts globally so its effect will be remembered.
% \changes{babel~3.7a}{1998/03/27}{Read pattern files in a group}
% \changes{babel~3.7c}{1999/04/05}{need to set hyphenmin values
%    globally}
% \changes{babel~3.7c}{1999/04/22}{Set \cs{lefthyphenmin} to \cs{m@ne}
%    \emph{inside} the group; explicitly set the hyphenmin parameters
%    for language 0}
%    \begin{macrocode}
    \input #2\relax
%    \end{macrocode}
%    Now we globally store the settings of |\lefthyphenmin| and
%    |\righthyphenmin| and close the group.
% \changes{babel~3.7c}{1999/04/25}{Only set hyphenmin values when the
%    pattern file changed them}
%    \begin{macrocode}
    \ifnum\lefthyphenmin=\m@ne
    \else
      \expandafter\xdef\csname #1hyphenmins\endcsname{%
        \the\lefthyphenmin\the\righthyphenmin}%
    \fi
  \endgroup
%    \end{macrocode}
%    If the counter |\language| is still equal to zero we set the
%    hyphenmin parameters to the values for the language loaded on
%    pattern register 0.
%    \begin{macrocode}
  \ifnum\the\language=\z@
    \expandafter\ifx\csname #1hyphenmins\endcsname\relax
      \set@hyphenmins\tw@\thr@@\relax
    \else
      \expandafter\expandafter\expandafter\set@hyphenmins
        \csname #1hyphenmins\endcsname
    \fi
%    \end{macrocode}
%    Now execute the contents of token register zero as it may
%    contain commands which set the hyphenmin parameters for synonyms
%    that were defined before the first pattern file is read in.
% \changes{babel~3.7c}{1999/04/27}{Added the execution of the contents
%    of \cs{toks@}}
%    \begin{macrocode}
    \the\toks@
  \fi
%    \end{macrocode}
%    Empty the token register after use.
%    \begin{macrocode}
  \toks@{}%
%    \end{macrocode}
%    When the hyphenation patterns have been processed we need to see
%    if a file with hyphenation exceptions needs to be read. This is
%    the case when the third argument is not empty and when it does
%    not contain a space token.
% \changes{babel~3.5b}{1995/04/28}{Added optional reading of file with
%    hyphenation exceptions}
% \changes{babel~3.5f}{1995/07/25}{Use \cs{empty} instead of
%    \cs{@empty} as the latter is unknown in plain}
%    \begin{macrocode}
  \def\bbl@tempa{#3}%
  \ifx\bbl@tempa\@empty
  \else
    \ifx\bbl@tempa\space
    \else
      \input #3\relax
    \fi
  \fi
  }
%    \end{macrocode}
%
%  \begin{macro}{\bbl@get@enc}
% \changes{babel~3.7f}{2000/02/18}{Added macro}
%  \begin{macro}{\bbl@hyph@enc}
%    The macro |\bbl@get@enc| extracts the font encoding from the
%    language name and stores it in |\bbl@hyph@enc|. It uses delimited
%    arguments to achieve this.
%    \begin{macrocode}
\def\bbl@get@enc#1:#2\@@@{%
%    \end{macrocode}
%    First store both arguments in temporary macros,
%    \begin{macrocode}
  \def\bbl@tempa{#1}%
  \def\bbl@tempb{#2}%
%    \end{macrocode}
%    then, if the second argument was empty, no font encoding was
%    specified and we're done.
%    \begin{macrocode}
  \ifx\bbl@tempb\@empty
    \let\bbl@hyph@enc\@empty
  \else
%    \end{macrocode}
%    But if the second argument was \emph{not} empty it will now have
%    a superfluous colon attached to it which we need to remove. This
%    done by feeding it to |\bbl@get@enc|. The string that we are
%    after will then be in the first argument and be stored in
%    |\bbl@tempa|.
%    \begin{macrocode}
    \bbl@get@enc#2\@@@
    \edef\bbl@hyph@enc{\bbl@tempa}%
  \fi}
%    \end{macrocode}
%  \end{macro}
%  \end{macro}
%  \end{macro}
%
%  \begin{macro}{\readconfigfile}
%    The configuration file can now be opened for reading.
%    \begin{macrocode}
\openin1 = language.dat
%    \end{macrocode}
%
%    See if the file exists, if not, use the default hyphenation file
%    \file{hyphen.tex}. The user will be informed about this.
%
%    \begin{macrocode}
\ifeof1
  \message{I couldn't find the file language.dat,\space
           I will try the file hyphen.tex}
  \input hyphen.tex\relax
\else
%    \end{macrocode}
%
%    Pattern registers are allocated using count register
%    |\last@language|. Its initial value is~0. The definition of the
%    macro |\newlanguage| is such that it first increments the count
%    register and then defines the language. In order to have the
%    first patterns loaded in pattern register number~0 we initialize
%    |\last@language| with the value~$-1$.
%
% \changes{babel~3.1}{1991/05/21}{Removed use of \cs{toks0}}
%    \begin{macrocode}
  \last@language\m@ne
%    \end{macrocode}
%
%    We now read lines from the file until the end is found
%
%    \begin{macrocode}
  \loop
%    \end{macrocode}
%
%    While reading from the input, it is useful to switch off
%    recognition of the end-of-line character. This saves us stripping
%    off spaces from the contents of the control sequence.
%
%    \begin{macrocode}
    \endlinechar\m@ne
    \read1 to \bbl@line
    \endlinechar`\^^M
%    \end{macrocode}
%
%    Empty lines are skipped.
%    \begin{macrocode}
    \ifx\bbl@line\@empty
    \else
%    \end{macrocode}
%
%    Now we add a space and a |/| character to the end of
%    |\bbl@line|. This is needed to be able to recognize the third,
%    optional, argument of |\process@language| later on.
% \changes{babel~3.5b}{1995/04/28}{Now add a \cs{space} and a /
%    character}
%    \begin{macrocode}
      \edef\bbl@line{\bbl@line\space/}%
      \expandafter\process@line\bbl@line
    \fi
%    \end{macrocode}
%
%    Check for the end of the file.  To avoid a new \texttt{if}
%    control sequence we create the necessary |\iftrue| or |\iffalse|
%    with the help of |\csname|.  But there is one complication with
%    this approach: when skipping the \texttt{loop...repeat} \TeX\ has
%    to read |\if|/|\fi| pairs.  So we have to insert a `dummy'
%    |\iftrue|.
% \changes{babel~3.1}{1991/10/31}{Removed the extra \texttt{if}
%    control sequence}
%    \begin{macrocode}
    \iftrue \csname fi\endcsname
    \csname if\ifeof1 false\else true\fi\endcsname
  \repeat
%    \end{macrocode}
%
%    Reactivate the default patterns,
%    \begin{macrocode}
  \language=0
\fi
%    \end{macrocode}
%    and close the configuration file.
% \changes{babel~3.2a}{1991/11/20}{Free macro space for
%    \cs{process@language}}
%    \begin{macrocode}
\closein1
%    \end{macrocode}
%    Also remove some macros from memory
%    \begin{macrocode}
\let\process@language\@undefined
\let\process@synonym\@undefined
\let\process@line\@undefined
\let\bbl@tempa\@undefined
\let\bbl@tempb\@undefined
\let\bbl@eq@\@undefined
\let\bbl@line\@undefined
\let\bbl@get@enc\@undefined
%    \end{macrocode}
%
% \changes{babel~3.5f}{1995/11/08}{Moved the fiddling with \cs{dump}
%     to \file{bbplain.dtx} as it is no longer needed for \LaTeX}
%    We add a message about the fact that babel is loaded in the
%    format and with which language patterns to the \cs{everyjob}
%    register.
% \changes{babel~3.6h}{1997/01/23}{Added a couple of \cs{expandafter}s
%    to copy the contents of \cs{toks8} into \cs{everyjob} instead of
%    the reference}
%    \begin{macrocode}
\ifx\addto@hook\@undefined
\else
  \expandafter\addto@hook\expandafter\everyjob\expandafter{%
    \expandafter\typeout\expandafter{\the\toks8 loaded.}}
\fi
%    \end{macrocode}
%    Here the code for ini\TeX\ ends.
%    \begin{macrocode}
%</patterns>
%</kernel>
%    \end{macrocode}
%  \end{macro}
%
% \subsection{Support for active characters}
%
%  \begin{macro}{\bbl@add@special}
% \changes{babel~3.2}{1991/11/10}{Added macro}
%    The macro |\bbl@add@special| is used to add a new character (or
%    single character control sequence) to the macro |\dospecials|
%    (and |\@sanitize| if \LaTeX\ is used).
%
%    To keep all changes local, we begin a new group.  Then we
%    redefine the macros |\do| and |\@makeother| to add themselves and
%    the given character without expansion.
%    \begin{macrocode}
%<*core|shorthands>
\def\bbl@add@special#1{\begingroup
    \def\do{\noexpand\do\noexpand}%
    \def\@makeother{\noexpand\@makeother\noexpand}%
%    \end{macrocode}
%    To add the character to the macros, we expand the original macros
%    with the additional character inside the redefinition of the
%    macros.  Because |\@sanitize| can be undefined, we put the
%    definition inside a conditional.
%    \begin{macrocode}
    \edef\x{\endgroup
      \def\noexpand\dospecials{\dospecials\do#1}%
      \expandafter\ifx\csname @sanitize\endcsname\relax \else
        \def\noexpand\@sanitize{\@sanitize\@makeother#1}%
      \fi}%
%    \end{macrocode}
%    The macro |\x| contains at this moment the following:\\
%    |\endgroup\def\dospecials{|\textit{old contents}%
%    |\do|\meta{char}|}|.\\
%    If |\@sanitize| is defined, it contains an additional definition
%    of this macro.  The last thing we have to do, is the expansion of
%    |\x|.  Then |\endgroup| is executed, which restores the old
%    meaning of |\x|, |\do| and |\@makeother|.  After the group is
%    closed, the new definition of |\dospecials| (and |\@sanitize|) is
%    assigned.
%    \begin{macrocode}
  \x}
%    \end{macrocode}
%  \end{macro}
%
%  \begin{macro}{\bbl@remove@special}
% \changes{babel~3.2}{1991/11/10}{Added macro}
%    The companion of the former macro is |\bbl@remove@special|.  It
%    is used to remove a character from the set macros |\dospecials|
%    and |\@sanitize|.
%
%    To keep all changes local, we begin a new group.  Then we define
%    a help macro |\x|, which expands to empty if the characters
%    match, otherwise it expands to its nonexpandable input.  Because
%    \TeX\ inserts a |\relax|, if the corresponding |\else| or |\fi|
%    is scanned before the comparison is evaluated, we provide a `stop
%    sign' which should expand to nothing.
%    \begin{macrocode}
\def\bbl@remove@special#1{\begingroup
    \def\x##1##2{\ifnum`#1=`##2\noexpand\@empty
                 \else\noexpand##1\noexpand##2\fi}%
%    \end{macrocode}
%    With the help of this macro we define |\do| and |\make@other|.
%    \begin{macrocode}
    \def\do{\x\do}%
    \def\@makeother{\x\@makeother}%
%    \end{macrocode}
%    The rest of the work is similar to |\bbl@add@special|.
%    \begin{macrocode}
    \edef\x{\endgroup
      \def\noexpand\dospecials{\dospecials}%
      \expandafter\ifx\csname @sanitize\endcsname\relax \else
        \def\noexpand\@sanitize{\@sanitize}%
      \fi}%
  \x}
%    \end{macrocode}
%  \end{macro}
%
%  \subsection{Shorthands}
%
%  \begin{macro}{\initiate@active@char}
% \changes{babel~3.5a}{1995/02/11}{Added macro}
% \changes{babel~3.5b}{1995/03/03}{Renamed macro}
%    A language definition file can call this macro to make a
%    character active. This macro takes one argument, the character
%    that is to be made active. When the character was already active
%    this macro does nothing. Otherwise, this macro defines the
%    control sequence |\normal@char|\m{char} to expand to the
%    character in its `normal state' and it defines the active
%    character to expand to |\normal@char|\m{char} by default
%    (\m{char} being the character to be made active). Later its
%    definition can be changed to expand to |\active@char|\m{char}
%    by calling |\bbl@activate{|\m{char}|}|.
%
%    For example, to make the double quote character active one could
%    have the following line in a language definition file:
%  \begin{verbatim}
%    \initiate@active@char{"}
%  \end{verbatim}
%
%  \begin{macro}{\bbl@afterelse}
%  \begin{macro}{\bbl@afterfi}
%    Because the code that is used in the handling of active
%    characters may need to look ahead, we take extra care to `throw'
%    it over the |\else| and |\fi| parts of an
%    |\if|-statement\footnote{This code is based on code presented in
%    TUGboat vol. 12, no2, June 1991 in ``An expansion Power Lemma''
%    by Sonja Maus.}. These macros will break if another |\if...\fi|
%    statement appears in one of the arguments.
% \changes{babel~3.6i}{1997/02/20}{Made \cs{bbl@afterelse} and
%    \cs{bbl@afterfi} \cs{long}}
%    \begin{macrocode}
\long\def\bbl@afterelse#1\else#2\fi{\fi#1}
\long\def\bbl@afterfi#1\fi{\fi#1}
%    \end{macrocode}
%  \end{macro}
%  \end{macro}
%
% \changes{babel~3.7a}{1997/02/23}{Commented out \c{peek@token} and
%    \cs{test@token} as shorthands are made expandable again}
%
%  \begin{macro}{\peek@token}
% \changes{babel~3.5f}{1995/12/06}{macro added}
% \changes{babel~3.6i}{1998/03/10}{Renamed \cs{test@token} to
%    \cs{bbl@test@token} to prevent a clash with Arab\TeX}
%    To prevent error messages when a shorthand, which
%    normally takes an argument, sees a |\par|, or |}|, or similar
%    tokens, we need to be able to `peek' at what is coming up next in
%    the input stream. Depending on the category code of the token
%    that is seen, we need to either continue the code for the active
%    character, or insert the non-active version of that character in
%    the output. The macro |\peek@token| therefore takes two
%    arguments, with which it constructs the control sequence to
%    expand next. It |\let|'s |\bbl@nexta| and |\bbl@nextb| to the two
%    possible macros. This is necessary for |\bbl@test@token| to take
%    the right decision.
%    \begin{macrocode}
%\def\peek@token#1#2{%
%  \expandafter\let\expandafter\bbl@nexta\csname #1\string#2\endcsname
%  \expandafter\let\expandafter\bbl@nextb
%    \csname system@active\string#2\endcsname
%  \futurelet\bbl@token\bbl@test@token}
%    \end{macrocode}
%
%  \begin{macro}{\bbl@test@token}
% \changes{babel~3.5f}{1995/12/06}{macro added}
% \changes{babel~3.6i}{1998/03/10}{renamed \cs{bbl@token} to
%    \cs{bbl@test@token} to prevent a clash with Arab\TeX}
%    When the result of peeking at the next token has yielded a token
%    with category `letter', `other' or `active' it is safe to proceed
%    with evaluating the code for the shorthand. When a token is found
%    with any other category code proceeding is unsafe and therefor
%    the original shorthand character is inserted in the output. The
%    macro that calls |\bbl@test@token| needs to setup |\bbl@nexta|
%    and |\bbl@nextb| in order to achieve this.
%    \begin{macrocode}
%\def\bbl@test@token{%
%  \let\bbl@next\bbl@nexta
%  \ifcat\noexpand\bbl@token a%
%  \else
%    \ifcat\noexpand\bbl@token=%
%    \else
%      \ifcat\noexpand\bbl@token\noexpand\bbl@next
%      \else
%        \let\bbl@next\bbl@nextb
%      \fi
%    \fi
%  \fi
%  \bbl@next}
%    \end{macrocode}
%  \end{macro}
%  \end{macro}
%
%^^A
%^^A Tekens met mathcode >"8000 zorgen voor problemen.
%^^A hier kan op getest worden door ze catcode13 te geven 
%^^A en te vragen of er een undefined macro ontstaat:
%^^A \ifx#1\undefined{matchcode<"8000}\else get active definition 
%^^A using \let \fi
%^^A
%    The macro |\initiate@active@char| takes all the necessary actions
%    to make its argument a shorthand character. The real work is
%    performed once for each character.
% \changes{babel~3.7c}{1999/04/30}{Only execute
%    \cs{initiate@active@char} once for each character}
%    \begin{macrocode}
\def\initiate@active@char#1{%
  \expandafter\ifx\csname active@char\string##1\endcsname\relax
    \bbl@afterfi{\@initiate@active@char{#1}}%
  \fi}
%    \end{macrocode}
%    Note that the definition of |\@initiate@active@char| needs an
%    active character, for this the |~| is used. Some of the changes
%    we need, do not have to become available later on, so we do it
%    inside a group.
%    \begin{macrocode}
\begingroup
  \catcode`\~\active
  \def\x{\endgroup
    \def\@initiate@active@char##1{%
%    \end{macrocode}
%    If the character is already active we provide the default
%    expansion under this shorthand mechanism.
% \changes{babel~3.5f}{1996/01/09}{Deal correctly with already active
%    characters, provide top level expansion and define all lower
%    level expansion macros outside of the \cs{else} branch.}
% \changes{babel~3.5g}{1996/08/13}{Top level expansion of
%    \cs{normal@char char} where char is already active, should be the
%    expansion of the active character, not the active character
%    itself as this causes an endless loop}
% \changes{babel~3.7d}{1999/08/19}{Make sure the active character
%    doesn't get expanded more then once by the \cs{edef} by adding
%    \cs{expandafter}\cs{strip@prefix}\cs{meaning}}
% \changes{babel~3.7e}{1999/09/06}{previous change was rubbish; use
%    \cs{let} instead of \cs{edef}}
%    \begin{macrocode}
      \ifcat\noexpand##1\noexpand~\relax
        \@ifundefined{normal@char\string##1}{%
          \expandafter\let\csname normal@char\string##1\endcsname##1%
          \expandafter\gdef
            \expandafter##1%
            \expandafter{%
              \expandafter\active@prefix\expandafter##1%
              \csname normal@char\string##1\endcsname}}{}%
      \else
%    \end{macrocode}
%    Otherwise we write a message in the transcript file,
%    \begin{macrocode}
        \@activated{##1}%
%    \end{macrocode}
%    and define |\normal@char|\m{char} to expand to the character in
%    its default state.
%    \begin{macrocode}
        \@namedef{normal@char\string##1}{##1}%
%    \end{macrocode}
%    If we are making the right quote active we need to change
%    |\pr@m@s| as well.
% \changes{babel~3.5a}{1995/03/10}{Added a check for right quote and
%    adapt \cs{pr@m@s} if necessary}
% \changes{babel~3.7f}{1999/12/18}{The redefinition needs to take
%    place one level higher, \cs{prim@s} needs to be redefined.}
%    \begin{macrocode}
        \ifx##1'%
          \let\prim@s\bbl@prim@s
%    \end{macrocode}
%    Also, make sure that a single |'| in math mode `does the right
%    thing'.
% \changes{babel~3.7f}{1999/12/18}{Insert a check for math mode in the
%    definition of \cs{normal@char'}}
% \changes{babel~3.7g}{2000/10/02}{use \cs{textormath} to get rid of
%    the \cs{fi} (PR 3266)}
%    \begin{macrocode}
          \@namedef{normal@char\string##1}{%
            \textormath{##1}{^\bgroup\prim@s}}%
        \fi
%    \end{macrocode}
%    If we are using the caret as a shorthand character special care
%    should be taken to make sure math still works. Therefor an extra
%    level of expansion is introduced with a check for math mode on
%    the upper level.
% \changes{babel~3.7f}{1999/12/18}{Introduced an extra level of
%    expansion in the definition of an active caret}
%    \begin{macrocode}
        \ifx##1^%
          \gdef\bbl@act@caret{%
            \ifmmode
              \csname normal@char\string^\endcsname
            \else
              \bbl@afterfi
              {\if@safe@actives
                \bbl@afterelse\csname normal@char\string##1\endcsname
               \else
                \bbl@afterfi\csname user@active\string##1\endcsname
               \fi}%
            \fi}
        \fi
%    \end{macrocode}
%    To prevent problems with the loading of other packages after
%    \babel\ we reset the catcode of the character at the end of the
%    package.
% \changes{babel~3.5f}{1995/12/01}{Restore the category code of a
%    shorthand char at end of package}
% \changes{babel~3.6f}{1997/01/14}{Made restoring of the category code
%    of shorthand characters optional}
% \changes{babel~3.7a}{1997/03/21}{Use \cs{@ifpackagewith} to
%    determine whether shorthand characters need to remain active}
%    \begin{macrocode}
        \@ifpackagewith{babel}{KeepShorthandsActive}{}{%
          \edef\bbl@tempa{\catcode`\noexpand##1\the\catcode`##1}%
          \expandafter\AtEndOfPackage\expandafter{\bbl@tempa}}%
%    \end{macrocode}
%    Now we set the lowercase code of the |~| equal to that of the
%    character to be made active and execute the rest of the code
%    inside a |\lowercase| `environment'.
% \changes{babel~3.5f}{1996/01/25}{store the \cs{lccode} of the tie
%    before changing it}
%    \begin{macrocode}
        \@tempcnta=\lccode`\~
        \lccode`~=`##1%
        \lowercase{%
%    \end{macrocode}
%    Make the character active and add it to |\dospecials| and
%    |\@sanitize|.
%    \begin{macrocode}
          \catcode`~\active
          \expandafter\bbl@add@special
            \csname \string##1\endcsname
%    \end{macrocode}
%    Also re-activate it again at |\begin{document}|.
%    \begin{macrocode}
            \AtBeginDocument{%
              \catcode`##1\active
%    \end{macrocode}
%    We also need to make sure that the shorthands are active during
%    the processing of the \file{.aux} file. Otherwise some citations
%    may give unexpected results in the printout when a shorthand was
%    used in the optional argument of |\bibitem| for example.
% \changes{babel~3.6i}{1997/03/01}{Make shorthands active during
%    \file{.aux} file processing}
%    \begin{macrocode}
              \if@filesw
                \immediate\write\@mainaux{%
                  \string\catcode`##1\string\active}%
              \fi}%
%    \end{macrocode}
%    Define the character to expand to
%    \begin{center}
%    |\active@prefix| \m{char} |\normal@char|\m{char}
%    \end{center}
%    (where |\active@char|\m{char} is \emph{one} control sequence!).
% \changes{babel~3.5f}{1996/01/25}{restore the \cs{lccode} of the tie}
%    \begin{macrocode}
          \expandafter\gdef
            \expandafter~%
            \expandafter{%
            \expandafter\active@prefix\expandafter##1%
            \csname normal@char\string##1\endcsname}}%
        \lccode`\~\@tempcnta
      \fi
%    \end{macrocode}
%    For the active caret we first expand to |\bbl@act@caret| in order
%    to be able to handle math mode correctly.
% \changes{babel~3.7f}{2000/09/25}{Make an exception for the active
%    caret which needs an extra level of expansion}
%    \begin{macrocode}
      \ifx##1^%
        \@namedef{active@char\string##1}{\bbl@act@caret}%
      \else
%    \end{macrocode}
%    We define the first level expansion of |\active@char|\m{char} to
%    check the status of the |@safe@actives| flag. If it is set to
%    true we expand to the `normal' version of this character,
%    otherwise we call |\@active@char|\m{char}.
%    \begin{macrocode}
        \@namedef{active@char\string##1}{%
          \if@safe@actives
            \bbl@afterelse\csname normal@char\string##1\endcsname
          \else
            \bbl@afterfi\csname user@active\string##1\endcsname
          \fi}%
      \fi
%    \end{macrocode}
%    The next level of the code checks whether a user has defined a
%    shorthand for himself with this character. First we check for a
%    single character shorthand. If that doesn't exist we check for a
%    shorthand with an argument.
% \changes{babel~3.5d}{1995/07/02}{Skip the user-level active char
%    with argument if no shorthands with arguments were defined}
% \changes{babel~3.8b}{2004/04/19}{Now use \cs{bbl@sh@select}}
%    \begin{macrocode}
      \@namedef{user@active\string##1}{%
        \expandafter\ifx
        \csname \user@group @sh@\string##1@\endcsname
        \relax
          \bbl@afterelse\bbl@sh@select\user@group##1%
        {user@active@arg\string##1}{language@active\string##1}%
        \else
          \bbl@afterfi\csname \user@group @sh@\string##1@\endcsname
        \fi}%
%    \end{macrocode}
%    When there is also no user-level shorthand with an argument we
%    will check whether there is a language defined shorthand for
%    this active character. Before the next token is absorbed as
%    argument we need to make sure that this is safe. Therefor
%    |\peek@token| is called to decide that.
% \changes{babel~3.5f}{1995/12/07}{use \cs{peek@token} to check whether
%    it is safe to proceed}
% \changes{babel~3.6i}{1997/02/20}{Remove the use of \cs{peek@token}
%    again and make the \cs{...active@arg...} commands \cs{long}}
% \changes{babel~3.7e}{1999/09/24}{pass the argument on with braces in
%    order to prevent it from breaking up}
% \changes{babel~3.7f}{2000/02/18}{remove the braces again}
%    \begin{macrocode}
      \long\@namedef{user@active@arg\string##1}####1{%
        \expandafter\ifx
        \csname \user@group @sh@\string##1@\string####1@\endcsname
        \relax
          \bbl@afterelse
          \csname language@active\string##1\endcsname####1%
        \else
          \bbl@afterfi
          \csname \user@group @sh@\string##1@\string####1@%
          \endcsname
        \fi}%
%    \end{macrocode}
%    In order to do the right thing when a shorthand with an argument
%    is used by itself at the end of the line we provide a definition
%    for the case of an empty argument. For that case we let the
%    shorthand character expand to its non-active self.
%    \begin{macrocode}
      \@namedef{\user@group @sh@\string##1@@}{%
        \csname normal@char\string##1\endcsname}
%    \end{macrocode}
%
%    Like the shorthands that can be defined by the user, a language
%    definition file can also define shorthands with and without an
%    argument, so we need two more macros to check if they exist.
% \changes{babel~3.5d}{1995/07/02}{Skip the language-level active char
%    with argument if no shorthands with arguments were defined}
% \changes{babel~3.8b}{2004/04/19}{Now use \cs{bbl@sh@select}}
%    \begin{macrocode}
      \@namedef{language@active\string##1}{%
        \expandafter\ifx
        \csname \language@group @sh@\string##1@\endcsname
        \relax
          \bbl@afterelse\bbl@sh@select\language@group##1%
          {language@active@arg\string##1}{system@active\string##1}%
        \else
          \bbl@afterfi
          \csname \language@group @sh@\string##1@\endcsname
        \fi}%
%    \end{macrocode}
% \changes{babel~3.5f}{1995/12/07}{use \cs{peek@token} to check whether
%    it is safe to proceed}
% \changes{babel~3.6i}{1997/02/20}{Remove the use of \cs{peek@token}
%    again}
% \changes{babel~3.7e}{1999/09/24}{pass the argument on with braces in
%    order to prevent it from breaking up}
% \changes{babel~3.7f}{2000/02/18}{remove the braces again}
%    \begin{macrocode}
      \long\@namedef{language@active@arg\string##1}####1{%
        \expandafter\ifx
        \csname \language@group @sh@\string##1@\string####1@\endcsname
        \relax
          \bbl@afterelse
          \csname system@active\string##1\endcsname####1%
        \else
          \bbl@afterfi
          \csname \language@group @sh@\string##1@\string####1@%
          \endcsname
        \fi}%
%    \end{macrocode}
%    And the same goes for the system level.
% \changes{babel~3.8b}{2004/04/19}{Now use \cs{bbl@sh@select}}
%    \begin{macrocode}
      \@namedef{system@active\string##1}{%
        \expandafter\ifx
        \csname \system@group @sh@\string##1@\endcsname
        \relax
          \bbl@afterelse\bbl@sh@select\system@group##1%
          {system@active@arg\string##1}{normal@char\string##1}%
        \else
          \bbl@afterfi\csname \system@group @sh@\string##1@\endcsname
        \fi}%
%    \end{macrocode}
%    When no shorthands were found the `normal' version of the active
%    character is inserted.
% \changes{babel~3.5f}{1995/12/07}{use \cs{peek@token} to check whether
%    it is safe to proceed}
% \changes{babel~3.6i}{1997/02/20}{Remove the use of \cs{peek@token}
%    again}
%    \begin{macrocode}
      \long\@namedef{system@active@arg\string##1}####1{%
        \expandafter\ifx
        \csname \system@group @sh@\string##1@\string####1@\endcsname
        \relax
          \bbl@afterelse\csname normal@char\string##1\endcsname####1%
        \else
          \bbl@afterfi
          \csname \system@group @sh@\string##1@\string####1@\endcsname
        \fi}%
%    \end{macrocode}
%    When a shorthand combination such as |''| ends up in a heading
%    \TeX\ would see |\protect'\protect'|. To prevent this from
%    happening a shorthand needs to be defined at user level.
% \changes{babel~3.7f}{1999/12/09}{Added an extra shorthand
%    combination on user level to catch an interfering \cs{protect}}
%    \begin{macrocode}
      \@namedef{user@sh@\string##1@\string\protect@}{%
        \csname user@active\string##1\endcsname}%
      }%
    }\x
%    \end{macrocode}
%  \end{macro}
%
%  \begin{macro}{\bbl@sh@select}
%    This command helps the shorthand supporting macros to select how
%    to proceed. Note that this macro needs to be expandable as do all
%    the shorthand macros in order for them to work in expansion-only
%    environments such as the argument of |\hyphenation|.
%
%    This macro expects the name of a group of shorthands in its first
%    argument and a shorthand character in its second argument. It
%    will expand to either |\bbl@firstcs| or |\bbl@scndcs|. Hence two
%    more arguments need to follow it.
% \changes{babel~3.8b}{2004/04/19}{Added command}
%    \begin{macrocode}
\def\bbl@sh@select#1#2{%
  \expandafter\ifx\csname#1@sh@\string#2@sel\endcsname\relax
    \bbl@afterelse\bbl@scndcs
  \else
    \bbl@afterfi\csname#1@sh@\string#2@sel\endcsname
  \fi
}
%    \end{macrocode}
%  \end{macro}
%
%  \begin{macro}{\active@prefix}
%    The command |\active@prefix| which is used in the expansion of
%    active characters has a function similar to |\OT1-cmd| in that it
%    |\protect|s the active character whenever |\protect| is
%    \emph{not} |\@typeset@protect|.
% \changes{babel~3.5d}{1995/07/02}{\cs{@protected@cmd} has vanished
%    from \file{ltoutenc.dtx}}
% \changes{babel~3.7o}{2003/11/17}{Added handling of the situation
%    where \cs{protect} is set to \cs{@unexpandable@protect}}
%    \begin{macrocode}
\def\active@prefix#1{%
  \ifx\protect\@typeset@protect
  \else
%    \end{macrocode}
%    When |\protect| is set to |\@unexpandable@protect| we make sure
%    that the active character is als \emph{not} expanded by inserting
%    |\noexpand| in front of it. The |\@gobble| is needed to remove
%    a token such as |\activechar:| (when the double colon was the
%    active character to be dealt with).
%    \begin{macrocode}
    \ifx\protect\@unexpandable@protect
      \bbl@afterelse\bbl@afterfi\noexpand#1\@gobble
    \else
      \bbl@afterfi\bbl@afterfi\protect#1\@gobble
    \fi
  \fi}
%    \end{macrocode}
%  \end{macro}
%
%  \begin{macro}{\if@safe@actives}
%    In some circumstances it is necessary to be able to change the
%    expansion of an active character on the fly. For this purpose the
%    switch |@safe@actives| is available. The setting of this switch
%    should be checked in the first level expansion of
%    |\active@char|\m{char}.
%    \begin{macrocode}
\newif\if@safe@actives
\@safe@activesfalse
%    \end{macrocode}
%  \end{macro}
%
%  \begin{macro}{\bbl@restore@actives}
% \changes{babel~3.7m}{2003/11/15}{New macro added}
%    When the output routine kicks in while the
%    active characters were made ``safe'' this must be undone in
%    the headers to prevent unexpected typeset results. For this
%    situation we define a command to make them ``unsafe'' again.
%    \begin{macrocode}
\def\bbl@restore@actives{\if@safe@actives\@safe@activesfalse\fi}
%    \end{macrocode}
%  \end{macro}
%
%  \begin{macro}{\bbl@activate}
% \changes{babel~3.5a}{1995/02/11}{Added macro}
%
%    This macro takes one argument, like |\initiate@active@char|. The
%    macro is used to change the definition of an active character to
%    expand to |\active@char|\m{char} instead of
%    |\normal@char|\m{char}.
%    \begin{macrocode}
\def\bbl@activate#1{%
  \expandafter\def
  \expandafter#1\expandafter{%
    \expandafter\active@prefix
    \expandafter#1\csname active@char\string#1\endcsname}%
}
%    \end{macrocode}
%  \end{macro}
%
%  \begin{macro}{\bbl@deactivate}
% \changes{babel~3.5a}{1995/02/11}{Added macro}
%    This macro takes one argument, like |\bbl@activate|. The macro
%    doesn't really make a character non-active; it changes its
%    definition to expand to |\normal@char|\m{char}.
%    \begin{macrocode}
\def\bbl@deactivate#1{%
  \expandafter\def
  \expandafter#1\expandafter{%
    \expandafter\active@prefix
    \expandafter#1\csname normal@char\string#1\endcsname}%
}
%    \end{macrocode}
%  \end{macro}
%
%  \begin{macro}{\bbl@firstcs}
%  \begin{macro}{\bbl@scndcs}
%    These macros have two arguments. They use one of their arguments
%    to build a control sequence from.
%    \begin{macrocode}
\def\bbl@firstcs#1#2{\csname#1\endcsname}
\def\bbl@scndcs#1#2{\csname#2\endcsname}
%    \end{macrocode}
%  \end{macro}
%  \end{macro}
%
%  \begin{macro}{\declare@shorthand}
%    The command |\declare@shorthand| is used to declare a shorthand
%    on a certain level. It takes three arguments:
%    \begin{enumerate}
%    \item a name for the collection of shorthands, i.e. `system', or
%      `dutch';
%    \item the character (sequence) that makes up the shorthand,
%      i.e. |~| or |"a|;
%    \item the code to be executed when the shorthand is encountered.
%    \end{enumerate}
% \changes{babel~3.5d}{1995/07/02}{Make a `note' when a shorthand with
%    an argument is defined.}
% \changes{babel~3.6i}{1997/02/23}{Make it possible to distinguish the
%    constructed control sequences for the case with argument}
% \changes{babel~3.8b}{2004/04/19}{We need to support shorthands with
%    and without argument in different groups; added the name of the
%    group to the storage macro}
%    \begin{macrocode}
\def\declare@shorthand#1#2{\@decl@short{#1}#2\@nil}
\def\@decl@short#1#2#3\@nil#4{%
  \def\bbl@tempa{#3}%
  \ifx\bbl@tempa\@empty
    \expandafter\let\csname #1@sh@\string#2@sel\endcsname\bbl@scndcs
    \@namedef{#1@sh@\string#2@}{#4}%
  \else
    \expandafter\let\csname #1@sh@\string#2@sel\endcsname\bbl@firstcs
    \@namedef{#1@sh@\string#2@\string#3@}{#4}%
  \fi}
%    \end{macrocode}
%  \end{macro}
%
%  \begin{macro}{\textormath}
%    Some of the shorthands that will be declared by the language
%    definition files have to be usable in both text and mathmode. To
%    achieve this the helper macro |\textormath| is provided.
%    \begin{macrocode}
\def\textormath#1#2{%
  \ifmmode
    \bbl@afterelse#2%
  \else
    \bbl@afterfi#1%
  \fi}
%    \end{macrocode}
%  \end{macro}
%
%  \begin{macro}{\user@group}
%  \begin{macro}{\language@group}
%  \begin{macro}{\system@group}
%    The current concept of `shorthands' supports three levels or
%    groups of shorthands. For each level the name of the level or
%    group is stored in a macro. The default is to have a user group;
%    use language group `english' and have a system group called
%    `system'.
% \changes{babel~3.6i}{1997/02/24}{Have a user group called `user' by
%    default}
%    \begin{macrocode}
\def\user@group{user}
\def\language@group{english}
\def\system@group{system}
%    \end{macrocode}
%  \end{macro}
%  \end{macro}
%  \end{macro}
%
%  \begin{macro}{\useshorthands}
%    This is the user level command to tell \LaTeX\ that user level
%    shorthands will be used in the document. It takes one argument,
%    the character that starts a shorthand.
% \changes{babel~3.7j}{2001/11/11}{When \TeX\ has seen a character
%    its category code is fixed; need to use a `stand-in' for the
%    call of \cs{bbl@activate}} 
%    \begin{macrocode}
\def\useshorthands#1{%
%    \end{macrocode}
%    First note that this is user level.
%    \begin{macrocode}
  \def\user@group{user}%
%    \end{macrocode}
%    Then initialize the character for use as a shorthand character.
%    \begin{macrocode}
  \initiate@active@char{#1}%
%    \end{macrocode}
%    Now that \TeX\ has seen the character its category code is
%    fixed, but for the actions of |\bbl@activate| to succeed we need
%    it to be active. Hence the trick with the |\lccode| to circumvent
%    this.
% \changes{babel~3.7j}{2003/09/11}{The change from 11/112001 was
%    incomplete} 
%    \begin{macrocode}
  \@tempcnta\lccode`\~
  \lccode`~=`#1%
  \lowercase{\catcode`~\active\bbl@activate{~}}%
  \lccode`\~\@tempcnta}
%    \end{macrocode}
%  \end{macro}
%
%  \begin{macro}{\defineshorthand}
%    Currently we only support one group of user level shorthands,
%    called `user'.
%    \begin{macrocode}
\def\defineshorthand{\declare@shorthand{user}}
%    \end{macrocode}
%  \end{macro}
%
%  \begin{macro}{\languageshorthands}
%    A user level command to change the language from which shorthands
%    are used.
%    \begin{macrocode}
\def\languageshorthands#1{\def\language@group{#1}}
%    \end{macrocode}
%  \end{macro}
%
%  \begin{macro}{\aliasshorthand}
% \changes{babel~3.5f}{1996/01/25}{New command}
%    \begin{macrocode}
\def\aliasshorthand#1#2{%
%    \end{macrocode}
%    First the new shorthand needs to be initialized,
%    \begin{macrocode}
  \expandafter\ifx\csname active@char\string#2\endcsname\relax
     \ifx\document\@notprerr
       \@notshorthand{#2}
     \else
       \initiate@active@char{#2}%
%    \end{macrocode}
%    Then we need to use the |\lccode| trick to make the new shorthand
%    behave like the old one. Therefore we save the current |\lccode|
%    of the |~|-character and restore it later. Then we |\let| the new
%    shorthand character be equal to the original. 
%    \begin{macrocode}
       \@tempcnta\lccode`\~
       \lccode`~=`#2%
       \lowercase{\let~#1}%
       \lccode`\~\@tempcnta
     \fi
   \fi
}
%    \end{macrocode}
%  \end{macro}
%
%  \begin{macro}{\@notshorthand}
% \changes{v3.8d}{2004/11/20}{Error message added}
%    
%    \begin{macrocode}
\def\@notshorthand#1{%
       \PackageError{babel}{%
         The character `\string #1' should be made
         a shorthand character;\MessageBreak
         add the command \string\useshorthands\string{#1\string} to
         the preamble.\MessageBreak
         I will ignore your instruction}{}%
   }
%    \end{macrocode}
%  \end{macro}
%
%  \begin{macro}{\shorthandon}
% \changes{babel~3.7a}{1998/06/07}{Added command}
%  \begin{macro}{\shorthandoff}
% \changes{babel~3.7a}{1998/06/07}{Added command}
%    The first level definition of these macros just passes the
%    argument on to |\bbl@switch@sh|, adding |\@nil| at the end to
%    denote the end of the list of characters.
%    \begin{macrocode}
\newcommand*\shorthandon[1]{\bbl@switch@sh{on}#1\@nil}
\newcommand*\shorthandoff[1]{\bbl@switch@sh{off}#1\@nil}
%    \end{macrocode}
%
%  \begin{macro}{\bbl@switch@sh}
% \changes{babel~3.7a}{1998/06/07}{Added command}
%    The macro |\bbl@switch@sh| takes the list of characters apart one
%    by  one and subsequently switches the category code of the
%    shorthand character according to the first argument of
%    |\bbl@switch@sh|.
%    \begin{macrocode}
\def\bbl@switch@sh#1#2#3\@nil{%
%    \end{macrocode}
%    But before any of this switching takes place we make sure that
%    the character we are dealing with is known as a shorthand
%    character. If it is, a macro such as |\active@char"| should
%    exist.
%    \begin{macrocode}
  \@ifundefined{active@char\string#2}{%
    \PackageError{babel}{%
      The character '\string #2' is not a shorthand character
      in \languagename}{%
      Maybe you made a typing mistake?\MessageBreak
      I will ignore your instruction}}{%
    \csname bbl@switch@sh@#1\endcsname#2}%
%    \end{macrocode}
%    Now that, as the first character in the list has been taken care
%    of, we pass the rest of the list back to |\bbl@switch@sh|.
%    \begin{macrocode}
  \ifx#3\@empty\else
    \bbl@afterfi\bbl@switch@sh{#1}#3\@nil
  \fi}
%    \end{macrocode}
%  \end{macro}
%
%  \begin{macro}{\bbl@switch@sh@off}
%    All that is left to do is define the actual switching
%    macros. Switching off is easy, we just set the category code to
%    `other' (12).
%    \begin{macrocode}
\def\bbl@switch@sh@off#1{\catcode`#112\relax}
%    \end{macrocode}
%  \end{macro}
%
%  \begin{macro}{\bbl@switch@sh@on}
%    But switching the shorthand character back on is a bit more
%    tricky. It involves making sure that we have an active character
%    to begin with when the macro is being defined. It also needs the
%    use of |\lowercase| and |\lccode| trickery to get everything to
%    work out as expected. And to keep things local that need to
%    remain local a group is opened, which is closed as soon as |\x|
%    gets executed.
% \changes{babel~3.8j}{2008/03/21}{Added a group in order to protect
%    the current lowercase code of the tilde (PR 3851)} 
%    \begin{macrocode}
\begingroup
  \catcode`\~\active
  \def\x{\endgroup
    \def\bbl@switch@sh@on##1{%
      \begingroup
      \lccode`~=`##1%
      \lowercase{\endgroup
        \catcode`~\active
        }%
      }%
    }
%    \end{macrocode}
%  \end{macro}
%    The next operation makes the above definition effective.
%    \begin{macrocode}
\x
%
%    \end{macrocode}
%  \end{macro}
%  \end{macro}
%
%    To prevent problems with constructs such as |\char"01A| when the
%    double quote is made active, we define a shorthand on
%    system level.
% \changes{babel~3.5a}{1995/03/10}{Replaced 16 system shorthands to
%    deal with hex numbers by one}
%    \begin{macrocode}
\declare@shorthand{system}{"}{\csname normal@char\string"\endcsname}
%    \end{macrocode}
%
%    When the right quote is made active we need to take care of
%    handling it correctly in mathmode. Therefore we define a
%    shorthand at system level to make it expand to a non-active right
%    quote in textmode, but expand to its original definition in
%    mathmode. (Note that the right quote is `active' in mathmode
%    because of its mathcode.)
% \changes{babel~3.5a}{1995/03/10}{Added a system shorthand for the
%    right quote}
%    \begin{macrocode}
\declare@shorthand{system}{'}{%
  \textormath{\csname normal@char\string'\endcsname}%
             {\sp\bgroup\prim@s}}
%    \end{macrocode}
%
%    When the left quote is  made active we need to take care of
%    handling it correctly when it is followed by for instance an open
%    brace token. Therefore we define a shorthand at system level to
%    make it expand to a non-active left quote.
% \changes{babel~3.5f}{1996/03/06}{Added a system shorthand for the
%    left quote}
%    \begin{macrocode}
\declare@shorthand{system}{`}{\csname normal@char\string`\endcsname}
%    \end{macrocode}
%
%  \begin{macro}{\bbl@prim@s}
% \changes{babel~3.7f}{1999/12/01}{Need to redefine \cs{prim@s} as
%    well as plain \TeX's definition uses \cs{next}}
%  \begin{macro}{\bbl@pr@m@s}
% \changes{babel~3.5a}{1995/03/10}{Added macro}
%    One of the internal macros that are involved in substituting
%    |\prime| for each right quote in mathmode is |\prim@s|. This
%    checks if the next character is a right quote. When the right
%    quote is active, the definition of this macro needs to be adapted
%    to look for an active right quote.
%    \begin{macrocode}
\def\bbl@prim@s{%
  \prime\futurelet\@let@token\bbl@pr@m@s}
\begingroup
  \catcode`\'\active\let'\relax
  \def\x{\endgroup
    \def\bbl@pr@m@s{%
      \ifx'\@let@token
        \expandafter\pr@@@s
      \else
        \ifx^\@let@token
          \expandafter\expandafter\expandafter\pr@@@t
        \else
          \egroup
        \fi
      \fi}%
    }
\x
%    \end{macrocode}
%  \end{macro}
%  \end{macro}
%
%    \begin{macrocode}
%</core|shorthands>
%    \end{macrocode}
%
%    Normally the |~| is active and expands to \verb*=\penalty\@M\ =.
%    When it is written to the \file{.aux} file it is written
%    expanded. To prevent that and to be able to use the character |~|
%    as a start character for a shorthand, it is redefined here as a
%    one character shorthand on system level.
% \changes{babel~3.5f}{1996/04/02}{No need to reset the category code
%    of the tilde as \cs{initiate@active@char} now correctly deals
%    with active characters}
%    \begin{macrocode}
%<*core>
\initiate@active@char{~}
\declare@shorthand{system}{~}{\leavevmode\nobreak\ }
\bbl@activate{~}
%    \end{macrocode}
%
%  \begin{macro}{\OT1dqpos}
%  \begin{macro}{\T1dqpos}
%    The position of the double quote character is different for the
%    OT1 and T1 encodings. It will later be selected using the
%    |\f@encoding| macro. Therefor we define two macros here to store
%    the position of the character in these encodings.
%    \begin{macrocode}
\expandafter\def\csname OT1dqpos\endcsname{127}
\expandafter\def\csname T1dqpos\endcsname{4}
%    \end{macrocode}
%    When the macro |\f@encoding| is undefined (as it is in plain
%    \TeX) we define it here to expand to \texttt{OT1}
%    \begin{macrocode}
\ifx\f@encoding\@undefined
  \def\f@encoding{OT1}
\fi
%    \end{macrocode}
%  \end{macro}
%  \end{macro}
%
%  \subsection{Language attributes}
%
%    Language attributes provide a means to give the user control over
%    which features of the language definition files he wants to
%    enable.
% \changes{babel~3.7c}{1998/07/02}{Added support for language
%    attributes}
%  \begin{macro}{\languageattribute}
%    The macro |\languageattribute| checks whether its arguments are
%    valid and then activates the selected language attribute.
%    \begin{macrocode}
\newcommand\languageattribute[2]{%
%    \end{macrocode}
%    First check whether the language is known.
%    \begin{macrocode}
  \expandafter\ifx\csname l@#1\endcsname\relax
    \@nolanerr{#1}%
  \else
%    \end{macrocode}
%    Than process each attribute in the list.
%    \begin{macrocode}
    \@for\bbl@attr:=#2\do{%
%    \end{macrocode}
%    We want to make sure that each attribute is selected only once;
%    therefor we store the already selected attributes in
%    |\bbl@known@attribs|. When that control sequence is not yet
%    defined this attribute is certainly not selected before.
%    \begin{macrocode}
      \ifx\bbl@known@attribs\@undefined
        \in@false
      \else
%    \end{macrocode}
%    Now we need to see if the attribute occurs in the list of
%    already selected attributes.
%    \begin{macrocode}
        \edef\bbl@tempa{\noexpand\in@{,#1-\bbl@attr,}%
          {,\bbl@known@attribs,}}%
        \bbl@tempa
      \fi
%    \end{macrocode}
%    When the attribute was in the list we issue a warning; this might
%    not be the users intention.
%    \begin{macrocode}
      \ifin@
        \PackageWarning{Babel}{%
          You have more than once selected the attribute
          '\bbl@attr'\MessageBreak for language #1}%
      \else
%    \end{macrocode}
%    When we end up here the attribute is not selected before. So, we
%    add it to the list of selected attributes and execute the
%    associated \TeX-code.
%    \begin{macrocode}
        \edef\bbl@tempa{%
          \noexpand\bbl@add@list\noexpand\bbl@known@attribs{#1-\bbl@attr}}%
        \bbl@tempa
        \edef\bbl@tempa{#1-\bbl@attr}%
        \expandafter\bbl@ifknown@ttrib\expandafter{\bbl@tempa}\bbl@attributes%
        {\csname#1@attr@\bbl@attr\endcsname}%
        {\@attrerr{#1}{\bbl@attr}}%
     \fi
      }
  \fi}
%    \end{macrocode}
%    This command should only be used in the preamble of a document.
%    \begin{macrocode}
\@onlypreamble\languageattribute
%    \end{macrocode}
%    The error text to be issued when an unknown attribute is
%    selected.
%    \begin{macrocode}
  \newcommand*{\@attrerr}[2]{%
    \PackageError{babel}%
                 {The attribute #2 is unknown for language #1.}%
        {Your command will be ignored, type <return> to proceed}}
%    \end{macrocode}
%  \end{macro}
%
%  \begin{macro}{\bbl@declare@ttribute}
%    This command adds the new language/attribute combination to the
%    list of known attributes.
%    \begin{macrocode}
\def\bbl@declare@ttribute#1#2#3{%
  \bbl@add@list\bbl@attributes{#1-#2}%
%    \end{macrocode}
%    Then it defines a control sequence to be executed when the
%    attribute is used in a document. The result of this should be
%    that the macro |\extras...| for the current language is extended,
%    otherwise the attribute will not work as its code is removed from
%    memory at |\begin{document}|.
%    \begin{macrocode}
  \expandafter\def\csname#1@attr@#2\endcsname{#3}%
  }
%    \end{macrocode}
%  \end{macro}
%
%  \begin{macro}{\bbl@ifattributeset}
% \changes{babel~3.7f}{2000/02/12}{macro added}
%    This internal macro has 4 arguments. It can be used to interpret
%    \TeX\ code based on whether a certain attribute was set. This
%    command should appear inside the argument to |\AtBeginDocument|
%    because the attributes are set in the document preamble,
%    \emph{after} \babel\ is loaded.
%
%    The first argument is the language, the second argument the
%    attribute being checked, and the third and fourth arguments are
%    the true and false clauses.
%    \begin{macrocode}
\def\bbl@ifattributeset#1#2#3#4{%
%    \end{macrocode}
%    First we need to find out if any attributes were set; if not
%    we're done.
%    \begin{macrocode}
  \ifx\bbl@known@attribs\@undefined
    \in@false
  \else
%    \end{macrocode}
%    The we need to check the list of known attributes.
%    \begin{macrocode}
    \edef\bbl@tempa{\noexpand\in@{,#1-#2,}%
      {,\bbl@known@attribs,}}%
    \bbl@tempa
  \fi
%    \end{macrocode}
%    When we're this far |\ifin@| has a value indicating if the
%    attribute in question was set or not. Just to be safe the code to
%    be executed is `thrown over the |\fi|'.
%    \begin{macrocode}
  \ifin@
    \bbl@afterelse#3%
  \else
    \bbl@afterfi#4%
  \fi
  }
%    \end{macrocode}
%  \end{macro}
%
%  \begin{macro}{\bbl@add@list}
%    This internal macro adds its second argument to a comma
%    separated list in its first argument. When the list is not
%    defined yet (or empty), it will be initiated
%    \begin{macrocode}
\def\bbl@add@list#1#2{%
  \ifx#1\@undefined
    \def#1{#2}%
  \else
    \ifx#1\@empty
      \def#1{#2}%
    \else
      \edef#1{#1,#2}%
    \fi
  \fi
  }
%    \end{macrocode}
%  \end{macro}
%
%  \begin{macro}{\bbl@ifknown@ttrib}
%    An internal macro to check whether a given language/attribute is
%    known. The macro takes 4 arguments, the language/attribute, the
%    attribute list, the \TeX-code to be executed when the attribute
%    is known and the \TeX-code to be executed otherwise.
%    \begin{macrocode}
\def\bbl@ifknown@ttrib#1#2{%
%    \end{macrocode}
%    We first assume the attribute is unknown.
%    \begin{macrocode}
  \let\bbl@tempa\@secondoftwo
%    \end{macrocode}
%    Then we loop over the list of known attributes, trying to find a
%    match.
%    \begin{macrocode}
  \@for\bbl@tempb:=#2\do{%
    \expandafter\in@\expandafter{\expandafter,\bbl@tempb,}{,#1,}%
    \ifin@
%    \end{macrocode}
%    When a match is found the definition of |\bbl@tempa| is changed.
%    \begin{macrocode}
      \let\bbl@tempa\@firstoftwo
    \else
    \fi}%
%    \end{macrocode}
%    Finally we execute |\bbl@tempa|.
%    \begin{macrocode}
  \bbl@tempa
}
%    \end{macrocode}
%  \end{macro}
%
%  \begin{macro}{\bbl@clear@ttribs}
%    This macro removes all the attribute code from \LaTeX's memory at
%    |\begin{document}| time (if any is present).
% \changes{babel~3.7e}{1999/09/24}{When \cs{bbl@attributes} is
%    undefined this should be a no-op} 
%    \begin{macrocode}
\def\bbl@clear@ttribs{%
  \ifx\bbl@attributes\@undefined\else
    \@for\bbl@tempa:=\bbl@attributes\do{%
      \expandafter\bbl@clear@ttrib\bbl@tempa.
      }%
    \let\bbl@attributes\@undefined
  \fi
  }
\def\bbl@clear@ttrib#1-#2.{%
  \expandafter\let\csname#1@attr@#2\endcsname\@undefined}
\AtBeginDocument{\bbl@clear@ttribs}
%    \end{macrocode}
%  \end{macro}
%
%  \subsection{Support for saving macro definitions}
%
%    To save the meaning of control sequences using |\babel@save|, we
%    use temporary control sequences.  To save hash table entries for
%    these control sequences, we don't use the name of the control
%    sequence to be saved to construct the temporary name.  Instead we
%    simply use the value of a counter, which is reset to zero each
%    time we begin to save new values.  This works well because we
%    release the saved meanings before we begin to save a new set of
%    control sequence meanings (see |\selectlanguage| and
%    |\originalTeX|).
%
%  \begin{macro}{\babel@savecnt}
% \changes{babel~3.2}{1991/11/10}{Added macro}
%  \begin{macro}{\babel@beginsave}
% \changes{babel~3.2}{1991/11/10}{Added macro}
%    The initialization of a new save cycle: reset the counter to
%    zero.
%    \begin{macrocode}
\def\babel@beginsave{\babel@savecnt\z@}
%    \end{macrocode}
%    Before it's forgotten, allocate the counter and initialize all.
%    \begin{macrocode}
\newcount\babel@savecnt
\babel@beginsave
%    \end{macrocode}
%  \end{macro}
%  \end{macro}
%
%  \begin{macro}{\babel@save}
% \changes{babel~3.2}{1991/11/10}{Added macro}
%    The macro |\babel@save|\meta{csname} saves the current meaning of
%    the control sequence \meta{csname} to
%    |\originalTeX|\footnote{\cs{originalTeX} has to be
%    expandable, i.\,e.\ you shouldn't let it to \cs{relax}.}.
%    To do this, we let the current meaning to a temporary control
%    sequence, the restore commands are appended to |\originalTeX| and
%    the counter is incremented.
% \changes{babel~3.2c}{1992/03/17}{missing backslash led to errors
%    when executing \cs{originalTeX}}
% \changes{babel~3.2d}{1992/07/02}{saving in \cs{babel@i} and
%    restoring from \cs{@babel@i} doesn't work very well...}
%    \begin{macrocode}
\def\babel@save#1{%
  \expandafter\let\csname babel@\number\babel@savecnt\endcsname #1\relax
  \begingroup
    \toks@\expandafter{\originalTeX \let#1=}%
    \edef\x{\endgroup
      \def\noexpand\originalTeX{\the\toks@ \expandafter\noexpand
         \csname babel@\number\babel@savecnt\endcsname\relax}}%
  \x
  \advance\babel@savecnt\@ne}
%    \end{macrocode}
%  \end{macro}
%
%  \begin{macro}{\babel@savevariable}
% \changes{babel~3.2}{1991/11/10}{Added macro}
%    The macro |\babel@savevariable|\meta{variable} saves the value of
%    the variable.  \meta{variable} can be anything allowed after the
%    |\the| primitive.
%    \begin{macrocode}
\def\babel@savevariable#1{\begingroup
    \toks@\expandafter{\originalTeX #1=}%
    \edef\x{\endgroup
      \def\noexpand\originalTeX{\the\toks@ \the#1\relax}}%
  \x}
%    \end{macrocode}
%  \end{macro}
%
%  \begin{macro}{\bbl@frenchspacing}
%  \begin{macro}{\bbl@nonfrenchspacing}
%    Some languages need to have |\frenchspacing| in effect. Others
%    don't want that. The command |\bbl@frenchspacing| switches it on
%    when it isn't already in effect and |\bbl@nonfrenchspacing|
%    switches it off if necessary.
%    \begin{macrocode}
\def\bbl@frenchspacing{%
  \ifnum\the\sfcode`\.=\@m
    \let\bbl@nonfrenchspacing\relax
  \else
    \frenchspacing
    \let\bbl@nonfrenchspacing\nonfrenchspacing
  \fi}
\let\bbl@nonfrenchspacing\nonfrenchspacing
%    \end{macrocode}
%  \end{macro}
%  \end{macro}
%
% \subsection{Support for extending macros}
%
%  \begin{macro}{\addto}
%    For each language four control sequences have to be defined that
%    control the language-specific definitions. To be able to add
%    something to these macro once they have been defined the macro
%    |\addto| is introduced. It takes two arguments, a \meta{control
%    sequence} and \TeX-code to be added to the \meta{control
%    sequence}.
%
%    If the \meta{control sequence} has not been defined before it is
%    defined now.
% \changes{babel~3.1}{1991/11/05}{Added macro}
% \changes{babel~3.4}{1994/02/04}{Changed to use toks register}
% \changes{babel~3.6b}{1996/12/30}{Also check if control sequence
%    expands to \cs{relax}}
%    \begin{macrocode}
\def\addto#1#2{%
  \ifx#1\@undefined
    \def#1{#2}%
  \else
%    \end{macrocode}
%    The control sequence could also expand to |\relax|, in which case
%    a circular definition results. The net result is a stack overflow.
%    \begin{macrocode}
    \ifx#1\relax
      \def#1{#2}%
    \else
%    \end{macrocode}
%    Otherwise the replacement text for the \meta{control sequence} is
%    expanded and stored in a token register, together with the
%    \TeX-code to be added.  Finally the \meta{control sequence} is
%    \emph{re}defined, using the contents of the token register.
%    \begin{macrocode}
      {\toks@\expandafter{#1#2}%
        \xdef#1{\the\toks@}}%
    \fi
  \fi
}
%    \end{macrocode}
%  \end{macro}
%
% \subsection{Macros common to a number of languages}
%
%  \begin{macro}{\allowhyphens}
% \changes{babel~3.2b}{1992/02/16}{Moved macro from language
%    definition files}
% \changes{babel~3.7a}{1998/03/12}{Make \cs{allowhyphens} a no-op for
%    T1 fontencoding}
%    This macro makes hyphenation possible. Basically its definition
%    is nothing more than |\nobreak| |\hskip| \texttt{0pt plus
%    0pt}\footnote{\TeX\ begins and ends a word for hyphenation at a
%    glue node. The penalty prevents a linebreak at this glue node.}.
%    \begin{macrocode}
\def\bbl@t@one{T1}
\def\allowhyphens{%
  \ifx\cf@encoding\bbl@t@one\else\bbl@allowhyphens\fi}
\def\bbl@allowhyphens{\nobreak\hskip\z@skip}
%    \end{macrocode}
%  \end{macro}
%
%  \begin{macro}{\set@low@box}
% \changes{babel~3.2b}{1992/02/16}{Moved macro from language
%    definition files}
%    The following macro is used to lower quotes to the same level as
%    the comma.  It prepares its argument in box register~0.
%    \begin{macrocode}
\def\set@low@box#1{\setbox\tw@\hbox{,}\setbox\z@\hbox{#1}%
    \dimen\z@\ht\z@ \advance\dimen\z@ -\ht\tw@%
    \setbox\z@\hbox{\lower\dimen\z@ \box\z@}\ht\z@\ht\tw@ \dp\z@\dp\tw@}
%    \end{macrocode}
%  \end{macro}
%
%  \begin{macro}{\save@sf@q}
% \changes{babel~3.2b}{1992/02/16}{Moved macro from language
%    definition files}
%    The macro |\save@sf@q| is used to save and reset the current
%    space factor.
% \changes{babel~3.7f}{2000/09/19}{PR3119, don't start a paragraph in
%    a local group}
%    \begin{macrocode}
\def\save@sf@q #1{\leavevmode
 \begingroup 
  \edef\@SF{\spacefactor \the\spacefactor}#1\@SF
 \endgroup
}
%    \end{macrocode}
%  \end{macro}
%
%  \begin{macro}{\bbl@disc}
% \changes{babel~3.5f}{1996/01/24}{Macro moved from language
%    definition files}
%    For some languages the macro |\bbl@disc| is used to ease the
%    insertion of discretionaries for letters that behave `abnormally'
%    at a breakpoint.
%    \begin{macrocode}
\def\bbl@disc#1#2{%
  \nobreak\discretionary{#2-}{}{#1}\allowhyphens}
%    \end{macrocode}
%  \end{macro}
%
% \changes{babel~3.5c}{1995/06/14}{Repaired a typo (itlaic, PR1652)}
%
%  \subsection{Making glyphs available}
%
%    The file \file{\filename}\footnote{The file described in this
%    section has version number \fileversion, and was last revised on
%    \filedate.} makes a number of glyphs available that either do not
%    exist in the \texttt{OT1} encoding and have to be `faked', or
%    that are not accessible through \file{T1enc.def}.
%
%  \subsection{Quotation marks}
%
%  \begin{macro}{\quotedblbase}
%    In the \texttt{T1} encoding the opening double quote at the
%    baseline is available as a separate character, accessible via
%    |\quotedblbase|. In the \texttt{OT1} encoding it is not
%    available, therefor we make it available by lowering the normal
%    open quote character to the baseline.
%    \begin{macrocode}
\ProvideTextCommand{\quotedblbase}{OT1}{%
  \save@sf@q{\set@low@box{\textquotedblright\/}%
    \box\z@\kern-.04em\allowhyphens}}
%    \end{macrocode}
%    Make sure that when an encoding other than \texttt{OT1} or
%    \texttt{T1} is used this glyph can still be typeset.
%    \begin{macrocode}
\ProvideTextCommandDefault{\quotedblbase}{%
  \UseTextSymbol{OT1}{\quotedblbase}}
%    \end{macrocode}
%  \end{macro}
%
%  \begin{macro}{\quotesinglbase}
%    We also need the single quote character at the baseline.
%    \begin{macrocode}
\ProvideTextCommand{\quotesinglbase}{OT1}{%
  \save@sf@q{\set@low@box{\textquoteright\/}%
    \box\z@\kern-.04em\allowhyphens}}
%    \end{macrocode}
%    Make sure that when an encoding other than \texttt{OT1} or
%    \texttt{T1} is used this glyph can still be typeset.
%    \begin{macrocode}
\ProvideTextCommandDefault{\quotesinglbase}{%
  \UseTextSymbol{OT1}{\quotesinglbase}}
%    \end{macrocode}
%  \end{macro}
%
%  \begin{macro}{\guillemotleft}
%  \begin{macro}{\guillemotright}
%    The guillemet characters are not available in \texttt{OT1}
%    encoding. They are faked.
%    \begin{macrocode}
\ProvideTextCommand{\guillemotleft}{OT1}{%
  \ifmmode
    \ll
  \else
    \save@sf@q{\nobreak
      \raise.2ex\hbox{$\scriptscriptstyle\ll$}\allowhyphens}%
  \fi}
\ProvideTextCommand{\guillemotright}{OT1}{%
  \ifmmode
    \gg
  \else
    \save@sf@q{\nobreak
      \raise.2ex\hbox{$\scriptscriptstyle\gg$}\allowhyphens}%
  \fi}
%    \end{macrocode}
%    Make sure that when an encoding other than \texttt{OT1} or
%    \texttt{T1} is used these glyphs can still be typeset.
%    \begin{macrocode}
\ProvideTextCommandDefault{\guillemotleft}{%
  \UseTextSymbol{OT1}{\guillemotleft}}
\ProvideTextCommandDefault{\guillemotright}{%
  \UseTextSymbol{OT1}{\guillemotright}}
%    \end{macrocode}
%  \end{macro}
%  \end{macro}
%
%  \begin{macro}{\guilsinglleft}
%  \begin{macro}{\guilsinglright}
%    The single guillemets are not available in \texttt{OT1}
%    encoding. They are faked.
%    \begin{macrocode}
\ProvideTextCommand{\guilsinglleft}{OT1}{%
  \ifmmode
    <%
  \else
    \save@sf@q{\nobreak
      \raise.2ex\hbox{$\scriptscriptstyle<$}\allowhyphens}%
  \fi}
\ProvideTextCommand{\guilsinglright}{OT1}{%
  \ifmmode
    >%
  \else
    \save@sf@q{\nobreak
      \raise.2ex\hbox{$\scriptscriptstyle>$}\allowhyphens}%
  \fi}
%    \end{macrocode}
%    Make sure that when an encoding other than \texttt{OT1} or
%    \texttt{T1} is used these glyphs can still be typeset.
%    \begin{macrocode}
\ProvideTextCommandDefault{\guilsinglleft}{%
  \UseTextSymbol{OT1}{\guilsinglleft}}
\ProvideTextCommandDefault{\guilsinglright}{%
  \UseTextSymbol{OT1}{\guilsinglright}}
%    \end{macrocode}
%  \end{macro}
%  \end{macro}
%
%
%  \subsection{Letters}
%
%  \begin{macro}{\ij}
%  \begin{macro}{\IJ}
%    The dutch language uses the letter `ij'. It is available in
%    \texttt{T1} encoded fonts, but not in the \texttt{OT1} encoded
%    fonts. Therefor we fake it for the \texttt{OT1} encoding.
% \changes{dutch-3.7a}{1995/02/04}{Changed the kerning in the faked ij
%    to match the dc-version of it}
%    \begin{macrocode}
\DeclareTextCommand{\ij}{OT1}{%
  \allowhyphens i\kern-0.02em j\allowhyphens}
\DeclareTextCommand{\IJ}{OT1}{%
  \allowhyphens I\kern-0.02em J\allowhyphens}
\DeclareTextCommand{\ij}{T1}{\char188}
\DeclareTextCommand{\IJ}{T1}{\char156}
%    \end{macrocode}
%    Make sure that when an encoding other than \texttt{OT1} or
%    \texttt{T1} is used these glyphs can still be typeset.
%    \begin{macrocode}
\ProvideTextCommandDefault{\ij}{%
  \UseTextSymbol{OT1}{\ij}}
\ProvideTextCommandDefault{\IJ}{%
  \UseTextSymbol{OT1}{\IJ}}
%    \end{macrocode}
%  \end{macro}
%  \end{macro}
%
%  \begin{macro}{\dj}
%  \begin{macro}{\DJ}
%    The croatian language needs the letters |\dj| and |\DJ|; they are
%    available in the \texttt{T1} encoding, but not in the
%    \texttt{OT1} encoding by default.
%
%    Some code to construct these glyphs for the \texttt{OT1} encoding
%    was made available to me by Stipcevic Mario,
%    (\texttt{stipcevic@olimp.irb.hr}).
% \changes{babel~3.5f}{1996/03/28}{New definition of \cs{dj}, see PR
%    2058}
%    \begin{macrocode}
\def\crrtic@{\hrule height0.1ex width0.3em}
\def\crttic@{\hrule height0.1ex width0.33em}
%
\def\ddj@{%
  \setbox0\hbox{d}\dimen@=\ht0
  \advance\dimen@1ex
  \dimen@.45\dimen@
  \dimen@ii\expandafter\rem@pt\the\fontdimen\@ne\font\dimen@
  \advance\dimen@ii.5ex
  \leavevmode\rlap{\raise\dimen@\hbox{\kern\dimen@ii\vbox{\crrtic@}}}}
\def\DDJ@{%
  \setbox0\hbox{D}\dimen@=.55\ht0
  \dimen@ii\expandafter\rem@pt\the\fontdimen\@ne\font\dimen@
  \advance\dimen@ii.15ex %            correction for the dash position
  \advance\dimen@ii-.15\fontdimen7\font %     correction for cmtt font
  \dimen\thr@@\expandafter\rem@pt\the\fontdimen7\font\dimen@
  \leavevmode\rlap{\raise\dimen@\hbox{\kern\dimen@ii\vbox{\crttic@}}}}
%
\DeclareTextCommand{\dj}{OT1}{\ddj@ d}
\DeclareTextCommand{\DJ}{OT1}{\DDJ@ D}
%    \end{macrocode}
%    Make sure that when an encoding other than \texttt{OT1} or
%    \texttt{T1} is used these glyphs can still be typeset.
%    \begin{macrocode}
\ProvideTextCommandDefault{\dj}{%
  \UseTextSymbol{OT1}{\dj}}
\ProvideTextCommandDefault{\DJ}{%
  \UseTextSymbol{OT1}{\DJ}}
%    \end{macrocode}
%  \end{macro}
%  \end{macro}
%
%  \begin{macro}{\SS}
%    For the \texttt{T1} encoding |\SS| is defined and selects a
%    specific glyph from the font, but for other encodings it is not
%    available. Therefor we make it available here.
%    \begin{macrocode}
\DeclareTextCommand{\SS}{OT1}{SS}
\ProvideTextCommandDefault{\SS}{\UseTextSymbol{OT1}{\SS}}
%    \end{macrocode}
%  \end{macro}
%
% \subsection{Shorthands for quotation marks}
%
%    Shorthands are provided for a number of different quotation
%    marks, which make them usable both outside and inside mathmode.
%
%  \begin{macro}{\glq}
%  \begin{macro}{\grq}
% \changes{babel~3.7a}{1997/04/25}{Make the definition of \cs{grq}
%    dependent on the font encoding}
% \changes{babel~3.8b}{2004/05/02}{Made \cs{glq} fontencoding
%    dependent as well} 
%    The `german' single quotes.
%    \begin{macrocode}
\ProvideTextCommand{\glq}{OT1}{%
  \textormath{\quotesinglbase}{\mbox{\quotesinglbase}}}
\ProvideTextCommand{\glq}{T1}{%
  \textormath{\quotesinglbase}{\mbox{\quotesinglbase}}}
\ProvideTextCommandDefault{\glq}{\UseTextSymbol{OT1}\glq}
%    \end{macrocode}
%    The definition of |\grq| depends on the fontencoding. With
%    \texttt{T1} encoding no extra kerning is needed.
%    \begin{macrocode}
\ProvideTextCommand{\grq}{T1}{%
  \textormath{\textquoteleft}{\mbox{\textquoteleft}}}
\ProvideTextCommand{\grq}{OT1}{%
  \save@sf@q{\kern-.0125em%
  \textormath{\textquoteleft}{\mbox{\textquoteleft}}%
  \kern.07em\relax}}
\ProvideTextCommandDefault{\grq}{\UseTextSymbol{OT1}\grq}
%    \end{macrocode}
%  \end{macro}
%  \end{macro}
%
%  \begin{macro}{\glqq}
%  \begin{macro}{\grqq}
% \changes{babel~3.7a}{1997/04/25}{Make the definition of \cs{grqq}
%    dependent on the font encoding}
% \changes{babel~3.8b}{2004/05/02}{Made \cs{grqq} fontencoding
%    dependent as well} 
%    The `german' double quotes.
%    \begin{macrocode}
\ProvideTextCommand{\glqq}{OT1}{%
  \textormath{\quotedblbase}{\mbox{\quotedblbase}}}
\ProvideTextCommand{\glqq}{T1}{%
  \textormath{\quotedblbase}{\mbox{\quotedblbase}}}
\ProvideTextCommandDefault{\glqq}{\UseTextSymbol{OT1}\glqq}
%    \end{macrocode}
%    The definition of |\grqq| depends on the fontencoding. With
%    \texttt{T1} encoding no extra kerning is needed.
%    \begin{macrocode}
\ProvideTextCommand{\grqq}{T1}{%
  \textormath{\textquotedblleft}{\mbox{\textquotedblleft}}}
\ProvideTextCommand{\grqq}{OT1}{%
  \save@sf@q{\kern-.07em%
  \textormath{\textquotedblleft}{\mbox{\textquotedblleft}}%
  \kern.07em\relax}}
\ProvideTextCommandDefault{\grqq}{\UseTextSymbol{OT1}\grqq}
%    \end{macrocode}
%  \end{macro}
%  \end{macro}
%
%  \begin{macro}{\flq}
%  \begin{macro}{\frq}
% \changes{babel~3.5f}{1995/08/07}{corrected spelling of
%    \cs{quilsingl...}}
% \changes{babel~3.5f}{1995/09/05}{now use \cs{textormath} in these
%    definitions}
% \changes{babel~3.8b}{2004/05/02}{Made \cs{flq} and \cs{frq}
%    fontencoding dependent} 
%    The `french' single guillemets.
%    \begin{macrocode}
\ProvideTextCommand{\flq}{OT1}{%
  \textormath{\guilsinglleft}{\mbox{\guilsinglleft}}}
\ProvideTextCommand{\flq}{T1}{%
  \textormath{\guilsinglleft}{\mbox{\guilsinglleft}}}
\ProvideTextCommandDefault{\flq}{\UseTextSymbol{OT1}\flq}
%    \end{macrocode}
%    
%    \begin{macrocode}
\ProvideTextCommand{\frq}{OT1}{%
  \textormath{\guilsinglright}{\mbox{\guilsinglright}}}
\ProvideTextCommand{\frq}{T1}{%
  \textormath{\guilsinglright}{\mbox{\guilsinglright}}}
\ProvideTextCommandDefault{\frq}{\UseTextSymbol{OT1}\frq}
%    \end{macrocode}
%  \end{macro}
%  \end{macro}
%
%  \begin{macro}{\flqq}
%  \begin{macro}{\frqq}
% \changes{babel~3.5f}{1995/08/07}{corrected spelling of
%    \cs{quillemot...}}
% \changes{babel~3.5f}{1995/09/05}{now use \cs{textormath} in these
%    definitions}
% \changes{babel~3.8b}{2004/05/02}{Made \cs{flqq} and \cs{frqq}
%    fontencoding dependent} 
%    The `french' double guillemets.
%    \begin{macrocode}
\ProvideTextCommand{\flqq}{OT1}{%
  \textormath{\guillemotleft}{\mbox{\guillemotleft}}}
\ProvideTextCommand{\flqq}{T1}{%
  \textormath{\guillemotleft}{\mbox{\guillemotleft}}}
\ProvideTextCommandDefault{\flqq}{\UseTextSymbol{OT1}\flqq}
%    \end{macrocode}
%    
%    \begin{macrocode}
\ProvideTextCommand{\frqq}{OT1}{%
  \textormath{\guillemotright}{\mbox{\guillemotright}}}
\ProvideTextCommand{\frqq}{T1}{%
  \textormath{\guillemotright}{\mbox{\guillemotright}}}
\ProvideTextCommandDefault{\frqq}{\UseTextSymbol{OT1}\frqq}
%    \end{macrocode}
%  \end{macro}
%  \end{macro}
%
%  \subsection{Umlauts and trema's}
%
%    The command |\"| needs to have a different effect for different
%    languages. For German for instance, the `umlaut' should be
%    positioned lower than the default position for placing it over
%    the letters a, o, u, A, O and U. When placed over an e, i, E or I
%    it can retain its normal position. For Dutch the same glyph is
%    always placed in the lower position.
%
%  \begin{macro}{\umlauthigh}
% \changes{v3.8a}{2004/02/19}{Use \cs{leavevmode}\cs{bgroup} to
%    prevent problems when this command occurs in vertical mode.}
%  \begin{macro}{\umlautlow}
%    To be able to provide both positions of |\"| we provide two
%    commands to switch the positioning, the default will be
%    |\umlauthigh| (the normal positioning).
%    \begin{macrocode}
\def\umlauthigh{%
  \def\bbl@umlauta##1{\leavevmode\bgroup%
      \expandafter\accent\csname\f@encoding dqpos\endcsname
      ##1\allowhyphens\egroup}%
  \let\bbl@umlaute\bbl@umlauta}
\def\umlautlow{%
  \def\bbl@umlauta{\protect\lower@umlaut}}
\def\umlautelow{%
  \def\bbl@umlaute{\protect\lower@umlaut}}
\umlauthigh
%    \end{macrocode}
%  \end{macro}
%  \end{macro}
%
%  \begin{macro}{\lower@umlaut}
%    The command |\lower@umlaut| is used to position the |\"| closer
%    the the letter.
%
%    We want the umlaut character lowered, nearer to the letter. To do
%    this we need an extra \meta{dimen} register.
%    \begin{macrocode}
\expandafter\ifx\csname U@D\endcsname\relax
  \csname newdimen\endcsname\U@D
\fi
%    \end{macrocode}
%    The following code fools \TeX's \texttt{make\_accent} procedure
%    about the current x-height of the font to force another placement
%    of the umlaut character.
%    \begin{macrocode}
\def\lower@umlaut#1{%
%    \end{macrocode}
%    First we have to save the current x-height of the font, because
%    we'll change this font dimension and this is always done
%    globally.
% \changes{v3.8a}{2004/02/19}{Use \cs{leavevmode}\cs{bgroup} to
%    prevent problems when this command occurs in vertical mode.}
%    \begin{macrocode}
  \leavevmode\bgroup
    \U@D 1ex%
%    \end{macrocode}
%    Then we compute the new x-height in such a way that the umlaut
%    character is lowered to the base character.  The value of
%    \texttt{.45ex} depends on the \MF\ parameters with which the
%    fonts were built.  (Just try out, which value will look best.)
%    \begin{macrocode}
    {\setbox\z@\hbox{%
      \expandafter\char\csname\f@encoding dqpos\endcsname}%
      \dimen@ -.45ex\advance\dimen@\ht\z@
%    \end{macrocode}
%    If the new x-height is too low, it is not changed.
%    \begin{macrocode}
      \ifdim 1ex<\dimen@ \fontdimen5\font\dimen@ \fi}%
%    \end{macrocode}
%    Finally we call the |\accent| primitive, reset the old x-height
%    and insert the base character in the argument.
% \changes{babel~3.5f}{1996/04/02}{Added a \cs{allowhyphens}}
% \changes{babel~3.5f}{1996/06/25}{removed \cs{allowhyphens}}
%    \begin{macrocode}
    \expandafter\accent\csname\f@encoding dqpos\endcsname
    \fontdimen5\font\U@D #1%
  \egroup}
%    \end{macrocode}
%  \end{macro}
%
%    For all vowels we declare |\"| to be a composite command which
%    uses |\bbl@umlauta| or |\bbl@umlaute| to position the umlaut
%    character. We need to be sure that these definitions override the
%    ones that are provided when the package \pkg{fontenc} with
%    option \Lopt{OT1} is used. Therefor these declarations are
%    postponed until the beginning of the document.
%    \begin{macrocode}
\AtBeginDocument{%
  \DeclareTextCompositeCommand{\"}{OT1}{a}{\bbl@umlauta{a}}%
  \DeclareTextCompositeCommand{\"}{OT1}{e}{\bbl@umlaute{e}}%
  \DeclareTextCompositeCommand{\"}{OT1}{i}{\bbl@umlaute{\i}}%
  \DeclareTextCompositeCommand{\"}{OT1}{\i}{\bbl@umlaute{\i}}%
  \DeclareTextCompositeCommand{\"}{OT1}{o}{\bbl@umlauta{o}}%
  \DeclareTextCompositeCommand{\"}{OT1}{u}{\bbl@umlauta{u}}%
  \DeclareTextCompositeCommand{\"}{OT1}{A}{\bbl@umlauta{A}}%
  \DeclareTextCompositeCommand{\"}{OT1}{E}{\bbl@umlaute{E}}%
  \DeclareTextCompositeCommand{\"}{OT1}{I}{\bbl@umlaute{I}}%
  \DeclareTextCompositeCommand{\"}{OT1}{O}{\bbl@umlauta{O}}%
  \DeclareTextCompositeCommand{\"}{OT1}{U}{\bbl@umlauta{U}}%
}
%    \end{macrocode}
%
% \subsection{The redefinition of the style commands}
%
%    The rest of the code in this file can only be processed by
%    \LaTeX, so we check the current format. If it is plain \TeX,
%    processing should stop here. But, because of the need to limit
%    the scope of the definition of |\format|, a macro that is used
%    locally in the following |\if|~statement, this comparison is done
%    inside a group. To prevent \TeX\ from complaining about an
%    unclosed group, the processing of the command |\endinput| is
%    deferred until after the group is closed. This is accomplished by
%    the command |\aftergroup|.
%    \begin{macrocode}
{\def\format{lplain}
\ifx\fmtname\format
\else
  \def\format{LaTeX2e}
  \ifx\fmtname\format
  \else
    \aftergroup\endinput
  \fi
\fi}
%    \end{macrocode}
%
%    Now that we're sure that the code is seen by \LaTeX\ only, we
%    have to find out what the main (primary) document style is
%    because we want to redefine some macros.  This is only necessary
%    for releases of \LaTeX\ dated before December~1991. Therefor
%    this part of the code can optionally be included in
%    \file{babel.def} by specifying the \texttt{docstrip} option
%    \texttt{names}.
%    \begin{macrocode}
%<*names>
%    \end{macrocode}
%
%    The standard styles can be distinguished by checking whether some
%    macros are defined. In table~\ref{styles} an overview is given of
%    the macros that can be used for this purpose.
%  \begin{table}[htb]
%  \begin{center}
% \DeleteShortVerb{\|}
%  \begin{tabular}{|lcp{8cm}|}
%   \hline
%   article         & : & both the \verb+\chapter+ and \verb+\opening+
%                         macros are undefined\\
%   report and book & : & the \verb+\chapter+ macro is defined and
%                         the \verb+\opening+ is undefined\\
%   letter          & : & the \verb+\chapter+ macro is undefined and
%                         the \verb+\opening+ is defined\\
%   \hline
%  \end{tabular}
% \caption{How to determine the main document style}\label{styles}
% \MakeShortVerb{\|}
%  \end{center}
%  \end{table}
%
%    \noindent The macros that have to be redefined for the
%    \texttt{report} and \texttt{book} document styles happen to be
%    the same, so there is no need to distinguish between those two
%    styles.
%
%  \begin{macro}{\doc@style}
%    First a parameter |\doc@style| is defined to identify the current
%    document style. This parameter might have been defined by a
%    document style that already uses macros instead of hard-wired
%    texts, such as \file{artikel1.sty}~\cite{BEP}, so the existence of
%    |\doc@style| is checked. If this macro is undefined, i.\,e., if
%    the document style is unknown and could therefore contain
%    hard-wired texts, |\doc@style| is defined to the default
%    value~`0'.
% \changes{babel~3.0d}{1991/10/29}{Removed use of \cs{@ifundefined}}
%    \begin{macrocode}
\ifx\@undefined\doc@style
  \def\doc@style{0}%
%    \end{macrocode}
%    This parameter is defined in the following \texttt{if}
%    construction (see table~\ref{styles}):
%
%    \begin{macrocode}
  \ifx\@undefined\opening
    \ifx\@undefined\chapter
      \def\doc@style{1}%
    \else
      \def\doc@style{2}%
    \fi
  \else
    \def\doc@style{3}%
  \fi%
\fi%
%    \end{macrocode}
%  \end{macro}
%
% \changes{babel~3.1}{1991/11/05}{Removed definition of
%    \cs{if@restonecol}}
%
%    \subsubsection{Redefinition of macros}
%
%    Now here comes the real work: we start to redefine things and
%    replace hard-wired texts by macros. These redefinitions should be
%    carried out conditionally, in case it has already been done.
%
%    For the \texttt{figure} and \texttt{table} environments we have
%    in all styles:
%    \begin{macrocode}
\@ifundefined{figurename}{\def\fnum@figure{\figurename{} \thefigure}}{}
\@ifundefined{tablename}{\def\fnum@table{\tablename{} \thetable}}{}
%    \end{macrocode}
%
%    The rest of the macros have to be treated differently for each
%    style.  When |\doc@style| still has its default value nothing
%    needs to be done.
%    \begin{macrocode}
\ifcase \doc@style\relax
\or
%    \end{macrocode}
%
%    This means that \file{babel.def} is read after the
%    \texttt{article} style, where no |\chapter| and |\opening|
%    commands are defined\footnote{A fact that was pointed out to me
%    by Nico Poppelier and was already used in Piet van Oostrum's
%    document style option~\texttt{nl}.}.
%
%    First we have the |\tableofcontents|,
%    |\listoffigures| and |\listoftables|:
%    \begin{macrocode}
\@ifundefined{contentsname}%
    {\def\tableofcontents{\section*{\contentsname\@mkboth
          {\uppercase{\contentsname}}{\uppercase{\contentsname}}}%
      \@starttoc{toc}}}{}

\@ifundefined{listfigurename}%
    {\def\listoffigures{\section*{\listfigurename\@mkboth
          {\uppercase{\listfigurename}}{\uppercase{\listfigurename}}}%
     \@starttoc{lof}}}{}

\@ifundefined{listtablename}%
    {\def\listoftables{\section*{\listtablename\@mkboth
          {\uppercase{\listtablename}}{\uppercase{\listtablename}}}%
      \@starttoc{lot}}}{}
%    \end{macrocode}
%
% Then the |\thebibliography| and |\theindex| environments.
%
%    \begin{macrocode}
\@ifundefined{refname}%
    {\def\thebibliography#1{\section*{\refname
      \@mkboth{\uppercase{\refname}}{\uppercase{\refname}}}%
      \list{[\arabic{enumi}]}{\settowidth\labelwidth{[#1]}%
        \leftmargin\labelwidth
        \advance\leftmargin\labelsep
        \usecounter{enumi}}%
        \def\newblock{\hskip.11em plus.33em minus.07em}%
        \sloppy\clubpenalty4000\widowpenalty\clubpenalty
        \sfcode`\.=1000\relax}}{}

\@ifundefined{indexname}%
    {\def\theindex{\@restonecoltrue\if@twocolumn\@restonecolfalse\fi
     \columnseprule \z@
     \columnsep 35pt\twocolumn[\section*{\indexname}]%
       \@mkboth{\uppercase{\indexname}}{\uppercase{\indexname}}%
       \thispagestyle{plain}%
       \parskip\z@ plus.3pt\parindent\z@\let\item\@idxitem}}{}
%    \end{macrocode}
%
% The |abstract| environment:
%
%    \begin{macrocode}
\@ifundefined{abstractname}%
    {\def\abstract{\if@twocolumn
    \section*{\abstractname}%
    \else \small
    \begin{center}%
    {\bf \abstractname\vspace{-.5em}\vspace{\z@}}%
    \end{center}%
    \quotation
    \fi}}{}
%    \end{macrocode}
%
% And last but not least, the macro |\part|:
%
%    \begin{macrocode}
\@ifundefined{partname}%
{\def\@part[#1]#2{\ifnum \c@secnumdepth >\m@ne
        \refstepcounter{part}%
        \addcontentsline{toc}{part}{\thepart
        \hspace{1em}#1}\else
      \addcontentsline{toc}{part}{#1}\fi
   {\parindent\z@ \raggedright
    \ifnum \c@secnumdepth >\m@ne
      \Large \bf \partname{} \thepart
      \par \nobreak
    \fi
    \huge \bf
    #2\markboth{}{}\par}%
    \nobreak
    \vskip 3ex\@afterheading}%
}{}
%    \end{macrocode}
%
%    This is all that needs to be done for the \texttt{article} style.
%
%    \begin{macrocode}
\or
%    \end{macrocode}
%
%    The next case is formed by the two styles \texttt{book} and
%    \texttt{report}.  Basically we have to do the same as for the
%    \texttt{article} style, except now we must also change the
%    |\chapter| command.
%
%    The tables of contents, figures and tables:
%    \begin{macrocode}
\@ifundefined{contentsname}%
    {\def\tableofcontents{\@restonecolfalse
      \if@twocolumn\@restonecoltrue\onecolumn
      \fi\chapter*{\contentsname\@mkboth
          {\uppercase{\contentsname}}{\uppercase{\contentsname}}}%
      \@starttoc{toc}%
      \csname if@restonecol\endcsname\twocolumn
      \csname fi\endcsname}}{}

\@ifundefined{listfigurename}%
    {\def\listoffigures{\@restonecolfalse
      \if@twocolumn\@restonecoltrue\onecolumn
      \fi\chapter*{\listfigurename\@mkboth
          {\uppercase{\listfigurename}}{\uppercase{\listfigurename}}}%
      \@starttoc{lof}%
      \csname if@restonecol\endcsname\twocolumn
      \csname fi\endcsname}}{}

\@ifundefined{listtablename}%
    {\def\listoftables{\@restonecolfalse
      \if@twocolumn\@restonecoltrue\onecolumn
      \fi\chapter*{\listtablename\@mkboth
          {\uppercase{\listtablename}}{\uppercase{\listtablename}}}%
      \@starttoc{lot}%
      \csname if@restonecol\endcsname\twocolumn
      \csname fi\endcsname}}{}
%    \end{macrocode}
%
%    Again, the |bibliography| and |index| environments; notice that
%    in this case we use |\bibname| instead of |\refname| as in the
%    definitions for the \texttt{article} style.  The reason for this
%    is that in the \texttt{article} document style the term
%    `References' is used in the definition of |\thebibliography|. In
%    the \texttt{report} and \texttt{book} document styles the term
%    `Bibliography' is used.
%    \begin{macrocode}
\@ifundefined{bibname}%
    {\def\thebibliography#1{\chapter*{\bibname
     \@mkboth{\uppercase{\bibname}}{\uppercase{\bibname}}}%
     \list{[\arabic{enumi}]}{\settowidth\labelwidth{[#1]}%
     \leftmargin\labelwidth \advance\leftmargin\labelsep
     \usecounter{enumi}}%
     \def\newblock{\hskip.11em plus.33em minus.07em}%
     \sloppy\clubpenalty4000\widowpenalty\clubpenalty
     \sfcode`\.=1000\relax}}{}

\@ifundefined{indexname}%
    {\def\theindex{\@restonecoltrue\if@twocolumn\@restonecolfalse\fi
    \columnseprule \z@
    \columnsep 35pt\twocolumn[\@makeschapterhead{\indexname}]%
      \@mkboth{\uppercase{\indexname}}{\uppercase{\indexname}}%
    \thispagestyle{plain}%
    \parskip\z@ plus.3pt\parindent\z@ \let\item\@idxitem}}{}
%    \end{macrocode}
%
% Here is the |abstract| environment:
%    \begin{macrocode}
\@ifundefined{abstractname}%
    {\def\abstract{\titlepage
    \null\vfil
    \begin{center}%
    {\bf \abstractname}%
    \end{center}}}{}
%    \end{macrocode}
%
%     And last but not least the |\chapter|, |\appendix| and
%    |\part| macros.
%    \begin{macrocode}
\@ifundefined{chaptername}{\def\@chapapp{\chaptername}}{}
%
\@ifundefined{appendixname}%
    {\def\appendix{\par
      \setcounter{chapter}{0}%
      \setcounter{section}{0}%
      \def\@chapapp{\appendixname}%
      \def\thechapter{\Alph{chapter}}}}{}
%
\@ifundefined{partname}%
    {\def\@part[#1]#2{\ifnum \c@secnumdepth >-2\relax
            \refstepcounter{part}%
            \addcontentsline{toc}{part}{\thepart
            \hspace{1em}#1}\else
            \addcontentsline{toc}{part}{#1}\fi
       \markboth{}{}%
       {\centering
        \ifnum \c@secnumdepth >-2\relax
          \huge\bf \partname{} \thepart
        \par
        \vskip 20pt \fi
        \Huge \bf
        #1\par}\@endpart}}{}%
%    \end{macrocode}
%
%    \begin{macrocode}
\or
%    \end{macrocode}
%
%    Now we address the case where \file{babel.def} is read after the
%    \texttt{letter} style. The \texttt{letter} document style
%    defines the macro |\opening| and some other macros that are
%    specific to \texttt{letter}. This means that we have to redefine
%    other macros, compared to the previous two cases.
%
%    First two macros for the material at the end of a letter, the
%    |\cc| and |\encl| macros.
%    \begin{macrocode}
\@ifundefined{ccname}%
    {\def\cc#1{\par\noindent
     \parbox[t]{\textwidth}%
     {\@hangfrom{\rm \ccname : }\ignorespaces #1\strut}\par}}{}

\@ifundefined{enclname}%
    {\def\encl#1{\par\noindent
     \parbox[t]{\textwidth}%
     {\@hangfrom{\rm \enclname : }\ignorespaces #1\strut}\par}}{}
%    \end{macrocode}
%
%    The last thing we have to do here is to redefine the
%    \texttt{headings} pagestyle:
% \changes{babel~3.3}{1993/07/11}{\cs{headpagename} should be
%    \cs{pagename}}
%    \begin{macrocode}
\@ifundefined{headtoname}%
    {\def\ps@headings{%
        \def\@oddhead{\sl \headtoname{} \ignorespaces\toname \hfil
                      \@date \hfil \pagename{} \thepage}%
        \def\@oddfoot{}}}{}
%    \end{macrocode}
%
%    This was the last of the four standard document styles, so if
%    |\doc@style| has another value we do nothing and just close the
%    \texttt{if} construction.
%    \begin{macrocode}
\fi
%    \end{macrocode}
%    Here ends the code that can be optionally included when a version
%    of \LaTeX\ is in use that is dated \emph{before} December~1991.
%    \begin{macrocode}
%</names>
%</core>
%    \end{macrocode}
%
% \subsection{Cross referencing macros}
%
%    The \LaTeX\ book states:
%  \begin{quote}
%    The \emph{key} argument is any sequence of letters, digits, and
%    punctuation symbols; upper- and lowercase letters are regarded as
%    different.
%  \end{quote}
%    When the above quote should still be true when a document is
%    typeset in a language that has active characters, special care
%    has to be taken of the category codes of these characters when
%    they appear in an argument of the cross referencing macros.
%
%    When a cross referencing command processes its argument, all
%    tokens in this argument should be character tokens with category
%    `letter' or `other'.
%
%    The only way to accomplish this in most cases is to use the trick
%    described in the \TeX book~\cite{DEK} (Appendix~D, page~382).
%    The primitive |\meaning| applied to a token expands to the
%    current meaning of this token.  For example, `|\meaning\A|' with
%    |\A| defined as `|\def\A#1{\B}|' expands to the characters
%    `|macro:#1->\B|' with all category codes set to `other' or
%    `space'.
%
%  \begin{macro}{\bbl@redefine}
% \changes{babel~3.5f}{1995/11/15}{Macro added}
%    To redefine a command, we save the old meaning of the macro.
%    Then we redefine it to call the original macro with the
%    `sanitized' argument.  The reason why we do it this way is that
%    we don't want to redefine the \LaTeX\ macros completely in case
%    their definitions change (they have changed in the past).
%
%    Because we need to redefine a number of commands we define the
%    command |\bbl@redefine| which takes care of this. It creates a
%    new control sequence, |\org@...|
%    \begin{macrocode}
%<*core|shorthands>
\def\bbl@redefine#1{%
  \edef\bbl@tempa{\expandafter\@gobble\string#1}%
  \expandafter\let\csname org@\bbl@tempa\endcsname#1
  \expandafter\def\csname\bbl@tempa\endcsname}
%    \end{macrocode}
%
%    This command should only be used in the preamble of the document.
%    \begin{macrocode}
\@onlypreamble\bbl@redefine
%    \end{macrocode}
%  \end{macro}
%
%  \begin{macro}{\bbl@redefine@long}
% \changes{babel~3.6f}{1997/01/14}{Macro added}
%    This version of |\babel@redefine| can be used to redefine |\long|
%    commands such as |\ifthenelse|.
%    \begin{macrocode}
\def\bbl@redefine@long#1{%
  \edef\bbl@tempa{\expandafter\@gobble\string#1}%
  \expandafter\let\csname org@\bbl@tempa\endcsname#1
  \expandafter\long\expandafter\def\csname\bbl@tempa\endcsname}
\@onlypreamble\bbl@redefine@long
%    \end{macrocode}
%  \end{macro}
%
%  \begin{macro}{\bbl@redefinerobust}
% \changes{babel~3.5f}{1995/11/15}{Macro added}
%    For commands that are redefined, but which \textit{might} be
%    robust we need a slightly more intelligent macro. A robust
%    command |foo| is defined to expand to |\protect|\verb*|\foo |. So
%    it is necessary to check whether \verb*|\foo | exists.
%    \begin{macrocode}
\def\bbl@redefinerobust#1{%
  \edef\bbl@tempa{\expandafter\@gobble\string#1}%
  \expandafter\ifx\csname \bbl@tempa\space\endcsname\relax
    \expandafter\let\csname org@\bbl@tempa\endcsname#1
    \expandafter\edef\csname\bbl@tempa\endcsname{\noexpand\protect
      \expandafter\noexpand\csname\bbl@tempa\space\endcsname}%
  \else
    \expandafter\let\csname org@\bbl@tempa\expandafter\endcsname
                    \csname\bbl@tempa\space\endcsname
  \fi
%    \end{macrocode}
%    The result of the code above is that the command that is being
%    redefined is always robust afterwards. Therefor all we need to do
%    now is define \verb*|\foo |.
% \changes{babel~3.5f}{1996/04/09}{Define \cs*{foo } instead of
%    \cs{foo}}
%    \begin{macrocode}
  \expandafter\def\csname\bbl@tempa\space\endcsname}
%    \end{macrocode}
%
%    This command should only be used in the preamble of the document.
%    \begin{macrocode}
\@onlypreamble\bbl@redefinerobust
%    \end{macrocode}
%  \end{macro}
%
%  \begin{macro}{\newlabel}
% \changes{babel~3.5f}{1995/11/15}{Now use \cs{bbl@redefine}}
%    The macro |\label| writes a line with a |\newlabel| command
%    into the |.aux| file to define labels.
%    \begin{macrocode}
%\bbl@redefine\newlabel#1#2{%
%  \@safe@activestrue\org@newlabel{#1}{#2}\@safe@activesfalse}
%    \end{macrocode}
%  \end{macro}
%
%  \begin{macro}{\@newl@bel}
% \changes{babel~3.6i}{1997/03/01}{Now redefine \cs{@newl@bel} instead
%    of \cs{@lbibitem} and \cs{newlabel}}
%    We need to change the definition of the \LaTeX-internal macro
%    |\@newl@bel|. This is needed because we need to make sure that
%    shorthand characters expand to their non-active version.
%
%^^A The following lines commented out in preparation for a change
%^^A in the LaTeX definition of \@enwl@bel... JLB 2000/10/01
%^^A
%^^A    To play it safe when redefining a \LaTeX-internal command we
%^^A    first check whether its definition didn't change.
%^^A    \begin{macrocode}
%^^A\CheckCommand*\@newl@bel[3]{%
%^^A  \@ifundefined{#1@#2}%
%^^A    \relax
%^^A    {\gdef \@multiplelabels {%
%^^A       \@latex@warning@no@line{There were multiply-defined labels}}%
%^^A     \@latex@warning@no@line{Label `#2' multiply defined}}%
%^^A  \global\@namedef{#1@#2}{#3}}
%^^A    \end{macrocode}
%^^A    Then we give the new definition.
%    \begin{macrocode}
\def\@newl@bel#1#2#3{%
%    \end{macrocode}
%    First we open a new group to keep the changed setting of
%    |\protect| local and then we set the |@safe@actives| switch to
%    true to make sure that any shorthand that appears in any of the
%    arguments immediately expands to its non-active self.
% \changes{babel~3.7a}{1997/12/19}{Call \cs{@safe@activestrue}
%    directly}
%    \begin{macrocode}
  {%
    \@safe@activestrue
    \@ifundefined{#1@#2}%
      \relax
      {%
        \gdef \@multiplelabels {%
          \@latex@warning@no@line{There were multiply-defined labels}}%
        \@latex@warning@no@line{Label `#2' multiply defined}%
      }%
    \global\@namedef{#1@#2}{#3}%
    }%
  }
%    \end{macrocode}
%  \end{macro}
%
%  \begin{macro}{\@testdef}
%    An internal \LaTeX\ macro used to test if the labels that have
%    been written on the |.aux| file have changed.  It is called by
%    the |\enddocument| macro. This macro needs to be completely
%    rewritten, using |\meaning|. The reason for this is that in some
%    cases the expansion of |\#1@#2| contains the same characters as
%    the |#3|; but the character codes differ. Therefor \LaTeX\ keeps
%    reporting that the labels may have changed.
% \changes{babel~3.4g}{1994/08/30}{Moved the \cs{def} inside the
%    macrocode environment}
% \changes{babel~3.5f}{1995/11/15}{Now use \cs{bbl@redefine}}
% \changes{babel~3.5f}{1996/01/09}{Complete rewrite of this macro as
%    the same character ended up with different category codes in the
%    labels that are being compared. Now use \cs{meaning}}
% \changes{babel~3.5f}{1996/01/16}{Use \cs{strip@prefix} only on
%    \cs{bbl@tempa} when it is not \cs{relax}}
% \changes{babel~3.6i}{1997/02/28}{Make sure that shorthands don't get
%    expanded at the wrong moment.}
% \changes{babel~3.6i}{1997/03/01}{\cs{@safe@activesfalse} is now
%    part of the label definition}
% \changes{babel~3.7a}{1998/03/13}{Removed \cs{@safe@activesfalse}
%    from the label definition}
%    \begin{macrocode}
\CheckCommand*\@testdef[3]{%
  \def\reserved@a{#3}%
  \expandafter \ifx \csname #1@#2\endcsname \reserved@a
  \else
    \@tempswatrue
  \fi}
%    \end{macrocode}
%    Now that we made sure that |\@testdef| still has the same
%    definition we can rewrite it. First we make the shorthands
%    `safe'.
%    \begin{macrocode}
\def\@testdef #1#2#3{%
  \@safe@activestrue
%    \end{macrocode}
%    Then we use |\bbl@tempa| as an `alias' for the macro that
%    contains the label which is being checked.
%    \begin{macrocode}
  \expandafter\let\expandafter\bbl@tempa\csname #1@#2\endcsname
%    \end{macrocode}
%    Then we define |\bbl@tempb| just as |\@newl@bel| does it.
%    \begin{macrocode}
  \def\bbl@tempb{#3}%
  \@safe@activesfalse
%    \end{macrocode}
%    When the label is defined we replace the definition of
%    |\bbl@tempa| by its meaning.
%    \begin{macrocode}
  \ifx\bbl@tempa\relax
  \else
    \edef\bbl@tempa{\expandafter\strip@prefix\meaning\bbl@tempa}%
  \fi
%    \end{macrocode}
%    We do the same for |\bbl@tempb|.
%    \begin{macrocode}
  \edef\bbl@tempb{\expandafter\strip@prefix\meaning\bbl@tempb}%
%    \end{macrocode}
%    If the label didn't change, |\bbl@tempa| and |\bbl@tempb| should
%    be identical macros.
%    \begin{macrocode}
  \ifx \bbl@tempa \bbl@tempb
  \else
    \@tempswatrue
  \fi}
%    \end{macrocode}
%  \end{macro}
%
%  \begin{macro}{\ref}
%  \begin{macro}{\pageref}
%    The same holds for the macro |\ref| that references a label
%    and |\pageref| to reference a page. So we redefine |\ref| and
%    |\pageref|. While we change these macros, we make them robust as
%    well (if they weren't already) to prevent problems if they should
%    become expanded at the wrong moment.
% \changes{babel~3.5b}{1995/03/07}{Made \cs{ref} and \cs{pageref}
%    robust (PR1353)}
% \changes{babel~3.5d}{1995/07/04}{use a different control sequence
%    while making \cs{ref} and \cs{pageref} robust}
% \changes{babel~3.5f}{1995/11/06}{redefine \cs*{ref } if it exists
%    instead of \cs{ref}}
% \changes{babel~3.5f}{1995/11/15}{Now use \cs{bbl@redefinerobust}}
% \changes{babel~3.5f}{1996/01/19}{redefine \cs{\@setref} instead of
%    \cs{ref} and \cs{pageref} in \LaTeXe.}
% \changes{babel~3.5f}{1996/01/21}{Reverse the previous change as it
%    inhibits the use of active characters in labels}
%    \begin{macrocode}
\bbl@redefinerobust\ref#1{%
  \@safe@activestrue\org@ref{#1}\@safe@activesfalse}
\bbl@redefinerobust\pageref#1{%
  \@safe@activestrue\org@pageref{#1}\@safe@activesfalse}
%    \end{macrocode}
%  \end{macro}
%  \end{macro}
%
%  \begin{macro}{\@citex}
% \changes{babel~3.5f}{1995/11/15}{Now use \cs{bbl@redefine}}
%    The macro used to cite from a bibliography, |\cite|, uses an
%    internal macro, |\@citex|.
%    It is this internal macro that picks up the argument(s),
%    so we redefine this internal macro and leave |\cite| alone. The
%    first argument is used for typesetting, so the shorthands need
%    only be deactivated in the second argument.
% \changes{babel~3.7g}{2000/10/01}{The shorthands need to be
%    deactivated for the second argument of \cs{@citex} only.}
%    \begin{macrocode}
\bbl@redefine\@citex[#1]#2{%
  \@safe@activestrue\edef\@tempa{#2}\@safe@activesfalse
  \org@@citex[#1]{\@tempa}}
%    \end{macrocode}
%    Unfortunately, the packages \pkg{natbib} and \pkg{cite} need a
%    different definition of |\@citex|...
%    To begin with, \pkg{natbib} has a definition for |\@citex| with
%    \emph{three} arguments... We only know that a package is loaded
%    when |\begin{document}| is executed, so we need to postpone the
%    different redefinition.
%    \begin{macrocode}
\AtBeginDocument{%
  \@ifpackageloaded{natbib}{%
%    \end{macrocode}
%    Notice that we use |\def| here instead of |\bbl@redefine| because
%    |\org@@citex| is already defined and we don't want to overwrite
%    that definition (it would result in parameter stack overflow
%    because of a circular definition).
%    \begin{macrocode}
    \def\@citex[#1][#2]#3{%
      \@safe@activestrue\edef\@tempa{#3}\@safe@activesfalse
      \org@@citex[#1][#2]{\@tempa}}%
  }{}}
%    \end{macrocode}
%    The package \pkg{cite} has a definition of |\@citex| where the
%    shorthands need to be turned off in both arguments.
%    \begin{macrocode}
\AtBeginDocument{%
  \@ifpackageloaded{cite}{%
    \def\@citex[#1]#2{%
      \@safe@activestrue\org@@citex[#1]{#2}\@safe@activesfalse}%
    }{}}
%    \end{macrocode}
%  \end{macro}
%
%  \begin{macro}{\nocite}
% \changes{babel~3.5f}{1995/11/15}{Now use \cs{bbl@redefine}}
%    The macro |\nocite| which is used to instruct BiB\TeX\ to
%    extract uncited references from the database.
%    \begin{macrocode}
\bbl@redefine\nocite#1{%
  \@safe@activestrue\org@nocite{#1}\@safe@activesfalse}
%    \end{macrocode}
%  \end{macro}
%
%  \begin{macro}{\bibcite}
% \changes{babel~3.5f}{1995/11/15}{Now use \cs{bbl@redefine}}
%    The macro that is used in the |.aux| file to define citation
%    labels. When packages such as \pkg{natbib} or \pkg{cite} are not
%    loaded its second argument is used to typeset the citation
%    label. In that case, this second argument can contain active
%    characters but is used in an environment where
%    |\@safe@activestrue| is in effect. This switch needs to be reset
%    inside the |\hbox| which contains the citation label. In order to
%    determine during \file{.aux} file processing which definition of
%    |\bibcite| is needed we define |\bibcite| in such a way that it
%    redefines itself with the proper definition.
% \changes{babel~3.6s}{1999/04/13}{Need to determine `online' which
%    definition of \cs{bibcite} is needed}
% \changes{babel~3.6v}{1999/04/21}{Also check for \pkg{cite} it can't
%    handle \cs{@safe@activesfalse} in its second argument}
%    \begin{macrocode}
\bbl@redefine\bibcite{%
%    \end{macrocode}
%    We call |\bbl@cite@choice| to select the proper definition for
%    |\bibcite|. This new definition is then activated.
%    \begin{macrocode}
  \bbl@cite@choice
  \bibcite}
%    \end{macrocode}
%  \end{macro}
%
%  \begin{macro}{\bbl@bibcite}
% \changes{babel~3.6v}{1999/04/21}{Macro \cs{bbl@bibcite} added}
%    The macro |\bbl@bibcite| holds the definition of |\bibcite|
%    needed when neither \pkg{natbib} nor \pkg{cite} is loaded.
%    \begin{macrocode}
\def\bbl@bibcite#1#2{%
  \org@bibcite{#1}{\@safe@activesfalse#2}}
%    \end{macrocode}
%  \end{macro}
%
%  \begin{macro}{\bbl@cite@choice}
% \changes{babel~3.6v}{1999/04/21}{Macro \cs{bbl@cite@choice} added}
%    The macro |\bbl@cite@choice| determines which definition of
%    |\bibcite| is needed.
%    \begin{macrocode}
\def\bbl@cite@choice{%
%    \end{macrocode}
%    First we give |\bibcite| its default definition.
%    \begin{macrocode}
  \global\let\bibcite\bbl@bibcite
%    \end{macrocode}
%    Then, when \pkg{natbib} is loaded we restore the original
%    definition of |\bibcite| .
%    \begin{macrocode}
  \@ifpackageloaded{natbib}{\global\let\bibcite\org@bibcite}{}%
%    \end{macrocode}
%    For \pkg{cite} we do the same.
%    \begin{macrocode}
  \@ifpackageloaded{cite}{\global\let\bibcite\org@bibcite}{}%
%    \end{macrocode}
%    Make sure this only happens once.
%    \begin{macrocode}
  \global\let\bbl@cite@choice\relax
  }
%    \end{macrocode}
%
%    When a document is run for the first time, no \file{.aux} file is
%    available, and |\bibcite| will not yet be properly defined. In
%    this case, this has to happen before the document starts.
%    \begin{macrocode}
\AtBeginDocument{\bbl@cite@choice}
%    \end{macrocode}
%  \end{macro}
%
%  \begin{macro}{\@bibitem}
% \changes{babel~3.5f}{1995/11/15}{Now use \cs{bbl@redefine}}
%    One of the two internal \LaTeX\ macros called by |\bibitem|
%    that write the citation label on the |.aux| file.
%    \begin{macrocode}
\bbl@redefine\@bibitem#1{%
  \@safe@activestrue\org@@bibitem{#1}\@safe@activesfalse}
%    \end{macrocode}
%  \end{macro}
%
%  \subsection{marks}
%
%  \begin{macro}{\markright}
% \changes{babel~3.6i}{1997/03/15}{Added redefinition of \cs{mark...}
%    commands}
%    Because the output routine is asynchronous, we must
%    pass the current language attribute to the head lines, together
%    with the text that is put into them. To achieve this we need to
%    adapt the definition of |\markright| and |\markboth| somewhat.
% \changes{babel~3.7c}{1999/04/08}{Removed the use of \cs{head@lang}
%    (PR 2990)}
% \changes{babel~3.7c}{1999/04/09}{Avoid expanding the arguments by
%    storing them in token registers}
% \changes{babel~3.7m}{2003/11/15}{added \cs{bbl@restore@actives} to
%    the mark}
% \changes{babel~3.8c}{2004/05/26}{No need to add \emph{anything} to
%    an empty mark; prevented this by checking the contents of the
%    argument}
% \changes{babel~3.8f}{2005/05/15}{Make the definition independent of
%    the original definition; expand \cs{languagename} before passing
%    it into the token registers} 
%    \begin{macrocode}
\bbl@redefine\markright#1{%
%    \end{macrocode}
%    First of all we temporarily store the language switching command,
%    using an expanded definition in order to get the current value of
%    |\languagename|. 
%    \begin{macrocode}
  \edef\bbl@tempb{\noexpand\protect
    \noexpand\foreignlanguage{\languagename}}%
%    \end{macrocode}
%    Then, we check whether the argument is empty; if it is, we
%    just make sure the scratch token register is empty.
%    \begin{macrocode}
  \def\bbl@arg{#1}%
  \ifx\bbl@arg\@empty
    \toks@{}%
  \else
%    \end{macrocode}
%    Next, we store the argument to |\markright| in the scratch token
%    register, together with the expansion of |\bbl@tempb| (containing
%    the language switching command) as defined before. This way
%    these commands will not be expanded by using |\edef| later
%    on, and we make sure that the text is typeset using the
%    correct language settings. While doing so, we make sure that
%    active characters that may end up in the mark are not disabled by
%    the output routine kicking in while \cs{@safe@activestrue} is in
%    effect.
%    \begin{macrocode}
    \expandafter\toks@\expandafter{%
             \bbl@tempb{\protect\bbl@restore@actives#1}}%
  \fi
%    \end{macrocode}
%    Then we define a temporary control sequence using |\edef|.
%    \begin{macrocode}
  \edef\bbl@tempa{%
%    \end{macrocode}
%     When |\bbl@tempa| is executed, only |\languagename| will be
%    expanded, because of the way the token register was filled.
%    \begin{macrocode}
    \noexpand\org@markright{\the\toks@}}%
  \bbl@tempa
}
%    \end{macrocode}
%  \end{macro}
%
%  \begin{macro}{\markboth}
%  \begin{macro}{\@mkboth}
%    The definition of |\markboth| is equivalent to that of
%    |\markright|, except that we need two token registers. The
%    documentclasses \cls{report} and \cls{book} define and set the
%    headings for the page. While doing so they also store a copy of
%    |\markboth| in |\@mkboth|. Therefor we need to check whether
%    |\@mkboth| has already been set. If so we neeed to do that again
%    with the new definition of |\makrboth|.
% \changes{babel~3.7m}{2003/11/15}{added \cs{bbl@restore@actives} to
%    the mark}
% \changes{babel~3.8c}{2004/05/26}{No need to add \emph{anything} to
%    an empty mark, prevented this by checking the contents of the
%    arguments} 
% \changes{babel~3.8f}{2005/05/15}{Make the definition independent of
%    the original definition; expand \cs{languagename} before passing
%    it into the token registers} 
% \changes{babel~3.8j}{2008/03/21}{Added setting of \cs{@mkboth} (PR
%    3826)} 
%    \begin{macrocode}
\ifx\@mkboth\markboth
  \def\bbl@tempc{\let\@mkboth\markboth}
\else
  \def\bbl@tempc{}
\fi
%    \end{macrocode}
%    Now we can start the new definition of |\markboth|
%    \begin{macrocode}
\bbl@redefine\markboth#1#2{%
  \edef\bbl@tempb{\noexpand\protect
    \noexpand\foreignlanguage{\languagename}}%
  \def\bbl@arg{#1}%
  \ifx\bbl@arg\@empty
    \toks@{}%
  \else
   \expandafter\toks@\expandafter{%
             \bbl@tempb{\protect\bbl@restore@actives#1}}%
  \fi
  \def\bbl@arg{#2}%
  \ifx\bbl@arg\@empty
    \toks8{}%
  \else
    \expandafter\toks8\expandafter{%
             \bbl@tempb{\protect\bbl@restore@actives#2}}%
  \fi
  \edef\bbl@tempa{%
    \noexpand\org@markboth{\the\toks@}{\the\toks8}}%
  \bbl@tempa
}
%    \end{macrocode}
%    and copy it to |\@mkboth| if necesary.
%    \begin{macrocode}
\bbl@tempc
%</core|shorthands>
%    \end{macrocode}
%  \end{macro}
%  \end{macro}
%
%  \subsection{Encoding issues (part 2)}
%
% \changes{babel~3.7c}{1999/04/16}{Removed redefinition of \cs{@roman}
%    and \cs{@Roman}}
%
%  \begin{macro}{\TeX}
%  \begin{macro}{\LaTeX}
% \changes{babel~3.7a}{1998/03/12}{Make \TeX\ and \LaTeX\ logos
%    encoding-independent}
%    Because documents may use font encodings other than one of the
%    latin encodings, we make sure that the logos of \TeX\ and
%    \LaTeX\ always come out in the right encoding.
%    \begin{macrocode}
%<*core>
\bbl@redefine\TeX{\textlatin{\org@TeX}}
\bbl@redefine\LaTeX{\textlatin{\org@LaTeX}}
%</core>
%    \end{macrocode}
%  \end{macro}
%  \end{macro}
%
%  \subsection{Preventing clashes with other packages}
%
%  \subsubsection{\pkg{ifthen}}
%
%  \begin{macro}{\ifthenelse}
% \changes{babel~3.5g}{1996/08/11}{Redefinition of \cs{ifthenelse}
%    added to circumvent problems with \cs{pageref} in the argument of
%    \cs{isodd}}
%    Sometimes a document writer wants to create a special effect
%    depending on the page a certain fragment of text appears on. This
%    can be achieved by the following piece of code:
% \begin{verbatim}
%    \ifthenelse{\isodd{\pageref{some:label}}}
%               {code for odd pages}
%               {code for even pages}
% \end{verbatim}
%    In order for this to work the argument of |\isodd| needs to be
%    fully expandable. With the above redefinition of |\pageref| it is
%    not in the case of this example. To overcome that, we add some
%    code to the definition of |\ifthenelse| to make things work.
%
%    The first thing we need to do is check if the package
%    \pkg{ifthen} is loaded. This should be done at |\begin{document}|
%    time.
%    \begin{macrocode}
%<*package>
\AtBeginDocument{%
  \@ifpackageloaded{ifthen}{%
%    \end{macrocode}
%    Then we can redefine |\ifthenelse|:
% \changes{babel~3.6f}{1997/01/14}{\cs{ifthenelse} needs to be long}
%    \begin{macrocode}
    \bbl@redefine@long\ifthenelse#1#2#3{%
%    \end{macrocode}
%    We want to revert the definition of |\pageref| to its original
%    definition for the duration of |\ifthenelse|, so we first need to
%    store its current meaning.
%    \begin{macrocode}
      \let\bbl@tempa\pageref
      \let\pageref\org@pageref
%    \end{macrocode}
%    Then we can set the |\@safe@actives| switch and call the original
%    |\ifthenelse|. In order to be able to use shorthands in the
%    second and third arguments of |\ifthenelse| the resetting of the
%    switch \emph{and} the definition of |\pageref| happens inside
%    those arguments. 
% \changes{babel~3.6i}{1997/02/25}{Now reset the @safe@actives switch
%    inside the 2nd and 3rd arguments of \cs{ifthenelse}}
% \changes{babel~3.7f}{2000/06/29}{\cs{pageref} needs to have its
%    babel definition reinstated in the second and third arguments}
%    \begin{macrocode}
      \@safe@activestrue
      \org@ifthenelse{#1}{%
        \let\pageref\bbl@tempa
        \@safe@activesfalse
        #2}{%
        \let\pageref\bbl@tempa
        \@safe@activesfalse
        #3}%
      }%
%    \end{macrocode}
%    When the package wasn't loaded we do nothing.
%    \begin{macrocode}
    }{}%
  }
%    \end{macrocode}
%  \end{macro}
%
%  \subsubsection{\pkg{varioref}}
%
%  \begin{macro}{\@@vpageref}
% \changes{babel~3.6a}{1996/10/29}{Redefinition of \cs{@@vpageref}
%    added to circumvent problems with active \texttt{:} in the
%    argument of \cs{vref} when \pkg{varioref} is used}
%  \begin{macro}{\vrefpagenum}
% \changes{babel~3.7o}{2003/11/18}{Added redefinition of
%    \cs{vrefpagenum} which deals with ranges of pages}
%  \begin{macro}{\Ref}
% \changes{babel~3.8g}{2005/05/21}{We also need to adapt \cs{Ref}
%    which needs to be able to uppercase the first letter of the
%    expansion of \cs{ref}} 
%    When the package varioref is in use we need to modify its
%    internal command |\@@vpageref| in order to prevent problems when
%    an active character ends up in the argument of |\vref|.
%    \begin{macrocode}
\AtBeginDocument{%
  \@ifpackageloaded{varioref}{%
    \bbl@redefine\@@vpageref#1[#2]#3{%
      \@safe@activestrue
      \org@@@vpageref{#1}[#2]{#3}%
      \@safe@activesfalse}%
%    \end{macrocode}
%    The same needs to happen for |\vrefpagenum|.
%    \begin{macrocode}
    \bbl@redefine\vrefpagenum#1#2{%
      \@safe@activestrue
      \org@vrefpagenum{#1}{#2}%
      \@safe@activesfalse}%
%    \end{macrocode}
%    The package \pkg{varioref} defines |\Ref| to be a robust command
%    wich uppercases the first character of the reference text. In
%    order to be able to do that it needs to access the exandable form
%    of |\ref|. So we employ a little trick here. We redefine the
%    (internal) command \verb*|\Ref | to call |\org@ref| instead of
%    |\ref|. The disadvantgage of this solution is that whenever the
%    derfinition of |\Ref| changes, this definition needs to be updated
%    as well.
%    \begin{macrocode}
    \expandafter\def\csname Ref \endcsname#1{%
      \protected@edef\@tempa{\org@ref{#1}}\expandafter\MakeUppercase\@tempa}
    }{}%
  }
%    \end{macrocode}
%  \end{macro}
%  \end{macro}
%  \end{macro}
%
%  \subsubsection{\pkg{hhline}}
%
%  \begin{macro}{\hhline}
%    Delaying the activation of the shorthand characters has introduced
%    a problem with the \pkg{hhline} package. The reason is that it
%    uses the `:' character which is made active by the french support
%    in \babel. Therefor we need to \emph{reload} the package when
%    the `:' is an active character.
%
%    So at |\begin{document}| we check whether \pkg{hhline} is loaded.
%    \begin{macrocode}
\AtBeginDocument{%
  \@ifpackageloaded{hhline}%
%    \end{macrocode}
%    Then we check whether the expansion of |\normal@char:| is not
%    equal to |\relax|.
% \changes{babel~3.8b}{2004/04/19}{added \cs{string} to prevent
%    unwanted expansion of the colon}
%    \begin{macrocode}
    {\expandafter\ifx\csname normal@char\string:\endcsname\relax
     \else
%    \end{macrocode}
%    In that case we simply reload the package. Note that this happens
%    \emph{after} the category code of the @-sign has been changed to
%    other, so we need to temporarily change it to letter again.
%    \begin{macrocode}
       \makeatletter
       \def\@currname{hhline}\input{hhline.sty}\makeatother
     \fi}%
    {}}
%    \end{macrocode}
%  \end{macro}
%
%  \subsubsection{\pkg{hyperref}}
%
%  \begin{macro}{\pdfstringdefDisableCommands}
% \changes{babel~3.8j}{2008/03/16}{Inform \pkg{hyperref} to use
%    shorthands at system level (PR4006)}
%    Although a number of interworking problems between \pkg{babel}
%    and \pkg{hyperref} are tackled by \pkg{hyperref} itself we need
%    to take care of correctly handling the shorthand characters.
%    When they get expanded inside a bookmark a warning will appear in
%    the log file which can be prevented. This is done by informing
%    \pkg{hyperref} that it should the shorthands as defined on the
%    system level rather than at the user level.
%    
%    \begin{macrocode}
\AtBeginDocument{%
  \@ifundefined{pdfstringdefDisableCommands}%
    {}%
    {\pdfstringdefDisableCommands{%
       \languageshorthands{system}}%
    }%
}
%    \end{macrocode}
%  \end{macro}
%
%
%  \subsubsection{General}
%
%  \begin{macro}{\FOREIGNLANGUAGE}
%    The package \pkg{fancyhdr} treats the running head and fout lines
%    somewhat differently as the standard classes. A symptom of this is
%    that the command |\foreignlanguage| which \babel\ adds to the
%    marks can end up inside the argument of |\MakeUppercase|. To
%    prevent unexpected results we need to define |\FOREIGNLANGUAGE|
%    here.
% \changes{babel~3.7j}{2003/05/23}{Define \cs{FOREIGNLANGUAGE}
%    unconditionally}
%    \begin{macrocode}
\DeclareRobustCommand{\FOREIGNLANGUAGE}[1]{%
  \lowercase{\foreignlanguage{#1}}}
%</package>
%    \end{macrocode}
%  \end{macro}
%
%  \begin{macro}{\nfss@catcodes}
% \changes{babel~3.5g}{1996/08/18}{Need to add the double quote and
%    acute characters to \cs{nfss@catcodes} to prevent problems when
%    reading in .fd files}
%    \LaTeX's font selection scheme sometimes wants to read font
%    definition files in the middle of processing the document. In
%    order to guard against any characters having the wrong
%    |\catcode|s it always calls |\nfss@catcodes| before loading a
%    file. Unfortunately, the characters |"| and |'| are not dealt
%    with. Therefor we have to add them until \LaTeX\ does that
%    herself.
%    \begin{macrocode}
%<*core|shorthands>
\ifx\nfss@catcodes\@undefined
\else
  \addto\nfss@catcodes{%
    \@makeother\'%
    \@makeother\"%
    }
\fi
%    \end{macrocode}
%  \end{macro}
%
%    \begin{macrocode}
%</core|shorthands>
%    \end{macrocode}
%
% \section{Local Language Configuration}
%
%  \begin{macro}{\loadlocalcfg}
%    At some sites it may be necessary to add site-specific actions to
%    a language definition file. This can be done by creating a file
%    with the same name as the language definition file, but with the
%    extension \file{.cfg}. For instance the file \file{norsk.cfg}
%    will be loaded when the language definition file \file{norsk.ldf}
%    is loaded.
%
% \changes{babel~3.5d}{1995/06/22}{Added macro}
%    \begin{macrocode}
%<*core>
%    \end{macrocode}
%    For plain-based formats we don't want to override the definition
%    of |\loadlocalcfg| from \file{plain.def}.
%    \begin{macrocode}
\ifx\loadlocalcfg\@undefined
  \def\loadlocalcfg#1{%
    \InputIfFileExists{#1.cfg}
           {\typeout{*************************************^^J%
                     * Local config file #1.cfg used^^J%
                     *}%
            }
           {}}
\fi
%    \end{macrocode}
%    Just to be compatible with \LaTeX$\:$2.09 we add a few more lines
%    of code:
%    \begin{macrocode}
\ifx\@unexpandable@protect\@undefined
  \def\@unexpandable@protect{\noexpand\protect\noexpand}
  \long\def \protected@write#1#2#3{%
        \begingroup
         \let\thepage\relax
         #2%
         \let\protect\@unexpandable@protect
         \edef\reserved@a{\write#1{#3}}%
         \reserved@a
        \endgroup
        \if@nobreak\ifvmode\nobreak\fi\fi
  }
\fi
%</core>
%    \end{macrocode}
%  \end{macro}
%
%
% \clearpage
% \section{Driver files for the documented source code}
%
%    Since \babel\ version 3.4 all source files that are part of the
%    \babel\ system can be typeset separately. But to typeset
%    them all in one document, the file \file{babel.drv} can be used.
%    If you only want the information on how to use the \babel\ system
%    and what goodies are provided by the language-specific files, you
%    can run the file \file{user.drv} through \LaTeX\ to get a user
%    guide.
%
% \changes{babel~3.4b}{1994/05/18}{Use the ltxdoc class instead of
%    article}
% \changes{babel~3.7a}{1997/05/21}{Now need packages t1enc and
%    supertabular to be loaded; the documentation for icelandic needs
%    its \file{.ldf} file to be present}
% \changes{babel~3.8a}{2004/02/20}{Also load package url}
%    \begin{macrocode}
%<*driver>
\documentclass{ltxdoc}
\usepackage{url,t1enc,supertabular}
\usepackage[icelandic,english]{babel}
\DoNotIndex{\!,\',\,,\.,\-,\:,\;,\?,\/,\^,\`,\@M}
\DoNotIndex{\@,\@ne,\@m,\@afterheading,\@date,\@endpart}
\DoNotIndex{\@hangfrom,\@idxitem,\@makeschapterhead,\@mkboth}
\DoNotIndex{\@oddfoot,\@oddhead,\@restonecolfalse,\@restonecoltrue}
\DoNotIndex{\@starttoc,\@unused}
\DoNotIndex{\accent,\active}
\DoNotIndex{\addcontentsline,\advance,\Alph,\arabic}
\DoNotIndex{\baselineskip,\begin,\begingroup,\bf,\box,\c@secnumdepth}
\DoNotIndex{\catcode,\centering,\char,\chardef,\clubpenalty}
\DoNotIndex{\columnsep,\columnseprule,\crcr,\csname}
\DoNotIndex{\day,\def,\dimen,\discretionary,\divide,\dp,\do}
\DoNotIndex{\edef,\else,\@empty,\end,\endgroup,\endcsname,\endinput}
\DoNotIndex{\errhelp,\errmessage,\expandafter,\fi,\filedate}
\DoNotIndex{\fileversion,\fmtname,\fnum@figure,\fnum@table,\fontdimen}
\DoNotIndex{\gdef,\global}
\DoNotIndex{\hbox,\hidewidth,\hfil,\hskip,\hspace,\ht,\Huge,\huge}
\DoNotIndex{\ialign,\if@twocolumn,\ifcase,\ifcat,\ifhmode,\ifmmode}
\DoNotIndex{\ifnum,\ifx,\immediate,\ignorespaces,\input,\item}
\DoNotIndex{\kern}
\DoNotIndex{\labelsep,\Large,\large,\labelwidth,\lccode,\leftmargin}
\DoNotIndex{\lineskip,\leavevmode,\let,\list,\ll,\long,\lower}
\DoNotIndex{\m@ne,\mathchar,\mathaccent,\markboth,\month,\multiply}
\DoNotIndex{\newblock,\newbox,\newcount,\newdimen,\newif,\newwrite}
\DoNotIndex{\nobreak,\noexpand,\noindent,\null,\number}
\DoNotIndex{\onecolumn,\or}
\DoNotIndex{\p@,par, \parbox,\parindent,\parskip,\penalty}
\DoNotIndex{\protect,\ps@headings}
\DoNotIndex{\quotation}
\DoNotIndex{\raggedright,\raise,\refstepcounter,\relax,\rm,\setbox}
\DoNotIndex{\section,\setcounter,\settowidth,\scriptscriptstyle}
\DoNotIndex{\sfcode,\sl,\sloppy,\small,\space,\spacefactor,\strut}
\DoNotIndex{\string}
\DoNotIndex{\textwidth,\the,\thechapter,\thefigure,\thepage,\thepart}
\DoNotIndex{\thetable,\thispagestyle,\titlepage,\tracingmacros}
\DoNotIndex{\tw@,\twocolumn,\typeout,\uppercase,\usecounter}
\DoNotIndex{\vbox,\vfil,\vskip,\vspace,\vss}
\DoNotIndex{\widowpenalty,\write,\xdef,\year,\z@,\z@skip}
%    \end{macrocode}
%
%     Here |\dlqq| is defined so that  an example of |"'| can be
%     given.
%    \begin{macrocode}
\makeatletter
\gdef\dlqq{{\setbox\tw@=\hbox{,}\setbox\z@=\hbox{''}%
  \dimen\z@=\ht\z@ \advance\dimen\z@-\ht\tw@
  \setbox\z@=\hbox{\lower\dimen\z@\box\z@}\ht\z@=\ht\tw@
  \dp\z@=\dp\tw@ \box\z@\kern-.04em}}
%    \end{macrocode}
%
%    The code lines are numbered within sections,
%    \begin{macrocode}
%<*!user>
\@addtoreset{CodelineNo}{section}
\renewcommand\theCodelineNo{%
  \reset@font\scriptsize\thesection.\arabic{CodelineNo}}
%    \end{macrocode}
%    which should also be visible in the index; hence this
%    redefinition of a macro from \file{doc.sty}.
%    \begin{macrocode}
\renewcommand\codeline@wrindex[1]{\if@filesw
        \immediate\write\@indexfile
            {\string\indexentry{#1}%
            {\number\c@section.\number\c@CodelineNo}}\fi}
%    \end{macrocode}
%
%    The glossary environment is used or the change log, but its
%    definition needs changing for this document.
%    \begin{macrocode}
\renewenvironment{theglossary}{%
    \glossary@prologue%
    \GlossaryParms \let\item\@idxitem \ignorespaces}%
   {}
%</!user>
\makeatother
%    \end{macrocode}
%
%    A few shorthands used in the documentation
% \changes{babel~3.5g}{1996/07/06}{Added definition of \cs{Babel}}
%    \begin{macrocode}
\font\manual=logo10 % font used for the METAFONT logo, etc.
\newcommand*\MF{{\manual META}\-{\manual FONT}}
\newcommand*\TeXhax{\TeX hax}
\newcommand*\babel{\textsf{babel}}
\newcommand*\Babel{\textsf{Babel}}
\newcommand*\m[1]{\mbox{$\langle$\it#1\/$\rangle$}}
\newcommand*\langvar{\m{lang}}
%    \end{macrocode}
%
%     Some more definitions needed in the documentation.
%    \begin{macrocode}
%\newcommand*\note[1]{\textbf{#1}}
\newcommand*\note[1]{}
\newcommand*\bsl{\protect\bslash}
\newcommand*\Lopt[1]{\textsf{#1}}
\newcommand*\Lenv[1]{\textsf{#1}}
\newcommand*\file[1]{\texttt{#1}}
\newcommand*\cls[1]{\texttt{#1}}
\newcommand*\pkg[1]{\texttt{#1}}
\newcommand*\langdeffile[1]{%
%<-user>  \clearpage
  \DocInput{#1}}
%    \end{macrocode}
%
%    When a full index should be generated uncomment the line with
%    |\EnableCrossrefs|. Beware, processing may take some time.
%    Use |\DisableCrossrefs| when the index is ready.
%    \begin{macrocode}
%  \EnableCrossrefs
\DisableCrossrefs
%    \end{macrocode}
%
%    Inlude the change log.
%    \begin{macrocode}
%<-user>\RecordChanges
%    \end{macrocode}
%    The index should use the linenumbers of the code.
%    \begin{macrocode}
%<-user>\CodelineIndex
%    \end{macrocode}
%
% Set everything in |\MacroFont| instead of |\AltMacroFont|
%    \begin{macrocode}
\setcounter{StandardModuleDepth}{1}
%    \end{macrocode}
%
%    For the user guide we only want the description parts of all the
%    files.
%    \begin{macrocode}
%<+user>\OnlyDescription
%    \end{macrocode}
%    Here starts the document
%    \begin{macrocode}
\begin{document}
\DocInput{babel.dtx}
%    \end{macrocode}
%
%    All the language definition files.
% \changes{babel~3.2e}{1992/07/07}{Added slovak}
% \changes{babel~3.3}{1993/07/11}{Added catalan and galician}
% \changes{babel~3.3}{1993/07/11}{Added turkish}
% \changes{babel~3.4}{1994/02/28}{Added bahasa}
% \changes{babel~3.5a}{1995/02/16}{Added breton, irish, scottish}
% \changes{babel~3.5b}{1995/05/19}{Added lsorbian, usorbian}
% \changes{babel~3.5c}{1995/06/14}{Changed the order of including the
%    language files somewhat (PR1652)}
% \changes{babel~3.5g}{1996/07/06}{Added greek}
% \changes{babel~3.6a}{1996/12/14}{Added welsh}
%^^A \changes{babel~3.6i}{1997/02/07}{Added sanskrit}
% \changes{babel~3.6i}{1997/02/22}{Added basque}
^^A% \changes{babel~3.6i}{1997/02/22}{Added kannada}
% \changes{babel~3.7a}{1997/05/21}{Added icelandic}
% \changes{babel~3.7b}{1998/06/25}{Added Latin}
% \changes{babel~3.7c}{1999/03/09}{Added ukrainian}
% \changes{babel~3.7c}{1999/05/09}{Added hebrew and serbian}
% \changes{babel~3.7e}{1999/11/22}{Added missing hebrew files}
% \changes{babel~3.7f}{2000/09/21}{Added bulgarian}
% \changes{babel~3.7f}{2000/09/26}{Added samin}
% \changes{babel~3.8a}{2004/02/20}{Added interlingua}
% \changes{babel~3.8h}{2005/11/23}{Added albanian and bahasam}
%    \begin{macrocode}
%<+user>\clearpage
\langdeffile{esperanto.dtx}
\langdeffile{interlingua.dtx}
%
\langdeffile{dutch.dtx}
\langdeffile{english.dtx}
\langdeffile{germanb.dtx}
\langdeffile{ngermanb.dtx}
%
\langdeffile{breton.dtx}
\langdeffile{welsh.dtx}
\langdeffile{irish.dtx}
\langdeffile{scottish.dtx}
%
\langdeffile{greek.dtx}
%
\langdeffile{frenchb.dtx}
\langdeffile{italian.dtx}
\langdeffile{latin.dtx}
\langdeffile{portuges.dtx}
\langdeffile{spanish.dtx}
\langdeffile{catalan.dtx}
\langdeffile{galician.dtx}
\langdeffile{basque.dtx}
\langdeffile{romanian.dtx}
%
\langdeffile{danish.dtx}
\langdeffile{icelandic.dtx}
\langdeffile{norsk.dtx}
\langdeffile{swedish.dtx}
\langdeffile{samin.dtx}
%
\langdeffile{finnish.dtx}
\langdeffile{magyar.dtx}
\langdeffile{estonian.dtx}
%
\langdeffile{albanian.dtx}
\langdeffile{croatian.dtx}
\langdeffile{czech.dtx}
\langdeffile{polish.dtx}
\langdeffile{serbian.dtx}
\langdeffile{slovak.dtx}
\langdeffile{slovene.dtx}
\langdeffile{russianb.dtx}
\langdeffile{bulgarian.dtx}
\langdeffile{ukraineb.dtx}
%
\langdeffile{lsorbian.dtx}
\langdeffile{usorbian.dtx}
\langdeffile{turkish.dtx}
%
\langdeffile{hebrew.dtx}
\DocInput{hebinp.dtx}
\DocInput{hebrew.fdd}
\DocInput{heb209.dtx}
\langdeffile{bahasa.dtx}
\langdeffile{bahasam.dtx}
%\langdeffile{sanskrit.dtx}
%\langdeffile{kannada.dtx}
%\langdeffile{nagari.dtx}
%\langdeffile{tamil.dtx}
\clearpage
\DocInput{bbplain.dtx}
%    \end{macrocode}
%    Finally print the index and change log (not for the user guide).
%    \begin{macrocode}
%<*!user>
\clearpage
\def\filename{index}
\PrintIndex
\clearpage
\def\filename{changes}
\PrintChanges
%</!user>
\end{document}
%</driver>
%    \end{macrocode}
%
% \Finale
%
%%
%% \CharacterTable
%%  {Upper-case    \A\B\C\D\E\F\G\H\I\J\K\L\M\N\O\P\Q\R\S\T\U\V\W\X\Y\Z
%%   Lower-case    \a\b\c\d\e\f\g\h\i\j\k\l\m\n\o\p\q\r\s\t\u\v\w\x\y\z
%%   Digits        \0\1\2\3\4\5\6\7\8\9
%%   Exclamation   \!     Double quote  \"     Hash (number) \#
%%   Dollar        \$     Percent       \%     Ampersand     \&
%%   Acute accent  \'     Left paren    \(     Right paren   \)
%%   Asterisk      \*     Plus          \+     Comma         \,
%%   Minus         \-     Point         \.     Solidus       \/
%%   Colon         \:     Semicolon     \;     Less than     \<
%%   Equals        \=     Greater than  \>     Question mark \?
%%   Commercial at \@     Left bracket  \[     Backslash     \\
%%   Right bracket \]     Circumflex    \^     Underscore    \_
%%   Grave accent  \`     Left brace    \{     Vertical bar  \|
%%   Right brace   \}     Tilde         \~}
\endinput

